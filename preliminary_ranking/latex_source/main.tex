% This must be in the first 5 lines to tell arXiv to use pdfLaTeX, which is strongly recommended.
\pdfoutput=1
% In particular, the hyperref package requires pdfLaTeX in order to break URLs across lines.

\documentclass[11pt]{article}

% Change "review" to "final" to generate the final (sometimes called camera-ready) version.
% Change to "preprint" to generate a non-anonymous version with page numbers.
\usepackage[final]{acl}

% Standard package includes
\usepackage{times}
\usepackage{latexsym}

% For proper rendering and hyphenation of words containing Latin characters (including in bib files)
\usepackage[T1]{fontenc}
% For Vietnamese characters
% \usepackage[T5]{fontenc}
% See https://www.latex-project.org/help/documentation/encguide.pdf for other character sets


% This assumes your files are encoded as UTF8
\usepackage[utf8]{inputenc}

% This is not strictly necessary, and may be commented out,
% but it will improve the layout of the manuscript,
% and will typically save some space.
\usepackage{microtype}

% This is also not strictly necessary, and may be commented out.
% However, it will improve the aesthetics of text in
% the typewriter font.
\usepackage{inconsolata}

%Including images in your LaTeX document requires adding
%additional package(s)
\usepackage{graphicx}

% added by VZ
\usepackage{enumitem}
\usepackage{geometry}

\usepackage{mathtools} % for \coloneqq and paired delimiters
\DeclareMathOperator*{\median}{median}
\newcommand{\Q}[2]{Q^{(#1)}_{#2}} % e.g., \Q{m}{25}
\DeclarePairedDelimiter\Set\{\}{}
\DeclarePairedDelimiter\Paren()   % auto-sizing with *-form


% If the title and author information does not fit in the area allocated, uncomment the following
%
\setlength\titlebox{8cm}
%
% and set <dim> to something 5cm or larger.

\usepackage{colortbl}
\usepackage{pifont}
\usepackage{booktabs}
\usepackage{amsmath}
\usepackage{cleveref}



% Define macros for row styles
\newcommand{\opentrack}[1]{\rowcolor{gray!20} #1}
\newcommand{\closedtrack}[1]{\rowcolor{gray!50} #1}
\newcommand{\nonsupporting}[1]{#1 \S}
\newcommand{\validated}{\colorbox{black}{\textcolor{white}{\ding{51}}}}


% Vilém: macro for custom highlighting
\newcommand{\hlc}[2][yellow]{{%
    \colorlet{foo}{#1}%
    \sethlcolor{foo}\hl{#2}}%
}

\usepackage[textsize=footnotesize]{todonotes}
\newcommand{\tk}[1]{\todo[inline,color=green!20!white]{Tom: #1}} 
\newcommand{\VZ}[1]{\todo[inline,color=pink!20!white]{Vilém: #1}} 
\newcommand{\vz}[1]{\todo[color=pink!20!white]{Vilém: #1}} 

\usepackage[normalem]{ulem} % sout, uline
\usepackage{xcolor}
\DeclareRobustCommand{\XXX}[1]{\textcolor{red}{XXX #1}}
  % use \XXX{blabla} for a red comment
\DeclareRobustCommand{\repl}[2]{%
\textcolor{red}{XXX\sout{#1}}%
\textcolor{blue}{\uline{#2}}}
% use \repl{what to replace}{what to use instead} inline for edit suggestions
% \repl{}{} emits also the string XXX so that the PDF can be searched for it
% and no outstanding change is forgotten


\usepackage{soul}
\usepackage{booktabs}
\usepackage{tabularx}
\usepackage[table]{xcolor}


\usepackage{pifont}   % ✓ ✗
\newcommand{\checkmark}{\textcolor{green!60!black}{\ding{51}}} % ✓
\newcommand{\crossmark}{\textcolor{red}{\ding{55}}}          % ✗
\newcommand{\unknown}{\textbf{?}}

\usepackage{amssymb} % gives \blacktriangle
\newcommand{\official}{\(\blacktriangle\)\,}

\newcolumntype{Y}{>{\raggedleft\arraybackslash}X}



\def\OLD{\color{red}}
\def\NEW{\color{black}}


\title{\underline{\textit{Preliminary}} Ranking of WMT25 General Machine Translation Systems}


\author{
  \null \AND
  Tom Kocmi  
  \And
  Eleftherios Avramidis 
  \And
  Rachel Bawden 
  \And
  Ond\v{r}ej Bojar
  \And
  Konstantin Dranch
  \AND
  Anton Dvorkovich 
  \And
  Sergey Dukanov 
  \And
  Natalia Fedorova 
  \And
  Mark Fishel  
  \And
  Markus Freitag
  \AND
  Thamme Gowda
  \And
  Roman Grundkiewicz
  \And
  Barry Haddow  
  \And
  Marzena Karpinska 
  \AND
  Philipp Koehn 
  \And
  Howard Lakougna
  \And
  Jessica Lundin
  \And
  Kenton Murray 
  \And
  Masaaki Nagata
  \AND
  Stefano Perrella 
  \And
  Lorenzo Proietti
  \And
  Martin Popel 
  \And 
  Maja Popovi\'{c}  
  \And
  Parker Riley 
  \AND
  Mariya Shmatova 
  \And
  Stein\th\'{o}r Steingr\'{i}msson 
  \And
  Lisa Yankovskaya 
  \And 
  Vilém Zouhar
% 
  \vspace{2cm}
}



\begin{document}
\maketitle


\section*{Introduction}

We present the \underline{\textbf{\textit{preliminary}}} ranking of the WMT25 General Machine Translation Shared Task, in which MT systems have been evaluated using automatic metrics. As this ranking is based on automatic evaluations, it may be biased in favor of systems that employ re-ranking techniques, such as Quality Estimation re-ranking or Minimum Bayes Risk decoding. The official WMT25 ranking will be based on human evaluation, which is more reliable and will supersede the automatic ranking.

The purpose of this report is not to present the final findings of the General MT task, but rather to share preliminary results with task participants, which may be useful when preparing their system submission papers.

\section*{Types of Systems}

We distinguish two types of MT systems participating in the shared task:
\begin{itemize}%[noitemsep,left=0mm]
\item \textbf{Constrained systems} are those using only publicly available training data and models. The maximum size of their parameter counts is 20B and participants are required to release their weights under an open license. 

\item \textbf{Unconstrained systems} (marked with \hlc[gray!30]{gray}) are all the remaining systems, with no limitations on their training data, model sizes and no requirements to publish their model weights. Systems where the relevant information is not publicly known also fall into the unconstrained category.
\end{itemize}


\section*{Evaluated Systems}

Details of all systems are going to be available in the upcoming WMT25 findings.
In addition to participating systems, we also collect open-weight and proprietary LLMs together with three popular commercial MT systems. For each provider, we selected their largest/best performing model for each of the subtracks (when applicable).

Constrained systems: 
AyaExpanse-8B, CommandR7B, EuroLLM-9B, Gemma-3-12B, Llama-3.1-8B, Mistral-7B, NLLB, Qwen2.5-7B, TowerPlus-9B

Unconstrained systems: AyaExpanse-32B, Claude-4, CommandA, DeepSeek-V3, EuroLLM-22B, Gemma-3-27B, Gemini-2.5-Pro, GPT-4.1, Llama-4-Maverick, Mistral-Medium, ONLINE-B, ONLINE-G, ONLINE-W, Qwen3-235B, TowerPlus-72B

We used a zero shot instruction following approach, translating data on a document-level whenever possible, having a paragraph-level backup for failed translations. The instructions were provided in the blindset. This may hurt some of the systems trained for specific MT instructions such as TowerLLM or EuroLLM, we mark them with [M].

We turned off the reasoning for Qwen3-235B, however, we didn't set the reasoning budget for Gemini-2.5-Pro which increased output tokens count 6.6 times making it the most expensive model in the evaluation.

The code for collecting translations is available at \href{https://github.com/wmt-conference/wmt-collect-translations}{github.com/wmt-conference/wmt-collect-translations} and we marked all systems collected by us with \official{}.

\section*{Evaluated Data}

We evaluated 32 language pairs: half of them will be evaluated by humans, while the other half belong to the multilingual subtrack and will rely solely on automatic ranking.

Most language pairs are in the English-to-X direction and contain approximately 37k words. Each segment contains about 100 words, which typically corresponds to paragraphs, although sometimes natural paragraphs had to be further segmented to meet this constraint. The data are aggregated into documents. The test sets combine material from four domains:
\begin{itemize}
    \item \textbf{News commentary}
    \item \textbf{Social} (texts from social networks, collected with screenshots)
    \item \textbf{Speech} (transcripts of speeches obtained automatically)
    \item \textbf{Literary} (two documents of roughly 5{,}000 words each)
\end{itemize}

Participants could use image and video modalities (when available) to improve their translations; however, their use was not required. Humans providing reference translations were offered these extra modalities, too.
Language pairs with a non-English source have a similar distribution but differ slightly in domains and sizes.

We do not provide sentence splitting; consequently, many segments contain multiple sentences.

We release all data, including references, system outputs, automatic segment scores, or latex sources of this document at: \href{https://github.com/wmt-conference/wmt25-general-mt}{github.com/wmt-conference/wmt25-general-mt}.


\section*{Automatic Ranking}
\label{sec:automatic-ranking}

We call the automatic ranking described in the rest of this section ``AutoRank''. % because AutoRank is later used without introduction
Compared to last year, both the set of automatic metrics and the aggregation procedure changed slightly.

\paragraph{Metrics used.}
For each language pair (except where noted below), we combine three families of evaluation methods:
\begin{itemize}
    \item \textbf{LLM-as-a-Judge (reference-less).} GEMBA-ESA \citep{kocmi-federmann-2023-large} with two independent judges: GPT-4.1 \citep{openai_gpt41_2025} and Command A \citep{cohere2025commandaenterprisereadylarge}, both used in a reference-less setting.
    \item \textbf{Trained reference-based metrics.} Two reference-based supervised metrics explicitly trained to approximate human judgments of translation quality: MetricX-24-Hybrid-XL \citep{juraska-etal-2024-metricx} and XCOMET-XL \citep{guerreiro-etal-2024-xcomet}.
    \item \textbf{Trained Quality Estimation (QE).} One QE metric trained to mimic human judgments without a reference: CometKiwi-XL \citep{rei-etal-2023-scaling}.
\end{itemize}
Including both reference-based and reference-less (or QE) methods balances complementary failure modes: reference-based metrics typically achieve higher correlation with human judgments when references are high-quality, whereas reference-less methods reduce susceptibility to reference bias when references are suboptimal \citep{freitag-etal-2023-results}. A known pitfall for multilingual QE is that it can be fooled by fluent output in the wrong target language; in contrast, the GEMBA-ESA prompt explicitly specify the target language, which should mitigate this issue.

The use of LLM-as-a-judge metrics (GEMBA-ESA) is intended to mitigate biases by models employing re-ranking or similar techniques during training or inference. Nevertheless, some systems incorporated GEMBA directly as their reward model.

For each metric and language pair, the system-level score of an MT system is computed as the average of the metric’s paragraph-level (segment-level) scores over all translations the system produced on the test set for that language pair. For language pairs without human references, we exclude CometKiwi-XL from the corresponding AutoRank computation, since MetricX-24-Hybrid-XL and XCOMET-XL are hybrid metrics and can be run in reference-less (QE) mode, thus already providing the QE signal from trained metrics for those pairs.

\paragraph{Low-resource exception.}
For the two most low-resource target languages, i.e., \textbf{Bhojpuri} and \textbf{Maasai}, we rely solely on \texttt{chrF++} \citep{popovic-2017-chrf} because it is not known whether the above metrics are reliable in these settings \citep{falcao-etal-2024-comet,singh-etal-2024-good,wang-etal-2024-evaluating,sindhujan-etal-2025-llms} and human references are available. We compute \texttt{chrF++} using \texttt{sacrebleu} \citep{post-2018-call}.
% \footnote{\url{https://github.com/mjpost/sacrebleu}.} .

\paragraph{From system-level scores to AutoRank}
To combine the metrics into a single score, we first normalize them using median-interpercentile scaling to address differences in scale and reduce the influence of low-performing outliers.
% Ondrej added the detail about the normalization, because simple normalization actually \emph{inflates} the influence of outliers on the overall evaluation in this sense: a single very bad outlier ``shifts the scales'', taking the lowest accepted score but pushing all other systems into a high score range which can get extremely narrow.
We then compute the average using equal weights. Finally, we linearly rescale the results to the range from 1 to $N$ systems. A detailed description is provided below:

Let $S$ be the set of submitted systems for a given language pair, $|S|=N$, and let $M$ be the set of automatic metrics used for that language pair (for Bhojpuri and Maasai, $|M|=1$). For each metric $m\in M$ and system $s\in S$, we compute a system-level score $x^{(m)}_s$ as the average of that metric over all available test segments. To combine scores across metrics, we first map them to a common scale; however, classical min–max normalization is highly sensitive to outliers. To downweight extremes without discarding any system, we apply a \emph{median--inter\-percentile} scaling to each metric $m$:
\begin{subequations}\label{eq:robust}
\begin{align}
\tilde{x}^{(m)} &= \median\,\Set*{x^{(m)}_s \mid s \in S},\\[2pt]
D^{(m)} &= \max\,\Paren*{\varepsilon,\, \Q{m}{100} - \Q{m}{25}},\\[2pt]
z^{(m)}_s &= \frac{x^{(m)}_s - \tilde{x}^{(m)}}{D^{(m)}}.
\end{align}
\end{subequations}
Where $\varepsilon>0$ and $\Q{m}{p}$ denotes the $p$-th percentile of $\{x^{(m)}_s : s\in S\}$. Importantly, Eq.~\eqref{eq:robust} is continuous and monotonic: it keeps all systems and preserves their order within each metric. Then, for each system, we average the robust-scaled values across metrics:
\begin{equation}
    \bar{z}_s \;=\; \frac{1}{|M|}\sum_{m\in M} z^{(m)}_s .
    \label{eq:avgz}
\end{equation}
Averaging after robust scaling yields a single comparable score that preserves the magnitude of performance differences between systems (in standardized units) while preventing any single metric’s outliers from dominating. Finally, for readability and to follow the WMT convention from last year (lower is better in AutoRank, i.e., $1$ is best and $N$ worst), we apply a final linear mapping to the set $\{\bar{z}_s\}_{s\in S}$. Specifically, within $\{\bar{z}_s\}_{s\in S}$ the system with the highest average score is assigned $1$, the system with the lowest average score is assigned $N$, and all remaining systems are placed linearly between these two endpoints. This remapping is applied only once—after the cross-metric aggregation—so it preserves the ordering and relative spacing between systems while retaining the outlier mitigation provided by the robust scaling. We refer to the resulting value as AutoRank in the various tables.


\section*{Human Evaluation}

This year, we received 36 unique teams,\footnote{We received 43 different teams, however, 7 of them have withdrew or been disqualified} the highest number of participants ever. We are not able to evaluate them all with human annotators.
Therefore, we select a subset of about 18 systems per language pair (some language pairs have this system count higher) which will be evaluated by humans with the Error Span Annotation protocol \citep{kocmi2024errorspanannotationbalanced}.
For the remaining systems, AutoRank is going to be the official final ranking.

When selecting systems for human evaluation, we prioritize constrained systems over unconstrained systems.
Therefore, we select the systems for human evaluation based on the following two rules:

\begin{enumerate}[noitemsep]
\item We select top eight constrained systems ignoring unconstrained systems.
\item Then, we take the top performing systems until we have total of 18 systems selected for human evaluation.
\end{enumerate}




\section*{Limitations}


A key limitation of our evaluation is that some models have been optimized for the very metrics we employ in AutoRank, either during training or at inference time \cite{freitag-etal-2022-high, finkelstein2024mbr}. This can result in artificially inflated scores that do not accurately reflect a model's true capabilities \cite{kovacs-etal-2024-mitigating}. To mitigate this issue, we aggregate the assessments from multiple learned metrics and LLM-as-a-judge approaches. However, even this strategy has shortcomings. First, scores from different learned metrics often exhibit high correlation among themselves. Second, LLM-as-a-judge approaches, including the Gemba-ESA we use, may also have been utilized to optimize machine translation models.

Another limitation is that we use automatic metrics to evaluate entire paragraphs, whereas their reliability is typically established at the sentence level. Additionally, learned metrics struggle when evaluating translation directions involving low-resource languages, such as English-to-Bhojpuri and English-to-Maasai. Therefore, we evaluate these language pairs using chrF++. However, chrF++ is a surface-level metric that, like BLEU, has been repeatedly shown to correlate poorly with human judgments \citep{kocmi-etal-2021-ship, freitag-etal-2022-results, freitag-etal-2023-results}.

Furthermore, our automatic evaluation is conducted at the paragraph level, without incorporating document-level context. This may lead to inflated scores for systems that translate the dataset paragraph by paragraph, disregarding dependencies and coherence across paragraphs.

The LLM-as-a-judge approach also depends on the language performance of the underlying LLMs. For our evaluation, we selected two top-performing multilingual systems: GPT-4.1 and Command A. Command A officially supports only 23 languages \citep{cohere2025commandaenterprisereadylarge}, while the set of languages supported by GPT-4.1 is not publicly documented. Nevertheless, as both metrics correlate well across all languages and show strong agreement with other evaluation metrics, we retained them as judges for all 30 language pairs.

Finally, using automatically generated speech recognition transcripts as source text in the speech domain introduces additional noise, as the evaluation metrics are unlikely to be robust to ASR errors. Consequently, systems that handle the speech domain well may receive lower scores if their outputs diverge from the ASR transcript, even when their translations are correct.

Given these issues, along with the well-documented biases and limitations of automatic metrics \citep{karpinska-etal-2022-demetr, moghe2024machine}, human evaluation remains indispensable. Therefore, the results from human assessments will supersede the automatic rankings presented here.


\section*{Acknowledgement}
This report would not have been possible without the partnership with Árni Magnússon Institute for Icelandic Studies, Charles University, Cohere, Custom.MT, Dubformer, Gates Foundation, Google, Institute of the Estonian Language, Microsoft, NTT, Toloka, University of Tartu, University of Tokyo.
Furthermore, we are grateful to Toshiaki Nakazawa.


% \newgeometry{top=0cm,left=0cm,right=0cm,bottom=0cm}
% added by VZ
\newcolumntype{L}[1]{>{\raggedright\let\newline\\\arraybackslash\hspace{0pt}}m{#1}}
\newcolumntype{C}[1]{>{\centering\let\newline\\\arraybackslash\hspace{0pt}}m{#1}}
\newcolumntype{R}[1]{>{\raggedleft\let\newline\\\arraybackslash\hspace{0pt}}m{#1}}



\begin{table*}
\small
\begin{tabularx}{\textwidth}{lYYYYYYYYY}
\toprule
\multicolumn{10}{c}{\textbf{English-Egyptian Arabic}} \\
\midrule
System Name & LP Supported & Params. (B) & Humeval? & AutoRank $\downarrow$ & CometKiwi-XL $\uparrow$ & GEMBA-ESA-CMDA $\uparrow$ & GEMBA-ESA-GPT4.1 $\uparrow$ & MetricX-24-Hybrid-XL $\uparrow$ & XCOMET-XL $\uparrow$ \\
\midrule
Shy-hunyuan-MT & \checkmark & 7 & \checkmark & \cellcolor{green!100!red!0}1.0 & \cellcolor{green!100!red!0}0.658 & \cellcolor{green!83!red!17}76.3 & \cellcolor{green!67!red!33}75.0 & \cellcolor{green!100!red!0}-5.7 & \cellcolor{green!100!red!0}0.388 \\
Wenyiil & \checkmark & 14 & \checkmark & \cellcolor{green!90!red!10}2.5 & \cellcolor{green!94!red!6}0.65 & \cellcolor{green!96!red!4}79.2 & \cellcolor{green!61!red!39}73.3 & \cellcolor{green!83!red!17}-6.4 & \cellcolor{green!70!red!30}0.337 \\
Algharb & \checkmark & 14 & \checkmark & \cellcolor{green!89!red!11}2.6 & \cellcolor{green!90!red!10}0.645 & \cellcolor{green!100!red!0}80.0 & \cellcolor{green!63!red!37}73.9 & \cellcolor{green!81!red!19}-6.5 & \cellcolor{green!64!red!36}0.328 \\
\rowcolor{gray!30}
GemTrans & \checkmark & 27 & \checkmark & \cellcolor{green!83!red!17}3.4 & \cellcolor{green!89!red!11}0.644 & \cellcolor{green!67!red!33}73.0 & \cellcolor{green!48!red!52}69.6 & \cellcolor{green!93!red!7}-6.0 & \cellcolor{green!74!red!26}0.345 \\
\rowcolor{gray!30}
CommandA-WMT & \checkmark & 111 & \checkmark & \cellcolor{green!79!red!21}4.0 & \cellcolor{green!72!red!28}0.621 & \cellcolor{green!90!red!10}77.8 & \cellcolor{green!68!red!32}75.4 & \cellcolor{green!69!red!31}-7.0 & \cellcolor{green!54!red!46}0.311 \\
\rowcolor{gray!30}
UvA-MT & \checkmark & 12 & \checkmark & \cellcolor{green!79!red!21}4.1 & \cellcolor{green!84!red!16}0.637 & \cellcolor{green!74!red!26}74.4 & \cellcolor{green!61!red!39}73.4 & \cellcolor{green!67!red!33}-7.1 & \cellcolor{green!62!red!38}0.325 \\
Yolu & \checkmark & 14 & \checkmark & \cellcolor{green!70!red!30}5.4 & \cellcolor{green!100!red!0}0.658 & \cellcolor{green!43!red!57}67.8 & \cellcolor{green!27!red!73}63.9 & \cellcolor{green!79!red!21}-6.6 & \cellcolor{green!61!red!39}0.323 \\
\rowcolor{gray!30}
\official Gemini-2.5-Pro & \checkmark & \unknown & \checkmark & \cellcolor{green!68!red!32}5.6 & \cellcolor{green!20!red!80}0.552 & \cellcolor{green!98!red!2}79.5 & \cellcolor{green!100!red!0}84.5 & \cellcolor{green!55!red!45}-7.6 & \cellcolor{green!28!red!72}0.267 \\
\rowcolor{gray!30}
\official ONLINE-B & \checkmark & \unknown & \checkmark & \cellcolor{green!63!red!37}6.4 & \cellcolor{green!77!red!23}0.627 & \cellcolor{green!55!red!45}70.4 & \cellcolor{green!40!red!60}67.4 & \cellcolor{green!67!red!33}-7.1 & \cellcolor{green!40!red!60}0.288 \\
\rowcolor{gray!30}
\official GPT-4.1 & \checkmark & \unknown & \checkmark & \cellcolor{green!62!red!38}6.5 & \cellcolor{green!6!red!94}0.534 & \cellcolor{green!92!red!8}78.4 & \cellcolor{green!99!red!1}84.1 & \cellcolor{green!50!red!50}-7.8 & \cellcolor{green!27!red!73}0.265 \\
\rowcolor{gray!30}
\official DeepSeek-V3 & \unknown & 671 & \checkmark & \cellcolor{green!59!red!41}6.9 & \cellcolor{green!36!red!64}0.573 & \cellcolor{green!73!red!27}74.2 & \cellcolor{green!69!red!31}75.7 & \cellcolor{green!52!red!48}-7.7 & \cellcolor{green!32!red!68}0.273 \\
\rowcolor{gray!30}
\official Mistral-Medium & \checkmark & \unknown & \checkmark & \cellcolor{green!55!red!45}7.5 & \cellcolor{green!45!red!55}0.586 & \cellcolor{green!61!red!39}71.7 & \cellcolor{green!52!red!48}71.0 & \cellcolor{green!50!red!50}-7.8 & \cellcolor{green!32!red!68}0.274 \\
\rowcolor{gray!30}
\official Claude-4 & \checkmark & \unknown & \checkmark & \cellcolor{green!54!red!46}7.6 & \cellcolor{green!20!red!80}0.552 & \cellcolor{green!84!red!16}76.5 & \cellcolor{green!84!red!16}80.0 & \cellcolor{green!33!red!67}-8.5 & \cellcolor{green!15!red!85}0.246 \\
SRPOL & \crossmark & 12 & \checkmark & \cellcolor{green!52!red!48}7.9 & \cellcolor{green!87!red!13}0.641 & \cellcolor{green!33!red!67}65.7 & \cellcolor{green!20!red!80}61.7 & \cellcolor{green!50!red!50}-7.8 & \cellcolor{green!39!red!61}0.286 \\
\rowcolor{gray!30}
\official CommandA & \checkmark & 111 & \checkmark & \cellcolor{green!50!red!50}8.3 & \cellcolor{green!5!red!95}0.533 & \cellcolor{green!80!red!20}75.8 & \cellcolor{green!84!red!16}80.0 & \cellcolor{green!33!red!67}-8.5 & \cellcolor{green!11!red!89}0.238 \\
\rowcolor{gray!30}
\official AyaExpanse-32B & \checkmark & 32 &  & \cellcolor{green!49!red!51}8.4 & \cellcolor{green!45!red!55}0.585 & \cellcolor{green!56!red!44}70.7 & \cellcolor{green!45!red!55}68.8 & \cellcolor{green!43!red!57}-8.1 & \cellcolor{green!24!red!76}0.261 \\
\rowcolor{gray!30}
\official ONLINE-W & \unknown & \unknown &  & \cellcolor{green!45!red!55}9.0 & \cellcolor{green!61!red!39}0.607 & \cellcolor{green!42!red!58}67.7 & \cellcolor{green!28!red!72}64.0 & \cellcolor{green!40!red!60}-8.2 & \cellcolor{green!23!red!77}0.258 \\
\official AyaExpanse-8B & \checkmark & 8 & \checkmark & \cellcolor{green!40!red!60}9.7 & \cellcolor{green!53!red!47}0.596 & \cellcolor{green!35!red!65}66.1 & \cellcolor{green!19!red!81}61.6 & \cellcolor{green!40!red!60}-8.2 & \cellcolor{green!23!red!77}0.259 \\
\rowcolor{gray!30}
\official Qwen3-235B & \checkmark & 235 &  & \cellcolor{green!33!red!67}10.7 & \cellcolor{green!34!red!66}0.571 & \cellcolor{green!35!red!65}66.1 & \cellcolor{green!28!red!72}64.1 & \cellcolor{green!29!red!71}-8.7 & \cellcolor{green!16!red!84}0.247 \\
\rowcolor{gray!30}
\official Gemma-3-27B & \checkmark & 27 &  & \cellcolor{green!33!red!67}10.7 & \cellcolor{green!17!red!83}0.549 & \cellcolor{green!29!red!71}64.8 & \cellcolor{green!25!red!75}63.3 & \cellcolor{green!31!red!69}-8.6 & \cellcolor{green!36!red!64}0.281 \\
\rowcolor{gray!30}
\official EuroLLM-22B-pre.[M] & \checkmark & 22 &  & \cellcolor{green!33!red!67}10.7 & \cellcolor{green!50!red!50}0.592 & \cellcolor{green!25!red!75}64.0 & \cellcolor{green!15!red!85}60.5 & \cellcolor{green!33!red!67}-8.5 & \cellcolor{green!15!red!85}0.246 \\
IRB-MT & \checkmark & 12 & \checkmark & \cellcolor{green!32!red!68}10.8 & \cellcolor{green!5!red!95}0.532 & \cellcolor{green!48!red!52}69.0 & \cellcolor{green!40!red!60}67.5 & \cellcolor{green!33!red!67}-8.5 & \cellcolor{green!10!red!90}0.236 \\
\rowcolor{gray!30}
\official Llama-4-Maverick & \checkmark & 400 &  & \cellcolor{green!30!red!70}11.1 & \cellcolor{green!0!red!100}0.526 & \cellcolor{green!43!red!57}67.9 & \cellcolor{green!49!red!51}70.0 & \cellcolor{green!26!red!74}-8.8 & \cellcolor{green!8!red!92}0.234 \\
\rowcolor{gray!30}
IR-MultiagentMT & \crossmark & \unknown &  & \cellcolor{green!29!red!71}11.3 & \cellcolor{green!13!red!87}0.543 & \cellcolor{green!34!red!66}66.0 & \cellcolor{green!29!red!71}64.2 & \cellcolor{green!29!red!71}-8.7 & \cellcolor{green!16!red!84}0.247 \\
\official CommandR7B & \checkmark & 7 & \checkmark & \cellcolor{green!29!red!71}11.3 & \cellcolor{green!47!red!53}0.588 & \cellcolor{green!19!red!81}62.7 & \cellcolor{green!10!red!90}59.0 & \cellcolor{green!26!red!74}-8.8 & \cellcolor{green!17!red!83}0.248 \\
\official Gemma-3-12B & \checkmark & 12 &  & \cellcolor{green!26!red!74}11.7 & \cellcolor{green!2!red!98}0.529 & \cellcolor{green!43!red!57}67.9 & \cellcolor{green!40!red!60}67.6 & \cellcolor{green!21!red!79}-9.0 & \cellcolor{green!0!red!100}0.22 \\
\official EuroLLM-9B[M] & \checkmark & 9 &  & \cellcolor{green!10!red!90}14.0 & \cellcolor{green!17!red!83}0.548 & \cellcolor{green!0!red!100}58.7 & \cellcolor{green!0!red!100}54.5 & \cellcolor{green!14!red!86}-9.3 & \cellcolor{green!8!red!92}0.233 \\
\rowcolor{gray!30}
\official TowerPlus-72B[M] & \crossmark & 72 &  & \cellcolor{green!0!red!100}15.5 & \cellcolor{green!6!red!94}0.534 & \cellcolor{green!0!red!100}58.2 & \cellcolor{green!0!red!100}54.0 & \cellcolor{green!0!red!100}-10.5 & \cellcolor{green!2!red!98}0.224 \\
\rowcolor{gray!30}
TranssionTranslate & \unknown & \unknown &  & \cellcolor{green!0!red!100}15.8 & \cellcolor{green!0!red!100}0.501 & \cellcolor{green!1!red!99}59.0 & \cellcolor{green!5!red!95}57.4 & \cellcolor{green!0!red!100}-9.9 & \cellcolor{green!0!red!100}0.2 \\
\rowcolor{gray!30}
TranssionMT & \checkmark & 1 &  & \cellcolor{green!0!red!100}16.9 & \cellcolor{green!0!red!100}0.488 & \cellcolor{green!0!red!100}58.5 & \cellcolor{green!0!red!100}56.1 & \cellcolor{green!0!red!100}-10.4 & \cellcolor{green!0!red!100}0.194 \\
\official NLLB & \checkmark & 1 &  & \cellcolor{green!0!red!100}18.0 & \cellcolor{green!0!red!100}0.499 & \cellcolor{green!0!red!100}53.9 & \cellcolor{green!0!red!100}51.2 & \cellcolor{green!0!red!100}-10.8 & \cellcolor{green!0!red!100}0.201 \\
SalamandraTA & \checkmark & 8 &  & \cellcolor{green!0!red!100}20.1 & \cellcolor{green!0!red!100}0.492 & \cellcolor{green!0!red!100}50.0 & \cellcolor{green!0!red!100}44.4 & \cellcolor{green!0!red!100}-11.4 & \cellcolor{green!0!red!100}0.195 \\
\rowcolor{gray!30}
\official ONLINE-G & \checkmark & \unknown &  & \cellcolor{green!0!red!100}22.6 & \cellcolor{green!0!red!100}0.445 & \cellcolor{green!0!red!100}53.5 & \cellcolor{green!0!red!100}48.3 & \cellcolor{green!0!red!100}-13.5 & \cellcolor{green!0!red!100}0.152 \\
\official Llama-3.1-8B & \crossmark & 8 &  & \cellcolor{green!0!red!100}22.8 & \cellcolor{green!0!red!100}0.458 & \cellcolor{green!0!red!100}45.5 & \cellcolor{green!0!red!100}41.8 & \cellcolor{green!0!red!100}-12.3 & \cellcolor{green!0!red!100}0.18 \\
\official Qwen2.5-7B & \checkmark & 7 &  & \cellcolor{green!0!red!100}24.0 & \cellcolor{green!0!red!100}0.436 & \cellcolor{green!0!red!100}44.5 & \cellcolor{green!0!red!100}39.3 & \cellcolor{green!0!red!100}-12.6 & \cellcolor{green!0!red!100}0.176 \\
\official TowerPlus-9B[M] & \crossmark & 9 &  & \cellcolor{green!0!red!100}31.9 & \cellcolor{green!0!red!100}0.337 & \cellcolor{green!0!red!100}31.1 & \cellcolor{green!0!red!100}26.9 & \cellcolor{green!0!red!100}-15.2 & \cellcolor{green!0!red!100}0.162 \\
\official Mistral-7B & \crossmark & 7 &  & \cellcolor{green!0!red!100}37.0 & \cellcolor{green!0!red!100}0.262 & \cellcolor{green!0!red!100}27.9 & \cellcolor{green!0!red!100}23.2 & \cellcolor{green!0!red!100}-18.4 & \cellcolor{green!0!red!100}0.157 \\
\bottomrule
\end{tabularx}
\end{table*}


\begin{table*}
\small
\begin{tabularx}{\textwidth}{lYYYYY}
\toprule
\multicolumn{6}{c}{\textbf{English-Bhojpuri}} \\
\midrule
System Name & LP Supported & Params. (B) & Humeval? & AutoRank $\downarrow$ & chrF++ $\uparrow$ \\
\midrule
\rowcolor{gray!30}
\official Gemini-2.5-Pro & \checkmark & \unknown & \checkmark & \cellcolor{green!100!red!0}1.0 & \cellcolor{green!100!red!0}40.6 \\
Wenyiil & \checkmark & 14 & \checkmark & \cellcolor{green!88!red!12}2.5 & \cellcolor{green!88!red!12}38.9 \\
Algharb & \checkmark & 14 & \checkmark & \cellcolor{green!85!red!15}2.8 & \cellcolor{green!85!red!15}38.6 \\
\rowcolor{gray!30}
\official ONLINE-B & \checkmark & \unknown & \checkmark & \cellcolor{green!75!red!25}4.1 & \cellcolor{green!75!red!25}37.1 \\
\rowcolor{gray!30}
TranssionTranslate & \unknown & \unknown & \checkmark & \cellcolor{green!72!red!28}4.4 & \cellcolor{green!73!red!27}36.9 \\
\rowcolor{gray!30}
\official Claude-4 & \unknown & \unknown & \checkmark & \cellcolor{green!71!red!29}4.5 & \cellcolor{green!72!red!28}36.7 \\
\rowcolor{gray!30}
\official DeepSeek-V3 & \unknown & 671 & \checkmark & \cellcolor{green!66!red!34}5.1 & \cellcolor{green!67!red!33}36.0 \\
\rowcolor{gray!30}
\official GPT-4.1 & \unknown & \unknown & \checkmark & \cellcolor{green!63!red!37}5.5 & \cellcolor{green!64!red!36}35.6 \\
Yolu & \checkmark & 14 & \checkmark & \cellcolor{green!62!red!38}5.6 & \cellcolor{green!62!red!38}35.4 \\
\rowcolor{gray!30}
TranssionMT & \checkmark & 1 & \checkmark & \cellcolor{green!57!red!43}6.2 & \cellcolor{green!58!red!42}34.8 \\
\rowcolor{gray!30}
\official Llama-4-Maverick & \checkmark & 400 & \checkmark & \cellcolor{green!55!red!45}6.5 & \cellcolor{green!55!red!45}34.4 \\
\rowcolor{gray!30}
\official CommandA & \crossmark & 111 & \checkmark & \cellcolor{green!55!red!45}6.5 & \cellcolor{green!55!red!45}34.4 \\
\official NLLB & \checkmark & 1 & \checkmark & \cellcolor{green!54!red!46}6.6 & \cellcolor{green!54!red!46}34.3 \\
\rowcolor{gray!30}
\official Gemma-3-27B & \unknown & 27 & \checkmark & \cellcolor{green!40!red!60}8.3 & \cellcolor{green!40!red!60}32.4 \\
\rowcolor{gray!30}
CommandA-WMT & \crossmark & 111 &  & \cellcolor{green!36!red!64}8.8 & \cellcolor{green!36!red!64}31.8 \\
COILD-BHO & \checkmark & 7 & \checkmark & \cellcolor{green!35!red!65}8.9 & \cellcolor{green!36!red!64}31.8 \\
\rowcolor{gray!30}
\official Mistral-Medium & \unknown & \unknown &  & \cellcolor{green!34!red!66}9.0 & \cellcolor{green!35!red!65}31.6 \\
\rowcolor{gray!30}
\official Qwen3-235B & \crossmark & 235 &  & \cellcolor{green!17!red!83}11.1 & \cellcolor{green!17!red!83}29.2 \\
IRB-MT & \checkmark & 12 & \checkmark & \cellcolor{green!15!red!85}11.4 & \cellcolor{green!15!red!85}28.9 \\
\rowcolor{gray!30}
\official AyaExpanse-32B & \crossmark & 32 &  & \cellcolor{green!15!red!85}11.4 & \cellcolor{green!15!red!85}28.9 \\
Shy-hunyuan-MT & \checkmark & 7 & \checkmark & \cellcolor{green!14!red!86}11.5 & \cellcolor{green!14!red!86}28.8 \\
\rowcolor{gray!30}
GemTrans & \checkmark & 27 &  & \cellcolor{green!11!red!89}11.9 & \cellcolor{green!11!red!89}28.3 \\
SalamandraTA & \checkmark & 8 & \checkmark & \cellcolor{green!9!red!91}12.1 & \cellcolor{green!10!red!90}28.2 \\
\official Gemma-3-12B & \unknown & 12 &  & \cellcolor{green!7!red!93}12.3 & \cellcolor{green!8!red!92}27.9 \\
\official TowerPlus-9B[M] & \crossmark & 9 &  & \cellcolor{green!4!red!96}12.7 & \cellcolor{green!4!red!96}27.4 \\
\rowcolor{gray!30}
\official TowerPlus-72B[M] & \crossmark & 72 &  & \cellcolor{green!3!red!97}12.8 & \cellcolor{green!3!red!97}27.3 \\
\rowcolor{gray!30}
\official EuroLLM-22B-pre.[M] & \crossmark & 22 &  & \cellcolor{green!0!red!100}13.6 & \cellcolor{green!0!red!100}26.4 \\
\official EuroLLM-9B[M] & \crossmark & 9 &  & \cellcolor{green!0!red!100}14.7 & \cellcolor{green!0!red!100}25.2 \\
\rowcolor{gray!30}
IR-MultiagentMT & \crossmark & \unknown &  & \cellcolor{green!0!red!100}15.9 & \cellcolor{green!0!red!100}23.9 \\
\official CommandR7B & \crossmark & 7 &  & \cellcolor{green!0!red!100}16.7 & \cellcolor{green!0!red!100}22.9 \\
\official AyaExpanse-8B & \crossmark & 8 &  & \cellcolor{green!0!red!100}16.7 & \cellcolor{green!0!red!100}22.9 \\
\official Qwen2.5-7B & \unknown & 7 &  & \cellcolor{green!0!red!100}17.7 & \cellcolor{green!0!red!100}21.8 \\
\official Mistral-7B & \crossmark & 7 &  & \cellcolor{green!0!red!100}20.9 & \cellcolor{green!0!red!100}18.2 \\
\rowcolor{gray!30}
UvA-MT & \checkmark & 12 &  & \cellcolor{green!0!red!100}28.4 & \cellcolor{green!0!red!100}9.7 \\
\official Llama-3.1-8B & \crossmark & 8 &  & \cellcolor{green!0!red!100}35.0 & \cellcolor{green!0!red!100}2.3 \\
\bottomrule
\end{tabularx}
\end{table*}


\begin{table*}
\small
\begin{tabularx}{\textwidth}{lYYYYYYYYY}
\toprule
\multicolumn{10}{c}{\textbf{English-Czech}} \\
\midrule
System Name & LP Supported & Params. (B) & Humeval? & AutoRank $\downarrow$ & CometKiwi-XL $\uparrow$ & GEMBA-ESA-CMDA $\uparrow$ & GEMBA-ESA-GPT4.1 $\uparrow$ & MetricX-24-Hybrid-XL $\uparrow$ & XCOMET-XL $\uparrow$ \\
\midrule
Shy-hunyuan-MT & \checkmark & 7 & \checkmark & \cellcolor{green!100!red!0}1.0 & \cellcolor{green!100!red!0}0.658 & \cellcolor{green!97!red!3}83.7 & \cellcolor{green!91!red!9}89.4 & \cellcolor{green!100!red!0}-5.5 & \cellcolor{green!100!red!0}0.639 \\
\rowcolor{gray!30}
\official Gemini-2.5-Pro & \checkmark & \unknown & \checkmark & \cellcolor{green!85!red!15}3.4 & \cellcolor{green!74!red!26}0.633 & \cellcolor{green!98!red!2}83.8 & \cellcolor{green!100!red!0}91.5 & \cellcolor{green!80!red!20}-6.2 & \cellcolor{green!64!red!36}0.574 \\
\rowcolor{gray!30}
CommandA-WMT & \checkmark & 111 & \checkmark & \cellcolor{green!84!red!16}3.5 & \cellcolor{green!87!red!13}0.645 & \cellcolor{green!83!red!17}81.3 & \cellcolor{green!77!red!23}86.2 & \cellcolor{green!86!red!14}-6.0 & \cellcolor{green!75!red!25}0.594 \\
\rowcolor{gray!30}
\official GPT-4.1 & \checkmark & \unknown & \checkmark & \cellcolor{green!82!red!18}3.9 & \cellcolor{green!71!red!29}0.63 & \cellcolor{green!100!red!0}84.2 & \cellcolor{green!92!red!8}89.7 & \cellcolor{green!68!red!32}-6.6 & \cellcolor{green!65!red!35}0.576 \\
Wenyiil & \checkmark & 14 & \checkmark & \cellcolor{green!79!red!21}4.4 & \cellcolor{green!87!red!13}0.645 & \cellcolor{green!72!red!28}79.4 & \cellcolor{green!78!red!22}86.3 & \cellcolor{green!74!red!26}-6.4 & \cellcolor{green!71!red!29}0.586 \\
\rowcolor{gray!30}
\official DeepSeek-V3 & \unknown & 671 & \checkmark & \cellcolor{green!75!red!25}5.0 & \cellcolor{green!69!red!31}0.628 & \cellcolor{green!84!red!16}81.4 & \cellcolor{green!81!red!19}87.0 & \cellcolor{green!71!red!29}-6.5 & \cellcolor{green!59!red!41}0.565 \\
\rowcolor{gray!30}
GemTrans & \checkmark & 27 & \checkmark & \cellcolor{green!75!red!25}5.0 & \cellcolor{green!77!red!23}0.636 & \cellcolor{green!56!red!44}76.6 & \cellcolor{green!58!red!42}81.8 & \cellcolor{green!91!red!9}-5.8 & \cellcolor{green!76!red!24}0.596 \\
Algharb & \checkmark & 14 & \checkmark & \cellcolor{green!68!red!32}6.2 & \cellcolor{green!68!red!32}0.627 & \cellcolor{green!72!red!28}79.4 & \cellcolor{green!72!red!28}85.0 & \cellcolor{green!60!red!40}-6.9 & \cellcolor{green!52!red!48}0.552 \\
Yolu & \checkmark & 14 & \checkmark & \cellcolor{green!67!red!33}6.3 & \cellcolor{green!93!red!7}0.651 & \cellcolor{green!44!red!56}74.6 & \cellcolor{green!44!red!56}78.6 & \cellcolor{green!71!red!29}-6.5 & \cellcolor{green!68!red!32}0.582 \\
\rowcolor{gray!30}
UvA-MT & \checkmark & 12 & \checkmark & \cellcolor{green!67!red!33}6.4 & \cellcolor{green!78!red!22}0.637 & \cellcolor{green!60!red!40}77.3 & \cellcolor{green!63!red!37}82.9 & \cellcolor{green!60!red!40}-6.9 & \cellcolor{green!57!red!43}0.562 \\
\rowcolor{gray!30}
\official Mistral-Medium & \unknown & \unknown & \checkmark & \cellcolor{green!63!red!37}7.0 & \cellcolor{green!62!red!38}0.621 & \cellcolor{green!66!red!34}78.4 & \cellcolor{green!69!red!31}84.4 & \cellcolor{green!54!red!46}-7.1 & \cellcolor{green!49!red!51}0.547 \\
SRPOL & \checkmark & 12 & \checkmark & \cellcolor{green!53!red!47}8.6 & \cellcolor{green!82!red!18}0.641 & \cellcolor{green!34!red!66}72.9 & \cellcolor{green!34!red!66}76.2 & \cellcolor{green!48!red!52}-7.3 & \cellcolor{green!52!red!48}0.552 \\
\rowcolor{gray!30}
\official CommandA & \checkmark & 111 & \checkmark & \cellcolor{green!53!red!47}8.6 & \cellcolor{green!49!red!51}0.609 & \cellcolor{green!65!red!35}78.2 & \cellcolor{green!61!red!39}82.5 & \cellcolor{green!40!red!60}-7.6 & \cellcolor{green!36!red!64}0.524 \\
Laniqo & \checkmark & 9 & \checkmark & \cellcolor{green!53!red!47}8.6 & \cellcolor{green!84!red!16}0.643 & \cellcolor{green!2!red!98}67.3 & \cellcolor{green!7!red!93}69.9 & \cellcolor{green!71!red!29}-6.5 & \cellcolor{green!83!red!17}0.608 \\
\rowcolor{gray!30}
\official Claude-4 & \unknown & \unknown & \checkmark & \cellcolor{green!52!red!48}8.8 & \cellcolor{green!46!red!54}0.606 & \cellcolor{green!68!red!32}78.6 & \cellcolor{green!63!red!37}83.0 & \cellcolor{green!31!red!69}-7.9 & \cellcolor{green!35!red!65}0.522 \\
\rowcolor{gray!30}
\official Gemma-3-27B & \checkmark & 27 & \checkmark & \cellcolor{green!50!red!50}9.0 & \cellcolor{green!46!red!54}0.606 & \cellcolor{green!58!red!42}76.9 & \cellcolor{green!57!red!43}81.5 & \cellcolor{green!42!red!58}-7.5 & \cellcolor{green!35!red!65}0.523 \\
\rowcolor{gray!30}
\official ONLINE-B & \checkmark & \unknown &  & \cellcolor{green!43!red!57}10.2 & \cellcolor{green!52!red!48}0.612 & \cellcolor{green!36!red!64}73.1 & \cellcolor{green!37!red!63}77.0 & \cellcolor{green!45!red!55}-7.4 & \cellcolor{green!30!red!70}0.513 \\
\rowcolor{gray!30}
\official AyaExpanse-32B & \checkmark & 32 &  & \cellcolor{green!43!red!57}10.2 & \cellcolor{green!44!red!56}0.604 & \cellcolor{green!42!red!58}74.2 & \cellcolor{green!46!red!54}78.9 & \cellcolor{green!34!red!66}-7.8 & \cellcolor{green!33!red!67}0.519 \\
SalamandraTA & \checkmark & 8 & \checkmark & \cellcolor{green!42!red!58}10.3 & \cellcolor{green!65!red!35}0.624 & \cellcolor{green!18!red!82}70.1 & \cellcolor{green!27!red!73}74.5 & \cellcolor{green!48!red!52}-7.3 & \cellcolor{green!38!red!62}0.528 \\
\rowcolor{gray!30}
\official Llama-4-Maverick & \checkmark & 400 &  & \cellcolor{green!37!red!63}11.1 & \cellcolor{green!35!red!65}0.595 & \cellcolor{green!48!red!52}75.3 & \cellcolor{green!49!red!51}79.7 & \cellcolor{green!19!red!81}-8.3 & \cellcolor{green!19!red!81}0.494 \\
\rowcolor{gray!30}
\official ONLINE-W & \unknown & \unknown &  & \cellcolor{green!37!red!63}11.2 & \cellcolor{green!42!red!58}0.602 & \cellcolor{green!44!red!56}74.5 & \cellcolor{green!41!red!59}77.9 & \cellcolor{green!19!red!81}-8.3 & \cellcolor{green!20!red!80}0.495 \\
\official TowerPlus-9B[M] & \checkmark & 9 & \checkmark & \cellcolor{green!36!red!64}11.4 & \cellcolor{green!45!red!55}0.605 & \cellcolor{green!29!red!71}72.0 & \cellcolor{green!32!red!68}75.8 & \cellcolor{green!31!red!69}-7.9 & \cellcolor{green!25!red!75}0.505 \\
\rowcolor{gray!30}
\official Qwen3-235B & \checkmark & 235 &  & \cellcolor{green!35!red!65}11.5 & \cellcolor{green!39!red!61}0.599 & \cellcolor{green!28!red!72}71.8 & \cellcolor{green!33!red!67}76.0 & \cellcolor{green!34!red!66}-7.8 & \cellcolor{green!25!red!75}0.505 \\
CUNI-MH-v2 & \checkmark & 9 & \checkmark & \cellcolor{green!32!red!68}11.9 & \cellcolor{green!49!red!51}0.609 & \cellcolor{green!13!red!87}69.2 & \cellcolor{green!22!red!78}73.4 & \cellcolor{green!31!red!69}-7.9 & \cellcolor{green!32!red!68}0.517 \\
\rowcolor{gray!30}
\official EuroLLM-22B-pre.[M] & \checkmark & 22 &  & \cellcolor{green!29!red!71}12.5 & \cellcolor{green!33!red!67}0.593 & \cellcolor{green!30!red!70}72.2 & \cellcolor{green!29!red!71}75.0 & \cellcolor{green!17!red!83}-8.4 & \cellcolor{green!16!red!84}0.488 \\
IRB-MT & \checkmark & 12 &  & \cellcolor{green!28!red!72}12.6 & \cellcolor{green!31!red!69}0.591 & \cellcolor{green!23!red!77}71.0 & \cellcolor{green!23!red!77}73.6 & \cellcolor{green!34!red!66}-7.8 & \cellcolor{green!14!red!86}0.484 \\
\rowcolor{gray!30}
\official TowerPlus-72B[M] & \checkmark & 72 &  & \cellcolor{green!26!red!74}12.9 & \cellcolor{green!32!red!68}0.592 & \cellcolor{green!22!red!78}70.8 & \cellcolor{green!28!red!72}74.9 & \cellcolor{green!17!red!83}-8.4 & \cellcolor{green!14!red!86}0.485 \\
\rowcolor{gray!30}
TranssionTranslate & \unknown & \unknown &  & \cellcolor{green!24!red!76}13.2 & \cellcolor{green!37!red!63}0.597 & \cellcolor{green!9!red!91}68.5 & \cellcolor{green!16!red!84}72.0 & \cellcolor{green!34!red!66}-7.8 & \cellcolor{green!12!red!88}0.48 \\
\official Gemma-3-12B & \checkmark & 12 &  & \cellcolor{green!23!red!77}13.4 & \cellcolor{green!22!red!78}0.583 & \cellcolor{green!27!red!73}71.6 & \cellcolor{green!25!red!75}74.1 & \cellcolor{green!14!red!86}-8.5 & \cellcolor{green!12!red!88}0.48 \\
CUNI-SFT & \checkmark & 9 &  & \cellcolor{green!8!red!92}15.9 & \cellcolor{green!14!red!86}0.575 & \cellcolor{green!0!red!100}66.9 & \cellcolor{green!0!red!100}68.2 & \cellcolor{green!2!red!98}-8.9 & \cellcolor{green!5!red!95}0.468 \\
\official AyaExpanse-8B & \checkmark & 8 &  & \cellcolor{green!7!red!93}16.0 & \cellcolor{green!11!red!89}0.572 & \cellcolor{green!1!red!99}67.1 & \cellcolor{green!0!red!100}67.9 & \cellcolor{green!8!red!92}-8.7 & \cellcolor{green!0!red!100}0.457 \\
CUNI-DocTransformer & \checkmark & <1 &  & \cellcolor{green!0!red!100}17.5 & \cellcolor{green!0!red!100}0.558 & \cellcolor{green!10!red!90}68.7 & \cellcolor{green!12!red!88}71.1 & \cellcolor{green!0!red!100}-10.0 & \cellcolor{green!0!red!100}0.425 \\
\rowcolor{gray!30}
IR-MultiagentMT & \crossmark & \unknown &  & \cellcolor{green!0!red!100}17.7 & \cellcolor{green!0!red!100}0.546 & \cellcolor{green!0!red!100}66.5 & \cellcolor{green!2!red!98}68.7 & \cellcolor{green!0!red!100}-9.3 & \cellcolor{green!0!red!100}0.442 \\
\official EuroLLM-9B[M] & \checkmark & 9 &  & \cellcolor{green!0!red!100}18.9 & \cellcolor{green!0!red!100}0.527 & \cellcolor{green!0!red!100}63.1 & \cellcolor{green!0!red!100}63.7 & \cellcolor{green!0!red!100}-9.0 & \cellcolor{green!4!red!96}0.466 \\
\official NLLB & \checkmark & 1 &  & \cellcolor{green!0!red!100}25.5 & \cellcolor{green!0!red!100}0.485 & \cellcolor{green!0!red!100}55.7 & \cellcolor{green!0!red!100}57.3 & \cellcolor{green!0!red!100}-10.8 & \cellcolor{green!0!red!100}0.392 \\
\official CommandR7B & \checkmark & 7 &  & \cellcolor{green!0!red!100}28.0 & \cellcolor{green!0!red!100}0.457 & \cellcolor{green!0!red!100}58.5 & \cellcolor{green!0!red!100}51.3 & \cellcolor{green!0!red!100}-11.6 & \cellcolor{green!0!red!100}0.369 \\
\rowcolor{gray!30}
\official ONLINE-G & \checkmark & \unknown &  & \cellcolor{green!0!red!100}28.7 & \cellcolor{green!0!red!100}0.472 & \cellcolor{green!0!red!100}58.1 & \cellcolor{green!0!red!100}58.0 & \cellcolor{green!0!red!100}-12.8 & \cellcolor{green!0!red!100}0.313 \\
\official Llama-3.1-8B & \crossmark & 8 &  & \cellcolor{green!0!red!100}28.9 & \cellcolor{green!0!red!100}0.48 & \cellcolor{green!0!red!100}55.7 & \cellcolor{green!0!red!100}52.0 & \cellcolor{green!0!red!100}-12.1 & \cellcolor{green!0!red!100}0.317 \\
\official Qwen2.5-7B & \unknown & 7 &  & \cellcolor{green!0!red!100}37.5 & \cellcolor{green!0!red!100}0.41 & \cellcolor{green!0!red!100}46.2 & \cellcolor{green!0!red!100}43.8 & \cellcolor{green!0!red!100}-14.2 & \cellcolor{green!0!red!100}0.239 \\
\official Mistral-7B & \crossmark & 7 &  & \cellcolor{green!0!red!100}40.8 & \cellcolor{green!0!red!100}0.374 & \cellcolor{green!0!red!100}45.8 & \cellcolor{green!0!red!100}41.2 & \cellcolor{green!0!red!100}-15.5 & \cellcolor{green!0!red!100}0.207 \\
ctpc\_nlp & \unknown & \unknown &  & \cellcolor{green!0!red!100}41.1 & \cellcolor{green!0!red!100}0.369 & \cellcolor{green!0!red!100}43.3 & \cellcolor{green!0!red!100}39.8 & \cellcolor{green!0!red!100}-14.8 & \cellcolor{green!0!red!100}0.207 \\
\rowcolor{gray!30}
TranssionMT & \checkmark & 1 &  & \cellcolor{green!0!red!100}42.0 & \cellcolor{green!0!red!100}0.364 & \cellcolor{green!0!red!100}45.4 & \cellcolor{green!0!red!100}45.4 & \cellcolor{green!0!red!100}-16.7 & \cellcolor{green!0!red!100}0.196 \\
\bottomrule
\end{tabularx}
\end{table*}


\begin{table*}
\small
\begin{tabularx}{\textwidth}{lYYYYYYYYY}
\toprule
\multicolumn{10}{c}{\textbf{English-Estonian}} \\
\midrule
System Name & LP Supported & Params. (B) & Humeval? & AutoRank $\downarrow$ & CometKiwi-XL $\uparrow$ & GEMBA-ESA-CMDA $\uparrow$ & GEMBA-ESA-GPT4.1 $\uparrow$ & MetricX-24-Hybrid-XL $\uparrow$ & XCOMET-XL $\uparrow$ \\
\midrule
Shy-hunyuan-MT & \checkmark & 7 & \checkmark & \cellcolor{green!100!red!0}1.0 & \cellcolor{green!100!red!0}0.72 & \cellcolor{green!100!red!0}78.8 & \cellcolor{green!93!red!7}87.8 & \cellcolor{green!100!red!0}-7.3 & \cellcolor{green!100!red!0}0.628 \\
\rowcolor{gray!30}
\official Gemini-2.5-Pro & \checkmark & \unknown & \checkmark & \cellcolor{green!90!red!10}2.5 & \cellcolor{green!90!red!10}0.7 & \cellcolor{green!79!red!21}74.1 & \cellcolor{green!100!red!0}90.7 & \cellcolor{green!92!red!8}-8.0 & \cellcolor{green!88!red!12}0.59 \\
Wenyiil & \checkmark & 14 & \checkmark & \cellcolor{green!90!red!10}2.6 & \cellcolor{green!94!red!6}0.708 & \cellcolor{green!80!red!20}74.4 & \cellcolor{green!89!red!11}86.0 & \cellcolor{green!92!red!8}-8.0 & \cellcolor{green!91!red!9}0.599 \\
\rowcolor{gray!30}
\official GPT-4.1 & \checkmark & \unknown & \checkmark & \cellcolor{green!87!red!13}3.0 & \cellcolor{green!87!red!13}0.695 & \cellcolor{green!84!red!16}75.2 & \cellcolor{green!93!red!7}87.9 & \cellcolor{green!84!red!16}-8.6 & \cellcolor{green!84!red!16}0.577 \\
Yolu & \checkmark & 14 & \checkmark & \cellcolor{green!83!red!17}3.7 & \cellcolor{green!100!red!0}0.72 & \cellcolor{green!70!red!30}72.1 & \cellcolor{green!68!red!32}77.4 & \cellcolor{green!88!red!12}-8.3 & \cellcolor{green!87!red!13}0.587 \\
Algharb & \checkmark & 14 & \checkmark & \cellcolor{green!82!red!18}3.8 & \cellcolor{green!86!red!14}0.692 & \cellcolor{green!76!red!24}73.6 & \cellcolor{green!84!red!16}84.1 & \cellcolor{green!83!red!17}-8.7 & \cellcolor{green!79!red!21}0.558 \\
\rowcolor{gray!30}
GemTrans & \checkmark & 27 & \checkmark & \cellcolor{green!75!red!25}4.9 & \cellcolor{green!84!red!16}0.689 & \cellcolor{green!64!red!36}70.8 & \cellcolor{green!60!red!40}74.3 & \cellcolor{green!88!red!12}-8.3 & \cellcolor{green!79!red!21}0.558 \\
Laniqo & \checkmark & 9 & \checkmark & \cellcolor{green!74!red!26}5.1 & \cellcolor{green!95!red!5}0.711 & \cellcolor{green!47!red!53}67.2 & \cellcolor{green!45!red!55}68.1 & \cellcolor{green!89!red!11}-8.2 & \cellcolor{green!92!red!8}0.602 \\
SRPOL & \checkmark & 12 & \checkmark & \cellcolor{green!71!red!29}5.5 & \cellcolor{green!92!red!8}0.705 & \cellcolor{green!62!red!38}70.5 & \cellcolor{green!60!red!40}74.2 & \cellcolor{green!71!red!29}-9.7 & \cellcolor{green!72!red!28}0.538 \\
\rowcolor{gray!30}
UvA-MT & \checkmark & 12 & \checkmark & \cellcolor{green!69!red!31}5.8 & \cellcolor{green!88!red!12}0.696 & \cellcolor{green!69!red!31}71.9 & \cellcolor{green!56!red!44}72.6 & \cellcolor{green!67!red!33}-10.0 & \cellcolor{green!70!red!30}0.531 \\
\rowcolor{gray!30}
\official ONLINE-B & \checkmark & \unknown & \checkmark & \cellcolor{green!69!red!31}5.8 & \cellcolor{green!79!red!21}0.678 & \cellcolor{green!60!red!40}69.9 & \cellcolor{green!65!red!35}76.5 & \cellcolor{green!77!red!23}-9.2 & \cellcolor{green!67!red!33}0.521 \\
\rowcolor{gray!30}
CommandA-WMT & \crossmark & 111 & \checkmark & \cellcolor{green!69!red!31}5.9 & \cellcolor{green!84!red!16}0.689 & \cellcolor{green!67!red!33}71.6 & \cellcolor{green!54!red!46}71.8 & \cellcolor{green!71!red!29}-9.7 & \cellcolor{green!69!red!31}0.527 \\
SalamandraTA & \checkmark & 8 & \checkmark & \cellcolor{green!68!red!32}6.1 & \cellcolor{green!87!red!13}0.695 & \cellcolor{green!53!red!47}68.4 & \cellcolor{green!53!red!47}71.5 & \cellcolor{green!76!red!24}-9.3 & \cellcolor{green!71!red!29}0.532 \\
\rowcolor{gray!30}
\official Claude-4 & \unknown & \unknown & \checkmark & \cellcolor{green!66!red!34}6.3 & \cellcolor{green!76!red!24}0.673 & \cellcolor{green!66!red!34}71.4 & \cellcolor{green!67!red!33}77.3 & \cellcolor{green!60!red!40}-10.6 & \cellcolor{green!62!red!38}0.505 \\
\rowcolor{gray!30}
TranssionTranslate & \unknown & \unknown & \checkmark & \cellcolor{green!61!red!39}7.2 & \cellcolor{green!74!red!26}0.669 & \cellcolor{green!42!red!58}66.1 & \cellcolor{green!57!red!43}73.2 & \cellcolor{green!73!red!27}-9.5 & \cellcolor{green!61!red!39}0.501 \\
\rowcolor{gray!30}
\official Gemma-3-27B & \checkmark & 27 & \checkmark & \cellcolor{green!59!red!41}7.4 & \cellcolor{green!70!red!30}0.662 & \cellcolor{green!61!red!39}70.2 & \cellcolor{green!54!red!46}71.8 & \cellcolor{green!58!red!42}-10.8 & \cellcolor{green!58!red!42}0.491 \\
\rowcolor{gray!30}
\official EuroLLM-22B-pre.[M] & \checkmark & 22 & \checkmark & \cellcolor{green!56!red!44}7.9 & \cellcolor{green!66!red!34}0.654 & \cellcolor{green!54!red!46}68.6 & \cellcolor{green!55!red!45}72.2 & \cellcolor{green!58!red!42}-10.8 & \cellcolor{green!54!red!46}0.479 \\
\rowcolor{gray!30}
\official Llama-4-Maverick & \checkmark & 400 &  & \cellcolor{green!55!red!45}8.0 & \cellcolor{green!67!red!33}0.655 & \cellcolor{green!56!red!44}69.0 & \cellcolor{green!54!red!46}71.9 & \cellcolor{green!54!red!46}-11.1 & \cellcolor{green!53!red!47}0.474 \\
\rowcolor{gray!30}
\official ONLINE-W & \unknown & \unknown &  & \cellcolor{green!52!red!48}8.6 & \cellcolor{green!66!red!34}0.654 & \cellcolor{green!51!red!49}67.9 & \cellcolor{green!50!red!50}70.3 & \cellcolor{green!48!red!52}-11.6 & \cellcolor{green!52!red!48}0.471 \\
\rowcolor{gray!30}
\official DeepSeek-V3 & \unknown & 671 &  & \cellcolor{green!42!red!58}10.1 & \cellcolor{green!46!red!54}0.613 & \cellcolor{green!33!red!67}64.0 & \cellcolor{green!41!red!59}66.5 & \cellcolor{green!50!red!50}-11.4 & \cellcolor{green!51!red!49}0.468 \\
IRB-MT & \checkmark & 12 & \checkmark & \cellcolor{green!36!red!64}11.1 & \cellcolor{green!44!red!56}0.609 & \cellcolor{green!40!red!60}65.6 & \cellcolor{green!26!red!74}60.5 & \cellcolor{green!45!red!55}-11.8 & \cellcolor{green!34!red!66}0.413 \\
\rowcolor{gray!30}
IR-MultiagentMT & \crossmark & \unknown &  & \cellcolor{green!36!red!64}11.1 & \cellcolor{green!41!red!59}0.605 & \cellcolor{green!35!red!65}64.5 & \cellcolor{green!32!red!68}62.7 & \cellcolor{green!44!red!56}-11.9 & \cellcolor{green!37!red!63}0.423 \\
\official Gemma-3-12B & \checkmark & 12 &  & \cellcolor{green!29!red!71}12.1 & \cellcolor{green!37!red!63}0.597 & \cellcolor{green!40!red!60}65.6 & \cellcolor{green!24!red!76}59.4 & \cellcolor{green!31!red!69}-13.0 & \cellcolor{green!26!red!74}0.387 \\
\official EuroLLM-9B[M] & \checkmark & 9 &  & \cellcolor{green!20!red!80}13.5 & \cellcolor{green!0!red!100}0.522 & \cellcolor{green!2!red!98}57.3 & \cellcolor{green!13!red!87}55.0 & \cellcolor{green!55!red!45}-11.0 & \cellcolor{green!49!red!51}0.463 \\
\rowcolor{gray!30}
\official Mistral-Medium & \unknown & \unknown &  & \cellcolor{green!18!red!82}13.9 & \cellcolor{green!26!red!74}0.574 & \cellcolor{green!14!red!86}59.9 & \cellcolor{green!13!red!87}54.8 & \cellcolor{green!25!red!75}-13.5 & \cellcolor{green!30!red!70}0.4 \\
\rowcolor{gray!30}
\official Qwen3-235B & \checkmark & 235 &  & \cellcolor{green!17!red!83}14.1 & \cellcolor{green!27!red!73}0.576 & \cellcolor{green!27!red!73}62.6 & \cellcolor{green!11!red!89}54.1 & \cellcolor{green!22!red!78}-13.7 & \cellcolor{green!14!red!86}0.349 \\
\rowcolor{gray!30}
\official CommandA & \crossmark & 111 &  & \cellcolor{green!5!red!95}15.9 & \cellcolor{green!11!red!89}0.546 & \cellcolor{green!32!red!68}63.9 & \cellcolor{green!0!red!100}48.4 & \cellcolor{green!2!red!98}-15.4 & \cellcolor{green!4!red!96}0.316 \\
\official NLLB & \checkmark & 1 &  & \cellcolor{green!4!red!96}16.1 & \cellcolor{green!2!red!98}0.528 & \cellcolor{green!0!red!100}56.6 & \cellcolor{green!9!red!91}53.4 & \cellcolor{green!16!red!84}-14.2 & \cellcolor{green!15!red!85}0.35 \\
\rowcolor{gray!30}
\official ONLINE-G & \checkmark & \unknown &  & \cellcolor{green!0!red!100}16.9 & \cellcolor{green!4!red!96}0.532 & \cellcolor{green!2!red!98}57.2 & \cellcolor{green!14!red!86}55.3 & \cellcolor{green!0!red!100}-15.6 & \cellcolor{green!0!red!100}0.297 \\
\rowcolor{gray!30}
\official TowerPlus-72B[M] & \crossmark & 72 &  & \cellcolor{green!0!red!100}20.2 & \cellcolor{green!0!red!100}0.491 & \cellcolor{green!0!red!100}54.7 & \cellcolor{green!0!red!100}40.3 & \cellcolor{green!0!red!100}-17.0 & \cellcolor{green!0!red!100}0.254 \\
\rowcolor{gray!30}
TranssionMT & \checkmark & 1 &  & \cellcolor{green!0!red!100}23.7 & \cellcolor{green!0!red!100}0.436 & \cellcolor{green!0!red!100}46.6 & \cellcolor{green!0!red!100}43.1 & \cellcolor{green!0!red!100}-19.1 & \cellcolor{green!0!red!100}0.176 \\
\official Llama-3.1-8B & \crossmark & 8 &  & \cellcolor{green!0!red!100}24.5 & \cellcolor{green!0!red!100}0.424 & \cellcolor{green!0!red!100}47.8 & \cellcolor{green!0!red!100}33.7 & \cellcolor{green!0!red!100}-18.7 & \cellcolor{green!0!red!100}0.166 \\
\official TowerPlus-9B[M] & \crossmark & 9 &  & \cellcolor{green!0!red!100}27.2 & \cellcolor{green!0!red!100}0.403 & \cellcolor{green!0!red!100}42.1 & \cellcolor{green!0!red!100}13.2 & \cellcolor{green!0!red!100}-19.0 & \cellcolor{green!0!red!100}0.19 \\
\rowcolor{gray!30}
\official AyaExpanse-32B & \crossmark & 32 &  & \cellcolor{green!0!red!100}32.3 & \cellcolor{green!0!red!100}0.284 & \cellcolor{green!0!red!100}33.4 & \cellcolor{green!0!red!100}20.2 & \cellcolor{green!0!red!100}-23.2 & \cellcolor{green!0!red!100}0.135 \\
\official Qwen2.5-7B & \unknown & 7 &  & \cellcolor{green!0!red!100}33.6 & \cellcolor{green!0!red!100}0.273 & \cellcolor{green!0!red!100}27.6 & \cellcolor{green!0!red!100}17.8 & \cellcolor{green!0!red!100}-23.6 & \cellcolor{green!0!red!100}0.144 \\
\official CommandR7B & \crossmark & 7 &  & \cellcolor{green!0!red!100}35.8 & \cellcolor{green!0!red!100}0.169 & \cellcolor{green!0!red!100}23.4 & \cellcolor{green!0!red!100}9.2 & \cellcolor{green!0!red!100}-22.6 & \cellcolor{green!0!red!100}0.193 \\
\official Mistral-7B & \crossmark & 7 &  & \cellcolor{green!0!red!100}37.4 & \cellcolor{green!0!red!100}0.182 & \cellcolor{green!0!red!100}18.1 & \cellcolor{green!0!red!100}11.4 & \cellcolor{green!0!red!100}-24.5 & \cellcolor{green!0!red!100}0.151 \\
\official AyaExpanse-8B & \crossmark & 8 &  & \cellcolor{green!0!red!100}38.0 & \cellcolor{green!0!red!100}0.151 & \cellcolor{green!0!red!100}17.4 & \cellcolor{green!0!red!100}10.1 & \cellcolor{green!0!red!100}-24.7 & \cellcolor{green!0!red!100}0.171 \\
\bottomrule
\end{tabularx}
\end{table*}


\begin{table*}
\small
\begin{tabularx}{\textwidth}{lYYYYYYYYY}
\toprule
\multicolumn{10}{c}{\textbf{English-Icelandic}} \\
\midrule
System Name & LP Supported & Params. (B) & Humeval? & AutoRank $\downarrow$ & CometKiwi-XL $\uparrow$ & GEMBA-ESA-CMDA $\uparrow$ & GEMBA-ESA-GPT4.1 $\uparrow$ & MetricX-24-Hybrid-XL $\uparrow$ & XCOMET-XL $\uparrow$ \\
\midrule
Shy-hunyuan-MT & \checkmark & 7 & \checkmark & \cellcolor{green!100!red!0}1.0 & \cellcolor{green!100!red!0}0.663 & \cellcolor{green!100!red!0}71.6 & \cellcolor{green!93!red!7}83.9 & \cellcolor{green!100!red!0}-7.5 & \cellcolor{green!100!red!0}0.543 \\
\rowcolor{gray!30}
\official Gemini-2.5-Pro & \checkmark & \unknown & \checkmark & \cellcolor{green!95!red!5}1.8 & \cellcolor{green!91!red!9}0.647 & \cellcolor{green!87!red!13}69.2 & \cellcolor{green!100!red!0}87.6 & \cellcolor{green!98!red!2}-7.7 & \cellcolor{green!90!red!10}0.512 \\
\rowcolor{gray!30}
\official GPT-4.1 & \checkmark & \unknown & \checkmark & \cellcolor{green!94!red!6}1.9 & \cellcolor{green!95!red!5}0.653 & \cellcolor{green!92!red!8}70.2 & \cellcolor{green!94!red!6}84.5 & \cellcolor{green!92!red!8}-8.3 & \cellcolor{green!91!red!9}0.516 \\
\rowcolor{gray!30}
Erlendur & \checkmark & 175 & \checkmark & \cellcolor{green!92!red!8}2.2 & \cellcolor{green!91!red!9}0.646 & \cellcolor{green!89!red!11}69.5 & \cellcolor{green!95!red!5}85.1 & \cellcolor{green!93!red!7}-8.2 & \cellcolor{green!88!red!12}0.506 \\
\official TowerPlus-9B[M] & \checkmark & 9 & \checkmark & \cellcolor{green!81!red!19}3.9 & \cellcolor{green!88!red!12}0.64 & \cellcolor{green!76!red!24}67.1 & \cellcolor{green!77!red!23}76.3 & \cellcolor{green!86!red!14}-8.8 & \cellcolor{green!76!red!24}0.471 \\
\rowcolor{gray!30}
\official ONLINE-B & \checkmark & \unknown & \checkmark & \cellcolor{green!78!red!22}4.4 & \cellcolor{green!86!red!14}0.636 & \cellcolor{green!70!red!30}66.1 & \cellcolor{green!72!red!28}73.5 & \cellcolor{green!86!red!14}-8.8 & \cellcolor{green!74!red!26}0.464 \\
\rowcolor{gray!30}
\official Claude-4 & \unknown & \unknown & \checkmark & \cellcolor{green!72!red!28}5.2 & \cellcolor{green!81!red!19}0.628 & \cellcolor{green!78!red!22}67.5 & \cellcolor{green!73!red!27}73.8 & \cellcolor{green!67!red!33}-10.6 & \cellcolor{green!63!red!37}0.43 \\
\rowcolor{gray!30}
\official TowerPlus-72B[M] & \checkmark & 72 & \checkmark & \cellcolor{green!69!red!31}5.7 & \cellcolor{green!78!red!22}0.621 & \cellcolor{green!74!red!26}66.7 & \cellcolor{green!60!red!40}67.7 & \cellcolor{green!73!red!27}-10.1 & \cellcolor{green!65!red!35}0.435 \\
\rowcolor{gray!30}
TranssionTranslate & \unknown & \unknown & \checkmark & \cellcolor{green!68!red!32}5.8 & \cellcolor{green!80!red!20}0.625 & \cellcolor{green!55!red!45}63.2 & \cellcolor{green!63!red!37}68.9 & \cellcolor{green!83!red!17}-9.1 & \cellcolor{green!63!red!37}0.43 \\
\rowcolor{gray!30}
UvA-MT & \checkmark & 12 & \checkmark & \cellcolor{green!62!red!38}6.8 & \cellcolor{green!81!red!19}0.627 & \cellcolor{green!81!red!19}68.1 & \cellcolor{green!43!red!57}59.1 & \cellcolor{green!57!red!43}-11.6 & \cellcolor{green!54!red!46}0.402 \\
\rowcolor{gray!30}
CommandA-WMT & \crossmark & 111 & \checkmark & \cellcolor{green!62!red!38}6.8 & \cellcolor{green!77!red!23}0.619 & \cellcolor{green!81!red!19}68.0 & \cellcolor{green!40!red!60}57.4 & \cellcolor{green!62!red!38}-11.1 & \cellcolor{green!55!red!45}0.404 \\
\rowcolor{gray!30}
GemTrans & \checkmark & 27 & \checkmark & \cellcolor{green!61!red!39}7.0 & \cellcolor{green!71!red!29}0.609 & \cellcolor{green!65!red!35}65.0 & \cellcolor{green!43!red!57}59.1 & \cellcolor{green!77!red!23}-9.7 & \cellcolor{green!54!red!46}0.401 \\
AMI & \checkmark & 3 & \checkmark & \cellcolor{green!58!red!42}7.4 & \cellcolor{green!81!red!19}0.627 & \cellcolor{green!35!red!65}59.6 & \cellcolor{green!41!red!59}58.1 & \cellcolor{green!77!red!23}-9.7 & \cellcolor{green!62!red!38}0.426 \\
SalamandraTA & \checkmark & 8 & \checkmark & \cellcolor{green!50!red!50}8.6 & \cellcolor{green!69!red!31}0.605 & \cellcolor{green!46!red!54}61.6 & \cellcolor{green!33!red!67}53.9 & \cellcolor{green!63!red!37}-11.0 & \cellcolor{green!49!red!51}0.386 \\
\rowcolor{gray!30}
\official Llama-4-Maverick & \checkmark & 400 &  & \cellcolor{green!49!red!51}8.8 & \cellcolor{green!60!red!40}0.587 & \cellcolor{green!63!red!37}64.7 & \cellcolor{green!43!red!57}58.8 & \cellcolor{green!49!red!51}-12.3 & \cellcolor{green!39!red!61}0.357 \\
\rowcolor{gray!30}
\official Mistral-Medium & \unknown & \unknown &  & \cellcolor{green!43!red!57}9.7 & \cellcolor{green!57!red!43}0.583 & \cellcolor{green!66!red!34}65.3 & \cellcolor{green!28!red!72}51.5 & \cellcolor{green!42!red!58}-13.0 & \cellcolor{green!33!red!67}0.337 \\
\rowcolor{gray!30}
\official Gemma-3-27B & \checkmark & 27 &  & \cellcolor{green!43!red!57}9.7 & \cellcolor{green!52!red!48}0.572 & \cellcolor{green!49!red!51}62.2 & \cellcolor{green!35!red!65}54.9 & \cellcolor{green!48!red!52}-12.4 & \cellcolor{green!42!red!58}0.364 \\
\rowcolor{gray!30}
\official DeepSeek-V3 & \unknown & 671 &  & \cellcolor{green!38!red!62}10.5 & \cellcolor{green!38!red!62}0.547 & \cellcolor{green!27!red!73}58.0 & \cellcolor{green!38!red!62}56.6 & \cellcolor{green!52!red!48}-12.1 & \cellcolor{green!46!red!54}0.378 \\
IRB-MT & \checkmark & 12 & \checkmark & \cellcolor{green!28!red!72}11.9 & \cellcolor{green!36!red!64}0.542 & \cellcolor{green!44!red!56}61.2 & \cellcolor{green!20!red!80}47.2 & \cellcolor{green!36!red!64}-13.6 & \cellcolor{green!23!red!77}0.306 \\
\rowcolor{gray!30}
IR-MultiagentMT & \crossmark & \unknown &  & \cellcolor{green!27!red!73}12.1 & \cellcolor{green!29!red!71}0.53 & \cellcolor{green!38!red!62}60.0 & \cellcolor{green!28!red!72}51.3 & \cellcolor{green!35!red!65}-13.7 & \cellcolor{green!24!red!76}0.31 \\
\rowcolor{gray!30}
\official Qwen3-235B & \crossmark & 235 &  & \cellcolor{green!18!red!82}13.5 & \cellcolor{green!27!red!73}0.525 & \cellcolor{green!40!red!60}60.5 & \cellcolor{green!8!red!92}41.5 & \cellcolor{green!21!red!79}-15.0 & \cellcolor{green!12!red!88}0.275 \\
\official Gemma-3-12B & \checkmark & 12 & \checkmark & \cellcolor{green!16!red!84}13.8 & \cellcolor{green!22!red!78}0.517 & \cellcolor{green!39!red!61}60.3 & \cellcolor{green!9!red!91}42.1 & \cellcolor{green!17!red!83}-15.4 & \cellcolor{green!10!red!90}0.268 \\
\official NLLB & \checkmark & 1 & \checkmark & \cellcolor{green!7!red!93}15.2 & \cellcolor{green!1!red!99}0.477 & \cellcolor{green!0!red!100}53.0 & \cellcolor{green!22!red!78}48.2 & \cellcolor{green!21!red!79}-15.0 & \cellcolor{green!11!red!89}0.27 \\
\rowcolor{gray!30}
\official ONLINE-G & \checkmark & \unknown &  & \cellcolor{green!3!red!97}15.8 & \cellcolor{green!1!red!99}0.477 & \cellcolor{green!2!red!98}53.4 & \cellcolor{green!24!red!76}49.2 & \cellcolor{green!9!red!91}-16.1 & \cellcolor{green!2!red!98}0.243 \\
\rowcolor{gray!30}
\official CommandA & \crossmark & 111 &  & \cellcolor{green!0!red!100}16.2 & \cellcolor{green!0!red!100}0.475 & \cellcolor{green!32!red!68}59.0 & \cellcolor{green!0!red!100}37.4 & \cellcolor{green!0!red!100}-17.0 & \cellcolor{green!0!red!100}0.221 \\
\official Llama-3.1-8B & \crossmark & 8 & \checkmark & \cellcolor{green!0!red!100}24.8 & \cellcolor{green!0!red!100}0.323 & \cellcolor{green!0!red!100}42.7 & \cellcolor{green!0!red!100}24.6 & \cellcolor{green!0!red!100}-21.3 & \cellcolor{green!0!red!100}0.133 \\
\official EuroLLM-9B[M] & \crossmark & 9 &  & \cellcolor{green!0!red!100}25.5 & \cellcolor{green!0!red!100}0.303 & \cellcolor{green!0!red!100}32.9 & \cellcolor{green!0!red!100}9.2 & \cellcolor{green!0!red!100}-17.4 & \cellcolor{green!0!red!100}0.237 \\
\rowcolor{gray!30}
\official AyaExpanse-32B & \crossmark & 32 &  & \cellcolor{green!0!red!100}28.0 & \cellcolor{green!0!red!100}0.275 & \cellcolor{green!0!red!100}35.2 & \cellcolor{green!0!red!100}18.4 & \cellcolor{green!0!red!100}-23.3 & \cellcolor{green!0!red!100}0.145 \\
\official CommandR7B & \crossmark & 7 &  & \cellcolor{green!0!red!100}30.3 & \cellcolor{green!0!red!100}0.2 & \cellcolor{green!0!red!100}23.4 & \cellcolor{green!0!red!100}9.1 & \cellcolor{green!0!red!100}-20.9 & \cellcolor{green!0!red!100}0.216 \\
\rowcolor{gray!30}
\official EuroLLM-22B-pre.[M] & \crossmark & 22 &  & \cellcolor{green!0!red!100}30.8 & \cellcolor{green!0!red!100}0.206 & \cellcolor{green!0!red!100}26.5 & \cellcolor{green!0!red!100}13.7 & \cellcolor{green!0!red!100}-23.7 & \cellcolor{green!0!red!100}0.171 \\
\official Mistral-7B & \crossmark & 7 &  & \cellcolor{green!0!red!100}31.8 & \cellcolor{green!0!red!100}0.177 & \cellcolor{green!0!red!100}25.2 & \cellcolor{green!0!red!100}14.3 & \cellcolor{green!0!red!100}-24.3 & \cellcolor{green!0!red!100}0.17 \\
\official Qwen2.5-7B & \unknown & 7 &  & \cellcolor{green!0!red!100}31.8 & \cellcolor{green!0!red!100}0.186 & \cellcolor{green!0!red!100}24.1 & \cellcolor{green!0!red!100}13.1 & \cellcolor{green!0!red!100}-24.3 & \cellcolor{green!0!red!100}0.174 \\
\official AyaExpanse-8B & \crossmark & 8 &  & \cellcolor{green!0!red!100}33.0 & \cellcolor{green!0!red!100}0.153 & \cellcolor{green!0!red!100}21.7 & \cellcolor{green!0!red!100}11.3 & \cellcolor{green!0!red!100}-24.6 & \cellcolor{green!0!red!100}0.177 \\
\bottomrule
\end{tabularx}
\end{table*}


\begin{table*}
\small
\begin{tabularx}{\textwidth}{lYYYYYYYY}
\toprule
\multicolumn{9}{c}{\textbf{English-Italian}} \\
\midrule
System Name & LP Supported & Params. (B) & Humeval? & AutoRank $\downarrow$ & GEMBA-ESA-CMDA $\uparrow$ & GEMBA-ESA-GPT4.1 $\uparrow$ & MetricX-24-Hybrid-XL $\uparrow$ & XCOMET-XL $\uparrow$ \\
\midrule
Shy-hunyuan-MT & \checkmark & 7 & \checkmark & \cellcolor{green!100!red!0}1.0 & \cellcolor{green!94!red!6}84.6 & \cellcolor{green!89!red!11}88.7 & \cellcolor{green!100!red!0}-4.7 & \cellcolor{green!92!red!8}0.62 \\
\rowcolor{gray!30}
CommandA-WMT & \checkmark & 111 & \checkmark & \cellcolor{green!89!red!11}2.6 & \cellcolor{green!87!red!13}83.4 & \cellcolor{green!85!red!15}88.0 & \cellcolor{green!95!red!5}-4.8 & \cellcolor{green!69!red!31}0.59 \\
\rowcolor{gray!30}
\official Gemini-2.5-Pro & \checkmark & \unknown & \checkmark & \cellcolor{green!76!red!24}4.4 & \cellcolor{green!100!red!0}85.5 & \cellcolor{green!100!red!0}90.5 & \cellcolor{green!57!red!43}-5.6 & \cellcolor{green!29!red!71}0.537 \\
\rowcolor{gray!30}
\official GPT-4.1 & \checkmark & \unknown & \checkmark & \cellcolor{green!75!red!25}4.5 & \cellcolor{green!97!red!3}85.0 & \cellcolor{green!96!red!4}89.8 & \cellcolor{green!48!red!52}-5.8 & \cellcolor{green!41!red!59}0.553 \\
\rowcolor{gray!30}
GemTrans & \checkmark & 27 & \checkmark & \cellcolor{green!70!red!30}5.2 & \cellcolor{green!54!red!46}78.2 & \cellcolor{green!58!red!42}83.5 & \cellcolor{green!90!red!10}-4.9 & \cellcolor{green!63!red!37}0.581 \\
\rowcolor{gray!30}
UvA-MT & \checkmark & 12 & \checkmark & \cellcolor{green!70!red!30}5.3 & \cellcolor{green!59!red!41}78.9 & \cellcolor{green!64!red!36}84.6 & \cellcolor{green!67!red!33}-5.4 & \cellcolor{green!73!red!27}0.595 \\
\rowcolor{gray!30}
\official DeepSeek-V3 & \unknown & 671 & \checkmark & \cellcolor{green!64!red!36}6.1 & \cellcolor{green!78!red!22}81.9 & \cellcolor{green!84!red!16}87.9 & \cellcolor{green!43!red!57}-5.9 & \cellcolor{green!34!red!66}0.543 \\
\rowcolor{gray!30}
\official Mistral-Medium & \unknown & \unknown & \checkmark & \cellcolor{green!57!red!43}7.1 & \cellcolor{green!65!red!35}79.9 & \cellcolor{green!75!red!25}86.4 & \cellcolor{green!38!red!62}-6.0 & \cellcolor{green!34!red!66}0.544 \\
\rowcolor{gray!30}
\official Qwen3-235B & \checkmark & 235 & \checkmark & \cellcolor{green!56!red!44}7.2 & \cellcolor{green!66!red!34}80.1 & \cellcolor{green!66!red!34}84.9 & \cellcolor{green!48!red!52}-5.8 & \cellcolor{green!32!red!68}0.541 \\
Laniqo & \checkmark & 9 & \checkmark & \cellcolor{green!54!red!46}7.6 & \cellcolor{green!6!red!94}70.5 & \cellcolor{green!8!red!92}75.3 & \cellcolor{green!90!red!10}-4.9 & \cellcolor{green!100!red!0}0.63 \\
\rowcolor{gray!30}
\official Claude-4 & \checkmark & \unknown & \checkmark & \cellcolor{green!48!red!52}8.4 & \cellcolor{green!76!red!24}81.7 & \cellcolor{green!68!red!32}85.2 & \cellcolor{green!19!red!81}-6.4 & \cellcolor{green!16!red!84}0.52 \\
\rowcolor{gray!30}
\official CommandA & \checkmark & 111 & \checkmark & \cellcolor{green!47!red!53}8.5 & \cellcolor{green!62!red!38}79.4 & \cellcolor{green!59!red!41}83.7 & \cellcolor{green!29!red!71}-6.2 & \cellcolor{green!29!red!71}0.537 \\
\rowcolor{gray!30}
\official ONLINE-B & \checkmark & \unknown &  & \cellcolor{green!41!red!59}9.4 & \cellcolor{green!45!red!55}76.7 & \cellcolor{green!28!red!72}78.6 & \cellcolor{green!57!red!43}-5.6 & \cellcolor{green!24!red!76}0.53 \\
\rowcolor{gray!30}
\official TowerPlus-72B[M] & \checkmark & 72 &  & \cellcolor{green!41!red!59}9.4 & \cellcolor{green!42!red!58}76.2 & \cellcolor{green!48!red!52}81.8 & \cellcolor{green!33!red!67}-6.1 & \cellcolor{green!31!red!69}0.539 \\
\rowcolor{gray!30}
\official AyaExpanse-32B & \checkmark & 32 &  & \cellcolor{green!36!red!64}10.1 & \cellcolor{green!39!red!61}75.7 & \cellcolor{green!42!red!58}80.9 & \cellcolor{green!33!red!67}-6.1 & \cellcolor{green!21!red!79}0.527 \\
\rowcolor{gray!30}
\official ONLINE-W & \unknown & \unknown &  & \cellcolor{green!36!red!64}10.1 & \cellcolor{green!32!red!68}74.6 & \cellcolor{green!43!red!57}81.1 & \cellcolor{green!38!red!62}-6.0 & \cellcolor{green!24!red!76}0.531 \\
IRB-MT & \checkmark & 12 & \checkmark & \cellcolor{green!35!red!65}10.2 & \cellcolor{green!27!red!73}73.8 & \cellcolor{green!36!red!64}79.8 & \cellcolor{green!52!red!48}-5.7 & \cellcolor{green!18!red!82}0.523 \\
SalamandraTA & \checkmark & 8 & \checkmark & \cellcolor{green!35!red!65}10.3 & \cellcolor{green!15!red!85}71.9 & \cellcolor{green!18!red!82}76.9 & \cellcolor{green!48!red!52}-5.8 & \cellcolor{green!47!red!53}0.561 \\
\rowcolor{gray!30}
\official EuroLLM-22B-pre.[M] & \checkmark & 22 &  & \cellcolor{green!30!red!70}11.0 & \cellcolor{green!29!red!71}74.2 & \cellcolor{green!36!red!64}79.8 & \cellcolor{green!19!red!81}-6.4 & \cellcolor{green!24!red!76}0.53 \\
\rowcolor{gray!30}
TranssionTranslate & \unknown & \unknown &  & \cellcolor{green!30!red!70}11.0 & \cellcolor{green!21!red!79}72.9 & \cellcolor{green!19!red!81}77.1 & \cellcolor{green!52!red!48}-5.7 & \cellcolor{green!18!red!82}0.523 \\
\official TowerPlus-9B[M] & \checkmark & 9 & \checkmark & \cellcolor{green!27!red!73}11.3 & \cellcolor{green!25!red!75}73.5 & \cellcolor{green!28!red!72}78.6 & \cellcolor{green!29!red!71}-6.2 & \cellcolor{green!21!red!79}0.526 \\
\rowcolor{gray!30}
\official Gemma-3-27B & \checkmark & 27 &  & \cellcolor{green!18!red!82}12.6 & \cellcolor{green!24!red!76}73.4 & \cellcolor{green!27!red!73}78.3 & \cellcolor{green!5!red!95}-6.7 & \cellcolor{green!11!red!89}0.513 \\
\rowcolor{gray!30}
IR-MultiagentMT & \crossmark & \unknown &  & \cellcolor{green!11!red!89}13.6 & \cellcolor{green!22!red!78}73.0 & \cellcolor{green!19!red!81}77.0 & \cellcolor{green!0!red!100}-6.8 & \cellcolor{green!0!red!100}0.499 \\
\official AyaExpanse-8B & \checkmark & 8 & \checkmark & \cellcolor{green!2!red!98}14.9 & \cellcolor{green!0!red!100}69.5 & \cellcolor{green!0!red!100}73.9 & \cellcolor{green!5!red!95}-6.7 & \cellcolor{green!2!red!98}0.502 \\
\official EuroLLM-9B[M] & \checkmark & 9 & \checkmark & \cellcolor{green!0!red!100}15.2 & \cellcolor{green!0!red!100}68.3 & \cellcolor{green!0!red!100}73.5 & \cellcolor{green!0!red!100}-6.8 & \cellcolor{green!8!red!92}0.509 \\
\official Gemma-3-12B & \checkmark & 12 & \checkmark & \cellcolor{green!0!red!100}15.5 & \cellcolor{green!1!red!99}69.7 & \cellcolor{green!5!red!95}74.7 & \cellcolor{green!0!red!100}-7.1 & \cellcolor{green!0!red!100}0.494 \\
\rowcolor{gray!30}
\official Llama-4-Maverick & \checkmark & 400 &  & \cellcolor{green!0!red!100}18.0 & \cellcolor{green!0!red!100}67.0 & \cellcolor{green!0!red!100}71.9 & \cellcolor{green!0!red!100}-7.5 & \cellcolor{green!0!red!100}0.479 \\
\official CommandR7B & \checkmark & 7 &  & \cellcolor{green!0!red!100}18.0 & \cellcolor{green!0!red!100}67.3 & \cellcolor{green!0!red!100}69.4 & \cellcolor{green!0!red!100}-7.4 & \cellcolor{green!0!red!100}0.486 \\
\official Llama-3.1-8B & \checkmark & 8 &  & \cellcolor{green!0!red!100}22.8 & \cellcolor{green!0!red!100}61.8 & \cellcolor{green!0!red!100}64.1 & \cellcolor{green!0!red!100}-8.1 & \cellcolor{green!0!red!100}0.449 \\
\official Qwen2.5-7B & \checkmark & 7 &  & \cellcolor{green!0!red!100}23.5 & \cellcolor{green!0!red!100}60.8 & \cellcolor{green!0!red!100}61.5 & \cellcolor{green!0!red!100}-7.8 & \cellcolor{green!0!red!100}0.44 \\
\official NLLB & \checkmark & 1 &  & \cellcolor{green!0!red!100}27.1 & \cellcolor{green!0!red!100}58.5 & \cellcolor{green!0!red!100}61.6 & \cellcolor{green!0!red!100}-9.3 & \cellcolor{green!0!red!100}0.421 \\
\rowcolor{gray!30}
\official ONLINE-G & \checkmark & \unknown &  & \cellcolor{green!0!red!100}30.0 & \cellcolor{green!0!red!100}58.7 & \cellcolor{green!0!red!100}60.6 & \cellcolor{green!0!red!100}-9.9 & \cellcolor{green!0!red!100}0.368 \\
\official Mistral-7B & \crossmark & 7 &  & \cellcolor{green!0!red!100}33.0 & \cellcolor{green!0!red!100}53.5 & \cellcolor{green!0!red!100}52.1 & \cellcolor{green!0!red!100}-9.8 & \cellcolor{green!0!red!100}0.363 \\
\bottomrule
\end{tabularx}
\end{table*}


\begin{table*}
\small
\begin{tabularx}{\textwidth}{lYYYYYYYYY}
\toprule
\multicolumn{10}{c}{\textbf{English-Japanese}} \\
\midrule
System Name & LP Supported & Params. (B) & Humeval? & AutoRank $\downarrow$ & CometKiwi-XL $\uparrow$ & GEMBA-ESA-CMDA $\uparrow$ & GEMBA-ESA-GPT4.1 $\uparrow$ & MetricX-24-Hybrid-XL $\uparrow$ & XCOMET-XL $\uparrow$ \\
\midrule
Shy-hunyuan-MT & \checkmark & 7 & \checkmark & \cellcolor{green!100!red!0}1.0 & \cellcolor{green!73!red!27}0.687 & \cellcolor{green!95!red!5}82.2 & \cellcolor{green!93!red!7}89.6 & \cellcolor{green!100!red!0}-5.5 & \cellcolor{green!100!red!0}0.592 \\
\rowcolor{gray!30}
In2x & \unknown & 72 & \checkmark & \cellcolor{green!92!red!8}2.3 & \cellcolor{green!100!red!0}0.711 & \cellcolor{green!78!red!22}78.4 & \cellcolor{green!78!red!22}86.3 & \cellcolor{green!79!red!21}-5.9 & \cellcolor{green!90!red!10}0.575 \\
\rowcolor{gray!30}
\official Gemini-2.5-Pro & \checkmark & \unknown & \checkmark & \cellcolor{green!91!red!9}2.4 & \cellcolor{green!56!red!44}0.672 & \cellcolor{green!100!red!0}83.2 & \cellcolor{green!100!red!0}91.2 & \cellcolor{green!90!red!10}-5.7 & \cellcolor{green!75!red!25}0.55 \\
\rowcolor{gray!30}
\official GPT-4.1 & \checkmark & \unknown & \checkmark & \cellcolor{green!88!red!12}2.9 & \cellcolor{green!58!red!42}0.674 & \cellcolor{green!93!red!7}81.8 & \cellcolor{green!93!red!7}89.7 & \cellcolor{green!79!red!21}-5.9 & \cellcolor{green!80!red!20}0.558 \\
Wenyiil & \checkmark & 14 & \checkmark & \cellcolor{green!88!red!12}2.9 & \cellcolor{green!67!red!33}0.682 & \cellcolor{green!83!red!17}79.6 & \cellcolor{green!89!red!11}88.6 & \cellcolor{green!90!red!10}-5.7 & \cellcolor{green!77!red!23}0.553 \\
KIKIS & \checkmark & 18 & \checkmark & \cellcolor{green!86!red!14}3.1 & \cellcolor{green!63!red!37}0.678 & \cellcolor{green!86!red!14}80.2 & \cellcolor{green!75!red!25}85.4 & \cellcolor{green!100!red!0}-5.5 & \cellcolor{green!76!red!24}0.551 \\
Algharb & \checkmark & 14 & \checkmark & \cellcolor{green!86!red!14}3.2 & \cellcolor{green!63!red!37}0.678 & \cellcolor{green!89!red!11}80.8 & \cellcolor{green!91!red!9}89.2 & \cellcolor{green!85!red!15}-5.8 & \cellcolor{green!70!red!30}0.541 \\
\rowcolor{gray!30}
CommandA-WMT & \checkmark & 111 & \checkmark & \cellcolor{green!83!red!17}3.6 & \cellcolor{green!81!red!19}0.694 & \cellcolor{green!69!red!31}76.5 & \cellcolor{green!75!red!25}85.5 & \cellcolor{green!85!red!15}-5.8 & \cellcolor{green!75!red!25}0.55 \\
\rowcolor{gray!30}
\official DeepSeek-V3 & \unknown & 671 & \checkmark & \cellcolor{green!77!red!23}4.6 & \cellcolor{green!50!red!50}0.667 & \cellcolor{green!87!red!13}80.5 & \cellcolor{green!85!red!15}87.7 & \cellcolor{green!64!red!36}-6.2 & \cellcolor{green!64!red!36}0.531 \\
\rowcolor{gray!30}
\official Mistral-Medium & \unknown & \unknown & \checkmark & \cellcolor{green!71!red!29}5.4 & \cellcolor{green!59!red!41}0.675 & \cellcolor{green!74!red!26}77.6 & \cellcolor{green!78!red!22}86.2 & \cellcolor{green!54!red!46}-6.4 & \cellcolor{green!64!red!36}0.532 \\
\rowcolor{gray!30}
GemTrans & \checkmark & 27 & \checkmark & \cellcolor{green!71!red!29}5.5 & \cellcolor{green!50!red!50}0.667 & \cellcolor{green!49!red!51}72.3 & \cellcolor{green!52!red!48}80.2 & \cellcolor{green!100!red!0}-5.5 & \cellcolor{green!77!red!23}0.553 \\
\rowcolor{gray!30}
\official Claude-4 & \checkmark & \unknown & \checkmark & \cellcolor{green!69!red!31}5.7 & \cellcolor{green!61!red!39}0.677 & \cellcolor{green!77!red!23}78.3 & \cellcolor{green!78!red!22}86.3 & \cellcolor{green!49!red!51}-6.5 & \cellcolor{green!55!red!45}0.516 \\
Yolu & \checkmark & 14 & \checkmark & \cellcolor{green!68!red!32}5.9 & \cellcolor{green!84!red!16}0.697 & \cellcolor{green!38!red!62}69.9 & \cellcolor{green!42!red!58}77.9 & \cellcolor{green!79!red!21}-5.9 & \cellcolor{green!70!red!30}0.541 \\
\rowcolor{gray!30}
\official ONLINE-B & \checkmark & \unknown & \checkmark & \cellcolor{green!67!red!33}6.1 & \cellcolor{green!69!red!31}0.684 & \cellcolor{green!49!red!51}72.4 & \cellcolor{green!51!red!49}80.0 & \cellcolor{green!74!red!26}-6.0 & \cellcolor{green!61!red!39}0.527 \\
\rowcolor{gray!30}
UvA-MT & \checkmark & 12 & \checkmark & \cellcolor{green!65!red!35}6.4 & \cellcolor{green!77!red!23}0.691 & \cellcolor{green!50!red!50}72.5 & \cellcolor{green!60!red!40}82.0 & \cellcolor{green!59!red!41}-6.3 & \cellcolor{green!55!red!45}0.517 \\
\rowcolor{gray!30}
bb88 & \unknown & \unknown &  & \cellcolor{green!60!red!40}7.2 & \cellcolor{green!58!red!42}0.674 & \cellcolor{green!60!red!40}74.6 & \cellcolor{green!62!red!38}82.6 & \cellcolor{green!49!red!51}-6.5 & \cellcolor{green!44!red!56}0.498 \\
\rowcolor{gray!30}
\official CommandA & \checkmark & 111 &  & \cellcolor{green!59!red!41}7.3 & \cellcolor{green!58!red!42}0.674 & \cellcolor{green!61!red!39}74.8 & \cellcolor{green!64!red!36}82.9 & \cellcolor{green!44!red!56}-6.6 & \cellcolor{green!48!red!52}0.504 \\
\rowcolor{gray!30}
\official Qwen3-235B & \checkmark & 235 &  & \cellcolor{green!59!red!41}7.3 & \cellcolor{green!50!red!50}0.667 & \cellcolor{green!58!red!42}74.3 & \cellcolor{green!65!red!35}83.2 & \cellcolor{green!54!red!46}-6.4 & \cellcolor{green!45!red!55}0.499 \\
Systran & \checkmark & 18 & \checkmark & \cellcolor{green!59!red!41}7.3 & \cellcolor{green!91!red!9}0.703 & \cellcolor{green!32!red!68}68.7 & \cellcolor{green!38!red!62}77.1 & \cellcolor{green!49!red!51}-6.5 & \cellcolor{green!59!red!41}0.523 \\
\rowcolor{gray!30}
\official Gemma-3-27B & \checkmark & 27 &  & \cellcolor{green!55!red!45}7.9 & \cellcolor{green!49!red!51}0.666 & \cellcolor{green!58!red!42}74.3 & \cellcolor{green!60!red!40}82.2 & \cellcolor{green!44!red!56}-6.6 & \cellcolor{green!44!red!56}0.497 \\
NTTSU & \checkmark & 14 & \checkmark & \cellcolor{green!55!red!45}8.0 & \cellcolor{green!60!red!40}0.676 & \cellcolor{green!27!red!73}67.7 & \cellcolor{green!26!red!74}74.3 & \cellcolor{green!95!red!5}-5.6 & \cellcolor{green!44!red!56}0.498 \\
\rowcolor{gray!30}
\official TowerPlus-72B[M] & \checkmark & 72 &  & \cellcolor{green!51!red!49}8.6 & \cellcolor{green!55!red!45}0.671 & \cellcolor{green!45!red!55}71.4 & \cellcolor{green!53!red!47}80.6 & \cellcolor{green!33!red!67}-6.8 & \cellcolor{green!45!red!55}0.499 \\
\rowcolor{gray!30}
\official Llama-4-Maverick & \checkmark & 400 &  & \cellcolor{green!47!red!53}9.1 & \cellcolor{green!43!red!57}0.661 & \cellcolor{green!45!red!55}71.5 & \cellcolor{green!56!red!44}81.1 & \cellcolor{green!33!red!67}-6.8 & \cellcolor{green!38!red!62}0.487 \\
Laniqo & \checkmark & 9 & \checkmark & \cellcolor{green!46!red!54}9.3 & \cellcolor{green!61!red!39}0.677 & \cellcolor{green!20!red!80}66.1 & \cellcolor{green!7!red!93}70.1 & \cellcolor{green!59!red!41}-6.3 & \cellcolor{green!63!red!37}0.529 \\
\rowcolor{gray!30}
\official AyaExpanse-32B & \checkmark & 32 &  & \cellcolor{green!43!red!57}9.8 & \cellcolor{green!44!red!56}0.662 & \cellcolor{green!42!red!58}70.9 & \cellcolor{green!44!red!56}78.5 & \cellcolor{green!33!red!67}-6.8 & \cellcolor{green!29!red!71}0.472 \\
IRB-MT & \checkmark & 12 &  & \cellcolor{green!40!red!60}10.3 & \cellcolor{green!23!red!77}0.643 & \cellcolor{green!38!red!62}70.0 & \cellcolor{green!42!red!58}77.9 & \cellcolor{green!49!red!51}-6.5 & \cellcolor{green!30!red!70}0.474 \\
\official TowerPlus-9B[M] & \checkmark & 9 &  & \cellcolor{green!39!red!61}10.4 & \cellcolor{green!48!red!52}0.665 & \cellcolor{green!32!red!68}68.7 & \cellcolor{green!35!red!65}76.3 & \cellcolor{green!28!red!72}-6.9 & \cellcolor{green!32!red!68}0.477 \\
SRPOL & \crossmark & 12 &  & \cellcolor{green!36!red!64}10.8 & \cellcolor{green!68!red!32}0.683 & \cellcolor{green!23!red!77}66.7 & \cellcolor{green!24!red!76}73.9 & \cellcolor{green!18!red!82}-7.1 & \cellcolor{green!29!red!71}0.472 \\
\rowcolor{gray!30}
TranssionTranslate & \unknown & \unknown &  & \cellcolor{green!29!red!71}12.0 & \cellcolor{green!51!red!49}0.668 & \cellcolor{green!13!red!87}64.6 & \cellcolor{green!15!red!85}71.8 & \cellcolor{green!28!red!72}-6.9 & \cellcolor{green!21!red!79}0.459 \\
\official Gemma-3-12B & \checkmark & 12 &  & \cellcolor{green!18!red!82}13.6 & \cellcolor{green!0!red!100}0.623 & \cellcolor{green!13!red!87}64.7 & \cellcolor{green!24!red!76}73.8 & \cellcolor{green!23!red!77}-7.0 & \cellcolor{green!22!red!78}0.461 \\
\official AyaExpanse-8B & \checkmark & 8 &  & \cellcolor{green!5!red!95}15.6 & \cellcolor{green!10!red!90}0.632 & \cellcolor{green!1!red!99}62.1 & \cellcolor{green!4!red!96}69.4 & \cellcolor{green!3!red!97}-7.4 & \cellcolor{green!0!red!100}0.422 \\
\rowcolor{gray!30}
\official ONLINE-W & \unknown & \unknown &  & \cellcolor{green!0!red!100}16.4 & \cellcolor{green!0!red!100}0.611 & \cellcolor{green!0!red!100}61.7 & \cellcolor{green!0!red!100}67.5 & \cellcolor{green!8!red!92}-7.3 & \cellcolor{green!5!red!95}0.432 \\
\rowcolor{gray!30}
SH & \checkmark & 56 &  & \cellcolor{green!0!red!100}16.4 & \cellcolor{green!20!red!80}0.641 & \cellcolor{green!0!red!100}59.9 & \cellcolor{green!0!red!100}65.6 & \cellcolor{green!0!red!100}-7.5 & \cellcolor{green!0!red!100}0.419 \\
\rowcolor{gray!30}
\official EuroLLM-22B-pre.[M] & \checkmark & 22 &  & \cellcolor{green!0!red!100}16.8 & \cellcolor{green!0!red!100}0.623 & \cellcolor{green!1!red!99}62.0 & \cellcolor{green!4!red!96}69.4 & \cellcolor{green!0!red!100}-7.9 & \cellcolor{green!1!red!99}0.425 \\
\official CommandR7B & \checkmark & 7 &  & \cellcolor{green!0!red!100}19.9 & \cellcolor{green!0!red!100}0.62 & \cellcolor{green!0!red!100}59.3 & \cellcolor{green!0!red!100}65.1 & \cellcolor{green!0!red!100}-8.6 & \cellcolor{green!0!red!100}0.379 \\
\rowcolor{gray!30}
IR-MultiagentMT & \crossmark & \unknown &  & \cellcolor{green!0!red!100}22.9 & \cellcolor{green!0!red!100}0.576 & \cellcolor{green!0!red!100}54.6 & \cellcolor{green!0!red!100}62.1 & \cellcolor{green!0!red!100}-8.6 & \cellcolor{green!0!red!100}0.373 \\
\official EuroLLM-9B[M] & \checkmark & 9 &  & \cellcolor{green!0!red!100}23.6 & \cellcolor{green!0!red!100}0.561 & \cellcolor{green!0!red!100}53.8 & \cellcolor{green!0!red!100}60.5 & \cellcolor{green!0!red!100}-8.6 & \cellcolor{green!0!red!100}0.391 \\
\official Qwen2.5-7B & \checkmark & 7 &  & \cellcolor{green!0!red!100}24.5 & \cellcolor{green!0!red!100}0.594 & \cellcolor{green!0!red!100}54.6 & \cellcolor{green!0!red!100}58.6 & \cellcolor{green!0!red!100}-9.2 & \cellcolor{green!0!red!100}0.338 \\
\official Llama-3.1-8B & \crossmark & 8 &  & \cellcolor{green!0!red!100}25.5 & \cellcolor{green!0!red!100}0.596 & \cellcolor{green!0!red!100}51.4 & \cellcolor{green!0!red!100}54.8 & \cellcolor{green!0!red!100}-9.0 & \cellcolor{green!0!red!100}0.32 \\
SalamandraTA & \checkmark & 8 &  & \cellcolor{green!0!red!100}26.4 & \cellcolor{green!0!red!100}0.603 & \cellcolor{green!0!red!100}51.8 & \cellcolor{green!0!red!100}53.1 & \cellcolor{green!0!red!100}-9.4 & \cellcolor{green!0!red!100}0.299 \\
\official NLLB & \checkmark & 1 &  & \cellcolor{green!0!red!100}37.9 & \cellcolor{green!0!red!100}0.479 & \cellcolor{green!0!red!100}42.7 & \cellcolor{green!0!red!100}46.8 & \cellcolor{green!0!red!100}-11.5 & \cellcolor{green!0!red!100}0.245 \\
\rowcolor{gray!30}
\official ONLINE-G & \checkmark & \unknown &  & \cellcolor{green!0!red!100}40.7 & \cellcolor{green!0!red!100}0.495 & \cellcolor{green!0!red!100}45.0 & \cellcolor{green!0!red!100}45.8 & \cellcolor{green!0!red!100}-13.2 & \cellcolor{green!0!red!100}0.207 \\
\official Mistral-7B & \crossmark & 7 &  & \cellcolor{green!0!red!100}43.0 & \cellcolor{green!0!red!100}0.462 & \cellcolor{green!0!red!100}39.2 & \cellcolor{green!0!red!100}40.2 & \cellcolor{green!0!red!100}-12.6 & \cellcolor{green!0!red!100}0.193 \\
\bottomrule
\end{tabularx}
\end{table*}


\begin{table*}
\small
\begin{tabularx}{\textwidth}{lYYYYYYYYY}
\toprule
\multicolumn{10}{c}{\textbf{English-Korean}} \\
\midrule
System Name & LP Supported & Params. (B) & Humeval? & AutoRank $\downarrow$ & CometKiwi-XL $\uparrow$ & GEMBA-ESA-CMDA $\uparrow$ & GEMBA-ESA-GPT4.1 $\uparrow$ & MetricX-24-Hybrid-XL $\uparrow$ & XCOMET-XL $\uparrow$ \\
\midrule
Shy-hunyuan-MT & \checkmark & 7 & \checkmark & \cellcolor{green!100!red!0}1.0 & \cellcolor{green!84!red!16}0.697 & \cellcolor{green!93!red!7}83.8 & \cellcolor{green!89!red!11}85.6 & \cellcolor{green!100!red!0}-4.9 & \cellcolor{green!100!red!0}0.624 \\
\rowcolor{gray!30}
\official Gemini-2.5-Pro & \checkmark & \unknown & \checkmark & \cellcolor{green!88!red!12}2.5 & \cellcolor{green!67!red!33}0.683 & \cellcolor{green!100!red!0}85.3 & \cellcolor{green!100!red!0}88.1 & \cellcolor{green!76!red!24}-5.6 & \cellcolor{green!74!red!26}0.571 \\
\rowcolor{gray!30}
CommandA-WMT & \checkmark & 111 & \checkmark & \cellcolor{green!86!red!14}2.8 & \cellcolor{green!100!red!0}0.711 & \cellcolor{green!74!red!26}79.6 & \cellcolor{green!75!red!25}82.3 & \cellcolor{green!76!red!24}-5.6 & \cellcolor{green!80!red!20}0.584 \\
\rowcolor{gray!30}
\official GPT-4.1 & \checkmark & \unknown & \checkmark & \cellcolor{green!86!red!14}2.8 & \cellcolor{green!71!red!29}0.686 & \cellcolor{green!92!red!8}83.6 & \cellcolor{green!92!red!8}86.3 & \cellcolor{green!72!red!28}-5.7 & \cellcolor{green!79!red!21}0.581 \\
Wenyiil & \checkmark & 14 & \checkmark & \cellcolor{green!85!red!15}2.9 & \cellcolor{green!77!red!23}0.691 & \cellcolor{green!85!red!15}82.1 & \cellcolor{green!87!red!13}85.0 & \cellcolor{green!76!red!24}-5.6 & \cellcolor{green!77!red!23}0.576 \\
Algharb & \checkmark & 14 & \checkmark & \cellcolor{green!84!red!16}3.0 & \cellcolor{green!72!red!28}0.687 & \cellcolor{green!90!red!10}83.2 & \cellcolor{green!91!red!9}85.9 & \cellcolor{green!72!red!28}-5.7 & \cellcolor{green!71!red!29}0.565 \\
\rowcolor{gray!30}
UvA-MT & \checkmark & 12 & \checkmark & \cellcolor{green!75!red!25}4.2 & \cellcolor{green!94!red!6}0.706 & \cellcolor{green!68!red!32}78.2 & \cellcolor{green!70!red!30}81.1 & \cellcolor{green!62!red!38}-6.0 & \cellcolor{green!66!red!34}0.554 \\
\rowcolor{gray!30}
\official Claude-4 & \checkmark & \unknown & \checkmark & \cellcolor{green!74!red!26}4.3 & \cellcolor{green!80!red!20}0.694 & \cellcolor{green!85!red!15}82.0 & \cellcolor{green!85!red!15}84.6 & \cellcolor{green!52!red!48}-6.3 & \cellcolor{green!57!red!43}0.536 \\
\rowcolor{gray!30}
GemTrans & \checkmark & 27 & \checkmark & \cellcolor{green!70!red!30}4.9 & \cellcolor{green!60!red!40}0.677 & \cellcolor{green!61!red!39}76.7 & \cellcolor{green!60!red!40}78.8 & \cellcolor{green!83!red!17}-5.4 & \cellcolor{green!73!red!27}0.568 \\
\rowcolor{gray!30}
\official DeepSeek-V3 & \unknown & 671 & \checkmark & \cellcolor{green!69!red!31}5.0 & \cellcolor{green!65!red!35}0.681 & \cellcolor{green!74!red!26}79.5 & \cellcolor{green!81!red!19}83.7 & \cellcolor{green!59!red!41}-6.1 & \cellcolor{green!59!red!41}0.539 \\
\rowcolor{gray!30}
\official CommandA & \checkmark & 111 & \checkmark & \cellcolor{green!63!red!37}5.8 & \cellcolor{green!78!red!22}0.692 & \cellcolor{green!66!red!34}77.8 & \cellcolor{green!69!red!31}80.9 & \cellcolor{green!41!red!59}-6.6 & \cellcolor{green!51!red!49}0.524 \\
\rowcolor{gray!30}
\official Mistral-Medium & \unknown & \unknown & \checkmark & \cellcolor{green!61!red!39}6.0 & \cellcolor{green!68!red!32}0.684 & \cellcolor{green!63!red!37}77.2 & \cellcolor{green!63!red!37}79.4 & \cellcolor{green!52!red!48}-6.3 & \cellcolor{green!54!red!46}0.53 \\
\rowcolor{gray!30}
\official Qwen3-235B & \checkmark & 235 & \checkmark & \cellcolor{green!59!red!41}6.3 & \cellcolor{green!61!red!39}0.678 & \cellcolor{green!63!red!37}77.1 & \cellcolor{green!65!red!35}80.0 & \cellcolor{green!55!red!45}-6.2 & \cellcolor{green!44!red!56}0.509 \\
Yolu & \checkmark & 14 & \checkmark & \cellcolor{green!55!red!45}6.8 & \cellcolor{green!88!red!12}0.701 & \cellcolor{green!31!red!69}70.1 & \cellcolor{green!35!red!65}73.0 & \cellcolor{green!66!red!34}-5.9 & \cellcolor{green!56!red!44}0.533 \\
\rowcolor{gray!30}
\official ONLINE-B & \checkmark & \unknown & \checkmark & \cellcolor{green!47!red!53}7.8 & \cellcolor{green!63!red!37}0.679 & \cellcolor{green!43!red!57}72.8 & \cellcolor{green!39!red!61}73.8 & \cellcolor{green!55!red!45}-6.2 & \cellcolor{green!41!red!59}0.504 \\
IRB-MT & \checkmark & 12 & \checkmark & \cellcolor{green!42!red!58}8.4 & \cellcolor{green!37!red!63}0.657 & \cellcolor{green!53!red!47}74.9 & \cellcolor{green!49!red!51}76.3 & \cellcolor{green!48!red!52}-6.4 & \cellcolor{green!34!red!66}0.489 \\
\rowcolor{gray!30}
\official TowerPlus-72B[M] & \checkmark & 72 &  & \cellcolor{green!42!red!58}8.5 & \cellcolor{green!68!red!32}0.684 & \cellcolor{green!40!red!60}72.2 & \cellcolor{green!45!red!55}75.2 & \cellcolor{green!34!red!66}-6.8 & \cellcolor{green!33!red!67}0.487 \\
\rowcolor{gray!30}
\official AyaExpanse-32B & \checkmark & 32 &  & \cellcolor{green!41!red!59}8.6 & \cellcolor{green!56!red!44}0.673 & \cellcolor{green!40!red!60}72.1 & \cellcolor{green!47!red!53}75.7 & \cellcolor{green!38!red!62}-6.7 & \cellcolor{green!36!red!64}0.493 \\
\rowcolor{gray!30}
\official Llama-4-Maverick & \checkmark & 400 &  & \cellcolor{green!39!red!61}8.8 & \cellcolor{green!46!red!54}0.665 & \cellcolor{green!47!red!53}73.6 & \cellcolor{green!44!red!56}75.0 & \cellcolor{green!38!red!62}-6.7 & \cellcolor{green!33!red!67}0.487 \\
Laniqo & \checkmark & 9 & \checkmark & \cellcolor{green!38!red!62}8.9 & \cellcolor{green!74!red!26}0.689 & \cellcolor{green!7!red!93}64.9 & \cellcolor{green!6!red!94}66.2 & \cellcolor{green!59!red!41}-6.1 & \cellcolor{green!59!red!41}0.54 \\
\official Gemma-3-12B & \checkmark & 12 & \checkmark & \cellcolor{green!38!red!62}9.0 & \cellcolor{green!49!red!51}0.667 & \cellcolor{green!47!red!53}73.6 & \cellcolor{green!52!red!48}77.0 & \cellcolor{green!28!red!72}-7.0 & \cellcolor{green!27!red!73}0.474 \\
\official TowerPlus-9B[M] & \checkmark & 9 & \checkmark & \cellcolor{green!31!red!69}9.8 & \cellcolor{green!61!red!39}0.678 & \cellcolor{green!31!red!69}70.2 & \cellcolor{green!37!red!63}73.5 & \cellcolor{green!21!red!79}-7.2 & \cellcolor{green!26!red!74}0.472 \\
\rowcolor{gray!30}
\official ONLINE-W & \unknown & \unknown &  & \cellcolor{green!27!red!73}10.4 & \cellcolor{green!57!red!43}0.674 & \cellcolor{green!20!red!80}67.7 & \cellcolor{green!20!red!80}69.4 & \cellcolor{green!34!red!66}-6.8 & \cellcolor{green!23!red!77}0.467 \\
\rowcolor{gray!30}
TranssionTranslate & \unknown & \unknown &  & \cellcolor{green!14!red!86}12.0 & \cellcolor{green!58!red!42}0.675 & \cellcolor{green!3!red!97}64.0 & \cellcolor{green!1!red!99}65.0 & \cellcolor{green!31!red!69}-6.9 & \cellcolor{green!10!red!90}0.439 \\
\official AyaExpanse-8B & \checkmark & 8 &  & \cellcolor{green!9!red!91}12.7 & \cellcolor{green!37!red!63}0.657 & \cellcolor{green!6!red!94}64.6 & \cellcolor{green!12!red!88}67.7 & \cellcolor{green!17!red!83}-7.3 & \cellcolor{green!7!red!93}0.434 \\
\rowcolor{gray!30}
\official Gemma-3-27B & \checkmark & 27 &  & \cellcolor{green!7!red!93}12.9 & \cellcolor{green!1!red!99}0.626 & \cellcolor{green!17!red!83}67.1 & \cellcolor{green!13!red!87}67.9 & \cellcolor{green!14!red!86}-7.4 & \cellcolor{green!28!red!72}0.477 \\
\rowcolor{gray!30}
\official EuroLLM-22B-pre.[M] & \checkmark & 22 &  & \cellcolor{green!6!red!94}13.0 & \cellcolor{green!33!red!67}0.654 & \cellcolor{green!14!red!86}66.4 & \cellcolor{green!18!red!82}68.9 & \cellcolor{green!3!red!97}-7.7 & \cellcolor{green!1!red!99}0.422 \\
\rowcolor{gray!30}
IR-MultiagentMT & \crossmark & \unknown &  & \cellcolor{green!0!red!100}16.3 & \cellcolor{green!0!red!100}0.614 & \cellcolor{green!0!red!100}61.2 & \cellcolor{green!0!red!100}64.2 & \cellcolor{green!0!red!100}-8.1 & \cellcolor{green!0!red!100}0.41 \\
\official CommandR7B & \checkmark & 7 &  & \cellcolor{green!0!red!100}18.2 & \cellcolor{green!0!red!100}0.619 & \cellcolor{green!0!red!100}59.8 & \cellcolor{green!0!red!100}61.3 & \cellcolor{green!0!red!100}-8.8 & \cellcolor{green!0!red!100}0.364 \\
\official EuroLLM-9B[M] & \checkmark & 9 &  & \cellcolor{green!0!red!100}19.1 & \cellcolor{green!0!red!100}0.594 & \cellcolor{green!0!red!100}56.8 & \cellcolor{green!0!red!100}57.7 & \cellcolor{green!0!red!100}-8.3 & \cellcolor{green!0!red!100}0.39 \\
SalamandraTA & \checkmark & 8 &  & \cellcolor{green!0!red!100}22.8 & \cellcolor{green!0!red!100}0.624 & \cellcolor{green!0!red!100}50.6 & \cellcolor{green!0!red!100}50.3 & \cellcolor{green!0!red!100}-9.7 & \cellcolor{green!0!red!100}0.29 \\
\official Llama-3.1-8B & \crossmark & 8 &  & \cellcolor{green!0!red!100}24.8 & \cellcolor{green!0!red!100}0.586 & \cellcolor{green!0!red!100}50.7 & \cellcolor{green!0!red!100}50.8 & \cellcolor{green!0!red!100}-10.2 & \cellcolor{green!0!red!100}0.278 \\
\official Qwen2.5-7B & \checkmark & 7 &  & \cellcolor{green!0!red!100}25.0 & \cellcolor{green!0!red!100}0.568 & \cellcolor{green!0!red!100}47.9 & \cellcolor{green!0!red!100}48.7 & \cellcolor{green!0!red!100}-9.4 & \cellcolor{green!0!red!100}0.291 \\
\official NLLB & \checkmark & 1 &  & \cellcolor{green!0!red!100}28.1 & \cellcolor{green!0!red!100}0.549 & \cellcolor{green!0!red!100}42.8 & \cellcolor{green!0!red!100}44.3 & \cellcolor{green!0!red!100}-10.4 & \cellcolor{green!0!red!100}0.286 \\
\rowcolor{gray!30}
\official ONLINE-G & \checkmark & \unknown &  & \cellcolor{green!0!red!100}32.4 & \cellcolor{green!0!red!100}0.532 & \cellcolor{green!0!red!100}44.8 & \cellcolor{green!0!red!100}44.3 & \cellcolor{green!0!red!100}-12.8 & \cellcolor{green!0!red!100}0.187 \\
\official Mistral-7B & \crossmark & 7 &  & \cellcolor{green!0!red!100}36.0 & \cellcolor{green!0!red!100}0.478 & \cellcolor{green!0!red!100}37.8 & \cellcolor{green!0!red!100}39.3 & \cellcolor{green!0!red!100}-12.7 & \cellcolor{green!0!red!100}0.174 \\
\bottomrule
\end{tabularx}
\end{table*}


\begin{table*}
\small
\begin{tabularx}{\textwidth}{lYYYYY}
\toprule
\multicolumn{6}{c}{\textbf{English-Maasai}} \\
\midrule
System Name & LP Supported & Params. (B) & Humeval? & AutoRank $\downarrow$ & chrF++ $\uparrow$ \\
\midrule
Shy-hunyuan-MT & \crossmark & 7 & \checkmark & \cellcolor{green!100!red!0}1.0 & \cellcolor{green!100!red!0}27.7 \\
\rowcolor{gray!30}
\official Claude-4 & \unknown & \unknown & \checkmark & \cellcolor{green!81!red!19}2.6 & \cellcolor{green!81!red!19}26.1 \\
\rowcolor{gray!30}
\official Qwen3-235B & \crossmark & 235 & \checkmark & \cellcolor{green!76!red!24}3.0 & \cellcolor{green!76!red!24}25.6 \\
\rowcolor{gray!30}
\official Llama-4-Maverick & \crossmark & 400 & \checkmark & \cellcolor{green!74!red!26}3.2 & \cellcolor{green!73!red!27}25.4 \\
\official CommandR7B & \crossmark & 7 & \checkmark & \cellcolor{green!60!red!40}4.3 & \cellcolor{green!60!red!40}24.3 \\
\official TowerPlus-9B[M] & \crossmark & 9 & \checkmark & \cellcolor{green!49!red!51}5.3 & \cellcolor{green!48!red!52}23.2 \\
\rowcolor{gray!30}
TranssionMT & \checkmark & 1 & \checkmark & \cellcolor{green!41!red!59}5.9 & \cellcolor{green!41!red!59}22.6 \\
\rowcolor{gray!30}
\official Gemini-2.5-Pro & \unknown & \unknown & \checkmark & \cellcolor{green!39!red!61}6.1 & \cellcolor{green!40!red!60}22.5 \\
\rowcolor{gray!30}
\official DeepSeek-V3 & \unknown & 671 & \checkmark & \cellcolor{green!38!red!62}6.2 & \cellcolor{green!38!red!62}22.4 \\
\rowcolor{gray!30}
CommandA-WMT & \crossmark & 111 & \checkmark & \cellcolor{green!35!red!65}6.4 & \cellcolor{green!36!red!64}22.2 \\
\rowcolor{gray!30}
\official AyaExpanse-32B & \crossmark & 32 & \checkmark & \cellcolor{green!27!red!73}7.1 & \cellcolor{green!27!red!73}21.4 \\
\rowcolor{gray!30}
\official CommandA & \crossmark & 111 & \checkmark & \cellcolor{green!17!red!83}7.9 & \cellcolor{green!17!red!83}20.6 \\
\official Llama-3.1-8B & \crossmark & 8 & \checkmark & \cellcolor{green!15!red!85}8.1 & \cellcolor{green!15!red!85}20.4 \\
\official EuroLLM-9B[M] & \crossmark & 9 & \checkmark & \cellcolor{green!14!red!86}8.2 & \cellcolor{green!14!red!86}20.3 \\
\rowcolor{gray!30}
\official EuroLLM-22B-pre.[M] & \crossmark & 22 & \checkmark & \cellcolor{green!14!red!86}8.2 & \cellcolor{green!14!red!86}20.3 \\
\official AyaExpanse-8B & \crossmark & 8 & \checkmark & \cellcolor{green!14!red!86}8.2 & \cellcolor{green!13!red!87}20.2 \\
\official Qwen2.5-7B & \unknown & 7 & \checkmark & \cellcolor{green!9!red!91}8.6 & \cellcolor{green!9!red!91}19.9 \\
\rowcolor{gray!30}
\official TowerPlus-72B[M] & \crossmark & 72 &  & \cellcolor{green!7!red!93}8.8 & \cellcolor{green!7!red!93}19.7 \\
\official Gemma-3-12B & \unknown & 12 & \checkmark & \cellcolor{green!7!red!93}8.8 & \cellcolor{green!6!red!94}19.6 \\
\rowcolor{gray!30}
IR-MultiagentMT & \crossmark & \unknown &  & \cellcolor{green!4!red!96}9.0 & \cellcolor{green!5!red!95}19.5 \\
IRB-MT & \checkmark & 12 &  & \cellcolor{green!0!red!100}9.7 & \cellcolor{green!0!red!100}18.7 \\
\official Mistral-7B & \crossmark & 7 &  & \cellcolor{green!0!red!100}11.3 & \cellcolor{green!0!red!100}17.1 \\
\rowcolor{gray!30}
\official Gemma-3-27B & \unknown & 27 &  & \cellcolor{green!0!red!100}13.3 & \cellcolor{green!0!red!100}15.1 \\
\rowcolor{gray!30}
UvA-MT & \checkmark & 12 &  & \cellcolor{green!0!red!100}14.7 & \cellcolor{green!0!red!100}13.6 \\
\rowcolor{gray!30}
\official GPT-4.1 & \unknown & \unknown &  & \cellcolor{green!0!red!100}14.9 & \cellcolor{green!0!red!100}13.4 \\
\rowcolor{gray!30}
GemTrans & \crossmark & 27 &  & \cellcolor{green!0!red!100}16.7 & \cellcolor{green!0!red!100}11.6 \\
\official NLLB & \crossmark & 1 &  & \cellcolor{green!0!red!100}27.0 & \cellcolor{green!0!red!100}0.9 \\
\bottomrule
\end{tabularx}
\end{table*}


\begin{table*}
\small
\begin{tabularx}{\textwidth}{lYYYYYYYYY}
\toprule
\multicolumn{10}{c}{\textbf{English-Russian}} \\
\midrule
System Name & LP Supported & Params. (B) & Humeval? & AutoRank $\downarrow$ & CometKiwi-XL $\uparrow$ & GEMBA-ESA-CMDA $\uparrow$ & GEMBA-ESA-GPT4.1 $\uparrow$ & MetricX-24-Hybrid-XL $\uparrow$ & XCOMET-XL $\uparrow$ \\
\midrule
Shy-hunyuan-MT & \checkmark & 7 & \checkmark & \cellcolor{green!100!red!0}1.0 & \cellcolor{green!92!red!8}0.657 & \cellcolor{green!91!red!9}84.3 & \cellcolor{green!91!red!9}85.9 & \cellcolor{green!100!red!0}-4.9 & \cellcolor{green!100!red!0}0.652 \\
\rowcolor{gray!30}
CommandA-WMT & \checkmark & 111 & \checkmark & \cellcolor{green!76!red!24}4.2 & \cellcolor{green!90!red!10}0.656 & \cellcolor{green!74!red!26}81.3 & \cellcolor{green!65!red!35}80.5 & \cellcolor{green!74!red!26}-5.8 & \cellcolor{green!70!red!30}0.607 \\
\rowcolor{gray!30}
\official Gemini-2.5-Pro & \checkmark & \unknown & \checkmark & \cellcolor{green!76!red!24}4.3 & \cellcolor{green!55!red!45}0.634 & \cellcolor{green!100!red!0}85.9 & \cellcolor{green!100!red!0}87.8 & \cellcolor{green!66!red!34}-6.1 & \cellcolor{green!48!red!52}0.575 \\
\rowcolor{gray!30}
Yandex & \checkmark & \unknown & \checkmark & \cellcolor{green!75!red!25}4.4 & \cellcolor{green!62!red!38}0.638 & \cellcolor{green!74!red!26}81.2 & \cellcolor{green!66!red!34}80.6 & \cellcolor{green!89!red!11}-5.3 & \cellcolor{green!77!red!23}0.617 \\
\rowcolor{gray!30}
UvA-MT & \checkmark & 12 & \checkmark & \cellcolor{green!74!red!26}4.5 & \cellcolor{green!100!red!0}0.662 & \cellcolor{green!60!red!40}78.6 & \cellcolor{green!65!red!35}80.5 & \cellcolor{green!66!red!34}-6.1 & \cellcolor{green!73!red!27}0.611 \\
Wenyiil & \checkmark & 14 & \checkmark & \cellcolor{green!73!red!27}4.7 & \cellcolor{green!71!red!29}0.644 & \cellcolor{green!81!red!19}82.5 & \cellcolor{green!82!red!18}84.1 & \cellcolor{green!66!red!34}-6.1 & \cellcolor{green!57!red!43}0.588 \\
\rowcolor{gray!30}
GemTrans & \checkmark & 27 & \checkmark & \cellcolor{green!70!red!30}5.1 & \cellcolor{green!63!red!37}0.639 & \cellcolor{green!55!red!45}77.8 & \cellcolor{green!60!red!40}79.5 & \cellcolor{green!89!red!11}-5.3 & \cellcolor{green!77!red!23}0.617 \\
Algharb & \checkmark & 14 & \checkmark & \cellcolor{green!70!red!30}5.1 & \cellcolor{green!60!red!40}0.637 & \cellcolor{green!92!red!8}84.4 & \cellcolor{green!89!red!11}85.5 & \cellcolor{green!57!red!43}-6.4 & \cellcolor{green!47!red!53}0.573 \\
\rowcolor{gray!30}
\official GPT-4.1 & \checkmark & \unknown & \checkmark & \cellcolor{green!68!red!32}5.3 & \cellcolor{green!50!red!50}0.631 & \cellcolor{green!93!red!7}84.6 & \cellcolor{green!90!red!10}85.8 & \cellcolor{green!55!red!45}-6.5 & \cellcolor{green!50!red!50}0.577 \\
\rowcolor{gray!30}
\official DeepSeek-V3 & \unknown & 671 & \checkmark & \cellcolor{green!66!red!34}5.6 & \cellcolor{green!52!red!48}0.632 & \cellcolor{green!91!red!9}84.2 & \cellcolor{green!85!red!15}84.7 & \cellcolor{green!57!red!43}-6.4 & \cellcolor{green!45!red!55}0.57 \\
Yolu & \checkmark & 14 & \checkmark & \cellcolor{green!56!red!44}6.9 & \cellcolor{green!94!red!6}0.658 & \cellcolor{green!29!red!71}73.1 & \cellcolor{green!32!red!68}73.6 & \cellcolor{green!69!red!31}-6.0 & \cellcolor{green!62!red!38}0.596 \\
\rowcolor{gray!30}
\official Claude-4 & \checkmark & \unknown & \checkmark & \cellcolor{green!44!red!56}8.5 & \cellcolor{green!31!red!69}0.619 & \cellcolor{green!78!red!22}82.0 & \cellcolor{green!70!red!30}81.6 & \cellcolor{green!26!red!74}-7.5 & \cellcolor{green!30!red!70}0.548 \\
\rowcolor{gray!30}
\official Qwen3-235B & \checkmark & 235 & \checkmark & \cellcolor{green!43!red!57}8.7 & \cellcolor{green!41!red!59}0.625 & \cellcolor{green!57!red!43}78.2 & \cellcolor{green!62!red!38}79.9 & \cellcolor{green!43!red!57}-6.9 & \cellcolor{green!27!red!73}0.543 \\
\rowcolor{gray!30}
\official Gemma-3-27B & \checkmark & 27 & \checkmark & \cellcolor{green!43!red!57}8.7 & \cellcolor{green!42!red!58}0.626 & \cellcolor{green!61!red!39}78.9 & \cellcolor{green!61!red!39}79.6 & \cellcolor{green!32!red!68}-7.3 & \cellcolor{green!32!red!68}0.551 \\
Laniqo & \checkmark & 9 & \checkmark & \cellcolor{green!43!red!57}8.7 & \cellcolor{green!79!red!21}0.649 & \cellcolor{green!0!red!100}67.9 & \cellcolor{green!0!red!100}67.0 & \cellcolor{green!69!red!31}-6.0 & \cellcolor{green!80!red!20}0.622 \\
RuZh & \unknown & 9 & \checkmark & \cellcolor{green!37!red!63}9.5 & \cellcolor{green!54!red!46}0.633 & \cellcolor{green!37!red!63}74.5 & \cellcolor{green!37!red!63}74.7 & \cellcolor{green!40!red!60}-7.0 & \cellcolor{green!37!red!63}0.558 \\
IRB-MT & \checkmark & 12 & \checkmark & \cellcolor{green!34!red!66}9.9 & \cellcolor{green!26!red!74}0.616 & \cellcolor{green!44!red!56}75.8 & \cellcolor{green!46!red!54}76.5 & \cellcolor{green!49!red!51}-6.7 & \cellcolor{green!26!red!74}0.541 \\
SRPOL & \checkmark & 12 & \checkmark & \cellcolor{green!29!red!71}10.5 & \cellcolor{green!76!red!24}0.647 & \cellcolor{green!22!red!78}71.8 & \cellcolor{green!22!red!78}71.6 & \cellcolor{green!21!red!79}-7.7 & \cellcolor{green!31!red!69}0.549 \\
\rowcolor{gray!30}
\official TowerPlus-72B[M] & \checkmark & 72 &  & \cellcolor{green!29!red!71}10.5 & \cellcolor{green!39!red!61}0.624 & \cellcolor{green!43!red!57}75.7 & \cellcolor{green!40!red!60}75.2 & \cellcolor{green!23!red!77}-7.6 & \cellcolor{green!27!red!73}0.543 \\
\rowcolor{gray!30}
\official CommandA & \checkmark & 111 &  & \cellcolor{green!26!red!74}11.0 & \cellcolor{green!30!red!70}0.618 & \cellcolor{green!50!red!50}76.9 & \cellcolor{green!45!red!55}76.3 & \cellcolor{green!9!red!91}-8.1 & \cellcolor{green!22!red!78}0.536 \\
\rowcolor{gray!30}
\official ONLINE-W & \unknown & \unknown &  & \cellcolor{green!22!red!78}11.5 & \cellcolor{green!39!red!61}0.624 & \cellcolor{green!32!red!68}73.7 & \cellcolor{green!32!red!68}73.6 & \cellcolor{green!15!red!85}-7.9 & \cellcolor{green!21!red!79}0.534 \\
\rowcolor{gray!30}
DLUT\_GTCOM & \checkmark & 27 &  & \cellcolor{green!21!red!79}11.6 & \cellcolor{green!42!red!58}0.626 & \cellcolor{green!18!red!82}71.1 & \cellcolor{green!21!red!79}71.2 & \cellcolor{green!32!red!68}-7.3 & \cellcolor{green!23!red!77}0.537 \\
\official TowerPlus-9B[M] & \checkmark & 9 &  & \cellcolor{green!20!red!80}11.8 & \cellcolor{green!28!red!72}0.617 & \cellcolor{green!30!red!70}73.2 & \cellcolor{green!28!red!72}72.7 & \cellcolor{green!26!red!74}-7.5 & \cellcolor{green!20!red!80}0.533 \\
\rowcolor{gray!30}
\official Llama-4-Maverick & \checkmark & 400 &  & \cellcolor{green!16!red!84}12.3 & \cellcolor{green!26!red!74}0.616 & \cellcolor{green!41!red!59}75.3 & \cellcolor{green!41!red!59}75.5 & \cellcolor{green!0!red!100}-8.5 & \cellcolor{green!7!red!93}0.513 \\
SalamandraTA & \checkmark & 8 &  & \cellcolor{green!16!red!84}12.3 & \cellcolor{green!52!red!48}0.632 & \cellcolor{green!10!red!90}69.6 & \cellcolor{green!6!red!94}68.2 & \cellcolor{green!23!red!77}-7.6 & \cellcolor{green!21!red!79}0.534 \\
\rowcolor{gray!30}
\official ONLINE-B & \checkmark & \unknown &  & \cellcolor{green!13!red!87}12.7 & \cellcolor{green!26!red!74}0.616 & \cellcolor{green!33!red!67}73.8 & \cellcolor{green!28!red!72}72.8 & \cellcolor{green!6!red!94}-8.2 & \cellcolor{green!10!red!90}0.517 \\
\rowcolor{gray!30}
TranssionTranslate & \unknown & \unknown &  & \cellcolor{green!13!red!87}12.7 & \cellcolor{green!30!red!70}0.618 & \cellcolor{green!16!red!84}70.7 & \cellcolor{green!20!red!80}71.0 & \cellcolor{green!26!red!74}-7.5 & \cellcolor{green!12!red!88}0.52 \\
\rowcolor{gray!30}
\official AyaExpanse-32B & \checkmark & 32 &  & \cellcolor{green!7!red!93}13.5 & \cellcolor{green!6!red!94}0.603 & \cellcolor{green!22!red!78}71.8 & \cellcolor{green!25!red!75}72.1 & \cellcolor{green!15!red!85}-7.9 & \cellcolor{green!10!red!90}0.517 \\
\rowcolor{gray!30}
\official ONLINE-G & \checkmark & \unknown &  & \cellcolor{green!2!red!98}14.2 & \cellcolor{green!22!red!78}0.613 & \cellcolor{green!0!red!100}67.8 & \cellcolor{green!0!red!100}66.6 & \cellcolor{green!23!red!77}-7.6 & \cellcolor{green!13!red!87}0.522 \\
\rowcolor{gray!30}
\official EuroLLM-22B-pre.[M] & \checkmark & 22 &  & \cellcolor{green!1!red!99}14.4 & \cellcolor{green!10!red!90}0.606 & \cellcolor{green!17!red!83}70.9 & \cellcolor{green!21!red!79}71.3 & \cellcolor{green!1!red!99}-8.4 & \cellcolor{green!0!red!100}0.502 \\
\official Gemma-3-12B & \checkmark & 12 &  & \cellcolor{green!0!red!100}14.7 & \cellcolor{green!0!red!100}0.589 & \cellcolor{green!26!red!74}72.6 & \cellcolor{green!30!red!70}73.2 & \cellcolor{green!1!red!99}-8.4 & \cellcolor{green!0!red!100}0.503 \\
\official AyaExpanse-8B & \checkmark & 8 &  & \cellcolor{green!0!red!100}17.4 & \cellcolor{green!0!red!100}0.589 & \cellcolor{green!0!red!100}67.1 & \cellcolor{green!0!red!100}66.2 & \cellcolor{green!0!red!100}-8.7 & \cellcolor{green!0!red!100}0.48 \\
\rowcolor{gray!30}
IR-MultiagentMT & \crossmark & \unknown &  & \cellcolor{green!0!red!100}19.4 & \cellcolor{green!0!red!100}0.564 & \cellcolor{green!0!red!100}65.7 & \cellcolor{green!0!red!100}65.7 & \cellcolor{green!0!red!100}-8.9 & \cellcolor{green!0!red!100}0.467 \\
\official EuroLLM-9B[M] & \checkmark & 9 &  & \cellcolor{green!0!red!100}21.5 & \cellcolor{green!0!red!100}0.547 & \cellcolor{green!0!red!100}63.6 & \cellcolor{green!0!red!100}63.1 & \cellcolor{green!0!red!100}-9.5 & \cellcolor{green!0!red!100}0.471 \\
\official Qwen2.5-7B & \checkmark & 7 &  & \cellcolor{green!0!red!100}24.8 & \cellcolor{green!0!red!100}0.546 & \cellcolor{green!0!red!100}60.2 & \cellcolor{green!0!red!100}57.5 & \cellcolor{green!0!red!100}-10.2 & \cellcolor{green!0!red!100}0.411 \\
\official Llama-3.1-8B & \crossmark & 8 &  & \cellcolor{green!0!red!100}29.7 & \cellcolor{green!0!red!100}0.521 & \cellcolor{green!0!red!100}58.6 & \cellcolor{green!0!red!100}55.1 & \cellcolor{green!0!red!100}-12.6 & \cellcolor{green!0!red!100}0.372 \\
\official NLLB & \checkmark & 1 &  & \cellcolor{green!0!red!100}31.5 & \cellcolor{green!0!red!100}0.483 & \cellcolor{green!0!red!100}54.3 & \cellcolor{green!0!red!100}53.3 & \cellcolor{green!0!red!100}-11.7 & \cellcolor{green!0!red!100}0.389 \\
\rowcolor{gray!30}
TranssionMT & \checkmark & 1 &  & \cellcolor{green!0!red!100}34.2 & \cellcolor{green!0!red!100}0.483 & \cellcolor{green!0!red!100}54.3 & \cellcolor{green!0!red!100}54.9 & \cellcolor{green!0!red!100}-13.7 & \cellcolor{green!0!red!100}0.332 \\
\official Mistral-7B & \crossmark & 7 &  & \cellcolor{green!0!red!100}39.0 & \cellcolor{green!0!red!100}0.45 & \cellcolor{green!0!red!100}52.4 & \cellcolor{green!0!red!100}46.3 & \cellcolor{green!0!red!100}-14.4 & \cellcolor{green!0!red!100}0.288 \\
\official CommandR7B & \checkmark & 7 &  & \cellcolor{green!0!red!100}40.0 & \cellcolor{green!0!red!100}0.41 & \cellcolor{green!0!red!100}52.1 & \cellcolor{green!0!red!100}39.9 & \cellcolor{green!0!red!100}-13.6 & \cellcolor{green!0!red!100}0.347 \\
\bottomrule
\end{tabularx}
\end{table*}


\begin{table*}
\small
\begin{tabularx}{\textwidth}{lYYYYYYYYY}
\toprule
\multicolumn{10}{c}{\textbf{English-Serbian (Cyrilics)}} \\
\midrule
System Name & LP Supported & Params. (B) & Humeval? & AutoRank $\downarrow$ & CometKiwi-XL $\uparrow$ & GEMBA-ESA-CMDA $\uparrow$ & GEMBA-ESA-GPT4.1 $\uparrow$ & MetricX-24-Hybrid-XL $\uparrow$ & XCOMET-XL $\uparrow$ \\
\midrule
Shy-hunyuan-MT & \checkmark & 7 & \checkmark & \cellcolor{green!100!red!0}1.0 & \cellcolor{green!100!red!0}0.687 & \cellcolor{green!100!red!0}76.6 & \cellcolor{green!92!red!8}83.3 & \cellcolor{green!100!red!0}-4.2 & \cellcolor{green!100!red!0}0.64 \\
\rowcolor{gray!30}
\official Gemini-2.5-Pro & \checkmark & \unknown & \checkmark & \cellcolor{green!90!red!10}3.0 & \cellcolor{green!89!red!11}0.663 & \cellcolor{green!92!red!8}74.6 & \cellcolor{green!100!red!0}87.2 & \cellcolor{green!83!red!17}-5.1 & \cellcolor{green!81!red!19}0.566 \\
\rowcolor{gray!30}
\official GPT-4.1 & \checkmark & \unknown & \checkmark & \cellcolor{green!88!red!12}3.4 & \cellcolor{green!86!red!14}0.655 & \cellcolor{green!91!red!9}74.4 & \cellcolor{green!92!red!8}83.4 & \cellcolor{green!81!red!19}-5.2 & \cellcolor{green!85!red!15}0.582 \\
\rowcolor{gray!30}
GemTrans & \checkmark & 27 & \checkmark & \cellcolor{green!82!red!18}4.6 & \cellcolor{green!89!red!11}0.663 & \cellcolor{green!79!red!21}71.6 & \cellcolor{green!73!red!27}74.5 & \cellcolor{green!87!red!13}-4.9 & \cellcolor{green!78!red!22}0.554 \\
\rowcolor{gray!30}
UvA-MT & \checkmark & 12 & \checkmark & \cellcolor{green!76!red!24}5.8 & \cellcolor{green!87!red!13}0.658 & \cellcolor{green!78!red!22}71.4 & \cellcolor{green!63!red!37}70.2 & \cellcolor{green!94!red!6}-4.5 & \cellcolor{green!53!red!47}0.46 \\
\rowcolor{gray!30}
\official ONLINE-B & \checkmark & \unknown & \checkmark & \cellcolor{green!75!red!25}6.1 & \cellcolor{green!81!red!19}0.644 & \cellcolor{green!76!red!24}71.0 & \cellcolor{green!74!red!26}75.2 & \cellcolor{green!71!red!29}-5.7 & \cellcolor{green!68!red!32}0.517 \\
\rowcolor{gray!30}
\official Claude-4 & \unknown & \unknown & \checkmark & \cellcolor{green!71!red!29}6.8 & \cellcolor{green!73!red!27}0.628 & \cellcolor{green!83!red!17}72.6 & \cellcolor{green!79!red!21}77.4 & \cellcolor{green!54!red!46}-6.6 & \cellcolor{green!64!red!36}0.503 \\
\rowcolor{gray!30}
CommandA-WMT & \crossmark & 111 & \checkmark & \cellcolor{green!70!red!30}7.0 & \cellcolor{green!79!red!21}0.641 & \cellcolor{green!80!red!20}71.8 & \cellcolor{green!57!red!43}67.3 & \cellcolor{green!65!red!35}-6.0 & \cellcolor{green!67!red!33}0.512 \\
\rowcolor{gray!30}
TranssionTranslate & \unknown & \unknown & \checkmark & \cellcolor{green!65!red!35}8.0 & \cellcolor{green!75!red!25}0.631 & \cellcolor{green!61!red!39}67.3 & \cellcolor{green!65!red!35}70.9 & \cellcolor{green!65!red!35}-6.0 & \cellcolor{green!59!red!41}0.484 \\
\rowcolor{gray!30}
\official DeepSeek-V3 & \unknown & 671 & \checkmark & \cellcolor{green!62!red!38}8.6 & \cellcolor{green!62!red!38}0.603 & \cellcolor{green!64!red!36}68.0 & \cellcolor{green!67!red!33}72.0 & \cellcolor{green!54!red!46}-6.6 & \cellcolor{green!64!red!36}0.501 \\
SalamandraTA & \checkmark & 8 & \checkmark & \cellcolor{green!61!red!39}8.8 & \cellcolor{green!77!red!23}0.635 & \cellcolor{green!56!red!44}66.2 & \cellcolor{green!53!red!47}65.1 & \cellcolor{green!62!red!38}-6.2 & \cellcolor{green!58!red!42}0.48 \\
\rowcolor{gray!30}
DLUT\_GTCOM & \checkmark & 27 & \checkmark & \cellcolor{green!59!red!41}9.3 & \cellcolor{green!69!red!31}0.618 & \cellcolor{green!59!red!41}66.9 & \cellcolor{green!59!red!41}68.1 & \cellcolor{green!54!red!46}-6.6 & \cellcolor{green!54!red!46}0.463 \\
IRB-MT & \checkmark & 12 & \checkmark & \cellcolor{green!56!red!44}9.9 & \cellcolor{green!63!red!37}0.604 & \cellcolor{green!63!red!37}67.9 & \cellcolor{green!51!red!49}64.2 & \cellcolor{green!56!red!44}-6.5 & \cellcolor{green!46!red!54}0.435 \\
\rowcolor{gray!30}
\official Llama-4-Maverick & \checkmark & 400 &  & \cellcolor{green!55!red!45}10.0 & \cellcolor{green!62!red!38}0.603 & \cellcolor{green!64!red!36}68.2 & \cellcolor{green!61!red!39}68.9 & \cellcolor{green!40!red!60}-7.3 & \cellcolor{green!49!red!51}0.444 \\
\rowcolor{gray!30}
\official Qwen3-235B & \checkmark & 235 &  & \cellcolor{green!46!red!54}11.9 & \cellcolor{green!57!red!43}0.591 & \cellcolor{green!54!red!46}65.8 & \cellcolor{green!42!red!58}60.1 & \cellcolor{green!35!red!65}-7.6 & \cellcolor{green!44!red!56}0.425 \\
\official Gemma-3-12B & \checkmark & 12 & \checkmark & \cellcolor{green!45!red!55}12.1 & \cellcolor{green!53!red!47}0.583 & \cellcolor{green!54!red!46}65.7 & \cellcolor{green!45!red!55}61.8 & \cellcolor{green!38!red!62}-7.4 & \cellcolor{green!36!red!64}0.394 \\
\rowcolor{gray!30}
\official Gemma-3-27B & \checkmark & 27 &  & \cellcolor{green!44!red!56}12.2 & \cellcolor{green!53!red!47}0.583 & \cellcolor{green!37!red!63}61.7 & \cellcolor{green!46!red!54}62.1 & \cellcolor{green!38!red!62}-7.4 & \cellcolor{green!49!red!51}0.444 \\
CUNI-SFT & \checkmark & 9 & \checkmark & \cellcolor{green!38!red!62}13.5 & \cellcolor{green!47!red!53}0.569 & \cellcolor{green!34!red!66}61.1 & \cellcolor{green!25!red!75}52.4 & \cellcolor{green!69!red!31}-5.8 & \cellcolor{green!19!red!81}0.328 \\
\rowcolor{gray!30}
IR-MultiagentMT & \crossmark & \unknown &  & \cellcolor{green!35!red!65}14.1 & \cellcolor{green!38!red!62}0.548 & \cellcolor{green!44!red!56}63.3 & \cellcolor{green!40!red!60}59.2 & \cellcolor{green!25!red!75}-8.1 & \cellcolor{green!34!red!66}0.386 \\
\rowcolor{gray!30}
\official ONLINE-G & \checkmark & \unknown &  & \cellcolor{green!33!red!67}14.5 & \cellcolor{green!46!red!54}0.566 & \cellcolor{green!24!red!76}58.8 & \cellcolor{green!33!red!67}56.2 & \cellcolor{green!33!red!67}-7.7 & \cellcolor{green!33!red!67}0.383 \\
\rowcolor{gray!30}
\official CommandA & \crossmark & 111 &  & \cellcolor{green!17!red!83}17.6 & \cellcolor{green!28!red!72}0.527 & \cellcolor{green!42!red!58}62.9 & \cellcolor{green!22!red!78}50.8 & \cellcolor{green!0!red!100}-10.0 & \cellcolor{green!17!red!83}0.323 \\
\official Llama-3.1-8B & \crossmark & 8 & \checkmark & \cellcolor{green!8!red!92}19.4 & \cellcolor{green!11!red!89}0.489 & \cellcolor{green!4!red!96}53.9 & \cellcolor{green!8!red!92}44.2 & \cellcolor{green!37!red!63}-7.5 & \cellcolor{green!0!red!100}0.233 \\
\official NLLB & \checkmark & 1 & \checkmark & \cellcolor{green!6!red!94}19.8 & \cellcolor{green!2!red!98}0.468 & \cellcolor{green!2!red!98}53.5 & \cellcolor{green!21!red!79}50.3 & \cellcolor{green!0!red!100}-9.4 & \cellcolor{green!19!red!81}0.33 \\
\rowcolor{gray!30}
\official EuroLLM-22B-pre.[M] & \crossmark & 22 &  & \cellcolor{green!2!red!98}20.6 & \cellcolor{green!2!red!98}0.469 & \cellcolor{green!2!red!98}53.6 & \cellcolor{green!2!red!98}41.4 & \cellcolor{green!15!red!85}-8.6 & \cellcolor{green!3!red!97}0.269 \\
\official EuroLLM-9B[M] & \crossmark & 9 & \checkmark & \cellcolor{green!0!red!100}22.4 & \cellcolor{green!0!red!100}0.454 & \cellcolor{green!0!red!100}51.5 & \cellcolor{green!0!red!100}37.4 & \cellcolor{green!0!red!100}-9.4 & \cellcolor{green!2!red!98}0.265 \\
\rowcolor{gray!30}
\official TowerPlus-72B[M] & \crossmark & 72 &  & \cellcolor{green!0!red!100}26.0 & \cellcolor{green!0!red!100}0.424 & \cellcolor{green!0!red!100}51.6 & \cellcolor{green!0!red!100}36.9 & \cellcolor{green!0!red!100}-12.4 & \cellcolor{green!0!red!100}0.203 \\
\official TowerPlus-9B[M] & \crossmark & 9 &  & \cellcolor{green!0!red!100}26.7 & \cellcolor{green!0!red!100}0.368 & \cellcolor{green!0!red!100}43.7 & \cellcolor{green!0!red!100}29.2 & \cellcolor{green!6!red!94}-9.1 & \cellcolor{green!0!red!100}0.182 \\
\official Mistral-7B & \crossmark & 7 &  & \cellcolor{green!0!red!100}27.0 & \cellcolor{green!0!red!100}0.414 & \cellcolor{green!0!red!100}49.2 & \cellcolor{green!0!red!100}38.3 & \cellcolor{green!0!red!100}-13.0 & \cellcolor{green!0!red!100}0.207 \\
\official AyaExpanse-8B & \crossmark & 8 &  & \cellcolor{green!0!red!100}29.7 & \cellcolor{green!0!red!100}0.306 & \cellcolor{green!0!red!100}40.9 & \cellcolor{green!0!red!100}27.3 & \cellcolor{green!0!red!100}-10.1 & \cellcolor{green!0!red!100}0.157 \\
\official CommandR7B & \crossmark & 7 &  & \cellcolor{green!0!red!100}31.3 & \cellcolor{green!0!red!100}0.307 & \cellcolor{green!0!red!100}38.0 & \cellcolor{green!0!red!100}26.0 & \cellcolor{green!0!red!100}-11.6 & \cellcolor{green!0!red!100}0.171 \\
\rowcolor{gray!30}
\official AyaExpanse-32B & \crossmark & 32 &  & \cellcolor{green!0!red!100}31.8 & \cellcolor{green!0!red!100}0.354 & \cellcolor{green!0!red!100}46.4 & \cellcolor{green!0!red!100}29.6 & \cellcolor{green!0!red!100}-15.2 & \cellcolor{green!0!red!100}0.142 \\
\official Qwen2.5-7B & \unknown & 7 &  & \cellcolor{green!0!red!100}32.0 & \cellcolor{green!0!red!100}0.306 & \cellcolor{green!0!red!100}37.0 & \cellcolor{green!0!red!100}27.2 & \cellcolor{green!0!red!100}-11.9 & \cellcolor{green!0!red!100}0.144 \\
\bottomrule
\end{tabularx}
\end{table*}


\begin{table*}
\small
\begin{tabularx}{\textwidth}{lYYYYYYYYY}
\toprule
\multicolumn{10}{c}{\textbf{English-Ukrainian}} \\
\midrule
System Name & LP Supported & Params. (B) & Humeval? & AutoRank $\downarrow$ & CometKiwi-XL $\uparrow$ & GEMBA-ESA-CMDA $\uparrow$ & GEMBA-ESA-GPT4.1 $\uparrow$ & MetricX-24-Hybrid-XL $\uparrow$ & XCOMET-XL $\uparrow$ \\
\midrule
Shy-hunyuan-MT & \checkmark & 7 & \checkmark & \cellcolor{green!100!red!0}1.0 & \cellcolor{green!100!red!0}0.65 & \cellcolor{green!97!red!3}84.1 & \cellcolor{green!81!red!19}85.3 & \cellcolor{green!100!red!0}-5.0 & \cellcolor{green!100!red!0}0.662 \\
\rowcolor{gray!30}
\official Gemini-2.5-Pro & \checkmark & \unknown & \checkmark & \cellcolor{green!82!red!18}3.3 & \cellcolor{green!67!red!33}0.625 & \cellcolor{green!100!red!0}84.6 & \cellcolor{green!100!red!0}89.8 & \cellcolor{green!63!red!37}-6.3 & \cellcolor{green!61!red!39}0.59 \\
Wenyiil & \checkmark & 14 & \checkmark & \cellcolor{green!81!red!19}3.4 & \cellcolor{green!80!red!20}0.635 & \cellcolor{green!95!red!5}83.7 & \cellcolor{green!82!red!18}85.4 & \cellcolor{green!66!red!34}-6.2 & \cellcolor{green!65!red!35}0.597 \\
\rowcolor{gray!30}
\official GPT-4.1 & \checkmark & \unknown & \checkmark & \cellcolor{green!81!red!19}3.4 & \cellcolor{green!68!red!32}0.626 & \cellcolor{green!90!red!10}82.8 & \cellcolor{green!88!red!12}87.0 & \cellcolor{green!66!red!34}-6.2 & \cellcolor{green!72!red!28}0.611 \\
\rowcolor{gray!30}
CommandA-WMT & \checkmark & 111 & \checkmark & \cellcolor{green!78!red!22}3.8 & \cellcolor{green!88!red!12}0.641 & \cellcolor{green!77!red!23}80.4 & \cellcolor{green!69!red!31}82.4 & \cellcolor{green!71!red!29}-6.0 & \cellcolor{green!66!red!34}0.599 \\
Algharb & \checkmark & 14 & \checkmark & \cellcolor{green!75!red!25}4.1 & \cellcolor{green!67!red!33}0.625 & \cellcolor{green!92!red!8}83.2 & \cellcolor{green!84!red!16}86.0 & \cellcolor{green!57!red!43}-6.5 & \cellcolor{green!59!red!41}0.586 \\
\rowcolor{gray!30}
UvA-MT & \checkmark & 12 & \checkmark & \cellcolor{green!74!red!26}4.3 & \cellcolor{green!88!red!12}0.641 & \cellcolor{green!68!red!32}78.7 & \cellcolor{green!65!red!35}81.5 & \cellcolor{green!63!red!37}-6.3 & \cellcolor{green!66!red!34}0.6 \\
\rowcolor{gray!30}
GemTrans & \checkmark & 27 & \checkmark & \cellcolor{green!72!red!28}4.5 & \cellcolor{green!71!red!29}0.628 & \cellcolor{green!64!red!36}78.0 & \cellcolor{green!59!red!41}80.1 & \cellcolor{green!80!red!20}-5.7 & \cellcolor{green!70!red!30}0.606 \\
\rowcolor{gray!30}
\official DeepSeek-V3 & \unknown & 671 & \checkmark & \cellcolor{green!69!red!31}4.9 & \cellcolor{green!59!red!41}0.619 & \cellcolor{green!84!red!16}81.7 & \cellcolor{green!76!red!24}84.0 & \cellcolor{green!57!red!43}-6.5 & \cellcolor{green!52!red!48}0.574 \\
Yolu & \checkmark & 14 & \checkmark & \cellcolor{green!61!red!39}5.9 & \cellcolor{green!91!red!9}0.643 & \cellcolor{green!39!red!61}73.4 & \cellcolor{green!35!red!65}74.4 & \cellcolor{green!66!red!34}-6.2 & \cellcolor{green!60!red!40}0.589 \\
\rowcolor{gray!30}
\official Mistral-Medium & \unknown & \unknown & \checkmark & \cellcolor{green!61!red!39}5.9 & \cellcolor{green!56!red!44}0.617 & \cellcolor{green!74!red!26}79.8 & \cellcolor{green!68!red!32}82.1 & \cellcolor{green!46!red!54}-6.9 & \cellcolor{green!48!red!52}0.566 \\
\rowcolor{gray!30}
\official Claude-4 & \unknown & \unknown & \checkmark & \cellcolor{green!53!red!47}6.9 & \cellcolor{green!39!red!61}0.604 & \cellcolor{green!81!red!19}81.1 & \cellcolor{green!70!red!30}82.6 & \cellcolor{green!26!red!74}-7.6 & \cellcolor{green!36!red!64}0.544 \\
\rowcolor{gray!30}
\official CommandA & \checkmark & 111 & \checkmark & \cellcolor{green!50!red!50}7.3 & \cellcolor{green!47!red!53}0.61 & \cellcolor{green!65!red!35}78.1 & \cellcolor{green!58!red!42}79.8 & \cellcolor{green!31!red!69}-7.4 & \cellcolor{green!37!red!63}0.546 \\
Laniqo & \checkmark & 9 & \checkmark & \cellcolor{green!48!red!52}7.5 & \cellcolor{green!84!red!16}0.638 & \cellcolor{green!6!red!94}67.3 & \cellcolor{green!2!red!98}66.3 & \cellcolor{green!63!red!37}-6.3 & \cellcolor{green!73!red!27}0.613 \\
IRB-MT & \checkmark & 12 & \checkmark & \cellcolor{green!44!red!56}8.0 & \cellcolor{green!39!red!61}0.604 & \cellcolor{green!47!red!53}74.8 & \cellcolor{green!46!red!54}76.9 & \cellcolor{green!46!red!54}-6.9 & \cellcolor{green!33!red!67}0.539 \\
SRPOL & \checkmark & 12 & \checkmark & \cellcolor{green!43!red!57}8.2 & \cellcolor{green!75!red!25}0.631 & \cellcolor{green!26!red!74}70.9 & \cellcolor{green!29!red!71}72.9 & \cellcolor{green!34!red!66}-7.3 & \cellcolor{green!38!red!62}0.548 \\
\official TowerPlus-9B[M] & \checkmark & 9 & \checkmark & \cellcolor{green!40!red!60}8.6 & \cellcolor{green!37!red!63}0.603 & \cellcolor{green!39!red!61}73.4 & \cellcolor{green!39!red!61}75.2 & \cellcolor{green!37!red!63}-7.2 & \cellcolor{green!34!red!66}0.541 \\
\rowcolor{gray!30}
\official Llama-4-Maverick & \checkmark & 400 & \checkmark & \cellcolor{green!40!red!60}8.6 & \cellcolor{green!37!red!63}0.603 & \cellcolor{green!55!red!45}76.3 & \cellcolor{green!51!red!49}78.1 & \cellcolor{green!20!red!80}-7.8 & \cellcolor{green!22!red!78}0.519 \\
\rowcolor{gray!30}
CGFOKUS & \checkmark & 235 &  & \cellcolor{green!39!red!61}8.7 & \cellcolor{green!29!red!71}0.597 & \cellcolor{green!52!red!48}75.7 & \cellcolor{green!51!red!49}78.1 & \cellcolor{green!31!red!69}-7.4 & \cellcolor{green!19!red!81}0.513 \\
\rowcolor{gray!30}
\official ONLINE-B & \checkmark & \unknown &  & \cellcolor{green!38!red!62}8.8 & \cellcolor{green!45!red!55}0.609 & \cellcolor{green!38!red!62}73.2 & \cellcolor{green!31!red!69}73.3 & \cellcolor{green!34!red!66}-7.3 & \cellcolor{green!29!red!71}0.531 \\
\rowcolor{gray!30}
\official AyaExpanse-32B & \checkmark & 32 &  & \cellcolor{green!36!red!64}9.1 & \cellcolor{green!33!red!67}0.6 & \cellcolor{green!42!red!58}73.9 & \cellcolor{green!38!red!62}75.0 & \cellcolor{green!29!red!71}-7.5 & \cellcolor{green!27!red!73}0.528 \\
\rowcolor{gray!30}
\official ONLINE-W & \unknown & \unknown &  & \cellcolor{green!36!red!64}9.1 & \cellcolor{green!40!red!60}0.605 & \cellcolor{green!36!red!64}72.8 & \cellcolor{green!38!red!62}75.0 & \cellcolor{green!29!red!71}-7.5 & \cellcolor{green!27!red!73}0.527 \\
\rowcolor{gray!30}
\official Qwen3-235B & \checkmark & 235 &  & \cellcolor{green!34!red!66}9.3 & \cellcolor{green!28!red!72}0.596 & \cellcolor{green!41!red!59}73.8 & \cellcolor{green!41!red!59}75.7 & \cellcolor{green!29!red!71}-7.5 & \cellcolor{green!20!red!80}0.515 \\
SalamandraTA & \checkmark & 8 &  & \cellcolor{green!29!red!71}9.9 & \cellcolor{green!51!red!49}0.613 & \cellcolor{green!11!red!89}68.3 & \cellcolor{green!11!red!89}68.6 & \cellcolor{green!37!red!63}-7.2 & \cellcolor{green!27!red!73}0.528 \\
\rowcolor{gray!30}
\official TowerPlus-72B[M] & \checkmark & 72 &  & \cellcolor{green!27!red!73}10.2 & \cellcolor{green!23!red!77}0.592 & \cellcolor{green!34!red!66}72.4 & \cellcolor{green!33!red!67}73.8 & \cellcolor{green!17!red!83}-7.9 & \cellcolor{green!20!red!80}0.514 \\
\rowcolor{gray!30}
TranssionTranslate & \unknown & \unknown &  & \cellcolor{green!23!red!77}10.7 & \cellcolor{green!25!red!75}0.594 & \cellcolor{green!16!red!84}69.1 & \cellcolor{green!23!red!77}71.3 & \cellcolor{green!29!red!71}-7.5 & \cellcolor{green!15!red!85}0.505 \\
\rowcolor{gray!30}
DLUT\_GTCOM & \checkmark & 27 &  & \cellcolor{green!21!red!79}11.0 & \cellcolor{green!23!red!77}0.592 & \cellcolor{green!20!red!80}69.8 & \cellcolor{green!23!red!77}71.4 & \cellcolor{green!17!red!83}-7.9 & \cellcolor{green!11!red!89}0.498 \\
\rowcolor{gray!30}
\official Gemma-3-27B & \checkmark & 27 &  & \cellcolor{green!13!red!87}11.9 & \cellcolor{green!0!red!100}0.575 & \cellcolor{green!10!red!90}68.1 & \cellcolor{green!21!red!79}71.0 & \cellcolor{green!11!red!89}-8.1 & \cellcolor{green!18!red!82}0.51 \\
\rowcolor{gray!30}
\official EuroLLM-22B-pre.[M] & \checkmark & 22 &  & \cellcolor{green!9!red!91}12.5 & \cellcolor{green!3!red!97}0.577 & \cellcolor{green!15!red!85}68.9 & \cellcolor{green!15!red!85}69.4 & \cellcolor{green!0!red!100}-8.6 & \cellcolor{green!8!red!92}0.492 \\
\official AyaExpanse-8B & \checkmark & 8 &  & \cellcolor{green!4!red!96}13.1 & \cellcolor{green!1!red!99}0.576 & \cellcolor{green!2!red!98}66.5 & \cellcolor{green!8!red!92}67.8 & \cellcolor{green!3!red!97}-8.4 & \cellcolor{green!0!red!100}0.477 \\
CUNI-SFT & \checkmark & 9 &  & \cellcolor{green!2!red!98}13.3 & \cellcolor{green!5!red!95}0.579 & \cellcolor{green!0!red!100}66.1 & \cellcolor{green!0!red!100}65.5 & \cellcolor{green!0!red!100}-8.5 & \cellcolor{green!4!red!96}0.484 \\
\rowcolor{gray!30}
\official ONLINE-G & \checkmark & \unknown &  & \cellcolor{green!0!red!100}13.7 & \cellcolor{green!0!red!100}0.575 & \cellcolor{green!0!red!100}64.2 & \cellcolor{green!0!red!100}65.0 & \cellcolor{green!3!red!97}-8.4 & \cellcolor{green!1!red!99}0.479 \\
\rowcolor{gray!30}
IR-MultiagentMT & \crossmark & \unknown &  & \cellcolor{green!0!red!100}14.0 & \cellcolor{green!0!red!100}0.555 & \cellcolor{green!6!red!94}67.3 & \cellcolor{green!6!red!94}67.3 & \cellcolor{green!0!red!100}-8.5 & \cellcolor{green!0!red!100}0.467 \\
\official Gemma-3-12B & \checkmark & 12 &  & \cellcolor{green!0!red!100}14.4 & \cellcolor{green!0!red!100}0.559 & \cellcolor{green!0!red!100}64.9 & \cellcolor{green!0!red!100}65.8 & \cellcolor{green!0!red!100}-8.6 & \cellcolor{green!0!red!100}0.473 \\
\official EuroLLM-9B[M] & \checkmark & 9 &  & \cellcolor{green!0!red!100}17.0 & \cellcolor{green!0!red!100}0.518 & \cellcolor{green!0!red!100}63.1 & \cellcolor{green!0!red!100}61.8 & \cellcolor{green!0!red!100}-9.0 & \cellcolor{green!0!red!100}0.459 \\
\official NLLB & \checkmark & 1 &  & \cellcolor{green!0!red!100}24.0 & \cellcolor{green!0!red!100}0.467 & \cellcolor{green!0!red!100}53.2 & \cellcolor{green!0!red!100}53.6 & \cellcolor{green!0!red!100}-11.2 & \cellcolor{green!0!red!100}0.368 \\
\official Llama-3.1-8B & \crossmark & 8 &  & \cellcolor{green!0!red!100}24.3 & \cellcolor{green!0!red!100}0.488 & \cellcolor{green!0!red!100}55.5 & \cellcolor{green!0!red!100}51.0 & \cellcolor{green!0!red!100}-11.9 & \cellcolor{green!0!red!100}0.331 \\
\rowcolor{gray!30}
TranssionMT & \checkmark & 1 &  & \cellcolor{green!0!red!100}28.1 & \cellcolor{green!0!red!100}0.441 & \cellcolor{green!0!red!100}51.8 & \cellcolor{green!0!red!100}52.2 & \cellcolor{green!0!red!100}-13.5 & \cellcolor{green!0!red!100}0.286 \\
\official CommandR7B & \checkmark & 7 &  & \cellcolor{green!0!red!100}29.0 & \cellcolor{green!0!red!100}0.411 & \cellcolor{green!0!red!100}54.6 & \cellcolor{green!0!red!100}43.6 & \cellcolor{green!0!red!100}-13.2 & \cellcolor{green!0!red!100}0.323 \\
\official Mistral-7B & \crossmark & 7 &  & \cellcolor{green!0!red!100}29.3 & \cellcolor{green!0!red!100}0.428 & \cellcolor{green!0!red!100}52.2 & \cellcolor{green!0!red!100}46.2 & \cellcolor{green!0!red!100}-13.4 & \cellcolor{green!0!red!100}0.277 \\
\official Qwen2.5-7B & \unknown & 7 &  & \cellcolor{green!0!red!100}36.6 & \cellcolor{green!0!red!100}0.362 & \cellcolor{green!0!red!100}41.8 & \cellcolor{green!0!red!100}36.0 & \cellcolor{green!0!red!100}-15.2 & \cellcolor{green!0!red!100}0.2 \\
\rowcolor{gray!30}
KYUoM & \unknown & <1 &  & \cellcolor{green!0!red!100}42.0 & \cellcolor{green!0!red!100}0.265 & \cellcolor{green!0!red!100}35.9 & \cellcolor{green!0!red!100}34.7 & \cellcolor{green!0!red!100}-16.6 & \cellcolor{green!0!red!100}0.201 \\
\bottomrule
\end{tabularx}
\end{table*}


\begin{table*}
\small
\begin{tabularx}{\textwidth}{lYYYYYYYYY}
\toprule
\multicolumn{10}{c}{\textbf{English-Simplified Chinese}} \\
\midrule
System Name & LP Supported & Params. (B) & Humeval? & AutoRank $\downarrow$ & CometKiwi-XL $\uparrow$ & GEMBA-ESA-CMDA $\uparrow$ & GEMBA-ESA-GPT4.1 $\uparrow$ & MetricX-24-Hybrid-XL $\uparrow$ & XCOMET-XL $\uparrow$ \\
\midrule
Shy-hunyuan-MT & \checkmark & 7 & \checkmark & \cellcolor{green!100!red!0}1.0 & \cellcolor{green!75!red!25}0.67 & \cellcolor{green!100!red!0}87.2 & \cellcolor{green!98!red!2}88.3 & \cellcolor{green!100!red!0}-4.0 & \cellcolor{green!100!red!0}0.576 \\
Wenyiil & \checkmark & 14 & \checkmark & \cellcolor{green!76!red!24}3.9 & \cellcolor{green!64!red!36}0.663 & \cellcolor{green!86!red!14}84.2 & \cellcolor{green!95!red!5}87.7 & \cellcolor{green!60!red!40}-5.0 & \cellcolor{green!61!red!39}0.52 \\
\rowcolor{gray!30}
\official Gemini-2.5-Pro & \checkmark & \unknown & \checkmark & \cellcolor{green!76!red!24}4.0 & \cellcolor{green!55!red!45}0.657 & \cellcolor{green!91!red!9}85.2 & \cellcolor{green!100!red!0}88.7 & \cellcolor{green!64!red!36}-4.9 & \cellcolor{green!56!red!44}0.512 \\
Algharb & \checkmark & 14 & \checkmark & \cellcolor{green!75!red!25}4.1 & \cellcolor{green!60!red!40}0.66 & \cellcolor{green!88!red!12}84.7 & \cellcolor{green!95!red!5}87.8 & \cellcolor{green!60!red!40}-5.0 & \cellcolor{green!58!red!42}0.515 \\
\rowcolor{gray!30}
\official GPT-4.1 & \checkmark & \unknown & \checkmark & \cellcolor{green!71!red!29}4.6 & \cellcolor{green!48!red!52}0.652 & \cellcolor{green!89!red!11}84.9 & \cellcolor{green!90!red!10}86.8 & \cellcolor{green!60!red!40}-5.0 & \cellcolor{green!56!red!44}0.512 \\
\rowcolor{gray!30}
\official Qwen3-235B & \checkmark & 235 & \checkmark & \cellcolor{green!69!red!31}4.8 & \cellcolor{green!61!red!39}0.661 & \cellcolor{green!79!red!21}82.7 & \cellcolor{green!81!red!19}85.0 & \cellcolor{green!60!red!40}-5.0 & \cellcolor{green!56!red!44}0.513 \\
Yolu & \checkmark & 14 & \checkmark & \cellcolor{green!69!red!31}4.8 & \cellcolor{green!100!red!0}0.687 & \cellcolor{green!43!red!57}74.9 & \cellcolor{green!42!red!58}77.1 & \cellcolor{green!76!red!24}-4.6 & \cellcolor{green!76!red!24}0.542 \\
\rowcolor{gray!30}
GemTrans & \checkmark & 27 & \checkmark & \cellcolor{green!68!red!32}4.9 & \cellcolor{green!57!red!43}0.658 & \cellcolor{green!52!red!48}77.0 & \cellcolor{green!57!red!43}80.2 & \cellcolor{green!88!red!12}-4.3 & \cellcolor{green!79!red!21}0.546 \\
\rowcolor{gray!30}
\official Mistral-Medium & \unknown & \unknown & \checkmark & \cellcolor{green!68!red!32}4.9 & \cellcolor{green!57!red!43}0.658 & \cellcolor{green!78!red!22}82.4 & \cellcolor{green!81!red!19}84.9 & \cellcolor{green!60!red!40}-5.0 & \cellcolor{green!57!red!43}0.514 \\
\rowcolor{gray!30}
CommandA-WMT & \checkmark & 111 & \checkmark & \cellcolor{green!62!red!38}5.6 & \cellcolor{green!67!red!33}0.665 & \cellcolor{green!61!red!39}78.9 & \cellcolor{green!64!red!36}81.5 & \cellcolor{green!60!red!40}-5.0 & \cellcolor{green!53!red!47}0.508 \\
\rowcolor{gray!30}
UvA-MT & \checkmark & 12 & \checkmark & \cellcolor{green!57!red!43}6.3 & \cellcolor{green!76!red!24}0.671 & \cellcolor{green!51!red!49}76.8 & \cellcolor{green!61!red!39}81.0 & \cellcolor{green!43!red!57}-5.4 & \cellcolor{green!47!red!53}0.499 \\
\rowcolor{gray!30}
\official Claude-4 & \checkmark & \unknown & \checkmark & \cellcolor{green!51!red!49}7.0 & \cellcolor{green!43!red!57}0.649 & \cellcolor{green!68!red!32}80.4 & \cellcolor{green!70!red!30}82.8 & \cellcolor{green!35!red!65}-5.6 & \cellcolor{green!38!red!62}0.487 \\
\rowcolor{gray!30}
\official DeepSeek-V3 & \checkmark & 671 & \checkmark & \cellcolor{green!50!red!50}7.1 & \cellcolor{green!0!red!100}0.618 & \cellcolor{green!89!red!11}84.9 & \cellcolor{green!82!red!18}85.1 & \cellcolor{green!52!red!48}-5.2 & \cellcolor{green!29!red!71}0.473 \\
\rowcolor{gray!30}
\official Llama-4-Maverick & \checkmark & 400 & \checkmark & \cellcolor{green!43!red!57}8.0 & \cellcolor{green!45!red!55}0.65 & \cellcolor{green!43!red!57}74.9 & \cellcolor{green!53!red!47}79.4 & \cellcolor{green!39!red!61}-5.5 & \cellcolor{green!40!red!60}0.489 \\
\rowcolor{gray!30}
\official ONLINE-B & \checkmark & \unknown &  & \cellcolor{green!41!red!59}8.2 & \cellcolor{green!54!red!46}0.656 & \cellcolor{green!34!red!66}73.0 & \cellcolor{green!30!red!70}74.7 & \cellcolor{green!52!red!48}-5.2 & \cellcolor{green!42!red!58}0.492 \\
\rowcolor{gray!30}
\official Gemma-3-27B & \checkmark & 27 &  & \cellcolor{green!35!red!65}9.0 & \cellcolor{green!27!red!73}0.638 & \cellcolor{green!45!red!55}75.5 & \cellcolor{green!50!red!50}78.7 & \cellcolor{green!27!red!73}-5.8 & \cellcolor{green!30!red!70}0.475 \\
Laniqo & \checkmark & 9 & \checkmark & \cellcolor{green!34!red!66}9.1 & \cellcolor{green!67!red!33}0.665 & \cellcolor{green!0!red!100}65.6 & \cellcolor{green!0!red!100}67.4 & \cellcolor{green!64!red!36}-4.9 & \cellcolor{green!56!red!44}0.513 \\
IRB-MT & \checkmark & 12 & \checkmark & \cellcolor{green!32!red!68}9.3 & \cellcolor{green!20!red!80}0.633 & \cellcolor{green!37!red!63}73.7 & \cellcolor{green!44!red!56}77.5 & \cellcolor{green!47!red!53}-5.3 & \cellcolor{green!24!red!76}0.467 \\
\rowcolor{gray!30}
\official CommandA & \checkmark & 111 &  & \cellcolor{green!31!red!69}9.4 & \cellcolor{green!38!red!62}0.645 & \cellcolor{green!50!red!50}76.5 & \cellcolor{green!40!red!60}76.8 & \cellcolor{green!15!red!85}-6.1 & \cellcolor{green!22!red!78}0.464 \\
\rowcolor{gray!30}
\official TowerPlus-72B[M] & \checkmark & 72 &  & \cellcolor{green!28!red!72}9.8 & \cellcolor{green!38!red!62}0.645 & \cellcolor{green!35!red!65}73.3 & \cellcolor{green!39!red!61}76.6 & \cellcolor{green!15!red!85}-6.1 & \cellcolor{green!24!red!76}0.466 \\
SRPOL & \crossmark & 12 & \checkmark & \cellcolor{green!24!red!76}10.3 & \cellcolor{green!69!red!31}0.666 & \cellcolor{green!11!red!89}68.2 & \cellcolor{green!12!red!88}71.1 & \cellcolor{green!19!red!81}-6.0 & \cellcolor{green!20!red!80}0.461 \\
RuZh & \unknown & 9 & \checkmark & \cellcolor{green!23!red!77}10.4 & \cellcolor{green!42!red!58}0.648 & \cellcolor{green!25!red!75}71.2 & \cellcolor{green!27!red!73}74.2 & \cellcolor{green!23!red!77}-5.9 & \cellcolor{green!15!red!85}0.454 \\
\official Gemma-3-12B & \checkmark & 12 &  & \cellcolor{green!22!red!78}10.6 & \cellcolor{green!24!red!76}0.636 & \cellcolor{green!36!red!64}73.4 & \cellcolor{green!39!red!61}76.6 & \cellcolor{green!15!red!85}-6.1 & \cellcolor{green!10!red!90}0.446 \\
\official Qwen2.5-7B & \checkmark & 7 &  & \cellcolor{green!14!red!86}11.5 & \cellcolor{green!8!red!92}0.625 & \cellcolor{green!22!red!78}70.6 & \cellcolor{green!24!red!76}73.6 & \cellcolor{green!23!red!77}-5.9 & \cellcolor{green!13!red!87}0.451 \\
\rowcolor{gray!30}
\official AyaExpanse-32B & \checkmark & 32 &  & \cellcolor{green!13!red!87}11.6 & \cellcolor{green!17!red!83}0.631 & \cellcolor{green!24!red!76}70.9 & \cellcolor{green!29!red!71}74.6 & \cellcolor{green!7!red!93}-6.3 & \cellcolor{green!8!red!92}0.444 \\
\official TowerPlus-9B[M] & \checkmark & 9 &  & \cellcolor{green!11!red!89}11.9 & \cellcolor{green!21!red!79}0.634 & \cellcolor{green!19!red!81}69.9 & \cellcolor{green!15!red!85}71.7 & \cellcolor{green!11!red!89}-6.2 & \cellcolor{green!10!red!90}0.446 \\
\rowcolor{gray!30}
TranssionTranslate & \unknown & \unknown &  & \cellcolor{green!4!red!96}12.7 & \cellcolor{green!27!red!73}0.638 & \cellcolor{green!5!red!95}66.9 & \cellcolor{green!7!red!93}70.1 & \cellcolor{green!3!red!97}-6.4 & \cellcolor{green!4!red!96}0.438 \\
\rowcolor{gray!30}
\official EuroLLM-22B-pre.[M] & \checkmark & 22 &  & \cellcolor{green!4!red!96}12.8 & \cellcolor{green!11!red!89}0.627 & \cellcolor{green!14!red!86}68.9 & \cellcolor{green!14!red!86}71.5 & \cellcolor{green!3!red!97}-6.4 & \cellcolor{green!0!red!100}0.43 \\
\rowcolor{gray!30}
\official ONLINE-W & \unknown & \unknown &  & \cellcolor{green!0!red!100}13.4 & \cellcolor{green!11!red!89}0.627 & \cellcolor{green!3!red!97}66.4 & \cellcolor{green!2!red!98}69.2 & \cellcolor{green!0!red!100}-6.5 & \cellcolor{green!4!red!96}0.437 \\
\official AyaExpanse-8B & \checkmark & 8 &  & \cellcolor{green!0!red!100}15.0 & \cellcolor{green!0!red!100}0.615 & \cellcolor{green!0!red!100}65.1 & \cellcolor{green!0!red!100}68.6 & \cellcolor{green!0!red!100}-6.7 & \cellcolor{green!0!red!100}0.403 \\
\official EuroLLM-9B[M] & \checkmark & 9 &  & \cellcolor{green!0!red!100}16.4 & \cellcolor{green!0!red!100}0.604 & \cellcolor{green!0!red!100}63.6 & \cellcolor{green!0!red!100}66.6 & \cellcolor{green!0!red!100}-6.9 & \cellcolor{green!0!red!100}0.394 \\
\rowcolor{gray!30}
IR-MultiagentMT & \crossmark & \unknown &  & \cellcolor{green!0!red!100}17.2 & \cellcolor{green!0!red!100}0.575 & \cellcolor{green!0!red!100}64.0 & \cellcolor{green!0!red!100}66.0 & \cellcolor{green!0!red!100}-6.6 & \cellcolor{green!0!red!100}0.399 \\
SalamandraTA & \checkmark & 8 &  & \cellcolor{green!0!red!100}17.9 & \cellcolor{green!0!red!100}0.618 & \cellcolor{green!0!red!100}59.5 & \cellcolor{green!0!red!100}59.2 & \cellcolor{green!0!red!100}-7.1 & \cellcolor{green!0!red!100}0.376 \\
\official Llama-3.1-8B & \crossmark & 8 &  & \cellcolor{green!0!red!100}18.4 & \cellcolor{green!0!red!100}0.594 & \cellcolor{green!0!red!100}61.6 & \cellcolor{green!0!red!100}62.8 & \cellcolor{green!0!red!100}-7.4 & \cellcolor{green!0!red!100}0.379 \\
\official CommandR7B & \checkmark & 7 &  & \cellcolor{green!0!red!100}18.5 & \cellcolor{green!0!red!100}0.595 & \cellcolor{green!0!red!100}63.1 & \cellcolor{green!0!red!100}65.0 & \cellcolor{green!0!red!100}-7.9 & \cellcolor{green!0!red!100}0.376 \\
\rowcolor{gray!30}
\official ONLINE-G & \checkmark & \unknown &  & \cellcolor{green!0!red!100}31.2 & \cellcolor{green!0!red!100}0.508 & \cellcolor{green!0!red!100}52.2 & \cellcolor{green!0!red!100}51.7 & \cellcolor{green!0!red!100}-11.1 & \cellcolor{green!0!red!100}0.256 \\
\official Mistral-7B & \crossmark & 7 &  & \cellcolor{green!0!red!100}32.0 & \cellcolor{green!0!red!100}0.5 & \cellcolor{green!0!red!100}47.6 & \cellcolor{green!0!red!100}46.7 & \cellcolor{green!0!red!100}-10.4 & \cellcolor{green!0!red!100}0.257 \\
\official NLLB & \checkmark & 1 &  & \cellcolor{green!0!red!100}38.0 & \cellcolor{green!0!red!100}0.441 & \cellcolor{green!0!red!100}44.4 & \cellcolor{green!0!red!100}45.6 & \cellcolor{green!0!red!100}-12.8 & \cellcolor{green!0!red!100}0.238 \\
\bottomrule
\end{tabularx}
\end{table*}


\begin{table*}
\small
\begin{tabularx}{\textwidth}{lYYYYYYYYY}
\toprule
\multicolumn{10}{c}{\textbf{Czech-Ukrainian}} \\
\midrule
System Name & LP Supported & Params. (B) & Humeval? & AutoRank $\downarrow$ & CometKiwi-XL $\uparrow$ & GEMBA-ESA-CMDA $\uparrow$ & GEMBA-ESA-GPT4.1 $\uparrow$ & MetricX-24-Hybrid-XL $\uparrow$ & XCOMET-XL $\uparrow$ \\
\midrule
Shy-hunyuan-MT & \checkmark & 7 & \checkmark & \cellcolor{green!100!red!0}1.0 & \cellcolor{green!93!red!7}0.601 & \cellcolor{green!85!red!15}79.1 & \cellcolor{green!81!red!19}85.3 & \cellcolor{green!92!red!8}-5.0 & \cellcolor{green!100!red!0}0.681 \\
\rowcolor{gray!30}
\official Gemini-2.5-Pro & \checkmark & \unknown & \checkmark & \cellcolor{green!100!red!0}1.0 & \cellcolor{green!69!red!31}0.582 & \cellcolor{green!100!red!0}81.4 & \cellcolor{green!100!red!0}89.5 & \cellcolor{green!87!red!13}-5.1 & \cellcolor{green!93!red!7}0.671 \\
\rowcolor{gray!30}
CommandA-WMT & \checkmark & 111 & \checkmark & \cellcolor{green!98!red!2}1.3 & \cellcolor{green!83!red!17}0.593 & \cellcolor{green!93!red!7}80.3 & \cellcolor{green!77!red!23}84.3 & \cellcolor{green!100!red!0}-4.8 & \cellcolor{green!87!red!13}0.664 \\
\rowcolor{gray!30}
\official GPT-4.1 & \checkmark & \unknown & \checkmark & \cellcolor{green!98!red!2}1.3 & \cellcolor{green!82!red!18}0.592 & \cellcolor{green!93!red!7}80.4 & \cellcolor{green!98!red!2}89.0 & \cellcolor{green!79!red!21}-5.3 & \cellcolor{green!89!red!11}0.666 \\
\rowcolor{gray!30}
\official DeepSeek-V3 & \unknown & 671 & \checkmark & \cellcolor{green!84!red!16}3.2 & \cellcolor{green!64!red!36}0.578 & \cellcolor{green!86!red!14}79.3 & \cellcolor{green!77!red!23}84.3 & \cellcolor{green!71!red!29}-5.5 & \cellcolor{green!80!red!20}0.654 \\
\rowcolor{gray!30}
\official Claude-4 & \unknown & \unknown & \checkmark & \cellcolor{green!81!red!19}3.6 & \cellcolor{green!75!red!25}0.587 & \cellcolor{green!83!red!17}78.8 & \cellcolor{green!83!red!17}85.6 & \cellcolor{green!49!red!51}-6.0 & \cellcolor{green!73!red!27}0.645 \\
\rowcolor{gray!30}
\official Mistral-Medium & \unknown & \unknown & \checkmark & \cellcolor{green!77!red!23}4.1 & \cellcolor{green!66!red!34}0.58 & \cellcolor{green!76!red!24}77.9 & \cellcolor{green!74!red!26}83.5 & \cellcolor{green!58!red!42}-5.8 & \cellcolor{green!71!red!29}0.642 \\
\rowcolor{gray!30}
GemTrans & \checkmark & 27 & \checkmark & \cellcolor{green!76!red!24}4.3 & \cellcolor{green!66!red!34}0.58 & \cellcolor{green!61!red!39}75.6 & \cellcolor{green!55!red!45}79.2 & \cellcolor{green!83!red!17}-5.2 & \cellcolor{green!73!red!27}0.645 \\
\rowcolor{gray!30}
\official CommandA & \checkmark & 111 & \checkmark & \cellcolor{green!75!red!25}4.5 & \cellcolor{green!69!red!31}0.582 & \cellcolor{green!83!red!17}78.9 & \cellcolor{green!66!red!34}81.7 & \cellcolor{green!49!red!51}-6.0 & \cellcolor{green!67!red!33}0.637 \\
\rowcolor{gray!30}
\official Gemma-3-27B & \checkmark & 27 & \checkmark & \cellcolor{green!72!red!28}4.9 & \cellcolor{green!67!red!33}0.581 & \cellcolor{green!72!red!28}77.3 & \cellcolor{green!66!red!34}81.7 & \cellcolor{green!49!red!51}-6.0 & \cellcolor{green!62!red!38}0.63 \\
\rowcolor{gray!30}
UvA-MT & \checkmark & 12 & \checkmark & \cellcolor{green!71!red!29}5.0 & \cellcolor{green!88!red!12}0.597 & \cellcolor{green!55!red!45}74.7 & \cellcolor{green!54!red!46}79.1 & \cellcolor{green!49!red!51}-6.0 & \cellcolor{green!69!red!31}0.64 \\
Wenyiil & \checkmark & 14 & \checkmark & \cellcolor{green!69!red!31}5.3 & \cellcolor{green!73!red!27}0.585 & \cellcolor{green!61!red!39}75.6 & \cellcolor{green!54!red!46}79.1 & \cellcolor{green!54!red!46}-5.9 & \cellcolor{green!66!red!34}0.635 \\
Yolu & \checkmark & 14 & \checkmark & \cellcolor{green!64!red!36}5.9 & \cellcolor{green!100!red!0}0.606 & \cellcolor{green!37!red!63}72.1 & \cellcolor{green!31!red!69}73.8 & \cellcolor{green!54!red!46}-5.9 & \cellcolor{green!65!red!35}0.634 \\
Algharb & \checkmark & 14 & \checkmark & \cellcolor{green!56!red!44}7.1 & \cellcolor{green!56!red!44}0.572 & \cellcolor{green!50!red!50}74.0 & \cellcolor{green!56!red!44}79.5 & \cellcolor{green!33!red!67}-6.4 & \cellcolor{green!54!red!46}0.619 \\
\rowcolor{gray!30}
\official Llama-4-Maverick & \checkmark & 400 &  & \cellcolor{green!54!red!46}7.3 & \cellcolor{green!58!red!42}0.574 & \cellcolor{green!60!red!40}75.5 & \cellcolor{green!59!red!41}80.3 & \cellcolor{green!20!red!80}-6.7 & \cellcolor{green!40!red!60}0.601 \\
\rowcolor{gray!30}
\official AyaExpanse-32B & \checkmark & 32 &  & \cellcolor{green!53!red!47}7.4 & \cellcolor{green!53!red!47}0.57 & \cellcolor{green!46!red!54}73.4 & \cellcolor{green!41!red!59}76.1 & \cellcolor{green!45!red!55}-6.1 & \cellcolor{green!53!red!47}0.618 \\
Laniqo & \checkmark & 9 & \checkmark & \cellcolor{green!53!red!47}7.5 & \cellcolor{green!87!red!13}0.596 & \cellcolor{green!11!red!89}68.1 & \cellcolor{green!8!red!92}68.6 & \cellcolor{green!54!red!46}-5.9 & \cellcolor{green!73!red!27}0.645 \\
SRPOL & \checkmark & 12 & \checkmark & \cellcolor{green!52!red!48}7.6 & \cellcolor{green!92!red!8}0.6 & \cellcolor{green!33!red!67}71.4 & \cellcolor{green!29!red!71}73.3 & \cellcolor{green!24!red!76}-6.6 & \cellcolor{green!53!red!47}0.618 \\
\official TowerPlus-9B[M] & \checkmark & 9 & \checkmark & \cellcolor{green!51!red!49}7.7 & \cellcolor{green!53!red!47}0.57 & \cellcolor{green!50!red!50}74.0 & \cellcolor{green!44!red!56}76.7 & \cellcolor{green!33!red!67}-6.4 & \cellcolor{green!46!red!54}0.608 \\
\rowcolor{gray!30}
\official TowerPlus-72B[M] & \checkmark & 72 &  & \cellcolor{green!44!red!56}8.7 & \cellcolor{green!49!red!51}0.567 & \cellcolor{green!42!red!58}72.7 & \cellcolor{green!39!red!61}75.7 & \cellcolor{green!20!red!80}-6.7 & \cellcolor{green!41!red!59}0.602 \\
\rowcolor{gray!30}
\official EuroLLM-22B-pre.[M] & \checkmark & 22 &  & \cellcolor{green!44!red!56}8.7 & \cellcolor{green!48!red!52}0.566 & \cellcolor{green!42!red!58}72.8 & \cellcolor{green!35!red!65}74.7 & \cellcolor{green!20!red!80}-6.7 & \cellcolor{green!44!red!56}0.606 \\
IRB-MT & \checkmark & 12 & \checkmark & \cellcolor{green!43!red!57}8.9 & \cellcolor{green!39!red!61}0.559 & \cellcolor{green!39!red!61}72.4 & \cellcolor{green!35!red!65}74.8 & \cellcolor{green!33!red!67}-6.4 & \cellcolor{green!38!red!62}0.598 \\
\official Gemma-3-12B & \checkmark & 12 &  & \cellcolor{green!37!red!63}9.7 & \cellcolor{green!39!red!61}0.559 & \cellcolor{green!44!red!56}73.0 & \cellcolor{green!40!red!60}75.9 & \cellcolor{green!12!red!88}-6.9 & \cellcolor{green!27!red!73}0.583 \\
\rowcolor{gray!30}
\official Qwen3-235B & \checkmark & 235 &  & \cellcolor{green!32!red!68}10.4 & \cellcolor{green!36!red!64}0.557 & \cellcolor{green!33!red!67}71.5 & \cellcolor{green!30!red!70}73.6 & \cellcolor{green!12!red!88}-6.9 & \cellcolor{green!26!red!74}0.582 \\
\rowcolor{gray!30}
IR-MultiagentMT & \crossmark & \unknown &  & \cellcolor{green!26!red!74}11.2 & \cellcolor{green!19!red!81}0.544 & \cellcolor{green!24!red!76}70.1 & \cellcolor{green!22!red!78}71.8 & \cellcolor{green!20!red!80}-6.7 & \cellcolor{green!24!red!76}0.579 \\
\rowcolor{gray!30}
\official ONLINE-B & \checkmark & \unknown &  & \cellcolor{green!24!red!76}11.5 & \cellcolor{green!17!red!83}0.542 & \cellcolor{green!17!red!83}69.1 & \cellcolor{green!13!red!87}69.8 & \cellcolor{green!28!red!72}-6.5 & \cellcolor{green!23!red!77}0.578 \\
SalamandraTA & \checkmark & 8 &  & \cellcolor{green!22!red!78}11.7 & \cellcolor{green!43!red!57}0.562 & \cellcolor{green!3!red!97}66.9 & \cellcolor{green!0!red!100}66.5 & \cellcolor{green!20!red!80}-6.7 & \cellcolor{green!27!red!73}0.583 \\
\rowcolor{gray!30}
CUNI-EdUKate-v1 & \checkmark & 9 &  & \cellcolor{green!16!red!84}12.5 & \cellcolor{green!34!red!66}0.555 & \cellcolor{green!5!red!95}67.2 & \cellcolor{green!3!red!97}67.4 & \cellcolor{green!3!red!97}-7.1 & \cellcolor{green!20!red!80}0.573 \\
\official AyaExpanse-8B & \checkmark & 8 &  & \cellcolor{green!11!red!89}13.3 & \cellcolor{green!14!red!86}0.54 & \cellcolor{green!3!red!97}66.9 & \cellcolor{green!0!red!100}66.7 & \cellcolor{green!12!red!88}-6.9 & \cellcolor{green!14!red!86}0.565 \\
\rowcolor{gray!30}
\official ONLINE-W & \unknown & \unknown &  & \cellcolor{green!10!red!90}13.4 & \cellcolor{green!6!red!94}0.534 & \cellcolor{green!1!red!99}66.6 & \cellcolor{green!5!red!95}68.0 & \cellcolor{green!12!red!88}-6.9 & \cellcolor{green!13!red!87}0.564 \\
\rowcolor{gray!30}
TranssionTranslate & \unknown & \unknown &  & \cellcolor{green!1!red!99}14.6 & \cellcolor{green!0!red!100}0.521 & \cellcolor{green!0!red!100}66.5 & \cellcolor{green!2!red!98}67.2 & \cellcolor{green!7!red!93}-7.0 & \cellcolor{green!0!red!100}0.541 \\
\official EuroLLM-9B[M] & \checkmark & 9 &  & \cellcolor{green!0!red!100}14.8 & \cellcolor{green!5!red!95}0.533 & \cellcolor{green!1!red!99}66.6 & \cellcolor{green!2!red!98}67.3 & \cellcolor{green!0!red!100}-7.6 & \cellcolor{green!0!red!100}0.545 \\
\rowcolor{gray!30}
DLUT\_GTCOM & \checkmark & 27 &  & \cellcolor{green!0!red!100}15.0 & \cellcolor{green!0!red!100}0.523 & \cellcolor{green!0!red!100}65.9 & \cellcolor{green!2!red!98}67.3 & \cellcolor{green!0!red!100}-7.3 & \cellcolor{green!0!red!100}0.54 \\
CUNI-SFT & \checkmark & 9 &  & \cellcolor{green!0!red!100}15.2 & \cellcolor{green!0!red!100}0.528 & \cellcolor{green!0!red!100}63.7 & \cellcolor{green!0!red!100}64.8 & \cellcolor{green!0!red!100}-7.2 & \cellcolor{green!4!red!96}0.552 \\
\rowcolor{gray!30}
\official ONLINE-G & \checkmark & \unknown &  & \cellcolor{green!0!red!100}24.0 & \cellcolor{green!0!red!100}0.471 & \cellcolor{green!0!red!100}58.0 & \cellcolor{green!0!red!100}55.3 & \cellcolor{green!0!red!100}-8.8 & \cellcolor{green!0!red!100}0.458 \\
\official Llama-3.1-8B & \crossmark & 8 &  & \cellcolor{green!0!red!100}25.3 & \cellcolor{green!0!red!100}0.493 & \cellcolor{green!0!red!100}58.2 & \cellcolor{green!0!red!100}53.8 & \cellcolor{green!0!red!100}-10.0 & \cellcolor{green!0!red!100}0.432 \\
CUNI-Transformer & \checkmark & <1 &  & \cellcolor{green!0!red!100}25.7 & \cellcolor{green!0!red!100}0.47 & \cellcolor{green!0!red!100}58.0 & \cellcolor{green!0!red!100}56.3 & \cellcolor{green!0!red!100}-10.1 & \cellcolor{green!0!red!100}0.449 \\
\official CommandR7B & \checkmark & 7 &  & \cellcolor{green!0!red!100}25.8 & \cellcolor{green!0!red!100}0.481 & \cellcolor{green!0!red!100}57.1 & \cellcolor{green!0!red!100}52.2 & \cellcolor{green!0!red!100}-10.2 & \cellcolor{green!0!red!100}0.467 \\
\official NLLB & \checkmark & 1 &  & \cellcolor{green!0!red!100}28.5 & \cellcolor{green!0!red!100}0.46 & \cellcolor{green!0!red!100}51.7 & \cellcolor{green!0!red!100}50.8 & \cellcolor{green!0!red!100}-10.3 & \cellcolor{green!0!red!100}0.439 \\
\rowcolor{gray!30}
TranssionMT & \checkmark & 1 &  & \cellcolor{green!0!red!100}33.0 & \cellcolor{green!0!red!100}0.444 & \cellcolor{green!0!red!100}52.1 & \cellcolor{green!0!red!100}49.7 & \cellcolor{green!0!red!100}-12.1 & \cellcolor{green!0!red!100}0.371 \\
\official Mistral-7B & \crossmark & 7 &  & \cellcolor{green!0!red!100}33.5 & \cellcolor{green!0!red!100}0.443 & \cellcolor{green!0!red!100}52.7 & \cellcolor{green!0!red!100}47.2 & \cellcolor{green!0!red!100}-12.0 & \cellcolor{green!0!red!100}0.359 \\
\official Qwen2.5-7B & \unknown & 7 &  & \cellcolor{green!0!red!100}42.0 & \cellcolor{green!0!red!100}0.382 & \cellcolor{green!0!red!100}45.6 & \cellcolor{green!0!red!100}40.2 & \cellcolor{green!0!red!100}-13.8 & \cellcolor{green!0!red!100}0.287 \\
\bottomrule
\end{tabularx}
\end{table*}


\begin{table*}
\small
\begin{tabularx}{\textwidth}{lYYYYYYYYY}
\toprule
\multicolumn{10}{c}{\textbf{Czech-German}} \\
\midrule
System Name & LP Supported & Params. (B) & Humeval? & AutoRank $\downarrow$ & CometKiwi-XL $\uparrow$ & GEMBA-ESA-CMDA $\uparrow$ & GEMBA-ESA-GPT4.1 $\uparrow$ & MetricX-24-Hybrid-XL $\uparrow$ & XCOMET-XL $\uparrow$ \\
\midrule
Shy-hunyuan-MT & \checkmark & 7 & \checkmark & \cellcolor{green!100!red!0}1.0 & \cellcolor{green!100!red!0}0.596 & \cellcolor{green!92!red!8}78.4 & \cellcolor{green!87!red!13}88.3 & \cellcolor{green!74!red!26}-3.6 & \cellcolor{green!100!red!0}0.653 \\
\rowcolor{gray!30}
CommandA-WMT & \checkmark & 111 & \checkmark & \cellcolor{green!93!red!7}2.1 & \cellcolor{green!73!red!27}0.582 & \cellcolor{green!88!red!12}77.9 & \cellcolor{green!83!red!17}87.5 & \cellcolor{green!100!red!0}-3.2 & \cellcolor{green!79!red!21}0.634 \\
\rowcolor{gray!30}
\official GPT-4.1 & \checkmark & \unknown & \checkmark & \cellcolor{green!92!red!8}2.3 & \cellcolor{green!69!red!31}0.58 & \cellcolor{green!99!red!1}79.4 & \cellcolor{green!100!red!0}91.0 & \cellcolor{green!67!red!33}-3.7 & \cellcolor{green!79!red!21}0.634 \\
\rowcolor{gray!30}
\official Gemini-2.5-Pro & \checkmark & \unknown & \checkmark & \cellcolor{green!90!red!10}2.5 & \cellcolor{green!63!red!37}0.577 & \cellcolor{green!97!red!3}79.1 & \cellcolor{green!99!red!1}90.8 & \cellcolor{green!74!red!26}-3.6 & \cellcolor{green!78!red!22}0.633 \\
\rowcolor{gray!30}
\official DeepSeek-V3 & \unknown & 671 & \checkmark & \cellcolor{green!84!red!16}3.5 & \cellcolor{green!63!red!37}0.577 & \cellcolor{green!100!red!0}79.5 & \cellcolor{green!88!red!12}88.6 & \cellcolor{green!61!red!39}-3.8 & \cellcolor{green!68!red!32}0.624 \\
\rowcolor{gray!30}
\official Mistral-Medium & \checkmark & \unknown & \checkmark & \cellcolor{green!80!red!20}4.1 & \cellcolor{green!63!red!37}0.577 & \cellcolor{green!85!red!15}77.6 & \cellcolor{green!80!red!20}86.9 & \cellcolor{green!61!red!39}-3.8 & \cellcolor{green!71!red!29}0.627 \\
\rowcolor{gray!30}
\official CommandA & \checkmark & 111 & \checkmark & \cellcolor{green!76!red!24}4.7 & \cellcolor{green!67!red!33}0.579 & \cellcolor{green!87!red!13}77.8 & \cellcolor{green!73!red!27}85.5 & \cellcolor{green!48!red!52}-4.0 & \cellcolor{green!68!red!32}0.624 \\
\rowcolor{gray!30}
\official Claude-4 & \checkmark & \unknown & \checkmark & \cellcolor{green!76!red!24}4.7 & \cellcolor{green!63!red!37}0.577 & \cellcolor{green!86!red!14}77.7 & \cellcolor{green!81!red!19}87.1 & \cellcolor{green!54!red!46}-3.9 & \cellcolor{green!61!red!39}0.618 \\
\rowcolor{gray!30}
GemTrans & \checkmark & 27 & \checkmark & \cellcolor{green!67!red!33}6.2 & \cellcolor{green!48!red!52}0.569 & \cellcolor{green!65!red!35}75.0 & \cellcolor{green!57!red!43}82.1 & \cellcolor{green!67!red!33}-3.7 & \cellcolor{green!62!red!38}0.619 \\
\rowcolor{gray!30}
UvA-MT & \checkmark & 12 & \checkmark & \cellcolor{green!63!red!37}6.8 & \cellcolor{green!77!red!23}0.584 & \cellcolor{green!59!red!41}74.2 & \cellcolor{green!58!red!42}82.3 & \cellcolor{green!28!red!72}-4.3 & \cellcolor{green!60!red!40}0.617 \\
\rowcolor{gray!30}
\official Gemma-3-27B & \checkmark & 27 & \checkmark & \cellcolor{green!61!red!39}7.1 & \cellcolor{green!54!red!46}0.572 & \cellcolor{green!64!red!36}74.9 & \cellcolor{green!59!red!41}82.5 & \cellcolor{green!41!red!59}-4.1 & \cellcolor{green!54!red!46}0.612 \\
\rowcolor{gray!30}
\official Llama-4-Maverick & \checkmark & 400 &  & \cellcolor{green!58!red!42}7.5 & \cellcolor{green!48!red!52}0.569 & \cellcolor{green!63!red!37}74.7 & \cellcolor{green!69!red!31}84.7 & \cellcolor{green!34!red!66}-4.2 & \cellcolor{green!46!red!54}0.604 \\
\rowcolor{gray!30}
\official AyaExpanse-32B & \checkmark & 32 &  & \cellcolor{green!55!red!45}8.0 & \cellcolor{green!46!red!54}0.568 & \cellcolor{green!60!red!40}74.3 & \cellcolor{green!49!red!51}80.5 & \cellcolor{green!41!red!59}-4.1 & \cellcolor{green!48!red!52}0.606 \\
Yolu & \checkmark & 14 & \checkmark & \cellcolor{green!49!red!51}9.0 & \cellcolor{green!86!red!14}0.589 & \cellcolor{green!27!red!73}70.0 & \cellcolor{green!23!red!77}75.2 & \cellcolor{green!21!red!79}-4.4 & \cellcolor{green!56!red!44}0.613 \\
\rowcolor{gray!30}
\official TowerPlus-72B[M] & \checkmark & 72 &  & \cellcolor{green!47!red!53}9.3 & \cellcolor{green!54!red!46}0.572 & \cellcolor{green!51!red!49}73.1 & \cellcolor{green!39!red!61}78.5 & \cellcolor{green!21!red!79}-4.4 & \cellcolor{green!41!red!59}0.6 \\
\rowcolor{gray!30}
\official Qwen3-235B & \checkmark & 235 &  & \cellcolor{green!47!red!53}9.3 & \cellcolor{green!40!red!60}0.565 & \cellcolor{green!51!red!49}73.1 & \cellcolor{green!51!red!49}80.9 & \cellcolor{green!34!red!66}-4.2 & \cellcolor{green!34!red!66}0.594 \\
Laniqo & \checkmark & 9 & \checkmark & \cellcolor{green!42!red!58}10.1 & \cellcolor{green!83!red!17}0.587 & \cellcolor{green!7!red!93}67.5 & \cellcolor{green!0!red!100}70.3 & \cellcolor{green!34!red!66}-4.2 & \cellcolor{green!62!red!38}0.619 \\
\official TowerPlus-9B[M] & \checkmark & 9 & \checkmark & \cellcolor{green!42!red!58}10.1 & \cellcolor{green!46!red!54}0.568 & \cellcolor{green!40!red!60}71.7 & \cellcolor{green!34!red!66}77.4 & \cellcolor{green!21!red!79}-4.4 & \cellcolor{green!40!red!60}0.599 \\
Wenyiil & \checkmark & 14 & \checkmark & \cellcolor{green!38!red!62}10.7 & \cellcolor{green!29!red!71}0.559 & \cellcolor{green!35!red!65}71.1 & \cellcolor{green!36!red!64}77.9 & \cellcolor{green!28!red!72}-4.3 & \cellcolor{green!38!red!62}0.597 \\
SRPOL & \checkmark & 12 & \checkmark & \cellcolor{green!37!red!63}10.8 & \cellcolor{green!94!red!6}0.593 & \cellcolor{green!20!red!80}69.2 & \cellcolor{green!13!red!87}73.2 & \cellcolor{green!2!red!98}-4.7 & \cellcolor{green!31!red!69}0.591 \\
\rowcolor{gray!30}
\official EuroLLM-22B-pre.[M] & \checkmark & 22 &  & \cellcolor{green!36!red!64}11.0 & \cellcolor{green!44!red!56}0.567 & \cellcolor{green!32!red!68}70.7 & \cellcolor{green!34!red!66}77.4 & \cellcolor{green!8!red!92}-4.6 & \cellcolor{green!37!red!63}0.596 \\
\official Gemma-3-12B & \checkmark & 12 & \checkmark & \cellcolor{green!35!red!65}11.2 & \cellcolor{green!32!red!68}0.561 & \cellcolor{green!41!red!59}71.9 & \cellcolor{green!34!red!66}77.5 & \cellcolor{green!8!red!92}-4.6 & \cellcolor{green!32!red!68}0.592 \\
\rowcolor{gray!30}
\official ONLINE-B & \checkmark & \unknown &  & \cellcolor{green!32!red!68}11.7 & \cellcolor{green!21!red!79}0.555 & \cellcolor{green!21!red!79}69.3 & \cellcolor{green!19!red!81}74.4 & \cellcolor{green!41!red!59}-4.1 & \cellcolor{green!38!red!62}0.597 \\
IRB-MT & \checkmark & 12 & \checkmark & \cellcolor{green!29!red!71}12.1 & \cellcolor{green!25!red!75}0.557 & \cellcolor{green!31!red!69}70.6 & \cellcolor{green!24!red!76}75.4 & \cellcolor{green!15!red!85}-4.5 & \cellcolor{green!28!red!72}0.588 \\
Algharb & \checkmark & 14 & \checkmark & \cellcolor{green!24!red!76}12.9 & \cellcolor{green!13!red!87}0.551 & \cellcolor{green!33!red!67}70.8 & \cellcolor{green!32!red!68}77.1 & \cellcolor{green!2!red!98}-4.7 & \cellcolor{green!19!red!81}0.58 \\
\rowcolor{gray!30}
IR-MultiagentMT & \crossmark & \unknown &  & \cellcolor{green!23!red!77}13.0 & \cellcolor{green!29!red!71}0.559 & \cellcolor{green!11!red!89}68.0 & \cellcolor{green!24!red!76}75.3 & \cellcolor{green!2!red!98}-4.7 & \cellcolor{green!32!red!68}0.592 \\
CUNI-MH-v2 & \checkmark & 9 & \checkmark & \cellcolor{green!18!red!82}13.8 & \cellcolor{green!34!red!66}0.562 & \cellcolor{green!13!red!87}68.2 & \cellcolor{green!10!red!90}72.5 & \cellcolor{green!2!red!98}-4.7 & \cellcolor{green!16!red!84}0.577 \\
SalamandraTA & \checkmark & 8 &  & \cellcolor{green!8!red!92}15.3 & \cellcolor{green!19!red!81}0.554 & \cellcolor{green!0!red!100}65.8 & \cellcolor{green!0!red!100}69.5 & \cellcolor{green!8!red!92}-4.6 & \cellcolor{green!12!red!88}0.574 \\
\official AyaExpanse-8B & \checkmark & 8 &  & \cellcolor{green!8!red!92}15.4 & \cellcolor{green!21!red!79}0.555 & \cellcolor{green!0!red!100}66.4 & \cellcolor{green!2!red!98}70.9 & \cellcolor{green!2!red!98}-4.7 & \cellcolor{green!1!red!99}0.564 \\
\rowcolor{gray!30}
TranssionTranslate & \unknown & \unknown &  & \cellcolor{green!0!red!100}16.6 & \cellcolor{green!0!red!100}0.538 & \cellcolor{green!3!red!97}67.0 & \cellcolor{green!3!red!97}71.1 & \cellcolor{green!2!red!98}-4.7 & \cellcolor{green!0!red!100}0.56 \\
\rowcolor{gray!30}
\official ONLINE-W & \unknown & \unknown &  & \cellcolor{green!0!red!100}16.7 & \cellcolor{green!0!red!100}0.542 & \cellcolor{green!3!red!97}67.0 & \cellcolor{green!3!red!97}71.1 & \cellcolor{green!0!red!100}-4.9 & \cellcolor{green!0!red!100}0.56 \\
\rowcolor{gray!30}
DLUT\_GTCOM & \checkmark & 27 &  & \cellcolor{green!0!red!100}17.4 & \cellcolor{green!0!red!100}0.537 & \cellcolor{green!0!red!100}66.6 & \cellcolor{green!0!red!100}70.5 & \cellcolor{green!0!red!100}-4.8 & \cellcolor{green!0!red!100}0.553 \\
\official CommandR7B & \checkmark & 7 &  & \cellcolor{green!0!red!100}17.9 & \cellcolor{green!1!red!99}0.545 & \cellcolor{green!0!red!100}65.8 & \cellcolor{green!0!red!100}68.9 & \cellcolor{green!0!red!100}-5.1 & \cellcolor{green!0!red!100}0.556 \\
\official EuroLLM-9B[M] & \checkmark & 9 &  & \cellcolor{green!0!red!100}22.4 & \cellcolor{green!0!red!100}0.531 & \cellcolor{green!0!red!100}57.1 & \cellcolor{green!0!red!100}61.1 & \cellcolor{green!0!red!100}-5.6 & \cellcolor{green!18!red!82}0.579 \\
\official Llama-3.1-8B & \checkmark & 8 &  & \cellcolor{green!0!red!100}25.3 & \cellcolor{green!0!red!100}0.524 & \cellcolor{green!0!red!100}59.6 & \cellcolor{green!0!red!100}61.6 & \cellcolor{green!0!red!100}-5.8 & \cellcolor{green!0!red!100}0.508 \\
\rowcolor{gray!30}
\official ONLINE-G & \checkmark & \unknown &  & \cellcolor{green!0!red!100}32.1 & \cellcolor{green!0!red!100}0.492 & \cellcolor{green!0!red!100}58.1 & \cellcolor{green!0!red!100}58.4 & \cellcolor{green!0!red!100}-6.9 & \cellcolor{green!0!red!100}0.47 \\
\official Qwen2.5-7B & \checkmark & 7 &  & \cellcolor{green!0!red!100}32.1 & \cellcolor{green!0!red!100}0.503 & \cellcolor{green!0!red!100}54.0 & \cellcolor{green!0!red!100}54.5 & \cellcolor{green!0!red!100}-6.6 & \cellcolor{green!0!red!100}0.476 \\
\official NLLB & \checkmark & 1 &  & \cellcolor{green!0!red!100}33.4 & \cellcolor{green!0!red!100}0.5 & \cellcolor{green!0!red!100}52.5 & \cellcolor{green!0!red!100}54.2 & \cellcolor{green!0!red!100}-6.9 & \cellcolor{green!0!red!100}0.479 \\
\official Mistral-7B & \crossmark & 7 &  & \cellcolor{green!0!red!100}36.4 & \cellcolor{green!0!red!100}0.492 & \cellcolor{green!0!red!100}53.3 & \cellcolor{green!0!red!100}51.1 & \cellcolor{green!0!red!100}-7.1 & \cellcolor{green!0!red!100}0.434 \\
\rowcolor{gray!30}
TranssionMT & \checkmark & 1 &  & \cellcolor{green!0!red!100}40.0 & \cellcolor{green!0!red!100}0.473 & \cellcolor{green!0!red!100}51.2 & \cellcolor{green!0!red!100}52.4 & \cellcolor{green!0!red!100}-7.9 & \cellcolor{green!0!red!100}0.425 \\
\bottomrule
\end{tabularx}
\end{table*}


\begin{table*}
\small
\begin{tabularx}{\textwidth}{lYYYYYYYYY}
\toprule
\multicolumn{10}{c}{\textbf{Japanese-Simplified Chinese}} \\
\midrule
System Name & LP Supported & Params. (B) & Humeval? & AutoRank $\downarrow$ & CometKiwi-XL $\uparrow$ & GEMBA-ESA-CMDA $\uparrow$ & GEMBA-ESA-GPT4.1 $\uparrow$ & MetricX-24-Hybrid-XL $\uparrow$ & XCOMET-XL $\uparrow$ \\
\midrule
Shy-hunyuan-MT & \checkmark & 7 & \checkmark & \cellcolor{green!100!red!0}1.0 & \cellcolor{green!59!red!41}0.577 & \cellcolor{green!100!red!0}85.1 & \cellcolor{green!100!red!0}85.5 & \cellcolor{green!100!red!0}-4.2 & \cellcolor{green!100!red!0}0.629 \\
\rowcolor{gray!30}
In2x & \unknown & 72 & \checkmark & \cellcolor{green!87!red!13}3.0 & \cellcolor{green!100!red!0}0.624 & \cellcolor{green!62!red!38}77.0 & \cellcolor{green!66!red!34}77.7 & \cellcolor{green!77!red!23}-4.7 & \cellcolor{green!93!red!7}0.618 \\
\rowcolor{gray!30}
\official Gemini-2.5-Pro & \checkmark & \unknown & \checkmark & \cellcolor{green!86!red!14}3.2 & \cellcolor{green!35!red!65}0.549 & \cellcolor{green!99!red!1}84.8 & \cellcolor{green!97!red!3}84.8 & \cellcolor{green!82!red!18}-4.6 & \cellcolor{green!79!red!21}0.596 \\
\rowcolor{gray!30}
Kaze-MT & \checkmark & 72 & \checkmark & \cellcolor{green!82!red!18}3.8 & \cellcolor{green!52!red!48}0.569 & \cellcolor{green!83!red!17}81.5 & \cellcolor{green!84!red!16}81.8 & \cellcolor{green!73!red!27}-4.8 & \cellcolor{green!85!red!15}0.605 \\
Algharb & \checkmark & 14 & \checkmark & \cellcolor{green!79!red!21}4.2 & \cellcolor{green!33!red!67}0.547 & \cellcolor{green!93!red!7}83.5 & \cellcolor{green!94!red!6}84.1 & \cellcolor{green!73!red!27}-4.8 & \cellcolor{green!71!red!29}0.583 \\
\rowcolor{gray!30}
\official GPT-4.1 & \checkmark & \unknown & \checkmark & \cellcolor{green!78!red!22}4.4 & \cellcolor{green!35!red!65}0.549 & \cellcolor{green!94!red!6}83.8 & \cellcolor{green!97!red!3}84.7 & \cellcolor{green!59!red!41}-5.1 & \cellcolor{green!70!red!30}0.582 \\
Wenyiil & \checkmark & 14 & \checkmark & \cellcolor{green!77!red!23}4.5 & \cellcolor{green!40!red!60}0.555 & \cellcolor{green!83!red!17}81.4 & \cellcolor{green!84!red!16}81.9 & \cellcolor{green!73!red!27}-4.8 & \cellcolor{green!76!red!24}0.591 \\
\rowcolor{gray!30}
CommandA-WMT & \checkmark & 111 & \checkmark & \cellcolor{green!73!red!27}5.1 & \cellcolor{green!43!red!57}0.558 & \cellcolor{green!77!red!23}80.2 & \cellcolor{green!75!red!25}79.7 & \cellcolor{green!77!red!23}-4.7 & \cellcolor{green!66!red!34}0.575 \\
NTTSU & \checkmark & 14 & \checkmark & \cellcolor{green!68!red!32}5.8 & \cellcolor{green!47!red!53}0.563 & \cellcolor{green!64!red!36}77.5 & \cellcolor{green!53!red!47}74.8 & \cellcolor{green!82!red!18}-4.6 & \cellcolor{green!67!red!33}0.577 \\
\rowcolor{gray!30}
bb88 & \unknown & \unknown &  & \cellcolor{green!66!red!34}6.1 & \cellcolor{green!37!red!63}0.551 & \cellcolor{green!77!red!23}80.1 & \cellcolor{green!71!red!29}78.9 & \cellcolor{green!55!red!45}-5.2 & \cellcolor{green!64!red!36}0.573 \\
\rowcolor{gray!30}
\official Claude-4 & \checkmark & \unknown & \checkmark & \cellcolor{green!66!red!34}6.2 & \cellcolor{green!31!red!69}0.545 & \cellcolor{green!90!red!10}82.9 & \cellcolor{green!92!red!8}83.7 & \cellcolor{green!36!red!64}-5.6 & \cellcolor{green!54!red!46}0.556 \\
\rowcolor{gray!30}
\official DeepSeek-V3 & \checkmark & 671 & \checkmark & \cellcolor{green!65!red!35}6.3 & \cellcolor{green!22!red!78}0.534 & \cellcolor{green!90!red!10}82.9 & \cellcolor{green!80!red!20}80.9 & \cellcolor{green!59!red!41}-5.1 & \cellcolor{green!51!red!49}0.552 \\
\rowcolor{gray!30}
\official Mistral-Medium & \unknown & \unknown & \checkmark & \cellcolor{green!64!red!36}6.4 & \cellcolor{green!32!red!68}0.546 & \cellcolor{green!81!red!19}81.1 & \cellcolor{green!81!red!19}81.1 & \cellcolor{green!45!red!55}-5.4 & \cellcolor{green!55!red!45}0.558 \\
\rowcolor{gray!30}
GemTrans & \checkmark & 27 & \checkmark & \cellcolor{green!64!red!36}6.5 & \cellcolor{green!41!red!59}0.556 & \cellcolor{green!57!red!43}76.0 & \cellcolor{green!54!red!46}74.9 & \cellcolor{green!73!red!27}-4.8 & \cellcolor{green!68!red!32}0.579 \\
Yolu & \checkmark & 14 & \checkmark & \cellcolor{green!61!red!39}6.9 & \cellcolor{green!60!red!40}0.578 & \cellcolor{green!51!red!49}74.6 & \cellcolor{green!48!red!52}73.6 & \cellcolor{green!64!red!36}-5.0 & \cellcolor{green!59!red!41}0.565 \\
\rowcolor{gray!30}
\official Qwen3-235B & \checkmark & 235 & \checkmark & \cellcolor{green!57!red!43}7.5 & \cellcolor{green!35!red!65}0.549 & \cellcolor{green!69!red!31}78.4 & \cellcolor{green!63!red!37}77.0 & \cellcolor{green!45!red!55}-5.4 & \cellcolor{green!53!red!47}0.555 \\
\rowcolor{gray!30}
\official CommandA & \checkmark & 111 &  & \cellcolor{green!57!red!43}7.6 & \cellcolor{green!27!red!73}0.54 & \cellcolor{green!73!red!27}79.4 & \cellcolor{green!66!red!34}77.6 & \cellcolor{green!41!red!59}-5.5 & \cellcolor{green!54!red!46}0.556 \\
\rowcolor{gray!30}
UvA-MT & \checkmark & 12 &  & \cellcolor{green!52!red!48}8.3 & \cellcolor{green!48!red!52}0.564 & \cellcolor{green!48!red!52}73.9 & \cellcolor{green!55!red!45}75.2 & \cellcolor{green!36!red!64}-5.6 & \cellcolor{green!57!red!43}0.561 \\
\rowcolor{gray!30}
\official TowerPlus-72B[M] & \checkmark & 72 &  & \cellcolor{green!43!red!57}9.7 & \cellcolor{green!24!red!76}0.537 & \cellcolor{green!60!red!40}76.5 & \cellcolor{green!54!red!46}75.0 & \cellcolor{green!23!red!77}-5.9 & \cellcolor{green!41!red!59}0.536 \\
\rowcolor{gray!30}
\official AyaExpanse-32B & \checkmark & 32 &  & \cellcolor{green!36!red!64}10.7 & \cellcolor{green!24!red!76}0.537 & \cellcolor{green!44!red!56}73.2 & \cellcolor{green!41!red!59}72.0 & \cellcolor{green!27!red!73}-5.8 & \cellcolor{green!31!red!69}0.521 \\
Laniqo & \checkmark & 9 & \checkmark & \cellcolor{green!34!red!66}11.1 & \cellcolor{green!61!red!39}0.579 & \cellcolor{green!0!red!100}63.1 & \cellcolor{green!0!red!100}62.1 & \cellcolor{green!45!red!55}-5.4 & \cellcolor{green!54!red!46}0.557 \\
\official TowerPlus-9B[M] & \checkmark & 9 & \checkmark & \cellcolor{green!33!red!67}11.2 & \cellcolor{green!23!red!77}0.535 & \cellcolor{green!38!red!62}71.9 & \cellcolor{green!32!red!68}69.8 & \cellcolor{green!27!red!73}-5.8 & \cellcolor{green!32!red!68}0.523 \\
IRB-MT & \checkmark & 12 & \checkmark & \cellcolor{green!27!red!73}12.1 & \cellcolor{green!10!red!90}0.521 & \cellcolor{green!40!red!60}72.2 & \cellcolor{green!34!red!66}70.4 & \cellcolor{green!18!red!82}-6.0 & \cellcolor{green!24!red!76}0.509 \\
\rowcolor{gray!30}
\official Gemma-3-27B & \checkmark & 27 &  & \cellcolor{green!22!red!78}12.8 & \cellcolor{green!15!red!85}0.526 & \cellcolor{green!31!red!69}70.4 & \cellcolor{green!33!red!67}70.2 & \cellcolor{green!9!red!91}-6.2 & \cellcolor{green!20!red!80}0.503 \\
\rowcolor{gray!30}
\official Llama-4-Maverick & \checkmark & 400 &  & \cellcolor{green!20!red!80}13.1 & \cellcolor{green!13!red!87}0.524 & \cellcolor{green!36!red!64}71.5 & \cellcolor{green!16!red!84}66.1 & \cellcolor{green!5!red!95}-6.3 & \cellcolor{green!29!red!71}0.518 \\
\official Qwen2.5-7B & \checkmark & 7 &  & \cellcolor{green!17!red!83}13.6 & \cellcolor{green!13!red!87}0.524 & \cellcolor{green!24!red!76}68.9 & \cellcolor{green!21!red!79}67.4 & \cellcolor{green!5!red!95}-6.3 & \cellcolor{green!19!red!81}0.502 \\
\rowcolor{gray!30}
IR-MultiagentMT & \crossmark & \unknown &  & \cellcolor{green!16!red!84}13.7 & \cellcolor{green!12!red!88}0.523 & \cellcolor{green!19!red!81}67.8 & \cellcolor{green!26!red!74}68.5 & \cellcolor{green!9!red!91}-6.2 & \cellcolor{green!13!red!87}0.492 \\
SRPOL & \crossmark & 12 &  & \cellcolor{green!16!red!84}13.8 & \cellcolor{green!44!red!56}0.56 & \cellcolor{green!0!red!100}63.8 & \cellcolor{green!0!red!100}62.5 & \cellcolor{green!0!red!100}-6.4 & \cellcolor{green!32!red!68}0.522 \\
\rowcolor{gray!30}
\official EuroLLM-22B-pre.[M] & \checkmark & 22 &  & \cellcolor{green!10!red!90}14.7 & \cellcolor{green!10!red!90}0.521 & \cellcolor{green!13!red!87}66.4 & \cellcolor{green!16!red!84}66.2 & \cellcolor{green!5!red!95}-6.3 & \cellcolor{green!9!red!91}0.486 \\
\official AyaExpanse-8B & \checkmark & 8 &  & \cellcolor{green!5!red!95}15.5 & \cellcolor{green!8!red!92}0.518 & \cellcolor{green!9!red!91}65.6 & \cellcolor{green!8!red!92}64.4 & \cellcolor{green!0!red!100}-6.4 & \cellcolor{green!0!red!100}0.472 \\
\rowcolor{gray!30}
\official ONLINE-B & \checkmark & \unknown &  & \cellcolor{green!0!red!100}16.2 & \cellcolor{green!0!red!100}0.499 & \cellcolor{green!0!red!100}63.7 & \cellcolor{green!3!red!97}63.2 & \cellcolor{green!9!red!91}-6.2 & \cellcolor{green!0!red!100}0.472 \\
\official Gemma-3-12B & \checkmark & 12 &  & \cellcolor{green!0!red!100}17.1 & \cellcolor{green!0!red!100}0.509 & \cellcolor{green!6!red!94}65.0 & \cellcolor{green!7!red!93}64.1 & \cellcolor{green!0!red!100}-7.1 & \cellcolor{green!0!red!100}0.465 \\
\official CommandR7B & \checkmark & 7 &  & \cellcolor{green!0!red!100}18.4 & \cellcolor{green!0!red!100}0.496 & \cellcolor{green!0!red!100}59.8 & \cellcolor{green!0!red!100}58.5 & \cellcolor{green!0!red!100}-6.9 & \cellcolor{green!9!red!91}0.486 \\
\rowcolor{gray!30}
TranssionTranslate & \unknown & \unknown &  & \cellcolor{green!0!red!100}18.8 & \cellcolor{green!0!red!100}0.488 & \cellcolor{green!0!red!100}59.9 & \cellcolor{green!0!red!100}60.6 & \cellcolor{green!0!red!100}-6.7 & \cellcolor{green!0!red!100}0.45 \\
\official Llama-3.1-8B & \crossmark & 8 &  & \cellcolor{green!0!red!100}20.2 & \cellcolor{green!0!red!100}0.507 & \cellcolor{green!0!red!100}58.8 & \cellcolor{green!0!red!100}57.3 & \cellcolor{green!0!red!100}-7.2 & \cellcolor{green!0!red!100}0.423 \\
\official EuroLLM-9B[M] & \checkmark & 9 &  & \cellcolor{green!0!red!100}20.8 & \cellcolor{green!0!red!100}0.479 & \cellcolor{green!0!red!100}59.4 & \cellcolor{green!0!red!100}57.2 & \cellcolor{green!0!red!100}-7.6 & \cellcolor{green!0!red!100}0.461 \\
\rowcolor{gray!30}
\official ONLINE-W & \unknown & \unknown &  & \cellcolor{green!0!red!100}25.2 & \cellcolor{green!0!red!100}0.456 & \cellcolor{green!0!red!100}52.3 & \cellcolor{green!0!red!100}52.9 & \cellcolor{green!0!red!100}-7.9 & \cellcolor{green!0!red!100}0.387 \\
\official Mistral-7B & \crossmark & 7 &  & \cellcolor{green!0!red!100}32.8 & \cellcolor{green!0!red!100}0.445 & \cellcolor{green!0!red!100}42.9 & \cellcolor{green!0!red!100}43.4 & \cellcolor{green!0!red!100}-9.8 & \cellcolor{green!0!red!100}0.317 \\
SalamandraTA & \checkmark & 8 &  & \cellcolor{green!0!red!100}33.1 & \cellcolor{green!0!red!100}0.426 & \cellcolor{green!0!red!100}36.5 & \cellcolor{green!0!red!100}38.0 & \cellcolor{green!0!red!100}-8.6 & \cellcolor{green!0!red!100}0.328 \\
\rowcolor{gray!30}
\official ONLINE-G & \checkmark & \unknown &  & \cellcolor{green!0!red!100}40.8 & \cellcolor{green!0!red!100}0.352 & \cellcolor{green!0!red!100}39.5 & \cellcolor{green!0!red!100}39.8 & \cellcolor{green!0!red!100}-12.1 & \cellcolor{green!0!red!100}0.28 \\
\official NLLB & \checkmark & 1 &  & \cellcolor{green!0!red!100}41.0 & \cellcolor{green!0!red!100}0.371 & \cellcolor{green!0!red!100}35.5 & \cellcolor{green!0!red!100}35.8 & \cellcolor{green!0!red!100}-12.1 & \cellcolor{green!0!red!100}0.303 \\
\bottomrule
\end{tabularx}
\end{table*}


\begin{table*}
\small
\begin{tabularx}{\textwidth}{lYYYYYYY}
\toprule
\multicolumn{8}{c}{\textbf{English-Bengali}} \\
\midrule
System Name & LP Supported & Params. (B) & AutoRank $\downarrow$ & GEMBA-ESA-CMDA $\uparrow$ & GEMBA-ESA-GPT4.1 $\uparrow$ & MetricX-24-Hybrid-XL $\uparrow$ & XCOMET-XL $\uparrow$ \\
\midrule
Shy-hunyuan-MT & \crossmark & 7 & \cellcolor{green!100!red!0}1.0 & \cellcolor{green!100!red!0}67.9 & \cellcolor{green!93!red!7}83.2 & \cellcolor{green!100!red!0}-4.8 & \cellcolor{green!100!red!0}0.449 \\
\rowcolor{gray!30}
\official Gemini-2.5-Pro & \checkmark & \unknown & \cellcolor{green!93!red!7}2.2 & \cellcolor{green!93!red!7}66.5 & \cellcolor{green!100!red!0}86.6 & \cellcolor{green!93!red!7}-5.2 & \cellcolor{green!75!red!25}0.382 \\
\rowcolor{gray!30}
\official GPT-4.1 & \checkmark & \unknown & \cellcolor{green!87!red!13}3.2 & \cellcolor{green!95!red!5}66.9 & \cellcolor{green!90!red!10}81.6 & \cellcolor{green!81!red!19}-5.9 & \cellcolor{green!71!red!29}0.373 \\
\rowcolor{gray!30}
GemTrans & \checkmark & 27 & \cellcolor{green!85!red!15}3.5 & \cellcolor{green!83!red!17}64.3 & \cellcolor{green!76!red!24}75.1 & \cellcolor{green!97!red!3}-5.0 & \cellcolor{green!72!red!28}0.374 \\
\rowcolor{gray!30}
\official Mistral-Medium & \unknown & \unknown & \cellcolor{green!82!red!18}3.9 & \cellcolor{green!89!red!11}65.5 & \cellcolor{green!82!red!18}78.0 & \cellcolor{green!79!red!21}-6.0 & \cellcolor{green!69!red!31}0.366 \\
\rowcolor{gray!30}
\official Claude-4 & \checkmark & \unknown & \cellcolor{green!82!red!18}4.0 & \cellcolor{green!89!red!11}65.6 & \cellcolor{green!88!red!12}80.6 & \cellcolor{green!78!red!22}-6.1 & \cellcolor{green!62!red!38}0.348 \\
\rowcolor{gray!30}
UvA-MT & \unknown & 12 & \cellcolor{green!80!red!20}4.2 & \cellcolor{green!82!red!18}64.1 & \cellcolor{green!76!red!24}75.0 & \cellcolor{green!78!red!22}-6.1 & \cellcolor{green!74!red!26}0.381 \\
\rowcolor{gray!30}
\official DeepSeek-V3 & \unknown & 671 & \cellcolor{green!80!red!20}4.3 & \cellcolor{green!80!red!20}63.7 & \cellcolor{green!81!red!19}77.7 & \cellcolor{green!78!red!22}-6.1 & \cellcolor{green!68!red!32}0.364 \\
IRB-MT & \checkmark & 12 & \cellcolor{green!75!red!25}5.1 & \cellcolor{green!76!red!24}62.9 & \cellcolor{green!71!red!29}72.7 & \cellcolor{green!79!red!21}-6.0 & \cellcolor{green!59!red!41}0.34 \\
\rowcolor{gray!30}
CommandA-WMT & \crossmark & 111 & \cellcolor{green!74!red!26}5.3 & \cellcolor{green!79!red!21}63.5 & \cellcolor{green!64!red!36}69.4 & \cellcolor{green!76!red!24}-6.2 & \cellcolor{green!61!red!39}0.345 \\
\rowcolor{gray!30}
\official Llama-4-Maverick & \checkmark & 400 & \cellcolor{green!73!red!27}5.5 & \cellcolor{green!81!red!19}63.9 & \cellcolor{green!74!red!26}73.9 & \cellcolor{green!74!red!26}-6.3 & \cellcolor{green!50!red!50}0.315 \\
\rowcolor{gray!30}
\official Qwen3-235B & \checkmark & 235 & \cellcolor{green!69!red!31}6.0 & \cellcolor{green!75!red!25}62.7 & \cellcolor{green!68!red!32}71.2 & \cellcolor{green!73!red!27}-6.4 & \cellcolor{green!49!red!51}0.313 \\
\rowcolor{gray!30}
\official ONLINE-B & \checkmark & \unknown & \cellcolor{green!63!red!37}7.1 & \cellcolor{green!60!red!40}59.5 & \cellcolor{green!57!red!43}65.9 & \cellcolor{green!73!red!27}-6.4 & \cellcolor{green!45!red!55}0.304 \\
\rowcolor{gray!30}
TranssionTranslate & \unknown & \unknown & \cellcolor{green!62!red!38}7.3 & \cellcolor{green!60!red!40}59.4 & \cellcolor{green!53!red!47}63.9 & \cellcolor{green!73!red!27}-6.4 & \cellcolor{green!44!red!56}0.301 \\
\official Gemma-3-12B & \checkmark & 12 & \cellcolor{green!60!red!40}7.6 & \cellcolor{green!62!red!38}59.8 & \cellcolor{green!57!red!43}65.9 & \cellcolor{green!56!red!44}-7.4 & \cellcolor{green!50!red!50}0.316 \\
\rowcolor{gray!30}
\official Gemma-3-27B & \checkmark & 27 & \cellcolor{green!58!red!42}7.8 & \cellcolor{green!42!red!58}55.7 & \cellcolor{green!56!red!44}65.6 & \cellcolor{green!61!red!39}-7.1 & \cellcolor{green!57!red!43}0.335 \\
\rowcolor{gray!30}
\official CommandA & \crossmark & 111 & \cellcolor{green!50!red!50}9.2 & \cellcolor{green!66!red!34}60.8 & \cellcolor{green!43!red!57}59.2 & \cellcolor{green!45!red!55}-8.0 & \cellcolor{green!27!red!73}0.254 \\
\official NLLB & \checkmark & 1 & \cellcolor{green!36!red!64}11.4 & \cellcolor{green!34!red!66}53.9 & \cellcolor{green!35!red!65}55.5 & \cellcolor{green!35!red!65}-8.6 & \cellcolor{green!20!red!80}0.235 \\
\rowcolor{gray!30}
IR-MultiagentMT & \crossmark & \unknown & \cellcolor{green!36!red!64}11.5 & \cellcolor{green!33!red!67}53.7 & \cellcolor{green!35!red!65}55.5 & \cellcolor{green!35!red!65}-8.6 & \cellcolor{green!21!red!79}0.238 \\
\rowcolor{gray!30}
\official TowerPlus-72B[M] & \crossmark & 72 & \cellcolor{green!24!red!76}13.5 & \cellcolor{green!39!red!61}55.0 & \cellcolor{green!18!red!82}47.0 & \cellcolor{green!13!red!87}-9.9 & \cellcolor{green!2!red!98}0.189 \\
\official Llama-3.1-8B & \crossmark & 8 & \cellcolor{green!19!red!81}14.2 & \cellcolor{green!19!red!81}50.7 & \cellcolor{green!16!red!84}46.1 & \cellcolor{green!20!red!80}-9.5 & \cellcolor{green!0!red!100}0.176 \\
\rowcolor{gray!30}
\official ONLINE-G & \checkmark & \unknown & \cellcolor{green!10!red!90}15.8 & \cellcolor{green!7!red!93}48.3 & \cellcolor{green!20!red!80}48.1 & \cellcolor{green!0!red!100}-10.9 & \cellcolor{green!0!red!100}0.151 \\
\rowcolor{gray!30}
\official AyaExpanse-32B & \crossmark & 32 & \cellcolor{green!0!red!100}17.9 & \cellcolor{green!0!red!100}46.2 & \cellcolor{green!0!red!100}36.1 & \cellcolor{green!0!red!100}-11.7 & \cellcolor{green!0!red!100}0.143 \\
\official TowerPlus-9B[M] & \crossmark & 9 & \cellcolor{green!0!red!100}20.3 & \cellcolor{green!0!red!100}27.9 & \cellcolor{green!0!red!100}9.6 & \cellcolor{green!28!red!72}-9.0 & \cellcolor{green!17!red!83}0.228 \\
\official Qwen2.5-7B & \unknown & 7 & \cellcolor{green!0!red!100}21.1 & \cellcolor{green!0!red!100}36.6 & \cellcolor{green!0!red!100}30.8 & \cellcolor{green!0!red!100}-12.8 & \cellcolor{green!0!red!100}0.122 \\
\official CommandR7B & \crossmark & 7 & \cellcolor{green!0!red!100}22.7 & \cellcolor{green!0!red!100}30.6 & \cellcolor{green!0!red!100}22.4 & \cellcolor{green!0!red!100}-13.8 & \cellcolor{green!0!red!100}0.181 \\
\official AyaExpanse-8B & \crossmark & 8 & \cellcolor{green!0!red!100}25.1 & \cellcolor{green!0!red!100}27.7 & \cellcolor{green!0!red!100}21.5 & \cellcolor{green!0!red!100}-16.1 & \cellcolor{green!0!red!100}0.16 \\
\official EuroLLM-9B[M] & \crossmark & 9 & \cellcolor{green!0!red!100}27.5 & \cellcolor{green!0!red!100}15.5 & \cellcolor{green!0!red!100}6.2 & \cellcolor{green!0!red!100}-15.2 & \cellcolor{green!2!red!98}0.189 \\
\official Mistral-7B & \crossmark & 7 & \cellcolor{green!0!red!100}28.6 & \cellcolor{green!0!red!100}19.3 & \cellcolor{green!0!red!100}14.4 & \cellcolor{green!0!red!100}-18.6 & \cellcolor{green!0!red!100}0.175 \\
\rowcolor{gray!30}
\official EuroLLM-22B-pre.[M] & \crossmark & 22 & \cellcolor{green!0!red!100}30.0 & \cellcolor{green!0!red!100}16.5 & \cellcolor{green!0!red!100}12.7 & \cellcolor{green!0!red!100}-19.5 & \cellcolor{green!0!red!100}0.171 \\
\bottomrule
\end{tabularx}
\end{table*}


\begin{table*}
\small
\begin{tabularx}{\textwidth}{lYYYYYYY}
\toprule
\multicolumn{8}{c}{\textbf{English-German}} \\
\midrule
System Name & LP Supported & Params. (B) & AutoRank $\downarrow$ & GEMBA-ESA-CMDA $\uparrow$ & GEMBA-ESA-GPT4.1 $\uparrow$ & MetricX-24-Hybrid-XL $\uparrow$ & XCOMET-XL $\uparrow$ \\
\midrule
Shy-hunyuan-MT & \checkmark & 7 & \cellcolor{green!100!red!0}1.0 & \cellcolor{green!98!red!2}84.3 & \cellcolor{green!93!red!7}90.6 & \cellcolor{green!94!red!6}-3.1 & \cellcolor{green!100!red!0}0.703 \\
\rowcolor{gray!30}
CommandA-WMT & \checkmark & 111 & \cellcolor{green!90!red!10}2.2 & \cellcolor{green!87!red!13}82.8 & \cellcolor{green!84!red!16}89.0 & \cellcolor{green!100!red!0}-3.0 & \cellcolor{green!77!red!23}0.686 \\
\rowcolor{gray!30}
\official GPT-4.1 & \checkmark & \unknown & \cellcolor{green!81!red!19}3.2 & \cellcolor{green!100!red!0}84.6 & \cellcolor{green!98!red!2}91.4 & \cellcolor{green!61!red!39}-3.6 & \cellcolor{green!56!red!44}0.671 \\
\rowcolor{gray!30}
\official Gemini-2.5-Pro & \checkmark & \unknown & \cellcolor{green!79!red!21}3.4 & \cellcolor{green!96!red!4}84.0 & \cellcolor{green!100!red!0}91.7 & \cellcolor{green!68!red!32}-3.5 & \cellcolor{green!48!red!52}0.665 \\
\rowcolor{gray!30}
\official DeepSeek-V3 & \unknown & 671 & \cellcolor{green!77!red!23}3.6 & \cellcolor{green!96!red!4}84.0 & \cellcolor{green!90!red!10}90.0 & \cellcolor{green!61!red!39}-3.6 & \cellcolor{green!56!red!44}0.671 \\
\rowcolor{gray!30}
\official Mistral-Medium & \checkmark & \unknown & \cellcolor{green!76!red!24}3.8 & \cellcolor{green!93!red!7}83.6 & \cellcolor{green!79!red!21}88.2 & \cellcolor{green!61!red!39}-3.6 & \cellcolor{green!63!red!37}0.676 \\
\rowcolor{gray!30}
GemTrans & \checkmark & 27 & \cellcolor{green!68!red!32}4.7 & \cellcolor{green!58!red!42}78.7 & \cellcolor{green!59!red!41}84.8 & \cellcolor{green!94!red!6}-3.1 & \cellcolor{green!58!red!42}0.672 \\
\rowcolor{gray!30}
\official CommandA & \checkmark & 111 & \cellcolor{green!67!red!33}4.8 & \cellcolor{green!84!red!16}82.3 & \cellcolor{green!73!red!27}87.1 & \cellcolor{green!48!red!52}-3.8 & \cellcolor{green!58!red!42}0.672 \\
\rowcolor{gray!30}
\official ONLINE-B & \checkmark & \unknown & \cellcolor{green!62!red!38}5.4 & \cellcolor{green!51!red!49}77.7 & \cellcolor{green!51!red!49}83.4 & \cellcolor{green!81!red!19}-3.3 & \cellcolor{green!66!red!34}0.678 \\
\rowcolor{gray!30}
\official Claude-4 & \checkmark & \unknown & \cellcolor{green!61!red!39}5.5 & \cellcolor{green!77!red!23}81.4 & \cellcolor{green!71!red!29}86.9 & \cellcolor{green!42!red!58}-3.9 & \cellcolor{green!53!red!47}0.669 \\
\rowcolor{gray!30}
UvA-MT & \checkmark & 12 & \cellcolor{green!57!red!43}5.9 & \cellcolor{green!50!red!50}77.5 & \cellcolor{green!50!red!50}83.3 & \cellcolor{green!61!red!39}-3.6 & \cellcolor{green!67!red!33}0.679 \\
\rowcolor{gray!30}
\official Qwen3-235B & \checkmark & 235 & \cellcolor{green!55!red!45}6.2 & \cellcolor{green!68!red!32}80.0 & \cellcolor{green!63!red!37}85.4 & \cellcolor{green!55!red!45}-3.7 & \cellcolor{green!40!red!60}0.659 \\
\rowcolor{gray!30}
\official AyaExpanse-32B & \checkmark & 32 & \cellcolor{green!52!red!48}6.5 & \cellcolor{green!55!red!45}78.2 & \cellcolor{green!53!red!47}83.8 & \cellcolor{green!48!red!52}-3.8 & \cellcolor{green!53!red!47}0.669 \\
\rowcolor{gray!30}
\official Llama-4-Maverick & \checkmark & 400 & \cellcolor{green!48!red!52}7.0 & \cellcolor{green!61!red!39}79.0 & \cellcolor{green!51!red!49}83.5 & \cellcolor{green!42!red!58}-3.9 & \cellcolor{green!45!red!55}0.663 \\
\rowcolor{gray!30}
\official ONLINE-W & \unknown & \unknown & \cellcolor{green!39!red!61}8.0 & \cellcolor{green!42!red!58}76.4 & \cellcolor{green!36!red!64}80.9 & \cellcolor{green!42!red!58}-3.9 & \cellcolor{green!47!red!53}0.664 \\
\official TowerPlus-9B[M] & \checkmark & 9 & \cellcolor{green!38!red!62}8.2 & \cellcolor{green!39!red!61}76.0 & \cellcolor{green!30!red!70}80.0 & \cellcolor{green!42!red!58}-3.9 & \cellcolor{green!51!red!49}0.667 \\
\rowcolor{gray!30}
\official TowerPlus-72B[M] & \checkmark & 72 & \cellcolor{green!36!red!64}8.4 & \cellcolor{green!39!red!61}76.0 & \cellcolor{green!32!red!68}80.2 & \cellcolor{green!35!red!65}-4.0 & \cellcolor{green!48!red!52}0.665 \\
\rowcolor{gray!30}
TranssionTranslate & \unknown & \unknown & \cellcolor{green!33!red!67}8.7 & \cellcolor{green!22!red!78}73.5 & \cellcolor{green!19!red!81}78.1 & \cellcolor{green!74!red!26}-3.4 & \cellcolor{green!32!red!68}0.653 \\
\rowcolor{gray!30}
\official EuroLLM-22B-pre.[M] & \checkmark & 22 & \cellcolor{green!33!red!67}8.7 & \cellcolor{green!35!red!65}75.4 & \cellcolor{green!25!red!75}79.0 & \cellcolor{green!29!red!71}-4.1 & \cellcolor{green!53!red!47}0.669 \\
SalamandraTA & \checkmark & 8 & \cellcolor{green!24!red!76}9.8 & \cellcolor{green!14!red!86}72.4 & \cellcolor{green!3!red!97}75.4 & \cellcolor{green!48!red!52}-3.8 & \cellcolor{green!45!red!55}0.663 \\
IRB-MT & \checkmark & 12 & \cellcolor{green!24!red!76}9.8 & \cellcolor{green!31!red!69}74.8 & \cellcolor{green!25!red!75}79.0 & \cellcolor{green!55!red!45}-3.7 & \cellcolor{green!0!red!100}0.63 \\
\official Gemma-3-12B & \checkmark & 12 & \cellcolor{green!3!red!97}12.2 & \cellcolor{green!18!red!82}73.0 & \cellcolor{green!8!red!92}76.2 & \cellcolor{green!10!red!90}-4.4 & \cellcolor{green!4!red!96}0.633 \\
\official AyaExpanse-8B & \checkmark & 8 & \cellcolor{green!3!red!97}12.2 & \cellcolor{green!0!red!100}70.1 & \cellcolor{green!3!red!97}75.3 & \cellcolor{green!16!red!84}-4.3 & \cellcolor{green!19!red!81}0.644 \\
\official EuroLLM-9B[M] & \checkmark & 9 & \cellcolor{green!1!red!99}12.4 & \cellcolor{green!1!red!99}70.5 & \cellcolor{green!0!red!100}73.6 & \cellcolor{green!3!red!97}-4.5 & \cellcolor{green!33!red!67}0.654 \\
\rowcolor{gray!30}
IR-MultiagentMT & \crossmark & \unknown & \cellcolor{green!0!red!100}12.9 & \cellcolor{green!11!red!89}71.9 & \cellcolor{green!14!red!86}77.2 & \cellcolor{green!0!red!100}-4.7 & \cellcolor{green!0!red!100}0.63 \\
\official CommandR7B & \checkmark & 7 & \cellcolor{green!0!red!100}15.7 & \cellcolor{green!0!red!100}67.8 & \cellcolor{green!0!red!100}68.8 & \cellcolor{green!0!red!100}-4.8 & \cellcolor{green!0!red!100}0.628 \\
\rowcolor{gray!30}
\official Gemma-3-27B & \checkmark & 27 & \cellcolor{green!0!red!100}17.6 & \cellcolor{green!0!red!100}67.4 & \cellcolor{green!0!red!100}71.7 & \cellcolor{green!0!red!100}-5.1 & \cellcolor{green!0!red!100}0.589 \\
\rowcolor{gray!30}
\official ONLINE-G & \checkmark & \unknown & \cellcolor{green!0!red!100}17.9 & \cellcolor{green!0!red!100}66.5 & \cellcolor{green!0!red!100}67.7 & \cellcolor{green!0!red!100}-5.2 & \cellcolor{green!0!red!100}0.609 \\
\official Llama-3.1-8B & \checkmark & 8 & \cellcolor{green!0!red!100}20.9 & \cellcolor{green!0!red!100}64.3 & \cellcolor{green!0!red!100}62.6 & \cellcolor{green!0!red!100}-5.5 & \cellcolor{green!0!red!100}0.588 \\
\official Qwen2.5-7B & \checkmark & 7 & \cellcolor{green!0!red!100}23.1 & \cellcolor{green!0!red!100}60.0 & \cellcolor{green!0!red!100}59.1 & \cellcolor{green!0!red!100}-5.5 & \cellcolor{green!0!red!100}0.575 \\
\official NLLB & \checkmark & 1 & \cellcolor{green!0!red!100}26.1 & \cellcolor{green!0!red!100}58.1 & \cellcolor{green!0!red!100}59.3 & \cellcolor{green!0!red!100}-6.7 & \cellcolor{green!0!red!100}0.573 \\
\official Mistral-7B & \crossmark & 7 & \cellcolor{green!0!red!100}32.0 & \cellcolor{green!0!red!100}54.9 & \cellcolor{green!0!red!100}50.2 & \cellcolor{green!0!red!100}-7.0 & \cellcolor{green!0!red!100}0.51 \\
\bottomrule
\end{tabularx}
\end{table*}


\begin{table*}
\small
\begin{tabularx}{\textwidth}{lYYYYYYY}
\toprule
\multicolumn{8}{c}{\textbf{English-Greek}} \\
\midrule
System Name & LP Supported & Params. (B) & AutoRank $\downarrow$ & GEMBA-ESA-CMDA $\uparrow$ & GEMBA-ESA-GPT4.1 $\uparrow$ & MetricX-24-Hybrid-XL $\uparrow$ & XCOMET-XL $\uparrow$ \\
\midrule
Shy-hunyuan-MT & \checkmark & 7 & \cellcolor{green!100!red!0}1.0 & \cellcolor{green!85!red!15}80.3 & \cellcolor{green!90!red!10}85.8 & \cellcolor{green!100!red!0}-5.3 & \cellcolor{green!100!red!0}0.601 \\
\rowcolor{gray!30}
\official Gemini-2.5-Pro & \checkmark & \unknown & \cellcolor{green!92!red!8}1.9 & \cellcolor{green!100!red!0}84.3 & \cellcolor{green!100!red!0}88.7 & \cellcolor{green!80!red!20}-6.2 & \cellcolor{green!61!red!39}0.529 \\
\rowcolor{gray!30}
CommandA-WMT & \checkmark & 111 & \cellcolor{green!90!red!10}2.1 & \cellcolor{green!83!red!17}79.9 & \cellcolor{green!84!red!16}84.1 & \cellcolor{green!91!red!9}-5.7 & \cellcolor{green!78!red!22}0.56 \\
\rowcolor{gray!30}
\official GPT-4.1 & \checkmark & \unknown & \cellcolor{green!88!red!12}2.4 & \cellcolor{green!94!red!6}82.6 & \cellcolor{green!95!red!5}87.1 & \cellcolor{green!75!red!25}-6.4 & \cellcolor{green!60!red!40}0.528 \\
\rowcolor{gray!30}
GemTrans & \checkmark & 27 & \cellcolor{green!81!red!19}3.2 & \cellcolor{green!62!red!38}74.3 & \cellcolor{green!66!red!34}78.9 & \cellcolor{green!98!red!2}-5.4 & \cellcolor{green!68!red!32}0.543 \\
\rowcolor{gray!30}
UvA-MT & \unknown & 12 & \cellcolor{green!73!red!27}4.0 & \cellcolor{green!58!red!42}73.3 & \cellcolor{green!62!red!38}77.6 & \cellcolor{green!75!red!25}-6.4 & \cellcolor{green!70!red!30}0.545 \\
\rowcolor{gray!30}
\official CommandA & \checkmark & 111 & \cellcolor{green!71!red!29}4.3 & \cellcolor{green!74!red!26}77.4 & \cellcolor{green!73!red!27}80.7 & \cellcolor{green!60!red!40}-7.1 & \cellcolor{green!50!red!50}0.509 \\
\rowcolor{gray!30}
\official Claude-4 & \unknown & \unknown & \cellcolor{green!71!red!29}4.3 & \cellcolor{green!80!red!20}79.0 & \cellcolor{green!78!red!22}82.2 & \cellcolor{green!55!red!45}-7.3 & \cellcolor{green!43!red!57}0.496 \\
SalamandraTA & \checkmark & 8 & \cellcolor{green!67!red!33}4.7 & \cellcolor{green!50!red!50}71.0 & \cellcolor{green!55!red!45}75.4 & \cellcolor{green!78!red!22}-6.3 & \cellcolor{green!58!red!42}0.524 \\
\rowcolor{gray!30}
\official Mistral-Medium & \unknown & \unknown & \cellcolor{green!64!red!36}5.1 & \cellcolor{green!59!red!41}73.5 & \cellcolor{green!64!red!36}78.2 & \cellcolor{green!62!red!38}-7.0 & \cellcolor{green!44!red!56}0.498 \\
\rowcolor{gray!30}
\official ONLINE-B & \checkmark & \unknown & \cellcolor{green!60!red!40}5.5 & \cellcolor{green!50!red!50}71.2 & \cellcolor{green!53!red!47}75.0 & \cellcolor{green!69!red!31}-6.7 & \cellcolor{green!42!red!58}0.495 \\
\rowcolor{gray!30}
\official ONLINE-W & \unknown & \unknown & \cellcolor{green!60!red!40}5.5 & \cellcolor{green!61!red!39}74.0 & \cellcolor{green!61!red!39}77.3 & \cellcolor{green!53!red!47}-7.4 & \cellcolor{green!38!red!62}0.487 \\
\rowcolor{gray!30}
\official AyaExpanse-32B & \checkmark & 32 & \cellcolor{green!59!red!41}5.6 & \cellcolor{green!54!red!46}72.1 & \cellcolor{green!56!red!44}75.8 & \cellcolor{green!57!red!43}-7.2 & \cellcolor{green!42!red!58}0.494 \\
IRB-MT & \checkmark & 12 & \cellcolor{green!57!red!43}5.9 & \cellcolor{green!47!red!53}70.3 & \cellcolor{green!49!red!51}73.9 & \cellcolor{green!64!red!36}-6.9 & \cellcolor{green!38!red!62}0.486 \\
\rowcolor{gray!30}
\official DeepSeek-V3 & \unknown & 671 & \cellcolor{green!52!red!48}6.4 & \cellcolor{green!45!red!55}69.9 & \cellcolor{green!52!red!48}74.7 & \cellcolor{green!48!red!52}-7.6 & \cellcolor{green!34!red!66}0.48 \\
\rowcolor{gray!30}
\official EuroLLM-22B-pre.[M] & \checkmark & 22 & \cellcolor{green!50!red!50}6.6 & \cellcolor{green!43!red!57}69.3 & \cellcolor{green!43!red!57}72.1 & \cellcolor{green!53!red!47}-7.4 & \cellcolor{green!35!red!65}0.482 \\
\rowcolor{gray!30}
\official Llama-4-Maverick & \checkmark & 400 & \cellcolor{green!50!red!50}6.6 & \cellcolor{green!51!red!49}71.4 & \cellcolor{green!50!red!50}74.1 & \cellcolor{green!42!red!58}-7.9 & \cellcolor{green!29!red!71}0.471 \\
\rowcolor{gray!30}
TranssionTranslate & \unknown & \unknown & \cellcolor{green!50!red!50}6.7 & \cellcolor{green!37!red!63}67.7 & \cellcolor{green!41!red!59}71.5 & \cellcolor{green!64!red!36}-6.9 & \cellcolor{green!28!red!72}0.468 \\
\rowcolor{gray!30}
\official Qwen3-235B & \checkmark & 235 & \cellcolor{green!42!red!58}7.6 & \cellcolor{green!34!red!66}67.0 & \cellcolor{green!35!red!65}69.8 & \cellcolor{green!48!red!52}-7.6 & \cellcolor{green!21!red!79}0.455 \\
\official AyaExpanse-8B & \checkmark & 8 & \cellcolor{green!38!red!62}8.0 & \cellcolor{green!27!red!73}65.1 & \cellcolor{green!28!red!72}67.7 & \cellcolor{green!44!red!56}-7.8 & \cellcolor{green!23!red!77}0.46 \\
\official EuroLLM-9B[M] & \checkmark & 9 & \cellcolor{green!32!red!68}8.7 & \cellcolor{green!18!red!82}62.7 & \cellcolor{green!23!red!77}66.1 & \cellcolor{green!37!red!63}-8.1 & \cellcolor{green!20!red!80}0.454 \\
\rowcolor{gray!30}
IR-MultiagentMT & \crossmark & \unknown & \cellcolor{green!28!red!72}9.1 & \cellcolor{green!29!red!71}65.6 & \cellcolor{green!26!red!74}67.2 & \cellcolor{green!26!red!74}-8.6 & \cellcolor{green!1!red!99}0.419 \\
\official Gemma-3-12B & \checkmark & 12 & \cellcolor{green!21!red!79}9.9 & \cellcolor{green!10!red!90}60.6 & \cellcolor{green!12!red!88}62.9 & \cellcolor{green!19!red!81}-8.9 & \cellcolor{green!10!red!90}0.436 \\
\rowcolor{gray!30}
\official Gemma-3-27B & \checkmark & 27 & \cellcolor{green!3!red!97}12.0 & \cellcolor{green!0!red!100}54.9 & \cellcolor{green!0!red!100}56.9 & \cellcolor{green!1!red!99}-9.7 & \cellcolor{green!0!red!100}0.411 \\
\rowcolor{gray!30}
\official ONLINE-G & \checkmark & \unknown & \cellcolor{green!0!red!100}13.2 & \cellcolor{green!4!red!96}58.9 & \cellcolor{green!2!red!98}60.1 & \cellcolor{green!0!red!100}-10.9 & \cellcolor{green!0!red!100}0.333 \\
\official NLLB & \checkmark & 1 & \cellcolor{green!0!red!100}13.4 & \cellcolor{green!0!red!100}55.1 & \cellcolor{green!0!red!100}57.5 & \cellcolor{green!0!red!100}-11.1 & \cellcolor{green!0!red!100}0.373 \\
\official CommandR7B & \checkmark & 7 & \cellcolor{green!0!red!100}17.6 & \cellcolor{green!0!red!100}27.9 & \cellcolor{green!0!red!100}17.5 & \cellcolor{green!0!red!100}-9.9 & \cellcolor{green!38!red!62}0.487 \\
\official Llama-3.1-8B & \crossmark & 8 & \cellcolor{green!0!red!100}19.0 & \cellcolor{green!0!red!100}44.8 & \cellcolor{green!0!red!100}41.7 & \cellcolor{green!0!red!100}-13.2 & \cellcolor{green!0!red!100}0.254 \\
\rowcolor{gray!30}
\official TowerPlus-72B[M] & \crossmark & 72 & \cellcolor{green!0!red!100}22.4 & \cellcolor{green!0!red!100}36.5 & \cellcolor{green!0!red!100}33.6 & \cellcolor{green!0!red!100}-14.8 & \cellcolor{green!0!red!100}0.202 \\
\official TowerPlus-9B[M] & \crossmark & 9 & \cellcolor{green!0!red!100}26.5 & \cellcolor{green!0!red!100}26.8 & \cellcolor{green!0!red!100}22.9 & \cellcolor{green!0!red!100}-16.7 & \cellcolor{green!0!red!100}0.148 \\
\official Qwen2.5-7B & \unknown & 7 & \cellcolor{green!0!red!100}29.8 & \cellcolor{green!0!red!100}22.1 & \cellcolor{green!0!red!100}20.0 & \cellcolor{green!0!red!100}-20.0 & \cellcolor{green!0!red!100}0.109 \\
\official Mistral-7B & \crossmark & 7 & \cellcolor{green!0!red!100}32.0 & \cellcolor{green!0!red!100}19.2 & \cellcolor{green!0!red!100}14.3 & \cellcolor{green!0!red!100}-22.7 & \cellcolor{green!0!red!100}0.135 \\
\bottomrule
\end{tabularx}
\end{table*}


\begin{table*}
\small
\begin{tabularx}{\textwidth}{lYYYYYYY}
\toprule
\multicolumn{8}{c}{\textbf{English-Persian}} \\
\midrule
System Name & LP Supported & Params. (B) & AutoRank $\downarrow$ & GEMBA-ESA-CMDA $\uparrow$ & GEMBA-ESA-GPT4.1 $\uparrow$ & MetricX-24-Hybrid-XL $\uparrow$ & XCOMET-XL $\uparrow$ \\
\midrule
Shy-hunyuan-MT & \crossmark & 7 & \cellcolor{green!100!red!0}1.0 & \cellcolor{green!93!red!7}80.4 & \cellcolor{green!89!red!11}84.1 & \cellcolor{green!98!red!2}-4.6 & \cellcolor{green!100!red!0}0.553 \\
\rowcolor{gray!30}
\official Gemini-2.5-Pro & \checkmark & \unknown & \cellcolor{green!94!red!6}1.7 & \cellcolor{green!100!red!0}82.4 & \cellcolor{green!100!red!0}88.4 & \cellcolor{green!85!red!15}-5.2 & \cellcolor{green!73!red!27}0.476 \\
\rowcolor{gray!30}
\official GPT-4.1 & \checkmark & \unknown & \cellcolor{green!90!red!10}2.3 & \cellcolor{green!96!red!4}81.1 & \cellcolor{green!92!red!8}85.4 & \cellcolor{green!81!red!19}-5.4 & \cellcolor{green!71!red!29}0.47 \\
\rowcolor{gray!30}
CommandA-WMT & \checkmark & 111 & \cellcolor{green!87!red!13}2.6 & \cellcolor{green!83!red!17}77.4 & \cellcolor{green!79!red!21}80.4 & \cellcolor{green!89!red!11}-5.0 & \cellcolor{green!81!red!19}0.497 \\
\rowcolor{gray!30}
GemTrans & \checkmark & 27 & \cellcolor{green!85!red!15}2.9 & \cellcolor{green!72!red!28}74.1 & \cellcolor{green!70!red!30}77.0 & \cellcolor{green!100!red!0}-4.5 & \cellcolor{green!82!red!18}0.502 \\
\rowcolor{gray!30}
\official DeepSeek-V3 & \unknown & 671 & \cellcolor{green!82!red!18}3.3 & \cellcolor{green!85!red!15}78.1 & \cellcolor{green!80!red!20}80.6 & \cellcolor{green!79!red!21}-5.5 & \cellcolor{green!67!red!33}0.456 \\
\rowcolor{gray!30}
UvA-MT & \unknown & 12 & \cellcolor{green!79!red!21}3.7 & \cellcolor{green!69!red!31}73.2 & \cellcolor{green!68!red!32}76.2 & \cellcolor{green!83!red!17}-5.3 & \cellcolor{green!78!red!22}0.489 \\
\rowcolor{gray!30}
\official Gemma-3-27B & \checkmark & 27 & \cellcolor{green!78!red!22}3.8 & \cellcolor{green!78!red!22}75.8 & \cellcolor{green!75!red!25}79.0 & \cellcolor{green!77!red!23}-5.6 & \cellcolor{green!65!red!35}0.453 \\
\rowcolor{gray!30}
\official Mistral-Medium & \unknown & \unknown & \cellcolor{green!77!red!23}3.9 & \cellcolor{green!77!red!23}75.5 & \cellcolor{green!75!red!25}78.7 & \cellcolor{green!77!red!23}-5.6 & \cellcolor{green!65!red!35}0.453 \\
\rowcolor{gray!30}
\official Claude-4 & \unknown & \unknown & \cellcolor{green!73!red!27}4.4 & \cellcolor{green!83!red!17}77.5 & \cellcolor{green!77!red!23}79.6 & \cellcolor{green!62!red!38}-6.3 & \cellcolor{green!57!red!43}0.427 \\
\rowcolor{gray!30}
\official CommandA & \checkmark & 111 & \cellcolor{green!72!red!28}4.6 & \cellcolor{green!71!red!29}74.0 & \cellcolor{green!70!red!30}77.1 & \cellcolor{green!68!red!32}-6.0 & \cellcolor{green!61!red!39}0.439 \\
\rowcolor{gray!30}
\official ONLINE-B & \checkmark & \unknown & \cellcolor{green!70!red!30}4.8 & \cellcolor{green!60!red!40}70.7 & \cellcolor{green!58!red!42}72.3 & \cellcolor{green!81!red!19}-5.4 & \cellcolor{green!67!red!33}0.458 \\
IRB-MT & \checkmark & 12 & \cellcolor{green!68!red!32}5.1 & \cellcolor{green!64!red!36}71.7 & \cellcolor{green!60!red!40}73.1 & \cellcolor{green!77!red!23}-5.6 & \cellcolor{green!58!red!42}0.432 \\
\rowcolor{gray!30}
\official Llama-4-Maverick & \checkmark & 400 & \cellcolor{green!68!red!32}5.1 & \cellcolor{green!66!red!34}72.3 & \cellcolor{green!67!red!33}75.8 & \cellcolor{green!68!red!32}-6.0 & \cellcolor{green!56!red!44}0.425 \\
\rowcolor{gray!30}
TranssionTranslate & \unknown & \unknown & \cellcolor{green!64!red!36}5.6 & \cellcolor{green!53!red!47}68.5 & \cellcolor{green!50!red!50}69.3 & \cellcolor{green!79!red!21}-5.5 & \cellcolor{green!60!red!40}0.438 \\
\official Gemma-3-12B & \checkmark & 12 & \cellcolor{green!63!red!37}5.7 & \cellcolor{green!61!red!39}71.0 & \cellcolor{green!58!red!42}72.5 & \cellcolor{green!66!red!34}-6.1 & \cellcolor{green!53!red!47}0.417 \\
\rowcolor{gray!30}
\official AyaExpanse-32B & \checkmark & 32 & \cellcolor{green!63!red!37}5.7 & \cellcolor{green!59!red!41}70.4 & \cellcolor{green!58!red!42}72.3 & \cellcolor{green!66!red!34}-6.1 & \cellcolor{green!56!red!44}0.425 \\
\rowcolor{gray!30}
\official Qwen3-235B & \checkmark & 235 & \cellcolor{green!46!red!54}7.8 & \cellcolor{green!38!red!62}64.1 & \cellcolor{green!44!red!56}66.9 & \cellcolor{green!55!red!45}-6.6 & \cellcolor{green!40!red!60}0.378 \\
\rowcolor{gray!30}
IR-MultiagentMT & \crossmark & \unknown & \cellcolor{green!39!red!61}8.7 & \cellcolor{green!37!red!63}63.8 & \cellcolor{green!36!red!64}63.8 & \cellcolor{green!45!red!55}-7.1 & \cellcolor{green!33!red!67}0.359 \\
\official AyaExpanse-8B & \checkmark & 8 & \cellcolor{green!39!red!61}8.8 & \cellcolor{green!31!red!69}62.1 & \cellcolor{green!32!red!68}62.5 & \cellcolor{green!47!red!53}-7.0 & \cellcolor{green!36!red!64}0.369 \\
\official CommandR7B & \checkmark & 7 & \cellcolor{green!8!red!92}12.7 & \cellcolor{green!7!red!93}55.1 & \cellcolor{green!0!red!100}49.5 & \cellcolor{green!6!red!94}-8.9 & \cellcolor{green!17!red!83}0.312 \\
\rowcolor{gray!30}
\official ONLINE-G & \checkmark & \unknown & \cellcolor{green!2!red!98}13.4 & \cellcolor{green!5!red!95}54.6 & \cellcolor{green!8!red!92}53.2 & \cellcolor{green!0!red!100}-9.3 & \cellcolor{green!0!red!100}0.255 \\
\official NLLB & \checkmark & 1 & \cellcolor{green!0!red!100}13.8 & \cellcolor{green!0!red!100}52.5 & \cellcolor{green!6!red!94}52.4 & \cellcolor{green!0!red!100}-9.6 & \cellcolor{green!2!red!98}0.27 \\
\official Llama-3.1-8B & \crossmark & 8 & \cellcolor{green!0!red!100}13.8 & \cellcolor{green!0!red!100}51.5 & \cellcolor{green!0!red!100}49.2 & \cellcolor{green!6!red!94}-8.9 & \cellcolor{green!0!red!100}0.261 \\
\rowcolor{gray!30}
\official TowerPlus-72B[M] & \crossmark & 72 & \cellcolor{green!0!red!100}16.6 & \cellcolor{green!0!red!100}45.6 & \cellcolor{green!0!red!100}43.8 & \cellcolor{green!0!red!100}-10.3 & \cellcolor{green!0!red!100}0.203 \\
\official TowerPlus-9B[M] & \crossmark & 9 & \cellcolor{green!0!red!100}20.2 & \cellcolor{green!0!red!100}37.7 & \cellcolor{green!0!red!100}32.8 & \cellcolor{green!0!red!100}-12.0 & \cellcolor{green!0!red!100}0.16 \\
\official Qwen2.5-7B & \unknown & 7 & \cellcolor{green!0!red!100}21.6 & \cellcolor{green!0!red!100}32.4 & \cellcolor{green!0!red!100}32.0 & \cellcolor{green!0!red!100}-12.7 & \cellcolor{green!0!red!100}0.134 \\
\rowcolor{gray!30}
\official EuroLLM-22B-pre.[M] & \crossmark & 22 & \cellcolor{green!0!red!100}28.3 & \cellcolor{green!0!red!100}21.2 & \cellcolor{green!0!red!100}16.1 & \cellcolor{green!0!red!100}-18.8 & \cellcolor{green!0!red!100}0.165 \\
\official Mistral-7B & \crossmark & 7 & \cellcolor{green!0!red!100}28.6 & \cellcolor{green!0!red!100}21.9 & \cellcolor{green!0!red!100}17.6 & \cellcolor{green!0!red!100}-19.0 & \cellcolor{green!0!red!100}0.131 \\
\official EuroLLM-9B[M] & \crossmark & 9 & \cellcolor{green!0!red!100}30.0 & \cellcolor{green!0!red!100}14.5 & \cellcolor{green!0!red!100}9.8 & \cellcolor{green!0!red!100}-19.7 & \cellcolor{green!0!red!100}0.185 \\
\bottomrule
\end{tabularx}
\end{table*}


\begin{table*}
\small
\begin{tabularx}{\textwidth}{lYYYYYYY}
\toprule
\multicolumn{8}{c}{\textbf{English-Hindi}} \\
\midrule
System Name & LP Supported & Params. (B) & AutoRank $\downarrow$ & GEMBA-ESA-CMDA $\uparrow$ & GEMBA-ESA-GPT4.1 $\uparrow$ & MetricX-24-Hybrid-XL $\uparrow$ & XCOMET-XL $\uparrow$ \\
\midrule
Shy-hunyuan-MT & \checkmark & 7 & \cellcolor{green!100!red!0}1.0 & \cellcolor{green!93!red!7}77.0 & \cellcolor{green!84!red!16}82.3 & \cellcolor{green!100!red!0}-5.1 & \cellcolor{green!100!red!0}0.44 \\
\rowcolor{gray!30}
\official Gemini-2.5-Pro & \checkmark & \unknown & \cellcolor{green!91!red!9}1.9 & \cellcolor{green!100!red!0}78.3 & \cellcolor{green!100!red!0}86.3 & \cellcolor{green!77!red!23}-5.7 & \cellcolor{green!65!red!35}0.376 \\
\rowcolor{gray!30}
GemTrans & \checkmark & 27 & \cellcolor{green!84!red!16}2.6 & \cellcolor{green!69!red!31}72.4 & \cellcolor{green!69!red!31}78.4 & \cellcolor{green!96!red!4}-5.2 & \cellcolor{green!77!red!23}0.397 \\
\rowcolor{gray!30}
\official GPT-4.1 & \checkmark & \unknown & \cellcolor{green!83!red!17}2.7 & \cellcolor{green!86!red!14}75.7 & \cellcolor{green!93!red!7}84.5 & \cellcolor{green!69!red!31}-5.9 & \cellcolor{green!63!red!37}0.372 \\
\rowcolor{gray!30}
\official DeepSeek-V3 & \unknown & 671 & \cellcolor{green!80!red!20}3.0 & \cellcolor{green!89!red!11}76.2 & \cellcolor{green!85!red!15}82.4 & \cellcolor{green!69!red!31}-5.9 & \cellcolor{green!57!red!43}0.36 \\
\rowcolor{gray!30}
CommandA-WMT & \checkmark & 111 & \cellcolor{green!77!red!23}3.2 & \cellcolor{green!75!red!25}73.6 & \cellcolor{green!71!red!29}79.0 & \cellcolor{green!81!red!19}-5.6 & \cellcolor{green!65!red!35}0.375 \\
\rowcolor{gray!30}
UvA-MT & \unknown & 12 & \cellcolor{green!65!red!35}4.4 & \cellcolor{green!60!red!40}70.8 & \cellcolor{green!66!red!34}77.6 & \cellcolor{green!65!red!35}-6.0 & \cellcolor{green!54!red!46}0.355 \\
\rowcolor{gray!30}
\official Claude-4 & \checkmark & \unknown & \cellcolor{green!61!red!39}4.8 & \cellcolor{green!76!red!24}73.7 & \cellcolor{green!69!red!31}78.3 & \cellcolor{green!42!red!58}-6.6 & \cellcolor{green!43!red!57}0.334 \\
\rowcolor{gray!30}
\official Gemma-3-27B & \checkmark & 27 & \cellcolor{green!57!red!43}5.2 & \cellcolor{green!66!red!34}71.9 & \cellcolor{green!62!red!38}76.6 & \cellcolor{green!54!red!46}-6.3 & \cellcolor{green!35!red!65}0.319 \\
IRB-MT & \checkmark & 12 & \cellcolor{green!56!red!44}5.3 & \cellcolor{green!55!red!45}69.8 & \cellcolor{green!53!red!47}74.3 & \cellcolor{green!62!red!38}-6.1 & \cellcolor{green!40!red!60}0.33 \\
\rowcolor{gray!30}
\official ONLINE-B & \checkmark & \unknown & \cellcolor{green!53!red!47}5.6 & \cellcolor{green!46!red!54}68.0 & \cellcolor{green!54!red!46}74.5 & \cellcolor{green!58!red!42}-6.2 & \cellcolor{green!41!red!59}0.331 \\
\rowcolor{gray!30}
\official CommandA & \checkmark & 111 & \cellcolor{green!51!red!49}5.8 & \cellcolor{green!61!red!39}71.0 & \cellcolor{green!55!red!45}74.9 & \cellcolor{green!42!red!58}-6.6 & \cellcolor{green!32!red!68}0.314 \\
\rowcolor{gray!30}
TranssionTranslate & \unknown & \unknown & \cellcolor{green!44!red!56}6.5 & \cellcolor{green!28!red!72}64.7 & \cellcolor{green!37!red!63}70.3 & \cellcolor{green!62!red!38}-6.1 & \cellcolor{green!38!red!62}0.326 \\
\rowcolor{gray!30}
\official Llama-4-Maverick & \checkmark & 400 & \cellcolor{green!42!red!58}6.7 & \cellcolor{green!48!red!52}68.5 & \cellcolor{green!50!red!50}73.4 & \cellcolor{green!38!red!62}-6.7 & \cellcolor{green!22!red!78}0.296 \\
\rowcolor{gray!30}
\official Qwen3-235B & \checkmark & 235 & \cellcolor{green!41!red!59}6.8 & \cellcolor{green!44!red!56}67.8 & \cellcolor{green!44!red!56}72.1 & \cellcolor{green!42!red!58}-6.6 & \cellcolor{green!23!red!77}0.298 \\
\rowcolor{gray!30}
\official Mistral-Medium & \unknown & \unknown & \cellcolor{green!40!red!60}6.9 & \cellcolor{green!41!red!59}67.2 & \cellcolor{green!43!red!57}71.7 & \cellcolor{green!31!red!69}-6.9 & \cellcolor{green!36!red!64}0.322 \\
\official Gemma-3-12B & \checkmark & 12 & \cellcolor{green!38!red!62}7.1 & \cellcolor{green!39!red!61}66.8 & \cellcolor{green!37!red!63}70.1 & \cellcolor{green!35!red!65}-6.8 & \cellcolor{green!29!red!71}0.309 \\
\official TowerPlus-9B[M] & \checkmark & 9 & \cellcolor{green!34!red!66}7.5 & \cellcolor{green!41!red!59}67.1 & \cellcolor{green!39!red!61}70.8 & \cellcolor{green!27!red!73}-7.0 & \cellcolor{green!17!red!83}0.287 \\
\rowcolor{gray!30}
\official AyaExpanse-32B & \checkmark & 32 & \cellcolor{green!27!red!73}8.1 & \cellcolor{green!33!red!67}65.6 & \cellcolor{green!38!red!62}70.4 & \cellcolor{green!23!red!77}-7.1 & \cellcolor{green!8!red!92}0.27 \\
\rowcolor{gray!30}
\official TowerPlus-72B[M] & \checkmark & 72 & \cellcolor{green!15!red!85}9.3 & \cellcolor{green!21!red!79}63.3 & \cellcolor{green!22!red!78}66.3 & \cellcolor{green!12!red!88}-7.4 & \cellcolor{green!5!red!95}0.264 \\
\rowcolor{gray!30}
IR-MultiagentMT & \crossmark & \unknown & \cellcolor{green!7!red!93}10.1 & \cellcolor{green!13!red!87}61.9 & \cellcolor{green!8!red!92}62.9 & \cellcolor{green!4!red!96}-7.6 & \cellcolor{green!0!red!100}0.251 \\
\rowcolor{gray!30}
\official EuroLLM-22B-pre.[M] & \checkmark & 22 & \cellcolor{green!1!red!99}10.7 & \cellcolor{green!2!red!98}59.7 & \cellcolor{green!2!red!98}61.2 & \cellcolor{green!0!red!100}-7.7 & \cellcolor{green!2!red!98}0.259 \\
\official AyaExpanse-8B & \checkmark & 8 & \cellcolor{green!0!red!100}10.8 & \cellcolor{green!0!red!100}59.3 & \cellcolor{green!0!red!100}60.6 & \cellcolor{green!0!red!100}-7.7 & \cellcolor{green!0!red!100}0.254 \\
\official EuroLLM-9B[M] & \checkmark & 9 & \cellcolor{green!0!red!100}11.6 & \cellcolor{green!0!red!100}53.6 & \cellcolor{green!0!red!100}54.4 & \cellcolor{green!0!red!100}-7.8 & \cellcolor{green!24!red!76}0.3 \\
\official Llama-3.1-8B & \checkmark & 8 & \cellcolor{green!0!red!100}13.8 & \cellcolor{green!0!red!100}54.8 & \cellcolor{green!0!red!100}54.2 & \cellcolor{green!0!red!100}-8.6 & \cellcolor{green!0!red!100}0.195 \\
\official NLLB & \checkmark & 1 & \cellcolor{green!0!red!100}14.4 & \cellcolor{green!0!red!100}55.2 & \cellcolor{green!0!red!100}55.2 & \cellcolor{green!0!red!100}-9.4 & \cellcolor{green!0!red!100}0.199 \\
\rowcolor{gray!30}
\official ONLINE-G & \checkmark & \unknown & \cellcolor{green!0!red!100}15.4 & \cellcolor{green!0!red!100}54.5 & \cellcolor{green!0!red!100}51.8 & \cellcolor{green!0!red!100}-9.6 & \cellcolor{green!0!red!100}0.176 \\
\official CommandR7B & \checkmark & 7 & \cellcolor{green!0!red!100}15.9 & \cellcolor{green!0!red!100}49.6 & \cellcolor{green!0!red!100}49.6 & \cellcolor{green!0!red!100}-9.3 & \cellcolor{green!0!red!100}0.18 \\
\official Qwen2.5-7B & \unknown & 7 & \cellcolor{green!0!red!100}24.8 & \cellcolor{green!0!red!100}30.0 & \cellcolor{green!0!red!100}32.5 & \cellcolor{green!0!red!100}-12.8 & \cellcolor{green!0!red!100}0.107 \\
\official Mistral-7B & \crossmark & 7 & \cellcolor{green!0!red!100}30.0 & \cellcolor{green!0!red!100}25.0 & \cellcolor{green!0!red!100}23.2 & \cellcolor{green!0!red!100}-16.6 & \cellcolor{green!0!red!100}0.126 \\
\bottomrule
\end{tabularx}
\end{table*}


\begin{table*}
\small
\begin{tabularx}{\textwidth}{lYYYYYYY}
\toprule
\multicolumn{8}{c}{\textbf{English-Indonesian}} \\
\midrule
System Name & LP Supported & Params. (B) & AutoRank $\downarrow$ & GEMBA-ESA-CMDA $\uparrow$ & GEMBA-ESA-GPT4.1 $\uparrow$ & MetricX-24-Hybrid-XL $\uparrow$ & XCOMET-XL $\uparrow$ \\
\midrule
Shy-hunyuan-MT & \checkmark & 7 & \cellcolor{green!100!red!0}1.0 & \cellcolor{green!100!red!0}83.2 & \cellcolor{green!91!red!9}87.1 & \cellcolor{green!100!red!0}-4.4 & \cellcolor{green!100!red!0}0.677 \\
\rowcolor{gray!30}
\official Gemini-2.5-Pro & \checkmark & \unknown & \cellcolor{green!85!red!15}2.8 & \cellcolor{green!99!red!1}83.0 & \cellcolor{green!100!red!0}89.3 & \cellcolor{green!72!red!28}-5.6 & \cellcolor{green!62!red!38}0.576 \\
\rowcolor{gray!30}
\official GPT-4.1 & \checkmark & \unknown & \cellcolor{green!79!red!21}3.6 & \cellcolor{green!92!red!8}81.6 & \cellcolor{green!94!red!6}87.9 & \cellcolor{green!65!red!35}-5.9 & \cellcolor{green!57!red!43}0.564 \\
\rowcolor{gray!30}
GemTrans & \checkmark & 27 & \cellcolor{green!78!red!22}3.7 & \cellcolor{green!67!red!33}76.4 & \cellcolor{green!66!red!34}80.8 & \cellcolor{green!93!red!7}-4.7 & \cellcolor{green!79!red!21}0.622 \\
\rowcolor{gray!30}
CommandA-WMT & \checkmark & 111 & \cellcolor{green!76!red!24}4.0 & \cellcolor{green!75!red!25}78.2 & \cellcolor{green!78!red!22}83.7 & \cellcolor{green!74!red!26}-5.5 & \cellcolor{green!68!red!32}0.592 \\
\rowcolor{gray!30}
\official DeepSeek-V3 & \unknown & 671 & \cellcolor{green!75!red!25}4.1 & \cellcolor{green!90!red!10}81.2 & \cellcolor{green!83!red!17}85.1 & \cellcolor{green!65!red!35}-5.9 & \cellcolor{green!55!red!45}0.558 \\
\rowcolor{gray!30}
\official Qwen3-235B & \checkmark & 235 & \cellcolor{green!73!red!27}4.3 & \cellcolor{green!83!red!17}79.8 & \cellcolor{green!80!red!20}84.2 & \cellcolor{green!62!red!38}-6.0 & \cellcolor{green!58!red!42}0.566 \\
\rowcolor{gray!30}
UvA-MT & \unknown & 12 & \cellcolor{green!72!red!28}4.4 & \cellcolor{green!75!red!25}78.0 & \cellcolor{green!76!red!24}83.2 & \cellcolor{green!65!red!35}-5.9 & \cellcolor{green!65!red!35}0.584 \\
\rowcolor{gray!30}
\official Mistral-Medium & \unknown & \unknown & \cellcolor{green!67!red!33}5.1 & \cellcolor{green!75!red!25}78.2 & \cellcolor{green!77!red!23}83.6 & \cellcolor{green!55!red!45}-6.3 & \cellcolor{green!51!red!49}0.549 \\
\rowcolor{gray!30}
\official Gemma-3-27B & \checkmark & 27 & \cellcolor{green!64!red!36}5.4 & \cellcolor{green!76!red!24}78.3 & \cellcolor{green!75!red!25}83.1 & \cellcolor{green!53!red!47}-6.4 & \cellcolor{green!44!red!56}0.531 \\
IRB-MT & \checkmark & 12 & \cellcolor{green!63!red!37}5.5 & \cellcolor{green!64!red!36}75.8 & \cellcolor{green!65!red!35}80.6 & \cellcolor{green!67!red!33}-5.8 & \cellcolor{green!51!red!49}0.548 \\
\rowcolor{gray!30}
\official Claude-4 & \checkmark & \unknown & \cellcolor{green!60!red!40}5.9 & \cellcolor{green!78!red!22}78.8 & \cellcolor{green!74!red!26}82.8 & \cellcolor{green!41!red!59}-6.9 & \cellcolor{green!38!red!62}0.514 \\
\official Gemma-3-12B & \checkmark & 12 & \cellcolor{green!54!red!46}6.6 & \cellcolor{green!61!red!39}75.3 & \cellcolor{green!67!red!33}81.1 & \cellcolor{green!44!red!56}-6.8 & \cellcolor{green!38!red!62}0.515 \\
\rowcolor{gray!30}
\official ONLINE-B & \checkmark & \unknown & \cellcolor{green!51!red!49}7.0 & \cellcolor{green!49!red!51}72.7 & \cellcolor{green!50!red!50}76.7 & \cellcolor{green!55!red!45}-6.3 & \cellcolor{green!43!red!57}0.528 \\
\rowcolor{gray!30}
\official Llama-4-Maverick & \checkmark & 400 & \cellcolor{green!49!red!51}7.3 & \cellcolor{green!55!red!45}74.0 & \cellcolor{green!57!red!43}78.5 & \cellcolor{green!41!red!59}-6.9 & \cellcolor{green!35!red!65}0.507 \\
\rowcolor{gray!30}
\official CommandA & \checkmark & 111 & \cellcolor{green!47!red!53}7.5 & \cellcolor{green!59!red!41}74.9 & \cellcolor{green!54!red!46}77.7 & \cellcolor{green!36!red!64}-7.1 & \cellcolor{green!32!red!68}0.498 \\
\rowcolor{gray!30}
\official AyaExpanse-32B & \checkmark & 32 & \cellcolor{green!44!red!56}7.8 & \cellcolor{green!49!red!51}72.7 & \cellcolor{green!51!red!49}76.9 & \cellcolor{green!41!red!59}-6.9 & \cellcolor{green!33!red!67}0.5 \\
\rowcolor{gray!30}
\official ONLINE-W & \unknown & \unknown & \cellcolor{green!41!red!59}8.2 & \cellcolor{green!34!red!66}69.8 & \cellcolor{green!39!red!61}73.9 & \cellcolor{green!46!red!54}-6.7 & \cellcolor{green!41!red!59}0.522 \\
\rowcolor{gray!30}
TranssionTranslate & \unknown & \unknown & \cellcolor{green!39!red!61}8.5 & \cellcolor{green!29!red!71}68.7 & \cellcolor{green!34!red!66}72.7 & \cellcolor{green!55!red!45}-6.3 & \cellcolor{green!32!red!68}0.498 \\
\rowcolor{gray!30}
\official TowerPlus-72B[M] & \crossmark & 72 & \cellcolor{green!34!red!66}9.1 & \cellcolor{green!36!red!64}70.2 & \cellcolor{green!42!red!58}74.6 & \cellcolor{green!27!red!73}-7.5 & \cellcolor{green!25!red!75}0.479 \\
\official AyaExpanse-8B & \checkmark & 8 & \cellcolor{green!32!red!68}9.3 & \cellcolor{green!28!red!72}68.6 & \cellcolor{green!32!red!68}72.1 & \cellcolor{green!34!red!66}-7.2 & \cellcolor{green!28!red!72}0.487 \\
\rowcolor{gray!30}
IR-MultiagentMT & \crossmark & \unknown & \cellcolor{green!28!red!72}9.8 & \cellcolor{green!27!red!73}68.4 & \cellcolor{green!34!red!66}72.6 & \cellcolor{green!27!red!73}-7.5 & \cellcolor{green!19!red!81}0.464 \\
\rowcolor{gray!30}
\official ONLINE-G & \checkmark & \unknown & \cellcolor{green!3!red!97}12.9 & \cellcolor{green!3!red!97}63.4 & \cellcolor{green!7!red!93}65.9 & \cellcolor{green!0!red!100}-8.7 & \cellcolor{green!0!red!100}0.409 \\
\official Llama-3.1-8B & \crossmark & 8 & \cellcolor{green!0!red!100}13.6 & \cellcolor{green!0!red!100}62.2 & \cellcolor{green!0!red!100}62.4 & \cellcolor{green!0!red!100}-9.0 & \cellcolor{green!1!red!99}0.417 \\
\official Qwen2.5-7B & \unknown & 7 & \cellcolor{green!0!red!100}13.8 & \cellcolor{green!0!red!100}60.2 & \cellcolor{green!0!red!100}61.6 & \cellcolor{green!1!red!99}-8.6 & \cellcolor{green!0!red!100}0.412 \\
\official CommandR7B & \checkmark & 7 & \cellcolor{green!0!red!100}16.6 & \cellcolor{green!0!red!100}57.6 & \cellcolor{green!0!red!100}53.0 & \cellcolor{green!0!red!100}-10.2 & \cellcolor{green!0!red!100}0.392 \\
\official NLLB & \checkmark & 1 & \cellcolor{green!0!red!100}17.3 & \cellcolor{green!0!red!100}57.3 & \cellcolor{green!0!red!100}57.7 & \cellcolor{green!0!red!100}-10.9 & \cellcolor{green!0!red!100}0.333 \\
\official TowerPlus-9B[M] & \crossmark & 9 & \cellcolor{green!0!red!100}18.7 & \cellcolor{green!0!red!100}52.0 & \cellcolor{green!0!red!100}50.2 & \cellcolor{green!0!red!100}-10.6 & \cellcolor{green!0!red!100}0.339 \\
\rowcolor{gray!30}
\official EuroLLM-22B-pre.[M] & \crossmark & 22 & \cellcolor{green!0!red!100}25.5 & \cellcolor{green!0!red!100}40.6 & \cellcolor{green!0!red!100}39.4 & \cellcolor{green!0!red!100}-13.7 & \cellcolor{green!0!red!100}0.214 \\
\official Mistral-7B & \crossmark & 7 & \cellcolor{green!0!red!100}25.6 & \cellcolor{green!0!red!100}43.1 & \cellcolor{green!0!red!100}40.1 & \cellcolor{green!0!red!100}-14.2 & \cellcolor{green!0!red!100}0.197 \\
\official EuroLLM-9B[M] & \crossmark & 9 & \cellcolor{green!0!red!100}31.0 & \cellcolor{green!0!red!100}26.4 & \cellcolor{green!0!red!100}20.2 & \cellcolor{green!0!red!100}-16.0 & \cellcolor{green!0!red!100}0.275 \\
\bottomrule
\end{tabularx}
\end{table*}


\begin{table*}
\small
\begin{tabularx}{\textwidth}{lYYYYYYY}
\toprule
\multicolumn{8}{c}{\textbf{English-Kannada}} \\
\midrule
System Name & LP Supported & Params. (B) & AutoRank $\downarrow$ & GEMBA-ESA-CMDA $\uparrow$ & GEMBA-ESA-GPT4.1 $\uparrow$ & MetricX-24-Hybrid-XL $\uparrow$ & XCOMET-XL $\uparrow$ \\
\midrule
Shy-hunyuan-MT & \crossmark & 7 & \cellcolor{green!100!red!0}1.0 & \cellcolor{green!100!red!0}64.0 & \cellcolor{green!96!red!4}78.8 & \cellcolor{green!100!red!0}-6.0 & \cellcolor{green!100!red!0}0.446 \\
\rowcolor{gray!30}
\official Gemini-2.5-Pro & \checkmark & \unknown & \cellcolor{green!94!red!6}2.2 & \cellcolor{green!95!red!5}62.5 & \cellcolor{green!100!red!0}81.6 & \cellcolor{green!97!red!3}-6.3 & \cellcolor{green!82!red!18}0.399 \\
\rowcolor{gray!30}
\official Claude-4 & \unknown & \unknown & \cellcolor{green!82!red!18}5.0 & \cellcolor{green!91!red!9}61.3 & \cellcolor{green!91!red!9}76.1 & \cellcolor{green!84!red!16}-7.6 & \cellcolor{green!57!red!43}0.333 \\
\rowcolor{gray!30}
GemTrans & \checkmark & 27 & \cellcolor{green!81!red!19}5.2 & \cellcolor{green!83!red!17}58.7 & \cellcolor{green!77!red!23}67.3 & \cellcolor{green!93!red!7}-6.7 & \cellcolor{green!66!red!34}0.358 \\
\rowcolor{gray!30}
\official GPT-4.1 & \checkmark & \unknown & \cellcolor{green!78!red!22}5.9 & \cellcolor{green!88!red!12}60.2 & \cellcolor{green!84!red!16}71.3 & \cellcolor{green!81!red!19}-7.9 & \cellcolor{green!54!red!46}0.327 \\
\rowcolor{gray!30}
\official Mistral-Medium & \unknown & \unknown & \cellcolor{green!75!red!25}6.5 & \cellcolor{green!86!red!14}59.8 & \cellcolor{green!80!red!20}69.2 & \cellcolor{green!80!red!20}-8.0 & \cellcolor{green!49!red!51}0.312 \\
\rowcolor{gray!30}
\official Qwen3-235B & \checkmark & 235 & \cellcolor{green!74!red!26}6.7 & \cellcolor{green!87!red!13}60.1 & \cellcolor{green!77!red!23}67.4 & \cellcolor{green!81!red!19}-7.9 & \cellcolor{green!46!red!54}0.305 \\
\rowcolor{gray!30}
\official DeepSeek-V3 & \unknown & 671 & \cellcolor{green!74!red!26}6.7 & \cellcolor{green!78!red!22}57.3 & \cellcolor{green!81!red!19}69.9 & \cellcolor{green!76!red!24}-8.3 & \cellcolor{green!54!red!46}0.325 \\
\rowcolor{gray!30}
CommandA-WMT & \crossmark & 111 & \cellcolor{green!70!red!30}7.5 & \cellcolor{green!86!red!14}59.7 & \cellcolor{green!73!red!27}64.5 & \cellcolor{green!75!red!25}-8.4 & \cellcolor{green!42!red!58}0.295 \\
\rowcolor{gray!30}
\official ONLINE-B & \checkmark & \unknown & \cellcolor{green!69!red!31}7.7 & \cellcolor{green!77!red!23}57.0 & \cellcolor{green!75!red!25}66.0 & \cellcolor{green!81!red!19}-7.9 & \cellcolor{green!40!red!60}0.289 \\
\rowcolor{gray!30}
\official Gemma-3-27B & \checkmark & 27 & \cellcolor{green!69!red!31}7.8 & \cellcolor{green!78!red!22}57.2 & \cellcolor{green!75!red!25}66.1 & \cellcolor{green!76!red!24}-8.3 & \cellcolor{green!42!red!58}0.294 \\
\rowcolor{gray!30}
TranssionTranslate & \unknown & \unknown & \cellcolor{green!67!red!33}8.1 & \cellcolor{green!74!red!26}56.1 & \cellcolor{green!71!red!29}63.3 & \cellcolor{green!82!red!18}-7.8 & \cellcolor{green!39!red!61}0.286 \\
\rowcolor{gray!30}
\official Llama-4-Maverick & \checkmark & 400 & \cellcolor{green!67!red!33}8.1 & \cellcolor{green!81!red!19}58.1 & \cellcolor{green!76!red!24}66.5 & \cellcolor{green!75!red!25}-8.4 & \cellcolor{green!33!red!67}0.27 \\
\rowcolor{gray!30}
UvA-MT & \unknown & 12 & \cellcolor{green!64!red!36}8.8 & \cellcolor{green!65!red!35}53.3 & \cellcolor{green!66!red!34}60.5 & \cellcolor{green!73!red!27}-8.6 & \cellcolor{green!47!red!53}0.308 \\
IRB-MT & \checkmark & 12 & \cellcolor{green!54!red!46}11.0 & \cellcolor{green!62!red!38}52.4 & \cellcolor{green!62!red!38}57.6 & \cellcolor{green!66!red!34}-9.3 & \cellcolor{green!21!red!79}0.239 \\
\official NLLB & \checkmark & 1 & \cellcolor{green!49!red!51}12.1 & \cellcolor{green!62!red!38}52.3 & \cellcolor{green!56!red!44}54.2 & \cellcolor{green!61!red!39}-9.8 & \cellcolor{green!12!red!88}0.215 \\
\rowcolor{gray!30}
\official ONLINE-G & \checkmark & \unknown & \cellcolor{green!44!red!56}13.3 & \cellcolor{green!63!red!37}52.5 & \cellcolor{green!53!red!47}51.9 & \cellcolor{green!54!red!46}-10.5 & \cellcolor{green!1!red!99}0.186 \\
\official Gemma-3-12B & \checkmark & 12 & \cellcolor{green!43!red!57}13.4 & \cellcolor{green!42!red!58}46.1 & \cellcolor{green!49!red!51}49.4 & \cellcolor{green!55!red!45}-10.4 & \cellcolor{green!23!red!77}0.244 \\
\rowcolor{gray!30}
\official CommandA & \crossmark & 111 & \cellcolor{green!39!red!61}14.2 & \cellcolor{green!69!red!31}54.3 & \cellcolor{green!47!red!53}48.1 & \cellcolor{green!43!red!57}-11.6 & \cellcolor{green!0!red!100}0.175 \\
\official TowerPlus-9B[M] & \crossmark & 9 & \cellcolor{green!22!red!78}18.1 & \cellcolor{green!0!red!100}32.8 & \cellcolor{green!0!red!100}2.7 & \cellcolor{green!71!red!29}-8.8 & \cellcolor{green!37!red!63}0.281 \\
\official Llama-3.1-8B & \crossmark & 8 & \cellcolor{green!17!red!83}19.1 & \cellcolor{green!35!red!65}44.1 & \cellcolor{green!27!red!73}35.8 & \cellcolor{green!24!red!76}-13.4 & \cellcolor{green!0!red!100}0.12 \\
\rowcolor{gray!30}
IR-MultiagentMT & \crossmark & \unknown & \cellcolor{green!15!red!85}19.5 & \cellcolor{green!22!red!78}40.1 & \cellcolor{green!28!red!72}36.4 & \cellcolor{green!19!red!81}-13.9 & \cellcolor{green!0!red!100}0.149 \\
\rowcolor{gray!30}
\official AyaExpanse-32B & \crossmark & 32 & \cellcolor{green!0!red!100}23.9 & \cellcolor{green!4!red!96}34.3 & \cellcolor{green!11!red!89}25.5 & \cellcolor{green!0!red!100}-18.6 & \cellcolor{green!0!red!100}0.157 \\
\official CommandR7B & \crossmark & 7 & \cellcolor{green!0!red!100}25.6 & \cellcolor{green!0!red!100}17.1 & \cellcolor{green!0!red!100}10.6 & \cellcolor{green!0!red!100}-16.4 & \cellcolor{green!14!red!86}0.222 \\
\rowcolor{gray!30}
\official TowerPlus-72B[M] & \crossmark & 72 & \cellcolor{green!0!red!100}25.8 & \cellcolor{green!0!red!100}22.4 & \cellcolor{green!0!red!100}16.5 & \cellcolor{green!0!red!100}-18.3 & \cellcolor{green!5!red!95}0.197 \\
\official AyaExpanse-8B & \crossmark & 8 & \cellcolor{green!0!red!100}28.1 & \cellcolor{green!0!red!100}19.8 & \cellcolor{green!0!red!100}14.3 & \cellcolor{green!0!red!100}-20.5 & \cellcolor{green!0!red!100}0.174 \\
\official EuroLLM-9B[M] & \crossmark & 9 & \cellcolor{green!0!red!100}28.2 & \cellcolor{green!0!red!100}13.1 & \cellcolor{green!0!red!100}2.9 & \cellcolor{green!0!red!100}-16.9 & \cellcolor{green!0!red!100}0.179 \\
\rowcolor{gray!30}
\official EuroLLM-22B-pre.[M] & \crossmark & 22 & \cellcolor{green!0!red!100}28.7 & \cellcolor{green!0!red!100}12.8 & \cellcolor{green!0!red!100}5.0 & \cellcolor{green!0!red!100}-18.3 & \cellcolor{green!0!red!100}0.184 \\
\official Mistral-7B & \crossmark & 7 & \cellcolor{green!0!red!100}29.9 & \cellcolor{green!0!red!100}7.4 & \cellcolor{green!0!red!100}4.6 & \cellcolor{green!0!red!100}-19.2 & \cellcolor{green!6!red!94}0.199 \\
\official Qwen2.5-7B & \unknown & 7 & \cellcolor{green!0!red!100}30.0 & \cellcolor{green!0!red!100}15.1 & \cellcolor{green!0!red!100}10.2 & \cellcolor{green!0!red!100}-21.4 & \cellcolor{green!0!red!100}0.163 \\
\bottomrule
\end{tabularx}
\end{table*}


\begin{table*}
\small
\begin{tabularx}{\textwidth}{lYYYYYYY}
\toprule
\multicolumn{8}{c}{\textbf{English-Lithuanian}} \\
\midrule
System Name & LP Supported & Params. (B) & AutoRank $\downarrow$ & GEMBA-ESA-CMDA $\uparrow$ & GEMBA-ESA-GPT4.1 $\uparrow$ & MetricX-24-Hybrid-XL $\uparrow$ & XCOMET-XL $\uparrow$ \\
\midrule
Shy-hunyuan-MT & \checkmark & 7 & \cellcolor{green!100!red!0}1.0 & \cellcolor{green!100!red!0}77.6 & \cellcolor{green!92!red!8}84.1 & \cellcolor{green!100!red!0}-6.3 & \cellcolor{green!100!red!0}0.569 \\
\rowcolor{gray!30}
\official Gemini-2.5-Pro & \checkmark & \unknown & \cellcolor{green!92!red!8}2.3 & \cellcolor{green!94!red!6}76.1 & \cellcolor{green!100!red!0}87.3 & \cellcolor{green!90!red!10}-7.2 & \cellcolor{green!78!red!22}0.502 \\
\rowcolor{gray!30}
\official GPT-4.1 & \checkmark & \unknown & \cellcolor{green!88!red!12}2.9 & \cellcolor{green!91!red!9}75.3 & \cellcolor{green!94!red!6}84.8 & \cellcolor{green!85!red!15}-7.6 & \cellcolor{green!77!red!23}0.5 \\
\rowcolor{gray!30}
CommandA-WMT & \crossmark & 111 & \cellcolor{green!79!red!21}4.5 & \cellcolor{green!81!red!19}72.6 & \cellcolor{green!65!red!35}72.4 & \cellcolor{green!83!red!17}-7.8 & \cellcolor{green!79!red!21}0.506 \\
\rowcolor{gray!30}
GemTrans & \checkmark & 27 & \cellcolor{green!79!red!21}4.5 & \cellcolor{green!72!red!28}70.2 & \cellcolor{green!63!red!37}71.7 & \cellcolor{green!94!red!6}-6.8 & \cellcolor{green!79!red!21}0.505 \\
\rowcolor{gray!30}
\official ONLINE-B & \checkmark & \unknown & \cellcolor{green!73!red!27}5.4 & \cellcolor{green!68!red!32}69.1 & \cellcolor{green!61!red!39}70.9 & \cellcolor{green!84!red!16}-7.7 & \cellcolor{green!73!red!27}0.487 \\
\rowcolor{gray!30}
\official Claude-4 & \unknown & \unknown & \cellcolor{green!71!red!29}5.8 & \cellcolor{green!78!red!22}71.8 & \cellcolor{green!72!red!28}75.6 & \cellcolor{green!66!red!34}-9.3 & \cellcolor{green!62!red!38}0.455 \\
SalamandraTA & \checkmark & 8 & \cellcolor{green!69!red!31}6.1 & \cellcolor{green!59!red!41}66.9 & \cellcolor{green!52!red!48}67.0 & \cellcolor{green!82!red!18}-7.9 & \cellcolor{green!76!red!24}0.496 \\
\rowcolor{gray!30}
\official ONLINE-W & \unknown & \unknown & \cellcolor{green!65!red!35}6.7 & \cellcolor{green!62!red!38}67.6 & \cellcolor{green!58!red!42}69.5 & \cellcolor{green!68!red!32}-9.1 & \cellcolor{green!66!red!34}0.467 \\
\rowcolor{gray!30}
TranssionTranslate & \unknown & \unknown & \cellcolor{green!65!red!35}6.7 & \cellcolor{green!57!red!43}66.3 & \cellcolor{green!53!red!47}67.6 & \cellcolor{green!82!red!18}-7.9 & \cellcolor{green!62!red!38}0.454 \\
\rowcolor{gray!30}
\official Gemma-3-27B & \checkmark & 27 & \cellcolor{green!64!red!36}6.9 & \cellcolor{green!70!red!30}69.8 & \cellcolor{green!56!red!44}68.8 & \cellcolor{green!69!red!31}-9.0 & \cellcolor{green!55!red!45}0.434 \\
\rowcolor{gray!30}
\official Llama-4-Maverick & \checkmark & 400 & \cellcolor{green!64!red!36}6.9 & \cellcolor{green!68!red!32}69.1 & \cellcolor{green!63!red!37}71.5 & \cellcolor{green!66!red!34}-9.3 & \cellcolor{green!53!red!47}0.43 \\
\rowcolor{gray!30}
UvA-MT & \unknown & 12 & \cellcolor{green!63!red!37}7.1 & \cellcolor{green!65!red!35}68.5 & \cellcolor{green!44!red!56}63.7 & \cellcolor{green!69!red!31}-9.0 & \cellcolor{green!68!red!32}0.472 \\
\rowcolor{gray!30}
\official Qwen3-235B & \checkmark & 235 & \cellcolor{green!59!red!41}7.8 & \cellcolor{green!63!red!37}67.8 & \cellcolor{green!50!red!50}66.1 & \cellcolor{green!68!red!32}-9.1 & \cellcolor{green!48!red!52}0.414 \\
\rowcolor{gray!30}
\official EuroLLM-22B-pre.[M] & \checkmark & 22 & \cellcolor{green!56!red!44}8.3 & \cellcolor{green!51!red!49}64.7 & \cellcolor{green!50!red!50}66.1 & \cellcolor{green!61!red!39}-9.7 & \cellcolor{green!55!red!45}0.434 \\
IRB-MT & \checkmark & 12 & \cellcolor{green!52!red!48}8.9 & \cellcolor{green!56!red!44}66.0 & \cellcolor{green!38!red!62}61.2 & \cellcolor{green!64!red!36}-9.5 & \cellcolor{green!44!red!56}0.402 \\
\official EuroLLM-9B[M] & \checkmark & 9 & \cellcolor{green!50!red!50}9.3 & \cellcolor{green!37!red!63}61.0 & \cellcolor{green!29!red!71}57.5 & \cellcolor{green!64!red!36}-9.5 & \cellcolor{green!62!red!38}0.455 \\
\rowcolor{gray!30}
\official DeepSeek-V3 & \unknown & 671 & \cellcolor{green!47!red!53}9.7 & \cellcolor{green!35!red!65}60.4 & \cellcolor{green!37!red!63}60.8 & \cellcolor{green!60!red!40}-9.8 & \cellcolor{green!49!red!51}0.418 \\
\official Gemma-3-12B & \checkmark & 12 & \cellcolor{green!44!red!56}10.2 & \cellcolor{green!56!red!44}66.0 & \cellcolor{green!31!red!69}58.3 & \cellcolor{green!50!red!50}-10.7 & \cellcolor{green!33!red!67}0.368 \\
\rowcolor{gray!30}
IR-MultiagentMT & \crossmark & \unknown & \cellcolor{green!42!red!58}10.5 & \cellcolor{green!45!red!55}63.2 & \cellcolor{green!35!red!65}59.8 & \cellcolor{green!48!red!52}-10.9 & \cellcolor{green!35!red!65}0.374 \\
\rowcolor{gray!30}
\official Mistral-Medium & \unknown & \unknown & \cellcolor{green!41!red!59}10.8 & \cellcolor{green!50!red!50}64.5 & \cellcolor{green!27!red!73}56.7 & \cellcolor{green!48!red!52}-10.9 & \cellcolor{green!31!red!69}0.362 \\
\rowcolor{gray!30}
\official CommandA & \crossmark & 111 & \cellcolor{green!39!red!61}11.0 & \cellcolor{green!53!red!47}65.1 & \cellcolor{green!25!red!75}55.7 & \cellcolor{green!42!red!58}-11.4 & \cellcolor{green!30!red!70}0.361 \\
\rowcolor{gray!30}
\official ONLINE-G & \checkmark & \unknown & \cellcolor{green!12!red!88}15.5 & \cellcolor{green!17!red!83}55.7 & \cellcolor{green!17!red!83}52.4 & \cellcolor{green!10!red!90}-14.2 & \cellcolor{green!0!red!100}0.259 \\
\official NLLB & \checkmark & 1 & \cellcolor{green!10!red!90}15.9 & \cellcolor{green!7!red!93}53.0 & \cellcolor{green!10!red!90}49.3 & \cellcolor{green!10!red!90}-14.2 & \cellcolor{green!4!red!96}0.283 \\
\rowcolor{gray!30}
\official AyaExpanse-32B & \crossmark & 32 & \cellcolor{green!0!red!100}22.3 & \cellcolor{green!0!red!100}46.0 & \cellcolor{green!0!red!100}32.7 & \cellcolor{green!0!red!100}-17.9 & \cellcolor{green!0!red!100}0.143 \\
\rowcolor{gray!30}
\official TowerPlus-72B[M] & \crossmark & 72 & \cellcolor{green!0!red!100}23.6 & \cellcolor{green!0!red!100}41.4 & \cellcolor{green!0!red!100}28.6 & \cellcolor{green!0!red!100}-18.1 & \cellcolor{green!0!red!100}0.141 \\
\official Llama-3.1-8B & \crossmark & 8 & \cellcolor{green!0!red!100}23.9 & \cellcolor{green!0!red!100}40.4 & \cellcolor{green!0!red!100}31.2 & \cellcolor{green!0!red!100}-18.6 & \cellcolor{green!0!red!100}0.125 \\
\official TowerPlus-9B[M] & \crossmark & 9 & \cellcolor{green!0!red!100}26.3 & \cellcolor{green!0!red!100}34.3 & \cellcolor{green!0!red!100}18.3 & \cellcolor{green!0!red!100}-19.8 & \cellcolor{green!0!red!100}0.157 \\
\official CommandR7B & \crossmark & 7 & \cellcolor{green!0!red!100}27.0 & \cellcolor{green!0!red!100}21.5 & \cellcolor{green!0!red!100}4.6 & \cellcolor{green!0!red!100}-17.7 & \cellcolor{green!1!red!99}0.274 \\
\official AyaExpanse-8B & \checkmark & 8 & \cellcolor{green!0!red!100}29.3 & \cellcolor{green!0!red!100}25.8 & \cellcolor{green!0!red!100}17.5 & \cellcolor{green!0!red!100}-22.7 & \cellcolor{green!0!red!100}0.141 \\
\official Qwen2.5-7B & \unknown & 7 & \cellcolor{green!0!red!100}29.7 & \cellcolor{green!0!red!100}24.1 & \cellcolor{green!0!red!100}19.0 & \cellcolor{green!0!red!100}-22.7 & \cellcolor{green!0!red!100}0.116 \\
\official Mistral-7B & \crossmark & 7 & \cellcolor{green!0!red!100}32.0 & \cellcolor{green!0!red!100}16.0 & \cellcolor{green!0!red!100}11.4 & \cellcolor{green!0!red!100}-24.0 & \cellcolor{green!0!red!100}0.14 \\
\bottomrule
\end{tabularx}
\end{table*}


\begin{table*}
\small
\begin{tabularx}{\textwidth}{lYYYYYYY}
\toprule
\multicolumn{8}{c}{\textbf{English-Marathi}} \\
\midrule
System Name & LP Supported & Params. (B) & AutoRank $\downarrow$ & GEMBA-ESA-CMDA $\uparrow$ & GEMBA-ESA-GPT4.1 $\uparrow$ & MetricX-24-Hybrid-XL $\uparrow$ & XCOMET-XL $\uparrow$ \\
\midrule
Shy-hunyuan-MT & \crossmark & 7 & \cellcolor{green!100!red!0}1.0 & \cellcolor{green!100!red!0}70.8 & \cellcolor{green!94!red!6}81.6 & \cellcolor{green!100!red!0}-5.8 & \cellcolor{green!74!red!26}0.248 \\
\rowcolor{gray!30}
\official Gemini-2.5-Pro & \checkmark & \unknown & \cellcolor{green!89!red!11}2.7 & \cellcolor{green!86!red!14}68.1 & \cellcolor{green!100!red!0}84.7 & \cellcolor{green!92!red!8}-6.2 & \cellcolor{green!50!red!50}0.222 \\
\rowcolor{gray!30}
GemTrans & \checkmark & 27 & \cellcolor{green!81!red!19}4.0 & \cellcolor{green!82!red!18}67.3 & \cellcolor{green!63!red!37}65.2 & \cellcolor{green!100!red!0}-5.8 & \cellcolor{green!52!red!48}0.224 \\
\rowcolor{gray!30}
\official GPT-4.1 & \checkmark & \unknown & \cellcolor{green!77!red!23}4.6 & \cellcolor{green!84!red!16}67.6 & \cellcolor{green!90!red!10}79.4 & \cellcolor{green!82!red!18}-6.7 & \cellcolor{green!26!red!74}0.196 \\
\rowcolor{gray!30}
UvA-MT & \unknown & 12 & \cellcolor{green!73!red!27}5.2 & \cellcolor{green!83!red!17}67.4 & \cellcolor{green!77!red!23}72.4 & \cellcolor{green!86!red!14}-6.5 & \cellcolor{green!22!red!78}0.192 \\
\rowcolor{gray!30}
\official Claude-4 & \unknown & \unknown & \cellcolor{green!71!red!29}5.5 & \cellcolor{green!82!red!18}67.2 & \cellcolor{green!84!red!16}76.2 & \cellcolor{green!73!red!27}-7.2 & \cellcolor{green!23!red!77}0.193 \\
\rowcolor{gray!30}
\official DeepSeek-V3 & \unknown & 671 & \cellcolor{green!71!red!29}5.5 & \cellcolor{green!83!red!17}67.5 & \cellcolor{green!81!red!19}74.7 & \cellcolor{green!76!red!24}-7.0 & \cellcolor{green!21!red!79}0.19 \\
\rowcolor{gray!30}
\official Gemma-3-27B & \checkmark & 27 & \cellcolor{green!69!red!31}5.8 & \cellcolor{green!81!red!19}67.1 & \cellcolor{green!76!red!24}71.8 & \cellcolor{green!76!red!24}-7.0 & \cellcolor{green!21!red!79}0.191 \\
\rowcolor{gray!30}
\official Mistral-Medium & \unknown & \unknown & \cellcolor{green!62!red!38}6.9 & \cellcolor{green!80!red!20}66.8 & \cellcolor{green!73!red!27}70.6 & \cellcolor{green!71!red!29}-7.3 & \cellcolor{green!3!red!97}0.171 \\
\rowcolor{gray!30}
CommandA-WMT & \crossmark & 111 & \cellcolor{green!60!red!40}7.2 & \cellcolor{green!78!red!22}66.4 & \cellcolor{green!62!red!38}64.7 & \cellcolor{green!73!red!27}-7.2 & \cellcolor{green!10!red!90}0.178 \\
IRB-MT & \checkmark & 12 & \cellcolor{green!59!red!41}7.3 & \cellcolor{green!68!red!32}64.4 & \cellcolor{green!69!red!31}68.1 & \cellcolor{green!76!red!24}-7.0 & \cellcolor{green!7!red!93}0.175 \\
\rowcolor{gray!30}
\official Llama-4-Maverick & \checkmark & 400 & \cellcolor{green!56!red!44}7.8 & \cellcolor{green!66!red!34}64.0 & \cellcolor{green!71!red!29}69.4 & \cellcolor{green!67!red!33}-7.5 & \cellcolor{green!1!red!99}0.169 \\
\rowcolor{gray!30}
\official ONLINE-B & \checkmark & \unknown & \cellcolor{green!55!red!45}8.0 & \cellcolor{green!59!red!41}62.7 & \cellcolor{green!65!red!35}66.0 & \cellcolor{green!73!red!27}-7.2 & \cellcolor{green!4!red!96}0.172 \\
\rowcolor{gray!30}
TranssionTranslate & \unknown & \unknown & \cellcolor{green!54!red!46}8.1 & \cellcolor{green!59!red!41}62.6 & \cellcolor{green!63!red!37}65.0 & \cellcolor{green!75!red!25}-7.1 & \cellcolor{green!2!red!98}0.17 \\
\rowcolor{gray!30}
\official Qwen3-235B & \checkmark & 235 & \cellcolor{green!53!red!47}8.3 & \cellcolor{green!66!red!34}64.1 & \cellcolor{green!63!red!37}64.9 & \cellcolor{green!67!red!33}-7.5 & \cellcolor{green!0!red!100}0.167 \\
\official TowerPlus-9B[M] & \crossmark & 9 & \cellcolor{green!49!red!51}8.9 & \cellcolor{green!61!red!39}63.0 & \cellcolor{green!0!red!100}7.7 & \cellcolor{green!63!red!37}-7.7 & \cellcolor{green!100!red!0}0.277 \\
\official NLLB & \checkmark & 1 & \cellcolor{green!29!red!71}12.0 & \cellcolor{green!37!red!63}58.3 & \cellcolor{green!45!red!55}55.6 & \cellcolor{green!43!red!57}-8.7 & \cellcolor{green!0!red!100}0.148 \\
\rowcolor{gray!30}
IR-MultiagentMT & \crossmark & \unknown & \cellcolor{green!28!red!72}12.1 & \cellcolor{green!34!red!66}57.7 & \cellcolor{green!45!red!55}55.5 & \cellcolor{green!35!red!65}-9.1 & \cellcolor{green!0!red!100}0.156 \\
\official Gemma-3-12B & \checkmark & 12 & \cellcolor{green!26!red!74}12.4 & \cellcolor{green!8!red!92}52.5 & \cellcolor{green!38!red!62}51.8 & \cellcolor{green!29!red!71}-9.4 & \cellcolor{green!20!red!80}0.189 \\
\rowcolor{gray!30}
\official ONLINE-G & \checkmark & \unknown & \cellcolor{green!21!red!79}13.2 & \cellcolor{green!32!red!68}57.3 & \cellcolor{green!43!red!57}54.2 & \cellcolor{green!29!red!71}-9.4 & \cellcolor{green!0!red!100}0.138 \\
\rowcolor{gray!30}
\official CommandA & \crossmark & 111 & \cellcolor{green!20!red!80}13.3 & \cellcolor{green!50!red!50}60.9 & \cellcolor{green!34!red!66}49.6 & \cellcolor{green!24!red!76}-9.7 & \cellcolor{green!0!red!100}0.131 \\
\official EuroLLM-9B[M] & \crossmark & 9 & \cellcolor{green!15!red!85}14.1 & \cellcolor{green!8!red!92}52.5 & \cellcolor{green!0!red!100}10.9 & \cellcolor{green!33!red!67}-9.2 & \cellcolor{green!53!red!47}0.225 \\
\official Llama-3.1-8B & \crossmark & 8 & \cellcolor{green!0!red!100}17.2 & \cellcolor{green!0!red!100}50.4 & \cellcolor{green!19!red!81}41.6 & \cellcolor{green!0!red!100}-11.3 & \cellcolor{green!0!red!100}0.139 \\
\rowcolor{gray!30}
\official TowerPlus-72B[M] & \crossmark & 72 & \cellcolor{green!0!red!100}17.8 & \cellcolor{green!0!red!100}49.8 & \cellcolor{green!0!red!100}30.5 & \cellcolor{green!0!red!100}-12.5 & \cellcolor{green!7!red!93}0.175 \\
\rowcolor{gray!30}
\official EuroLLM-22B-pre.[M] & \crossmark & 22 & \cellcolor{green!0!red!100}18.1 & \cellcolor{green!0!red!100}47.0 & \cellcolor{green!0!red!100}15.3 & \cellcolor{green!0!red!100}-11.8 & \cellcolor{green!29!red!71}0.199 \\
\rowcolor{gray!30}
\official AyaExpanse-32B & \crossmark & 32 & \cellcolor{green!0!red!100}18.4 & \cellcolor{green!0!red!100}49.6 & \cellcolor{green!6!red!94}34.7 & \cellcolor{green!0!red!100}-13.0 & \cellcolor{green!0!red!100}0.163 \\
\official AyaExpanse-8B & \crossmark & 8 & \cellcolor{green!0!red!100}20.8 & \cellcolor{green!0!red!100}41.8 & \cellcolor{green!0!red!100}27.0 & \cellcolor{green!0!red!100}-14.5 & \cellcolor{green!20!red!80}0.189 \\
\official CommandR7B & \crossmark & 7 & \cellcolor{green!0!red!100}21.4 & \cellcolor{green!0!red!100}36.6 & \cellcolor{green!0!red!100}20.0 & \cellcolor{green!0!red!100}-13.1 & \cellcolor{green!18!red!82}0.187 \\
\official Qwen2.5-7B & \unknown & 7 & \cellcolor{green!0!red!100}27.8 & \cellcolor{green!0!red!100}27.6 & \cellcolor{green!0!red!100}19.8 & \cellcolor{green!0!red!100}-18.0 & \cellcolor{green!7!red!93}0.175 \\
\official Mistral-7B & \crossmark & 7 & \cellcolor{green!0!red!100}30.0 & \cellcolor{green!0!red!100}25.0 & \cellcolor{green!0!red!100}12.5 & \cellcolor{green!0!red!100}-17.9 & \cellcolor{green!0!red!100}0.146 \\
\bottomrule
\end{tabularx}
\end{table*}


\begin{table*}
\small
\begin{tabularx}{\textwidth}{lYYYYYYY}
\toprule
\multicolumn{8}{c}{\textbf{English-Romanian}} \\
\midrule
System Name & LP Supported & Params. (B) & AutoRank $\downarrow$ & GEMBA-ESA-CMDA $\uparrow$ & GEMBA-ESA-GPT4.1 $\uparrow$ & MetricX-24-Hybrid-XL $\uparrow$ & XCOMET-XL $\uparrow$ \\
\midrule
Shy-hunyuan-MT & \checkmark & 7 & \cellcolor{green!100!red!0}1.0 & \cellcolor{green!89!red!11}83.2 & \cellcolor{green!83!red!17}86.3 & \cellcolor{green!100!red!0}-5.7 & \cellcolor{green!100!red!0}0.651 \\
\rowcolor{gray!30}
CommandA-WMT & \checkmark & 111 & \cellcolor{green!91!red!9}1.8 & \cellcolor{green!85!red!15}82.5 & \cellcolor{green!82!red!18}86.0 & \cellcolor{green!89!red!11}-6.0 & \cellcolor{green!88!red!12}0.634 \\
\rowcolor{gray!30}
\official Gemini-2.5-Pro & \checkmark & \unknown & \cellcolor{green!85!red!15}2.3 & \cellcolor{green!100!red!0}85.0 & \cellcolor{green!100!red!0}89.3 & \cellcolor{green!67!red!33}-6.6 & \cellcolor{green!53!red!47}0.586 \\
\rowcolor{gray!30}
\official GPT-4.1 & \checkmark & \unknown & \cellcolor{green!81!red!19}2.7 & \cellcolor{green!91!red!9}83.5 & \cellcolor{green!94!red!6}88.2 & \cellcolor{green!59!red!41}-6.8 & \cellcolor{green!61!red!39}0.597 \\
\rowcolor{gray!30}
GemTrans & \checkmark & 27 & \cellcolor{green!75!red!25}3.2 & \cellcolor{green!56!red!44}77.7 & \cellcolor{green!52!red!48}80.7 & \cellcolor{green!100!red!0}-5.7 & \cellcolor{green!77!red!23}0.619 \\
\rowcolor{gray!30}
\official DeepSeek-V3 & \unknown & 671 & \cellcolor{green!64!red!36}4.2 & \cellcolor{green!71!red!29}80.1 & \cellcolor{green!73!red!27}84.4 & \cellcolor{green!59!red!41}-6.8 & \cellcolor{green!45!red!55}0.574 \\
\rowcolor{gray!30}
UvA-MT & \unknown & 12 & \cellcolor{green!58!red!42}4.7 & \cellcolor{green!55!red!45}77.6 & \cellcolor{green!51!red!49}80.5 & \cellcolor{green!56!red!44}-6.9 & \cellcolor{green!62!red!38}0.598 \\
\rowcolor{gray!30}
\official Mistral-Medium & \unknown & \unknown & \cellcolor{green!54!red!46}5.1 & \cellcolor{green!56!red!44}77.7 & \cellcolor{green!66!red!34}83.2 & \cellcolor{green!44!red!56}-7.2 & \cellcolor{green!41!red!59}0.568 \\
\rowcolor{gray!30}
\official CommandA & \checkmark & 111 & \cellcolor{green!53!red!47}5.2 & \cellcolor{green!66!red!34}79.4 & \cellcolor{green!64!red!36}82.9 & \cellcolor{green!37!red!63}-7.4 & \cellcolor{green!37!red!63}0.563 \\
\rowcolor{gray!30}
\official Gemma-3-27B & \checkmark & 27 & \cellcolor{green!52!red!48}5.3 & \cellcolor{green!63!red!37}78.9 & \cellcolor{green!61!red!39}82.3 & \cellcolor{green!37!red!63}-7.4 & \cellcolor{green!36!red!64}0.562 \\
\official TowerPlus-9B[M] & \checkmark & 9 & \cellcolor{green!44!red!56}6.0 & \cellcolor{green!39!red!61}74.8 & \cellcolor{green!47!red!53}79.9 & \cellcolor{green!48!red!52}-7.1 & \cellcolor{green!39!red!61}0.566 \\
\rowcolor{gray!30}
\official Claude-4 & \unknown & \unknown & \cellcolor{green!43!red!57}6.1 & \cellcolor{green!66!red!34}79.4 & \cellcolor{green!61!red!39}82.4 & \cellcolor{green!22!red!78}-7.8 & \cellcolor{green!18!red!82}0.536 \\
\rowcolor{gray!30}
\official Qwen3-235B & \checkmark & 235 & \cellcolor{green!42!red!58}6.2 & \cellcolor{green!44!red!56}75.7 & \cellcolor{green!42!red!58}78.9 & \cellcolor{green!44!red!56}-7.2 & \cellcolor{green!33!red!67}0.558 \\
\rowcolor{gray!30}
\official AyaExpanse-32B & \checkmark & 32 & \cellcolor{green!39!red!61}6.4 & \cellcolor{green!48!red!52}76.4 & \cellcolor{green!47!red!53}79.9 & \cellcolor{green!33!red!67}-7.5 & \cellcolor{green!25!red!75}0.546 \\
IRB-MT & \checkmark & 12 & \cellcolor{green!39!red!61}6.4 & \cellcolor{green!42!red!58}75.4 & \cellcolor{green!33!red!67}77.4 & \cellcolor{green!52!red!48}-7.0 & \cellcolor{green!26!red!74}0.548 \\
SalamandraTA & \checkmark & 8 & \cellcolor{green!38!red!62}6.5 & \cellcolor{green!15!red!85}70.9 & \cellcolor{green!21!red!79}75.2 & \cellcolor{green!63!red!37}-6.7 & \cellcolor{green!56!red!44}0.589 \\
\rowcolor{gray!30}
\official Llama-4-Maverick & \checkmark & 400 & \cellcolor{green!38!red!62}6.5 & \cellcolor{green!48!red!52}76.4 & \cellcolor{green!54!red!46}81.1 & \cellcolor{green!30!red!70}-7.6 & \cellcolor{green!21!red!79}0.541 \\
\rowcolor{gray!30}
\official ONLINE-B & \checkmark & \unknown & \cellcolor{green!36!red!64}6.7 & \cellcolor{green!31!red!69}73.5 & \cellcolor{green!31!red!69}76.9 & \cellcolor{green!48!red!52}-7.1 & \cellcolor{green!32!red!68}0.556 \\
\official Gemma-3-12B & \checkmark & 12 & \cellcolor{green!22!red!78}7.9 & \cellcolor{green!36!red!64}74.3 & \cellcolor{green!36!red!64}77.9 & \cellcolor{green!15!red!85}-8.0 & \cellcolor{green!9!red!91}0.524 \\
\rowcolor{gray!30}
\official EuroLLM-22B-pre.[M] & \checkmark & 22 & \cellcolor{green!19!red!81}8.2 & \cellcolor{green!19!red!81}71.6 & \cellcolor{green!28!red!72}76.4 & \cellcolor{green!22!red!78}-7.8 & \cellcolor{green!15!red!85}0.533 \\
\rowcolor{gray!30}
TranssionTranslate & \unknown & \unknown & \cellcolor{green!10!red!90}9.0 & \cellcolor{green!0!red!100}68.4 & \cellcolor{green!4!red!96}72.1 & \cellcolor{green!41!red!59}-7.3 & \cellcolor{green!7!red!93}0.521 \\
\rowcolor{gray!30}
\official TowerPlus-72B[M] & \checkmark & 72 & \cellcolor{green!7!red!93}9.3 & \cellcolor{green!12!red!88}70.3 & \cellcolor{green!10!red!90}73.2 & \cellcolor{green!15!red!85}-8.0 & \cellcolor{green!0!red!100}0.512 \\
\rowcolor{gray!30}
\official ONLINE-W & \unknown & \unknown & \cellcolor{green!6!red!94}9.4 & \cellcolor{green!23!red!77}72.2 & \cellcolor{green!22!red!78}75.4 & \cellcolor{green!0!red!100}-8.7 & \cellcolor{green!0!red!100}0.51 \\
\official AyaExpanse-8B & \checkmark & 8 & \cellcolor{green!1!red!99}9.8 & \cellcolor{green!0!red!100}68.3 & \cellcolor{green!2!red!98}71.7 & \cellcolor{green!15!red!85}-8.0 & \cellcolor{green!3!red!97}0.516 \\
\official EuroLLM-9B[M] & \checkmark & 9 & \cellcolor{green!0!red!100}10.2 & \cellcolor{green!1!red!99}68.6 & \cellcolor{green!0!red!100}70.6 & \cellcolor{green!4!red!96}-8.3 & \cellcolor{green!0!red!100}0.512 \\
\rowcolor{gray!30}
IR-MultiagentMT & \crossmark & \unknown & \cellcolor{green!0!red!100}16.5 & \cellcolor{green!0!red!100}57.6 & \cellcolor{green!0!red!100}59.3 & \cellcolor{green!0!red!100}-10.0 & \cellcolor{green!0!red!100}0.425 \\
\official CommandR7B & \checkmark & 7 & \cellcolor{green!0!red!100}16.9 & \cellcolor{green!0!red!100}59.9 & \cellcolor{green!0!red!100}54.3 & \cellcolor{green!0!red!100}-10.2 & \cellcolor{green!0!red!100}0.434 \\
\official Llama-3.1-8B & \crossmark & 8 & \cellcolor{green!0!red!100}18.0 & \cellcolor{green!0!red!100}56.9 & \cellcolor{green!0!red!100}56.7 & \cellcolor{green!0!red!100}-10.3 & \cellcolor{green!0!red!100}0.38 \\
\rowcolor{gray!30}
\official ONLINE-G & \checkmark & \unknown & \cellcolor{green!0!red!100}18.7 & \cellcolor{green!0!red!100}59.5 & \cellcolor{green!0!red!100}60.6 & \cellcolor{green!0!red!100}-11.6 & \cellcolor{green!0!red!100}0.359 \\
\official NLLB & \checkmark & 1 & \cellcolor{green!0!red!100}19.3 & \cellcolor{green!0!red!100}55.0 & \cellcolor{green!0!red!100}57.2 & \cellcolor{green!0!red!100}-11.6 & \cellcolor{green!0!red!100}0.39 \\
\official Mistral-7B & \crossmark & 7 & \cellcolor{green!0!red!100}28.1 & \cellcolor{green!0!red!100}44.6 & \cellcolor{green!0!red!100}41.7 & \cellcolor{green!0!red!100}-14.0 & \cellcolor{green!0!red!100}0.224 \\
\official Qwen2.5-7B & \unknown & 7 & \cellcolor{green!0!red!100}32.0 & \cellcolor{green!0!red!100}37.7 & \cellcolor{green!0!red!100}34.9 & \cellcolor{green!0!red!100}-15.2 & \cellcolor{green!0!red!100}0.177 \\
\bottomrule
\end{tabularx}
\end{table*}


\begin{table*}
\small
\begin{tabularx}{\textwidth}{lYYYYYYY}
\toprule
\multicolumn{8}{c}{\textbf{English-Thai}} \\
\midrule
System Name & LP Supported & Params. (B) & AutoRank $\downarrow$ & GEMBA-ESA-CMDA $\uparrow$ & GEMBA-ESA-GPT4.1 $\uparrow$ & MetricX-24-Hybrid-XL $\uparrow$ & XCOMET-XL $\uparrow$ \\
\midrule
Shy-hunyuan-MT & \checkmark & 7 & \cellcolor{green!100!red!0}1.0 & \cellcolor{green!100!red!0}71.3 & \cellcolor{green!95!red!5}87.9 & \cellcolor{green!100!red!0}-5.1 & \cellcolor{green!100!red!0}0.603 \\
\rowcolor{gray!30}
\official Gemini-2.5-Pro & \checkmark & \unknown & \cellcolor{green!92!red!8}2.2 & \cellcolor{green!91!red!9}69.0 & \cellcolor{green!100!red!0}90.6 & \cellcolor{green!92!red!8}-5.6 & \cellcolor{green!81!red!19}0.533 \\
\rowcolor{gray!30}
GemTrans & \checkmark & 27 & \cellcolor{green!89!red!11}2.7 & \cellcolor{green!87!red!13}67.9 & \cellcolor{green!80!red!20}80.4 & \cellcolor{green!95!red!5}-5.4 & \cellcolor{green!88!red!12}0.558 \\
\rowcolor{gray!30}
UvA-MT & \unknown & 12 & \cellcolor{green!86!red!14}3.2 & \cellcolor{green!93!red!7}69.5 & \cellcolor{green!79!red!21}79.7 & \cellcolor{green!85!red!15}-6.0 & \cellcolor{green!83!red!17}0.54 \\
\rowcolor{gray!30}
\official GPT-4.1 & \checkmark & \unknown & \cellcolor{green!86!red!14}3.2 & \cellcolor{green!95!red!5}69.9 & \cellcolor{green!93!red!7}87.2 & \cellcolor{green!82!red!18}-6.2 & \cellcolor{green!70!red!30}0.489 \\
\rowcolor{gray!30}
\official Qwen3-235B & \checkmark & 235 & \cellcolor{green!84!red!16}3.5 & \cellcolor{green!91!red!9}68.8 & \cellcolor{green!81!red!19}80.9 & \cellcolor{green!84!red!16}-6.1 & \cellcolor{green!75!red!25}0.51 \\
\rowcolor{gray!30}
\official DeepSeek-V3 & \unknown & 671 & \cellcolor{green!84!red!16}3.6 & \cellcolor{green!94!red!6}69.6 & \cellcolor{green!85!red!15}82.9 & \cellcolor{green!81!red!19}-6.3 & \cellcolor{green!71!red!29}0.493 \\
\rowcolor{gray!30}
\official Gemma-3-27B & \checkmark & 27 & \cellcolor{green!80!red!20}4.1 & \cellcolor{green!89!red!11}68.3 & \cellcolor{green!84!red!16}82.2 & \cellcolor{green!77!red!23}-6.5 & \cellcolor{green!68!red!32}0.482 \\
\rowcolor{gray!30}
\official Mistral-Medium & \unknown & \unknown & \cellcolor{green!80!red!20}4.2 & \cellcolor{green!91!red!9}68.9 & \cellcolor{green!79!red!21}79.8 & \cellcolor{green!76!red!24}-6.6 & \cellcolor{green!69!red!31}0.486 \\
\rowcolor{gray!30}
\official Claude-4 & \unknown & \unknown & \cellcolor{green!78!red!22}4.5 & \cellcolor{green!90!red!10}68.5 & \cellcolor{green!81!red!19}80.7 & \cellcolor{green!73!red!27}-6.8 & \cellcolor{green!64!red!36}0.466 \\
IRB-MT & \checkmark & 12 & \cellcolor{green!76!red!24}4.8 & \cellcolor{green!81!red!19}66.2 & \cellcolor{green!74!red!26}77.1 & \cellcolor{green!79!red!21}-6.4 & \cellcolor{green!66!red!34}0.475 \\
\rowcolor{gray!30}
\official Llama-4-Maverick & \checkmark & 400 & \cellcolor{green!75!red!25}5.0 & \cellcolor{green!84!red!16}67.0 & \cellcolor{green!72!red!28}76.2 & \cellcolor{green!76!red!24}-6.6 & \cellcolor{green!63!red!37}0.463 \\
\rowcolor{gray!30}
\official ONLINE-B & \checkmark & \unknown & \cellcolor{green!75!red!25}5.0 & \cellcolor{green!77!red!23}65.1 & \cellcolor{green!64!red!36}72.2 & \cellcolor{green!84!red!16}-6.1 & \cellcolor{green!68!red!32}0.484 \\
\rowcolor{gray!30}
CommandA-WMT & \crossmark & 111 & \cellcolor{green!70!red!30}5.8 & \cellcolor{green!83!red!17}66.8 & \cellcolor{green!60!red!40}70.0 & \cellcolor{green!73!red!27}-6.8 & \cellcolor{green!59!red!41}0.449 \\
\rowcolor{gray!30}
TranssionTranslate & \unknown & \unknown & \cellcolor{green!67!red!33}6.3 & \cellcolor{green!66!red!34}62.2 & \cellcolor{green!55!red!45}67.7 & \cellcolor{green!79!red!21}-6.4 & \cellcolor{green!60!red!40}0.453 \\
\rowcolor{gray!30}
\official TowerPlus-72B[M] & \crossmark & 72 & \cellcolor{green!64!red!36}6.7 & \cellcolor{green!73!red!27}64.1 & \cellcolor{green!61!red!39}70.3 & \cellcolor{green!66!red!34}-7.2 & \cellcolor{green!52!red!48}0.424 \\
\official Gemma-3-12B & \checkmark & 12 & \cellcolor{green!49!red!51}9.1 & \cellcolor{green!40!red!60}55.1 & \cellcolor{green!46!red!54}62.6 & \cellcolor{green!52!red!48}-8.1 & \cellcolor{green!53!red!47}0.427 \\
\rowcolor{gray!30}
IR-MultiagentMT & \crossmark & \unknown & \cellcolor{green!46!red!54}9.5 & \cellcolor{green!45!red!55}56.6 & \cellcolor{green!34!red!66}56.5 & \cellcolor{green!55!red!45}-7.9 & \cellcolor{green!47!red!53}0.404 \\
\rowcolor{gray!30}
\official CommandA & \crossmark & 111 & \cellcolor{green!37!red!63}11.0 & \cellcolor{green!60!red!40}60.6 & \cellcolor{green!30!red!70}54.6 & \cellcolor{green!31!red!69}-9.4 & \cellcolor{green!22!red!78}0.311 \\
\official Qwen2.5-7B & \checkmark & 7 & \cellcolor{green!32!red!68}11.7 & \cellcolor{green!45!red!55}56.4 & \cellcolor{green!23!red!77}51.1 & \cellcolor{green!34!red!66}-9.2 & \cellcolor{green!24!red!76}0.319 \\
\official Llama-3.1-8B & \checkmark & 8 & \cellcolor{green!27!red!73}12.6 & \cellcolor{green!42!red!58}55.6 & \cellcolor{green!23!red!77}51.2 & \cellcolor{green!26!red!74}-9.7 & \cellcolor{green!13!red!87}0.277 \\
\official NLLB & \checkmark & 1 & \cellcolor{green!13!red!87}14.8 & \cellcolor{green!25!red!75}51.2 & \cellcolor{green!18!red!82}48.4 & \cellcolor{green!0!red!100}-11.3 & \cellcolor{green!5!red!95}0.247 \\
\official TowerPlus-9B[M] & \crossmark & 9 & \cellcolor{green!0!red!100}17.5 & \cellcolor{green!0!red!100}42.1 & \cellcolor{green!0!red!100}34.2 & \cellcolor{green!0!red!100}-11.3 & \cellcolor{green!0!red!100}0.221 \\
\rowcolor{gray!30}
\official ONLINE-G & \checkmark & \unknown & \cellcolor{green!0!red!100}21.1 & \cellcolor{green!0!red!100}35.8 & \cellcolor{green!0!red!100}36.1 & \cellcolor{green!0!red!100}-14.9 & \cellcolor{green!0!red!100}0.176 \\
\rowcolor{gray!30}
\official AyaExpanse-32B & \crossmark & 32 & \cellcolor{green!0!red!100}21.6 & \cellcolor{green!0!red!100}36.6 & \cellcolor{green!0!red!100}30.5 & \cellcolor{green!0!red!100}-14.9 & \cellcolor{green!0!red!100}0.154 \\
\official Mistral-7B & \crossmark & 7 & \cellcolor{green!0!red!100}24.6 & \cellcolor{green!0!red!100}31.0 & \cellcolor{green!0!red!100}24.9 & \cellcolor{green!0!red!100}-17.2 & \cellcolor{green!0!red!100}0.132 \\
\official CommandR7B & \crossmark & 7 & \cellcolor{green!0!red!100}25.8 & \cellcolor{green!0!red!100}27.0 & \cellcolor{green!0!red!100}21.6 & \cellcolor{green!0!red!100}-18.2 & \cellcolor{green!0!red!100}0.156 \\
\official AyaExpanse-8B & \crossmark & 8 & \cellcolor{green!0!red!100}26.4 & \cellcolor{green!0!red!100}24.6 & \cellcolor{green!0!red!100}20.1 & \cellcolor{green!0!red!100}-18.5 & \cellcolor{green!0!red!100}0.167 \\
\rowcolor{gray!30}
\official EuroLLM-22B-pre.[M] & \crossmark & 22 & \cellcolor{green!0!red!100}29.1 & \cellcolor{green!0!red!100}19.0 & \cellcolor{green!0!red!100}15.2 & \cellcolor{green!0!red!100}-20.9 & \cellcolor{green!0!red!100}0.169 \\
\official EuroLLM-9B[M] & \crossmark & 9 & \cellcolor{green!0!red!100}30.0 & \cellcolor{green!0!red!100}12.8 & \cellcolor{green!0!red!100}7.4 & \cellcolor{green!0!red!100}-20.2 & \cellcolor{green!0!red!100}0.185 \\
\bottomrule
\end{tabularx}
\end{table*}


\begin{table*}
\small
\begin{tabularx}{\textwidth}{lYYYYYYY}
\toprule
\multicolumn{8}{c}{\textbf{English-Serbian (Latin)}} \\
\midrule
System Name & LP Supported & Params. (B) & AutoRank $\downarrow$ & GEMBA-ESA-CMDA $\uparrow$ & GEMBA-ESA-GPT4.1 $\uparrow$ & MetricX-24-Hybrid-XL $\uparrow$ & XCOMET-XL $\uparrow$ \\
\midrule
Shy-hunyuan-MT & \checkmark & 7 & \cellcolor{green!100!red!0}1.0 & \cellcolor{green!100!red!0}80.1 & \cellcolor{green!93!red!7}84.2 & \cellcolor{green!100!red!0}-3.4 & \cellcolor{green!100!red!0}0.583 \\
Wenyiil & \checkmark & 14 & \cellcolor{green!87!red!13}2.5 & \cellcolor{green!90!red!10}77.8 & \cellcolor{green!94!red!6}84.6 & \cellcolor{green!84!red!16}-3.8 & \cellcolor{green!73!red!27}0.513 \\
Algharb & \checkmark & 14 & \cellcolor{green!84!red!16}2.8 & \cellcolor{green!91!red!9}77.9 & \cellcolor{green!99!red!1}86.5 & \cellcolor{green!76!red!24}-4.0 & \cellcolor{green!65!red!35}0.493 \\
\rowcolor{gray!30}
GemTrans & \checkmark & 27 & \cellcolor{green!84!red!16}2.9 & \cellcolor{green!77!red!23}74.6 & \cellcolor{green!72!red!28}75.3 & \cellcolor{green!100!red!0}-3.4 & \cellcolor{green!78!red!22}0.528 \\
\rowcolor{gray!30}
\official GPT-4.1 & \checkmark & \unknown & \cellcolor{green!84!red!16}2.9 & \cellcolor{green!94!red!6}78.6 & \cellcolor{green!96!red!4}85.3 & \cellcolor{green!71!red!29}-4.1 & \cellcolor{green!68!red!32}0.501 \\
\rowcolor{gray!30}
\official DeepSeek-V3 & \unknown & 671 & \cellcolor{green!84!red!16}2.9 & \cellcolor{green!93!red!7}78.5 & \cellcolor{green!84!red!16}80.3 & \cellcolor{green!80!red!20}-3.9 & \cellcolor{green!73!red!27}0.514 \\
\rowcolor{gray!30}
UvA-MT & \checkmark & 12 & \cellcolor{green!84!red!16}2.9 & \cellcolor{green!78!red!22}75.0 & \cellcolor{green!71!red!29}75.0 & \cellcolor{green!88!red!12}-3.7 & \cellcolor{green!92!red!8}0.562 \\
\rowcolor{gray!30}
\official Gemini-2.5-Pro & \checkmark & \unknown & \cellcolor{green!84!red!16}2.9 & \cellcolor{green!90!red!10}77.7 & \cellcolor{green!100!red!0}86.9 & \cellcolor{green!71!red!29}-4.1 & \cellcolor{green!63!red!37}0.488 \\
Yolu & \checkmark & 14 & \cellcolor{green!83!red!17}3.0 & \cellcolor{green!70!red!30}73.0 & \cellcolor{green!66!red!34}73.1 & \cellcolor{green!100!red!0}-3.4 & \cellcolor{green!88!red!12}0.553 \\
\rowcolor{gray!30}
\official Claude-4 & \unknown & \unknown & \cellcolor{green!67!red!33}4.8 & \cellcolor{green!76!red!24}74.5 & \cellcolor{green!75!red!25}76.6 & \cellcolor{green!55!red!45}-4.5 & \cellcolor{green!56!red!44}0.471 \\
SalamandraTA & \checkmark & 8 & \cellcolor{green!65!red!35}5.0 & \cellcolor{green!52!red!48}68.8 & \cellcolor{green!55!red!45}68.3 & \cellcolor{green!84!red!16}-3.8 & \cellcolor{green!64!red!36}0.491 \\
\rowcolor{gray!30}
\official Llama-4-Maverick & \checkmark & 400 & \cellcolor{green!56!red!44}6.1 & \cellcolor{green!63!red!37}71.3 & \cellcolor{green!60!red!40}70.4 & \cellcolor{green!47!red!53}-4.7 & \cellcolor{green!47!red!53}0.448 \\
IRB-MT & \checkmark & 12 & \cellcolor{green!54!red!46}6.3 & \cellcolor{green!53!red!47}69.0 & \cellcolor{green!51!red!49}66.7 & \cellcolor{green!63!red!37}-4.3 & \cellcolor{green!44!red!56}0.441 \\
\rowcolor{gray!30}
\official Qwen3-235B & \checkmark & 235 & \cellcolor{green!53!red!47}6.4 & \cellcolor{green!52!red!48}68.8 & \cellcolor{green!48!red!52}65.4 & \cellcolor{green!63!red!37}-4.3 & \cellcolor{green!44!red!56}0.439 \\
\rowcolor{gray!30}
IR-MultiagentMT & \crossmark & \unknown & \cellcolor{green!50!red!50}6.8 & \cellcolor{green!57!red!43}70.0 & \cellcolor{green!50!red!50}66.2 & \cellcolor{green!47!red!53}-4.7 & \cellcolor{green!43!red!57}0.437 \\
\rowcolor{gray!30}
CommandA-WMT & \crossmark & 111 & \cellcolor{green!46!red!54}7.2 & \cellcolor{green!59!red!41}70.4 & \cellcolor{green!41!red!59}62.5 & \cellcolor{green!10!red!90}-5.6 & \cellcolor{green!70!red!30}0.506 \\
\rowcolor{gray!30}
\official ONLINE-B & \checkmark & \unknown & \cellcolor{green!46!red!54}7.2 & \cellcolor{green!56!red!44}69.6 & \cellcolor{green!46!red!54}64.6 & \cellcolor{green!10!red!90}-5.6 & \cellcolor{green!66!red!34}0.497 \\
\official Gemma-3-12B & \checkmark & 12 & \cellcolor{green!43!red!57}7.6 & \cellcolor{green!48!red!52}67.8 & \cellcolor{green!43!red!57}63.6 & \cellcolor{green!35!red!65}-5.0 & \cellcolor{green!39!red!61}0.427 \\
\rowcolor{gray!30}
\official CommandA & \crossmark & 111 & \cellcolor{green!43!red!57}7.6 & \cellcolor{green!47!red!53}67.7 & \cellcolor{green!34!red!66}59.9 & \cellcolor{green!51!red!49}-4.6 & \cellcolor{green!32!red!68}0.41 \\
\rowcolor{gray!30}
\official Gemma-3-27B & \checkmark & 27 & \cellcolor{green!31!red!69}9.0 & \cellcolor{green!32!red!68}64.1 & \cellcolor{green!43!red!57}63.6 & \cellcolor{green!2!red!98}-5.8 & \cellcolor{green!43!red!57}0.438 \\
\rowcolor{gray!30}
\official EuroLLM-22B-pre.[M] & \crossmark & 22 & \cellcolor{green!31!red!69}9.0 & \cellcolor{green!17!red!83}60.6 & \cellcolor{green!21!red!79}54.4 & \cellcolor{green!39!red!61}-4.9 & \cellcolor{green!40!red!60}0.43 \\
CUNI-SFT & \checkmark & 9 & \cellcolor{green!27!red!73}9.4 & \cellcolor{green!21!red!79}61.5 & \cellcolor{green!18!red!82}53.2 & \cellcolor{green!39!red!61}-4.9 & \cellcolor{green!25!red!75}0.392 \\
\official EuroLLM-9B[M] & \crossmark & 9 & \cellcolor{green!20!red!80}10.2 & \cellcolor{green!10!red!90}58.8 & \cellcolor{green!5!red!95}47.7 & \cellcolor{green!27!red!73}-5.2 & \cellcolor{green!36!red!64}0.42 \\
\rowcolor{gray!30}
TranssionTranslate & \unknown & \unknown & \cellcolor{green!16!red!84}10.7 & \cellcolor{green!18!red!82}60.8 & \cellcolor{green!34!red!66}59.6 & \cellcolor{green!0!red!100}-5.9 & \cellcolor{green!8!red!92}0.349 \\
\rowcolor{gray!30}
TranssionMT & \checkmark & 1 & \cellcolor{green!13!red!87}11.0 & \cellcolor{green!3!red!97}57.3 & \cellcolor{green!17!red!83}52.8 & \cellcolor{green!18!red!82}-5.4 & \cellcolor{green!12!red!88}0.358 \\
\rowcolor{gray!30}
\official ONLINE-G & \checkmark & \unknown & \cellcolor{green!0!red!100}12.5 & \cellcolor{green!6!red!94}57.8 & \cellcolor{green!17!red!83}52.9 & \cellcolor{green!0!red!100}-6.9 & \cellcolor{green!19!red!81}0.375 \\
\rowcolor{gray!30}
\official TowerPlus-72B[M] & \crossmark & 72 & \cellcolor{green!0!red!100}12.6 & \cellcolor{green!0!red!100}55.7 & \cellcolor{green!0!red!100}43.1 & \cellcolor{green!14!red!86}-5.5 & \cellcolor{green!0!red!100}0.306 \\
\official Llama-3.1-8B & \crossmark & 8 & \cellcolor{green!0!red!100}13.4 & \cellcolor{green!0!red!100}54.7 & \cellcolor{green!0!red!100}43.8 & \cellcolor{green!0!red!100}-6.0 & \cellcolor{green!0!red!100}0.29 \\
\rowcolor{gray!30}
\official AyaExpanse-32B & \crossmark & 32 & \cellcolor{green!0!red!100}13.8 & \cellcolor{green!0!red!100}52.9 & \cellcolor{green!0!red!100}40.4 & \cellcolor{green!6!red!94}-5.7 & \cellcolor{green!0!red!100}0.259 \\
\official TowerPlus-9B[M] & \crossmark & 9 & \cellcolor{green!0!red!100}17.6 & \cellcolor{green!0!red!100}43.0 & \cellcolor{green!0!red!100}29.2 & \cellcolor{green!0!red!100}-6.4 & \cellcolor{green!0!red!100}0.181 \\
\official Mistral-7B & \crossmark & 7 & \cellcolor{green!0!red!100}17.6 & \cellcolor{green!0!red!100}49.4 & \cellcolor{green!0!red!100}37.0 & \cellcolor{green!0!red!100}-7.8 & \cellcolor{green!0!red!100}0.213 \\
\official Qwen2.5-7B & \unknown & 7 & \cellcolor{green!0!red!100}20.5 & \cellcolor{green!0!red!100}39.3 & \cellcolor{green!0!red!100}29.0 & \cellcolor{green!0!red!100}-8.1 & \cellcolor{green!0!red!100}0.144 \\
\official AyaExpanse-8B & \crossmark & 8 & \cellcolor{green!0!red!100}20.7 & \cellcolor{green!0!red!100}37.5 & \cellcolor{green!0!red!100}25.9 & \cellcolor{green!0!red!100}-7.9 & \cellcolor{green!0!red!100}0.143 \\
\official CommandR7B & \crossmark & 7 & \cellcolor{green!0!red!100}21.1 & \cellcolor{green!0!red!100}38.5 & \cellcolor{green!0!red!100}25.7 & \cellcolor{green!0!red!100}-8.9 & \cellcolor{green!0!red!100}0.203 \\
\official NLLB & \checkmark & 1 & \cellcolor{green!0!red!100}35.0 & \cellcolor{green!0!red!100}0.8 & \cellcolor{green!0!red!100}0.1 & \cellcolor{green!0!red!100}-15.2 & \cellcolor{green!0!red!100}0.195 \\
\bottomrule
\end{tabularx}
\end{table*}


\begin{table*}
\small
\begin{tabularx}{\textwidth}{lYYYYYYY}
\toprule
\multicolumn{8}{c}{\textbf{English-Swedish}} \\
\midrule
System Name & LP Supported & Params. (B) & AutoRank $\downarrow$ & GEMBA-ESA-CMDA $\uparrow$ & GEMBA-ESA-GPT4.1 $\uparrow$ & MetricX-24-Hybrid-XL $\uparrow$ & XCOMET-XL $\uparrow$ \\
\midrule
Shy-hunyuan-MT & \checkmark & 7 & \cellcolor{green!100!red!0}1.0 & \cellcolor{green!100!red!0}84.2 & \cellcolor{green!95!red!5}91.0 & \cellcolor{green!100!red!0}-4.7 & \cellcolor{green!100!red!0}0.685 \\
\rowcolor{gray!30}
\official Gemini-2.5-Pro & \checkmark & \unknown & \cellcolor{green!87!red!13}2.5 & \cellcolor{green!94!red!6}83.1 & \cellcolor{green!100!red!0}92.3 & \cellcolor{green!79!red!21}-5.4 & \cellcolor{green!72!red!28}0.638 \\
\rowcolor{gray!30}
GemTrans & \checkmark & 27 & \cellcolor{green!83!red!17}2.9 & \cellcolor{green!72!red!28}79.2 & \cellcolor{green!71!red!29}85.1 & \cellcolor{green!100!red!0}-4.7 & \cellcolor{green!83!red!17}0.656 \\
\rowcolor{gray!30}
\official GPT-4.1 & \checkmark & \unknown & \cellcolor{green!80!red!20}3.2 & \cellcolor{green!85!red!15}81.5 & \cellcolor{green!98!red!2}91.7 & \cellcolor{green!63!red!37}-5.9 & \cellcolor{green!70!red!30}0.635 \\
\rowcolor{gray!30}
\official DeepSeek-V3 & \unknown & 671 & \cellcolor{green!72!red!28}4.1 & \cellcolor{green!82!red!18}81.0 & \cellcolor{green!78!red!22}86.8 & \cellcolor{green!63!red!37}-5.9 & \cellcolor{green!62!red!38}0.621 \\
\rowcolor{gray!30}
CommandA-WMT & \crossmark & 111 & \cellcolor{green!70!red!30}4.4 & \cellcolor{green!66!red!34}78.2 & \cellcolor{green!59!red!41}81.9 & \cellcolor{green!82!red!18}-5.3 & \cellcolor{green!67!red!33}0.63 \\
\rowcolor{gray!30}
UvA-MT & \unknown & 12 & \cellcolor{green!69!red!31}4.5 & \cellcolor{green!71!red!29}79.0 & \cellcolor{green!63!red!37}82.9 & \cellcolor{green!69!red!31}-5.7 & \cellcolor{green!71!red!29}0.636 \\
\rowcolor{gray!30}
\official Mistral-Medium & \unknown & \unknown & \cellcolor{green!68!red!32}4.6 & \cellcolor{green!81!red!19}80.8 & \cellcolor{green!73!red!27}85.6 & \cellcolor{green!57!red!43}-6.1 & \cellcolor{green!58!red!42}0.614 \\
\rowcolor{gray!30}
\official Gemma-3-27B & \checkmark & 27 & \cellcolor{green!64!red!36}5.0 & \cellcolor{green!74!red!26}79.5 & \cellcolor{green!70!red!30}84.7 & \cellcolor{green!57!red!43}-6.1 & \cellcolor{green!55!red!45}0.61 \\
\rowcolor{gray!30}
\official Claude-4 & \unknown & \unknown & \cellcolor{green!62!red!38}5.3 & \cellcolor{green!82!red!18}80.9 & \cellcolor{green!73!red!27}85.4 & \cellcolor{green!42!red!58}-6.6 & \cellcolor{green!50!red!50}0.601 \\
IRB-MT & \checkmark & 12 & \cellcolor{green!57!red!43}5.8 & \cellcolor{green!56!red!44}76.3 & \cellcolor{green!53!red!47}80.4 & \cellcolor{green!66!red!34}-5.8 & \cellcolor{green!53!red!47}0.606 \\
\official TowerPlus-9B[M] & \checkmark & 9 & \cellcolor{green!55!red!45}6.0 & \cellcolor{green!60!red!40}77.0 & \cellcolor{green!56!red!44}81.3 & \cellcolor{green!54!red!46}-6.2 & \cellcolor{green!51!red!49}0.602 \\
\rowcolor{gray!30}
\official ONLINE-B & \checkmark & \unknown & \cellcolor{green!54!red!46}6.1 & \cellcolor{green!55!red!45}76.2 & \cellcolor{green!53!red!47}80.5 & \cellcolor{green!57!red!43}-6.1 & \cellcolor{green!49!red!51}0.599 \\
SalamandraTA & \checkmark & 8 & \cellcolor{green!54!red!46}6.1 & \cellcolor{green!48!red!52}75.0 & \cellcolor{green!43!red!57}78.0 & \cellcolor{green!60!red!40}-6.0 & \cellcolor{green!62!red!38}0.621 \\
\rowcolor{gray!30}
\official Llama-4-Maverick & \checkmark & 400 & \cellcolor{green!54!red!46}6.2 & \cellcolor{green!68!red!32}78.4 & \cellcolor{green!58!red!42}81.8 & \cellcolor{green!42!red!58}-6.6 & \cellcolor{green!44!red!56}0.591 \\
\rowcolor{gray!30}
\official ONLINE-W & \unknown & \unknown & \cellcolor{green!45!red!55}7.2 & \cellcolor{green!52!red!48}75.6 & \cellcolor{green!54!red!46}80.7 & \cellcolor{green!30!red!70}-7.0 & \cellcolor{green!44!red!56}0.591 \\
\rowcolor{gray!30}
\official TowerPlus-72B[M] & \checkmark & 72 & \cellcolor{green!42!red!58}7.5 & \cellcolor{green!50!red!50}75.2 & \cellcolor{green!41!red!59}77.6 & \cellcolor{green!36!red!64}-6.8 & \cellcolor{green!38!red!62}0.58 \\
\rowcolor{gray!30}
\official Qwen3-235B & \checkmark & 235 & \cellcolor{green!36!red!64}8.2 & \cellcolor{green!42!red!58}73.9 & \cellcolor{green!32!red!68}75.3 & \cellcolor{green!36!red!64}-6.8 & \cellcolor{green!32!red!68}0.571 \\
\rowcolor{gray!30}
\official EuroLLM-22B-pre.[M] & \checkmark & 22 & \cellcolor{green!35!red!65}8.3 & \cellcolor{green!45!red!55}74.4 & \cellcolor{green!38!red!62}76.6 & \cellcolor{green!24!red!76}-7.2 & \cellcolor{green!30!red!70}0.568 \\
\rowcolor{gray!30}
IR-MultiagentMT & \crossmark & \unknown & \cellcolor{green!35!red!65}8.3 & \cellcolor{green!46!red!54}74.6 & \cellcolor{green!39!red!61}76.9 & \cellcolor{green!24!red!76}-7.2 & \cellcolor{green!28!red!72}0.564 \\
\rowcolor{gray!30}
TranssionTranslate & \unknown & \unknown & \cellcolor{green!34!red!66}8.4 & \cellcolor{green!19!red!81}69.7 & \cellcolor{green!34!red!66}75.7 & \cellcolor{green!54!red!46}-6.2 & \cellcolor{green!27!red!73}0.563 \\
\rowcolor{gray!30}
\official CommandA & \crossmark & 111 & \cellcolor{green!29!red!71}8.9 & \cellcolor{green!46!red!54}74.6 & \cellcolor{green!31!red!69}75.0 & \cellcolor{green!21!red!79}-7.3 & \cellcolor{green!20!red!80}0.551 \\
\official EuroLLM-9B[M] & \checkmark & 9 & \cellcolor{green!21!red!79}9.8 & \cellcolor{green!25!red!75}70.8 & \cellcolor{green!20!red!80}72.3 & \cellcolor{green!21!red!79}-7.3 & \cellcolor{green!23!red!77}0.555 \\
\official Gemma-3-12B & \checkmark & 12 & \cellcolor{green!7!red!93}11.4 & \cellcolor{green!5!red!95}67.2 & \cellcolor{green!8!red!92}69.2 & \cellcolor{green!8!red!92}-7.7 & \cellcolor{green!7!red!93}0.528 \\
\official Llama-3.1-8B & \crossmark & 8 & \cellcolor{green!0!red!100}14.6 & \cellcolor{green!0!red!100}63.8 & \cellcolor{green!0!red!100}61.1 & \cellcolor{green!0!red!100}-8.8 & \cellcolor{green!0!red!100}0.483 \\
\rowcolor{gray!30}
\official ONLINE-G & \checkmark & \unknown & \cellcolor{green!0!red!100}17.7 & \cellcolor{green!0!red!100}61.8 & \cellcolor{green!0!red!100}59.9 & \cellcolor{green!0!red!100}-10.6 & \cellcolor{green!0!red!100}0.422 \\
\official NLLB & \checkmark & 1 & \cellcolor{green!0!red!100}18.1 & \cellcolor{green!0!red!100}58.9 & \cellcolor{green!0!red!100}58.2 & \cellcolor{green!0!red!100}-10.5 & \cellcolor{green!0!red!100}0.436 \\
\official Mistral-7B & \crossmark & 7 & \cellcolor{green!0!red!100}21.0 & \cellcolor{green!0!red!100}56.2 & \cellcolor{green!0!red!100}50.3 & \cellcolor{green!0!red!100}-11.2 & \cellcolor{green!0!red!100}0.374 \\
\rowcolor{gray!30}
\official AyaExpanse-32B & \crossmark & 32 & \cellcolor{green!0!red!100}21.1 & \cellcolor{green!0!red!100}55.1 & \cellcolor{green!0!red!100}49.6 & \cellcolor{green!0!red!100}-11.1 & \cellcolor{green!0!red!100}0.376 \\
\official Qwen2.5-7B & \unknown & 7 & \cellcolor{green!0!red!100}26.0 & \cellcolor{green!0!red!100}47.5 & \cellcolor{green!0!red!100}42.0 & \cellcolor{green!0!red!100}-12.9 & \cellcolor{green!0!red!100}0.304 \\
\official CommandR7B & \crossmark & 7 & \cellcolor{green!0!red!100}27.8 & \cellcolor{green!0!red!100}41.0 & \cellcolor{green!0!red!100}31.7 & \cellcolor{green!0!red!100}-12.7 & \cellcolor{green!0!red!100}0.316 \\
\official AyaExpanse-8B & \crossmark & 8 & \cellcolor{green!0!red!100}32.0 & \cellcolor{green!0!red!100}40.1 & \cellcolor{green!0!red!100}33.7 & \cellcolor{green!0!red!100}-15.5 & \cellcolor{green!0!red!100}0.211 \\
\bottomrule
\end{tabularx}
\end{table*}


\begin{table*}
\small
\begin{tabularx}{\textwidth}{lYYYYYYY}
\toprule
\multicolumn{8}{c}{\textbf{English-Turkish}} \\
\midrule
System Name & LP Supported & Params. (B) & AutoRank $\downarrow$ & GEMBA-ESA-CMDA $\uparrow$ & GEMBA-ESA-GPT4.1 $\uparrow$ & MetricX-24-Hybrid-XL $\uparrow$ & XCOMET-XL $\uparrow$ \\
\midrule
Shy-hunyuan-MT & \checkmark & 7 & \cellcolor{green!100!red!0}1.0 & \cellcolor{green!93!red!7}81.4 & \cellcolor{green!91!red!9}85.2 & \cellcolor{green!100!red!0}-7.2 & \cellcolor{green!100!red!0}0.542 \\
\rowcolor{gray!30}
\official Gemini-2.5-Pro & \checkmark & \unknown & \cellcolor{green!86!red!14}2.7 & \cellcolor{green!98!red!2}82.7 & \cellcolor{green!100!red!0}87.9 & \cellcolor{green!70!red!30}-8.4 & \cellcolor{green!63!red!37}0.462 \\
\rowcolor{gray!30}
\official GPT-4.1 & \checkmark & \unknown & \cellcolor{green!84!red!16}3.0 & \cellcolor{green!100!red!0}83.1 & \cellcolor{green!94!red!6}86.1 & \cellcolor{green!65!red!35}-8.6 & \cellcolor{green!65!red!35}0.465 \\
\rowcolor{gray!30}
CommandA-WMT & \checkmark & 111 & \cellcolor{green!81!red!19}3.3 & \cellcolor{green!79!red!21}77.9 & \cellcolor{green!74!red!26}80.2 & \cellcolor{green!85!red!15}-7.8 & \cellcolor{green!76!red!24}0.491 \\
\rowcolor{gray!30}
GemTrans & \checkmark & 27 & \cellcolor{green!81!red!19}3.3 & \cellcolor{green!65!red!35}74.6 & \cellcolor{green!69!red!31}78.8 & \cellcolor{green!95!red!5}-7.4 & \cellcolor{green!83!red!17}0.506 \\
\rowcolor{gray!30}
\official DeepSeek-V3 & \unknown & 671 & \cellcolor{green!80!red!20}3.4 & \cellcolor{green!92!red!8}81.2 & \cellcolor{green!88!red!12}84.5 & \cellcolor{green!65!red!35}-8.6 & \cellcolor{green!63!red!37}0.461 \\
\rowcolor{gray!30}
\official Mistral-Medium & \unknown & \unknown & \cellcolor{green!65!red!35}5.3 & \cellcolor{green!71!red!29}76.0 & \cellcolor{green!69!red!31}78.7 & \cellcolor{green!54!red!46}-9.0 & \cellcolor{green!53!red!47}0.44 \\
\rowcolor{gray!30}
\official Claude-4 & \checkmark & \unknown & \cellcolor{green!63!red!37}5.5 & \cellcolor{green!78!red!22}77.7 & \cellcolor{green!73!red!27}80.1 & \cellcolor{green!44!red!56}-9.4 & \cellcolor{green!46!red!54}0.424 \\
\rowcolor{gray!30}
UvA-MT & \unknown & 12 & \cellcolor{green!62!red!38}5.6 & \cellcolor{green!57!red!43}72.6 & \cellcolor{green!59!red!41}76.0 & \cellcolor{green!59!red!41}-8.8 & \cellcolor{green!62!red!38}0.46 \\
\rowcolor{gray!30}
\official ONLINE-W & \unknown & \unknown & \cellcolor{green!58!red!42}6.2 & \cellcolor{green!62!red!38}73.7 & \cellcolor{green!60!red!40}76.1 & \cellcolor{green!52!red!48}-9.1 & \cellcolor{green!49!red!51}0.431 \\
\rowcolor{gray!30}
\official ONLINE-B & \checkmark & \unknown & \cellcolor{green!50!red!50}7.1 & \cellcolor{green!56!red!44}72.4 & \cellcolor{green!47!red!53}72.5 & \cellcolor{green!47!red!53}-9.3 & \cellcolor{green!41!red!59}0.414 \\
IRB-MT & \checkmark & 12 & \cellcolor{green!49!red!51}7.2 & \cellcolor{green!46!red!54}69.9 & \cellcolor{green!45!red!55}71.8 & \cellcolor{green!57!red!43}-8.9 & \cellcolor{green!41!red!59}0.415 \\
\rowcolor{gray!30}
\official Llama-4-Maverick & \checkmark & 400 & \cellcolor{green!49!red!51}7.3 & \cellcolor{green!57!red!43}72.5 & \cellcolor{green!55!red!45}74.7 & \cellcolor{green!34!red!66}-9.8 & \cellcolor{green!39!red!61}0.409 \\
\rowcolor{gray!30}
TranssionTranslate & \unknown & \unknown & \cellcolor{green!46!red!54}7.6 & \cellcolor{green!39!red!61}68.1 & \cellcolor{green!44!red!56}71.6 & \cellcolor{green!52!red!48}-9.1 & \cellcolor{green!41!red!59}0.413 \\
\rowcolor{gray!30}
\official Qwen3-235B & \checkmark & 235 & \cellcolor{green!45!red!55}7.7 & \cellcolor{green!44!red!56}69.4 & \cellcolor{green!43!red!57}71.2 & \cellcolor{green!47!red!53}-9.3 & \cellcolor{green!38!red!62}0.408 \\
\rowcolor{gray!30}
\official CommandA & \checkmark & 111 & \cellcolor{green!42!red!58}8.1 & \cellcolor{green!53!red!47}71.5 & \cellcolor{green!46!red!54}72.2 & \cellcolor{green!32!red!68}-9.9 & \cellcolor{green!31!red!69}0.393 \\
\official Gemma-3-12B & \checkmark & 12 & \cellcolor{green!37!red!63}8.7 & \cellcolor{green!41!red!59}68.5 & \cellcolor{green!37!red!63}69.4 & \cellcolor{green!34!red!66}-9.8 & \cellcolor{green!30!red!70}0.391 \\
\rowcolor{gray!30}
\official EuroLLM-22B-pre.[M] & \checkmark & 22 & \cellcolor{green!35!red!65}9.0 & \cellcolor{green!31!red!69}66.1 & \cellcolor{green!36!red!64}69.1 & \cellcolor{green!32!red!68}-9.9 & \cellcolor{green!33!red!67}0.397 \\
\rowcolor{gray!30}
\official Gemma-3-27B & \checkmark & 27 & \cellcolor{green!34!red!66}9.1 & \cellcolor{green!34!red!66}66.9 & \cellcolor{green!37!red!63}69.6 & \cellcolor{green!27!red!73}-10.1 & \cellcolor{green!32!red!68}0.394 \\
\rowcolor{gray!30}
\official AyaExpanse-32B & \checkmark & 32 & \cellcolor{green!27!red!73}9.9 & \cellcolor{green!24!red!76}64.5 & \cellcolor{green!25!red!75}66.1 & \cellcolor{green!24!red!76}-10.2 & \cellcolor{green!27!red!73}0.383 \\
\official EuroLLM-9B[M] & \checkmark & 9 & \cellcolor{green!20!red!80}10.8 & \cellcolor{green!4!red!96}59.6 & \cellcolor{green!6!red!94}60.5 & \cellcolor{green!24!red!76}-10.2 & \cellcolor{green!39!red!61}0.409 \\
\rowcolor{gray!30}
IR-MultiagentMT & \crossmark & \unknown & \cellcolor{green!19!red!81}10.9 & \cellcolor{green!23!red!77}64.1 & \cellcolor{green!23!red!77}65.3 & \cellcolor{green!14!red!86}-10.6 & \cellcolor{green!12!red!88}0.351 \\
\official AyaExpanse-8B & \checkmark & 8 & \cellcolor{green!2!red!98}13.0 & \cellcolor{green!0!red!100}58.6 & \cellcolor{green!0!red!100}58.7 & \cellcolor{green!4!red!96}-11.0 & \cellcolor{green!0!red!100}0.325 \\
\rowcolor{gray!30}
\official TowerPlus-72B[M] & \crossmark & 72 & \cellcolor{green!0!red!100}13.5 & \cellcolor{green!0!red!100}58.5 & \cellcolor{green!0!red!100}56.7 & \cellcolor{green!0!red!100}-11.3 & \cellcolor{green!0!red!100}0.325 \\
\rowcolor{gray!30}
\official ONLINE-G & \checkmark & \unknown & \cellcolor{green!0!red!100}14.3 & \cellcolor{green!0!red!100}58.0 & \cellcolor{green!0!red!100}58.6 & \cellcolor{green!0!red!100}-11.9 & \cellcolor{green!0!red!100}0.294 \\
\official NLLB & \checkmark & 1 & \cellcolor{green!0!red!100}15.5 & \cellcolor{green!0!red!100}53.3 & \cellcolor{green!0!red!100}55.3 & \cellcolor{green!0!red!100}-12.4 & \cellcolor{green!0!red!100}0.304 \\
\official Llama-3.1-8B & \crossmark & 8 & \cellcolor{green!0!red!100}17.8 & \cellcolor{green!0!red!100}51.1 & \cellcolor{green!0!red!100}48.4 & \cellcolor{green!0!red!100}-12.9 & \cellcolor{green!0!red!100}0.248 \\
\official CommandR7B & \crossmark & 7 & \cellcolor{green!0!red!100}18.0 & \cellcolor{green!0!red!100}48.4 & \cellcolor{green!0!red!100}42.9 & \cellcolor{green!0!red!100}-12.8 & \cellcolor{green!0!red!100}0.291 \\
\official TowerPlus-9B[M] & \crossmark & 9 & \cellcolor{green!0!red!100}22.1 & \cellcolor{green!0!red!100}43.6 & \cellcolor{green!0!red!100}36.9 & \cellcolor{green!0!red!100}-14.6 & \cellcolor{green!0!red!100}0.192 \\
\official Qwen2.5-7B & \unknown & 7 & \cellcolor{green!0!red!100}22.7 & \cellcolor{green!0!red!100}41.2 & \cellcolor{green!0!red!100}38.5 & \cellcolor{green!0!red!100}-14.9 & \cellcolor{green!0!red!100}0.174 \\
\official Mistral-7B & \crossmark & 7 & \cellcolor{green!0!red!100}31.0 & \cellcolor{green!0!red!100}27.1 & \cellcolor{green!0!red!100}22.2 & \cellcolor{green!0!red!100}-20.2 & \cellcolor{green!0!red!100}0.138 \\
\bottomrule
\end{tabularx}
\end{table*}


\begin{table*}
\small
\begin{tabularx}{\textwidth}{lYYYYYYY}
\toprule
\multicolumn{8}{c}{\textbf{English-Vietnamese}} \\
\midrule
System Name & LP Supported & Params. (B) & AutoRank $\downarrow$ & GEMBA-ESA-CMDA $\uparrow$ & GEMBA-ESA-GPT4.1 $\uparrow$ & MetricX-24-Hybrid-XL $\uparrow$ & XCOMET-XL $\uparrow$ \\
\midrule
Shy-hunyuan-MT & \checkmark & 7 & \cellcolor{green!100!red!0}1.0 & \cellcolor{green!100!red!0}83.1 & \cellcolor{green!95!red!5}87.3 & \cellcolor{green!100!red!0}-4.5 & \cellcolor{green!100!red!0}0.623 \\
\rowcolor{gray!30}
\official Gemini-2.5-Pro & \checkmark & \unknown & \cellcolor{green!84!red!16}2.7 & \cellcolor{green!97!red!3}82.3 & \cellcolor{green!100!red!0}88.6 & \cellcolor{green!74!red!26}-5.6 & \cellcolor{green!61!red!39}0.539 \\
\rowcolor{gray!30}
CommandA-WMT & \checkmark & 111 & \cellcolor{green!84!red!16}2.7 & \cellcolor{green!80!red!20}78.4 & \cellcolor{green!81!red!19}83.2 & \cellcolor{green!91!red!9}-4.9 & \cellcolor{green!79!red!21}0.577 \\
\rowcolor{gray!30}
\official GPT-4.1 & \checkmark & \unknown & \cellcolor{green!83!red!17}2.8 & \cellcolor{green!99!red!1}82.9 & \cellcolor{green!98!red!2}88.1 & \cellcolor{green!72!red!28}-5.7 & \cellcolor{green!58!red!42}0.533 \\
\rowcolor{gray!30}
\official DeepSeek-V3 & \unknown & 671 & \cellcolor{green!79!red!21}3.2 & \cellcolor{green!93!red!7}81.5 & \cellcolor{green!89!red!11}85.5 & \cellcolor{green!72!red!28}-5.7 & \cellcolor{green!58!red!42}0.533 \\
\rowcolor{gray!30}
\official Qwen3-235B & \checkmark & 235 & \cellcolor{green!78!red!22}3.3 & \cellcolor{green!86!red!14}79.9 & \cellcolor{green!84!red!16}84.1 & \cellcolor{green!76!red!24}-5.5 & \cellcolor{green!61!red!39}0.539 \\
\rowcolor{gray!30}
GemTrans & \checkmark & 27 & \cellcolor{green!77!red!23}3.4 & \cellcolor{green!64!red!36}74.8 & \cellcolor{green!70!red!30}80.3 & \cellcolor{green!93!red!7}-4.8 & \cellcolor{green!76!red!24}0.572 \\
\rowcolor{gray!30}
UvA-MT & \unknown & 12 & \cellcolor{green!74!red!26}3.7 & \cellcolor{green!74!red!26}77.0 & \cellcolor{green!72!red!28}80.8 & \cellcolor{green!76!red!24}-5.5 & \cellcolor{green!70!red!30}0.559 \\
\rowcolor{gray!30}
\official Mistral-Medium & \unknown & \unknown & \cellcolor{green!74!red!26}3.7 & \cellcolor{green!81!red!19}78.7 & \cellcolor{green!83!red!17}83.8 & \cellcolor{green!69!red!31}-5.8 & \cellcolor{green!57!red!43}0.53 \\
\rowcolor{gray!30}
\official Claude-4 & \unknown & \unknown & \cellcolor{green!62!red!38}5.0 & \cellcolor{green!79!red!21}78.2 & \cellcolor{green!74!red!26}81.5 & \cellcolor{green!48!red!52}-6.7 & \cellcolor{green!40!red!60}0.494 \\
IRB-MT & \checkmark & 12 & \cellcolor{green!61!red!39}5.1 & \cellcolor{green!61!red!39}74.1 & \cellcolor{green!61!red!39}77.7 & \cellcolor{green!69!red!31}-5.8 & \cellcolor{green!46!red!54}0.506 \\
\rowcolor{gray!30}
\official AyaExpanse-32B & \checkmark & 32 & \cellcolor{green!55!red!45}5.8 & \cellcolor{green!54!red!46}72.3 & \cellcolor{green!58!red!42}76.9 & \cellcolor{green!58!red!42}-6.3 & \cellcolor{green!42!red!58}0.498 \\
\rowcolor{gray!30}
\official Llama-4-Maverick & \checkmark & 400 & \cellcolor{green!47!red!53}6.6 & \cellcolor{green!53!red!47}72.2 & \cellcolor{green!56!red!44}76.4 & \cellcolor{green!44!red!56}-6.9 & \cellcolor{green!29!red!71}0.47 \\
\rowcolor{gray!30}
\official ONLINE-B & \checkmark & \unknown & \cellcolor{green!47!red!53}6.6 & \cellcolor{green!48!red!52}70.9 & \cellcolor{green!49!red!51}74.4 & \cellcolor{green!51!red!49}-6.6 & \cellcolor{green!33!red!67}0.478 \\
\rowcolor{gray!30}
TranssionTranslate & \unknown & \unknown & \cellcolor{green!41!red!59}7.2 & \cellcolor{green!28!red!72}66.4 & \cellcolor{green!36!red!64}70.7 & \cellcolor{green!62!red!38}-6.1 & \cellcolor{green!32!red!68}0.476 \\
\official AyaExpanse-8B & \checkmark & 8 & \cellcolor{green!36!red!64}7.8 & \cellcolor{green!28!red!72}66.3 & \cellcolor{green!33!red!67}70.1 & \cellcolor{green!48!red!52}-6.7 & \cellcolor{green!27!red!73}0.465 \\
\official Gemma-3-12B & \checkmark & 12 & \cellcolor{green!33!red!67}8.1 & \cellcolor{green!32!red!68}67.2 & \cellcolor{green!36!red!64}70.9 & \cellcolor{green!36!red!64}-7.2 & \cellcolor{green!19!red!81}0.448 \\
\rowcolor{gray!30}
\official CommandA & \checkmark & 111 & \cellcolor{green!27!red!73}8.7 & \cellcolor{green!30!red!70}66.8 & \cellcolor{green!30!red!70}69.2 & \cellcolor{green!25!red!75}-7.7 & \cellcolor{green!16!red!84}0.442 \\
\rowcolor{gray!30}
IR-MultiagentMT & \crossmark & \unknown & \cellcolor{green!27!red!73}8.7 & \cellcolor{green!31!red!69}67.0 & \cellcolor{green!33!red!67}70.0 & \cellcolor{green!29!red!71}-7.5 & \cellcolor{green!8!red!92}0.424 \\
\rowcolor{gray!30}
\official TowerPlus-72B[M] & \crossmark & 72 & \cellcolor{green!26!red!74}8.8 & \cellcolor{green!24!red!76}65.3 & \cellcolor{green!16!red!84}65.4 & \cellcolor{green!34!red!66}-7.3 & \cellcolor{green!25!red!75}0.46 \\
\rowcolor{gray!30}
\official Gemma-3-27B & \checkmark & 27 & \cellcolor{green!16!red!84}9.9 & \cellcolor{green!13!red!87}62.8 & \cellcolor{green!17!red!83}65.5 & \cellcolor{green!20!red!80}-7.9 & \cellcolor{green!6!red!94}0.42 \\
\official Qwen2.5-7B & \checkmark & 7 & \cellcolor{green!7!red!93}10.8 & \cellcolor{green!7!red!93}61.4 & \cellcolor{green!1!red!99}61.2 & \cellcolor{green!11!red!89}-8.3 & \cellcolor{green!1!red!99}0.41 \\
\official Llama-3.1-8B & \crossmark & 8 & \cellcolor{green!0!red!100}11.8 & \cellcolor{green!0!red!100}59.3 & \cellcolor{green!0!red!100}60.7 & \cellcolor{green!0!red!100}-8.9 & \cellcolor{green!0!red!100}0.385 \\
\official CommandR7B & \checkmark & 7 & \cellcolor{green!0!red!100}13.1 & \cellcolor{green!0!red!100}55.7 & \cellcolor{green!0!red!100}52.0 & \cellcolor{green!0!red!100}-9.6 & \cellcolor{green!0!red!100}0.406 \\
\official NLLB & \checkmark & 1 & \cellcolor{green!0!red!100}15.5 & \cellcolor{green!0!red!100}54.0 & \cellcolor{green!0!red!100}53.9 & \cellcolor{green!0!red!100}-11.4 & \cellcolor{green!0!red!100}0.303 \\
\official TowerPlus-9B[M] & \crossmark & 9 & \cellcolor{green!0!red!100}16.8 & \cellcolor{green!0!red!100}46.2 & \cellcolor{green!0!red!100}42.1 & \cellcolor{green!0!red!100}-10.7 & \cellcolor{green!0!red!100}0.319 \\
\rowcolor{gray!30}
\official ONLINE-G & \checkmark & \unknown & \cellcolor{green!0!red!100}17.4 & \cellcolor{green!0!red!100}52.5 & \cellcolor{green!0!red!100}51.0 & \cellcolor{green!0!red!100}-12.6 & \cellcolor{green!0!red!100}0.238 \\
\official Mistral-7B & \crossmark & 7 & \cellcolor{green!0!red!100}24.4 & \cellcolor{green!0!red!100}33.7 & \cellcolor{green!0!red!100}33.6 & \cellcolor{green!0!red!100}-15.9 & \cellcolor{green!0!red!100}0.139 \\
\official EuroLLM-9B[M] & \crossmark & 9 & \cellcolor{green!0!red!100}27.3 & \cellcolor{green!0!red!100}18.8 & \cellcolor{green!0!red!100}9.4 & \cellcolor{green!0!red!100}-17.9 & \cellcolor{green!0!red!100}0.327 \\
\rowcolor{gray!30}
\official EuroLLM-22B-pre.[M] & \crossmark & 22 & \cellcolor{green!0!red!100}30.0 & \cellcolor{green!0!red!100}22.2 & \cellcolor{green!0!red!100}20.8 & \cellcolor{green!0!red!100}-20.5 & \cellcolor{green!0!red!100}0.113 \\
\bottomrule
\end{tabularx}
\end{table*}


% \restoregeometry

\clearpage
\bibliography{anthology.min.bib,custom}



\appendix
\onecolumn

\section{Metrics correlations}

To examine how the metrics used for AutoRank correlate with each other, we calculated the Pearson correlation between paragraph-level scores for all systems, resulting in a sample size of around 14k scores per each language pair.

The results show that GEMBA-ESA on CmdA and GPT-4.1 exhibit the highest correlations for almost all languages. In contrast, the weakest correlations are generally observed between xComet and both GEMBA-ESA variants.

When examining results by language pair, Bhojpuri, Maasai, and Marathi show the lowest correlations. This is why we use chrF++ for the first two language pairs. Unfortunately, no reference translations are available for Marathi, so we must rely on QE metrics for its evaluation.

\begin{table*}[h]
\small
\centering
% Vilém: reduce column spacing to fit the page width
\setlength{\tabcolsep}{4.9pt}
\begin{tabular}{lllllllllll}
\toprule
 & \shortstack{Kiwi\\G-CmdA} & \shortstack{Kiwi\\G-GPT} & \shortstack{Kiwi\\MetX} & \shortstack{Kiwi\\xComet} & \shortstack{G-CmdA\\G-GPT} & \shortstack{G-CmdA\\MetX} & \shortstack{G-CmdA\\xComet} & \shortstack{G-GPT\\MetX} & \shortstack{G-GPT\\xComet} & \shortstack{MetX\\xComet} \\
\midrule
cs-de\_DE & \cellcolor[HTML]{E9F6A1}\textcolor[HTML]{000000}{0.441} & \cellcolor[HTML]{D9EF8B}\textcolor[HTML]{000000}{0.484} & \cellcolor[HTML]{BBE278}\textcolor[HTML]{000000}{0.541} & \cellcolor[HTML]{48AE5C}\textcolor[HTML]{000000}{0.709} & \cellcolor[HTML]{36A657}\textcolor[HTML]{FFFFFF}{0.732} & \cellcolor[HTML]{A2D76A}\textcolor[HTML]{000000}{0.583} & \cellcolor[HTML]{FAFDB8}\textcolor[HTML]{000000}{0.403} & \cellcolor[HTML]{7FC866}\textcolor[HTML]{000000}{0.636} & \cellcolor[HTML]{ECF7A6}\textcolor[HTML]{000000}{0.436} & \cellcolor[HTML]{AFDD70}\textcolor[HTML]{000000}{0.560} \\
cs-uk\_UA & \cellcolor[HTML]{BFE47A}\textcolor[HTML]{000000}{0.531} & \cellcolor[HTML]{98D368}\textcolor[HTML]{000000}{0.600} & \cellcolor[HTML]{54B45F}\textcolor[HTML]{000000}{0.696} & \cellcolor[HTML]{128A49}\textcolor[HTML]{FFFFFF}{0.794} & \cellcolor[HTML]{48AE5C}\textcolor[HTML]{000000}{0.708} & \cellcolor[HTML]{A9DA6C}\textcolor[HTML]{000000}{0.571} & \cellcolor[HTML]{C7E77F}\textcolor[HTML]{000000}{0.517} & \cellcolor[HTML]{73C264}\textcolor[HTML]{000000}{0.654} & \cellcolor[HTML]{A9DA6C}\textcolor[HTML]{000000}{0.573} & \cellcolor[HTML]{48AE5C}\textcolor[HTML]{000000}{0.710} \\
en-ar\_EG & \cellcolor[HTML]{91D068}\textcolor[HTML]{000000}{0.610} & \cellcolor[HTML]{A9DA6C}\textcolor[HTML]{000000}{0.573} & \cellcolor[HTML]{279F53}\textcolor[HTML]{FFFFFF}{0.750} & \cellcolor[HTML]{D3EC87}\textcolor[HTML]{000000}{0.494} & \cellcolor[HTML]{30A356}\textcolor[HTML]{FFFFFF}{0.740} & \cellcolor[HTML]{87CB67}\textcolor[HTML]{000000}{0.624} & \cellcolor[HTML]{FFF2AA}\textcolor[HTML]{000000}{0.350} & \cellcolor[HTML]{93D168}\textcolor[HTML]{000000}{0.605} & \cellcolor[HTML]{FED27F}\textcolor[HTML]{000000}{0.268} & \cellcolor[HTML]{ADDC6F}\textcolor[HTML]{000000}{0.565} \\
en-bho\_IN & \cellcolor[HTML]{E0F295}\textcolor[HTML]{000000}{0.465} & \cellcolor[HTML]{F16640}\textcolor[HTML]{000000}{0.093} & \cellcolor[HTML]{C7E77F}\textcolor[HTML]{000000}{0.517} & \cellcolor[HTML]{DD3D2D}\textcolor[HTML]{FFFFFF}{0.030} & \cellcolor[HTML]{CFEB85}\textcolor[HTML]{000000}{0.503} & \cellcolor[HTML]{89CC67}\textcolor[HTML]{000000}{0.621} & \cellcolor[HTML]{E34933}\textcolor[HTML]{FFFFFF}{0.051} & \cellcolor[HTML]{EFF8AA}\textcolor[HTML]{000000}{0.428} & \cellcolor[HTML]{CC2627}\textcolor[HTML]{FFFFFF}{-0.008} & \cellcolor[HTML]{FCAA5F}\textcolor[HTML]{000000}{0.194} \\
en-bn\_BD & \cellcolor[HTML]{2DA155}\textcolor[HTML]{FFFFFF}{0.742} & \cellcolor[HTML]{279F53}\textcolor[HTML]{FFFFFF}{0.752} & \cellcolor[HTML]{0B7D42}\textcolor[HTML]{FFFFFF}{0.822} & \cellcolor[HTML]{D1EC86}\textcolor[HTML]{000000}{0.498} & \cellcolor[HTML]{108647}\textcolor[HTML]{FFFFFF}{0.802} & \cellcolor[HTML]{33A456}\textcolor[HTML]{FFFFFF}{0.735} & \cellcolor[HTML]{ECF7A6}\textcolor[HTML]{000000}{0.435} & \cellcolor[HTML]{36A657}\textcolor[HTML]{FFFFFF}{0.730} & \cellcolor[HTML]{E8F59F}\textcolor[HTML]{000000}{0.448} & \cellcolor[HTML]{A2D76A}\textcolor[HTML]{000000}{0.584} \\
en-cs\_CZ & \cellcolor[HTML]{8CCD67}\textcolor[HTML]{000000}{0.617} & \cellcolor[HTML]{54B45F}\textcolor[HTML]{000000}{0.696} & \cellcolor[HTML]{39A758}\textcolor[HTML]{FFFFFF}{0.728} & \cellcolor[HTML]{2AA054}\textcolor[HTML]{FFFFFF}{0.747} & \cellcolor[HTML]{219C52}\textcolor[HTML]{FFFFFF}{0.757} & \cellcolor[HTML]{7AC665}\textcolor[HTML]{000000}{0.642} & \cellcolor[HTML]{BFE47A}\textcolor[HTML]{000000}{0.533} & \cellcolor[HTML]{5DB961}\textcolor[HTML]{000000}{0.682} & \cellcolor[HTML]{BDE379}\textcolor[HTML]{000000}{0.535} & \cellcolor[HTML]{45AD5B}\textcolor[HTML]{000000}{0.712} \\
en-de\_DE & \cellcolor[HTML]{DAF08D}\textcolor[HTML]{000000}{0.481} & \cellcolor[HTML]{B7E075}\textcolor[HTML]{000000}{0.546} & \cellcolor[HTML]{8ECF67}\textcolor[HTML]{000000}{0.612} & \cellcolor[HTML]{138C4A}\textcolor[HTML]{FFFFFF}{0.789} & \cellcolor[HTML]{2DA155}\textcolor[HTML]{FFFFFF}{0.742} & \cellcolor[HTML]{A5D86A}\textcolor[HTML]{000000}{0.578} & \cellcolor[HTML]{FFF2AA}\textcolor[HTML]{000000}{0.350} & \cellcolor[HTML]{9BD469}\textcolor[HTML]{000000}{0.593} & \cellcolor[HTML]{FFF5AE}\textcolor[HTML]{000000}{0.358} & \cellcolor[HTML]{B1DE71}\textcolor[HTML]{000000}{0.559} \\
en-el\_GR & \cellcolor[HTML]{33A456}\textcolor[HTML]{FFFFFF}{0.736} & \cellcolor[HTML]{17934E}\textcolor[HTML]{FFFFFF}{0.777} & \cellcolor[HTML]{148E4B}\textcolor[HTML]{FFFFFF}{0.787} & \cellcolor[HTML]{57B65F}\textcolor[HTML]{000000}{0.691} & \cellcolor[HTML]{006837}\textcolor[HTML]{FFFFFF}{0.863} & \cellcolor[HTML]{42AC5A}\textcolor[HTML]{000000}{0.716} & \cellcolor[HTML]{B9E176}\textcolor[HTML]{000000}{0.542} & \cellcolor[HTML]{2DA155}\textcolor[HTML]{FFFFFF}{0.743} & \cellcolor[HTML]{B9E176}\textcolor[HTML]{000000}{0.544} & \cellcolor[HTML]{30A356}\textcolor[HTML]{FFFFFF}{0.741} \\
en-et\_EE & \cellcolor[HTML]{15904C}\textcolor[HTML]{FFFFFF}{0.783} & \cellcolor[HTML]{07753E}\textcolor[HTML]{FFFFFF}{0.837} & \cellcolor[HTML]{0A7B41}\textcolor[HTML]{FFFFFF}{0.825} & \cellcolor[HTML]{3FAA59}\textcolor[HTML]{000000}{0.720} & \cellcolor[HTML]{148E4B}\textcolor[HTML]{FFFFFF}{0.787} & \cellcolor[HTML]{33A456}\textcolor[HTML]{FFFFFF}{0.736} & \cellcolor[HTML]{A2D76A}\textcolor[HTML]{000000}{0.583} & \cellcolor[HTML]{128A49}\textcolor[HTML]{FFFFFF}{0.795} & \cellcolor[HTML]{73C264}\textcolor[HTML]{000000}{0.655} & \cellcolor[HTML]{108647}\textcolor[HTML]{FFFFFF}{0.802} \\
en-fa\_IR & \cellcolor[HTML]{0D8044}\textcolor[HTML]{FFFFFF}{0.814} & \cellcolor[HTML]{07753E}\textcolor[HTML]{FFFFFF}{0.834} & \cellcolor[HTML]{006837}\textcolor[HTML]{FFFFFF}{0.862} & \cellcolor[HTML]{4EB15D}\textcolor[HTML]{000000}{0.703} & \cellcolor[HTML]{036E3A}\textcolor[HTML]{FFFFFF}{0.852} & \cellcolor[HTML]{15904C}\textcolor[HTML]{FFFFFF}{0.785} & \cellcolor[HTML]{9BD469}\textcolor[HTML]{000000}{0.596} & \cellcolor[HTML]{128A49}\textcolor[HTML]{FFFFFF}{0.793} & \cellcolor[HTML]{9DD569}\textcolor[HTML]{000000}{0.589} & \cellcolor[HTML]{57B65F}\textcolor[HTML]{000000}{0.689} \\
en-hi\_IN & \cellcolor[HTML]{75C465}\textcolor[HTML]{000000}{0.651} & \cellcolor[HTML]{6EC064}\textcolor[HTML]{000000}{0.663} & \cellcolor[HTML]{73C264}\textcolor[HTML]{000000}{0.654} & \cellcolor[HTML]{E9F6A1}\textcolor[HTML]{000000}{0.443} & \cellcolor[HTML]{249D53}\textcolor[HTML]{FFFFFF}{0.754} & \cellcolor[HTML]{70C164}\textcolor[HTML]{000000}{0.658} & \cellcolor[HTML]{EEF8A8}\textcolor[HTML]{000000}{0.432} & \cellcolor[HTML]{60BA62}\textcolor[HTML]{000000}{0.681} & \cellcolor[HTML]{E3F399}\textcolor[HTML]{000000}{0.459} & \cellcolor[HTML]{7FC866}\textcolor[HTML]{000000}{0.634} \\
en-id\_ID & \cellcolor[HTML]{54B45F}\textcolor[HTML]{000000}{0.696} & \cellcolor[HTML]{17934E}\textcolor[HTML]{FFFFFF}{0.777} & \cellcolor[HTML]{4BB05C}\textcolor[HTML]{000000}{0.705} & \cellcolor[HTML]{60BA62}\textcolor[HTML]{000000}{0.680} & \cellcolor[HTML]{17934E}\textcolor[HTML]{FFFFFF}{0.775} & \cellcolor[HTML]{82C966}\textcolor[HTML]{000000}{0.633} & \cellcolor[HTML]{B9E176}\textcolor[HTML]{000000}{0.542} & \cellcolor[HTML]{73C264}\textcolor[HTML]{000000}{0.653} & \cellcolor[HTML]{B3DF72}\textcolor[HTML]{000000}{0.552} & \cellcolor[HTML]{17934E}\textcolor[HTML]{FFFFFF}{0.775} \\
en-is\_IS & \cellcolor[HTML]{148E4B}\textcolor[HTML]{FFFFFF}{0.787} & \cellcolor[HTML]{0E8245}\textcolor[HTML]{FFFFFF}{0.811} & \cellcolor[HTML]{06733D}\textcolor[HTML]{FFFFFF}{0.839} & \cellcolor[HTML]{70C164}\textcolor[HTML]{000000}{0.659} & \cellcolor[HTML]{249D53}\textcolor[HTML]{FFFFFF}{0.756} & \cellcolor[HTML]{45AD5B}\textcolor[HTML]{000000}{0.713} & \cellcolor[HTML]{D3EC87}\textcolor[HTML]{000000}{0.495} & \cellcolor[HTML]{148E4B}\textcolor[HTML]{FFFFFF}{0.787} & \cellcolor[HTML]{89CC67}\textcolor[HTML]{000000}{0.620} & \cellcolor[HTML]{30A356}\textcolor[HTML]{FFFFFF}{0.741} \\
en-it\_IT & \cellcolor[HTML]{B5DF74}\textcolor[HTML]{000000}{0.549} & \cellcolor[HTML]{9BD469}\textcolor[HTML]{000000}{0.596} & \cellcolor[HTML]{57B65F}\textcolor[HTML]{000000}{0.691} & \cellcolor[HTML]{16914D}\textcolor[HTML]{FFFFFF}{0.780} & \cellcolor[HTML]{33A456}\textcolor[HTML]{FFFFFF}{0.735} & \cellcolor[HTML]{ADDC6F}\textcolor[HTML]{000000}{0.566} & \cellcolor[HTML]{DFF293}\textcolor[HTML]{000000}{0.470} & \cellcolor[HTML]{A2D76A}\textcolor[HTML]{000000}{0.583} & \cellcolor[HTML]{E3F399}\textcolor[HTML]{000000}{0.456} & \cellcolor[HTML]{42AC5A}\textcolor[HTML]{000000}{0.716} \\
en-ja\_JP & \cellcolor[HTML]{7AC665}\textcolor[HTML]{000000}{0.644} & \cellcolor[HTML]{69BE63}\textcolor[HTML]{000000}{0.668} & \cellcolor[HTML]{42AC5A}\textcolor[HTML]{000000}{0.717} & \cellcolor[HTML]{57B65F}\textcolor[HTML]{000000}{0.691} & \cellcolor[HTML]{279F53}\textcolor[HTML]{FFFFFF}{0.752} & \cellcolor[HTML]{87CB67}\textcolor[HTML]{000000}{0.626} & \cellcolor[HTML]{B9E176}\textcolor[HTML]{000000}{0.543} & \cellcolor[HTML]{7FC866}\textcolor[HTML]{000000}{0.637} & \cellcolor[HTML]{D3EC87}\textcolor[HTML]{000000}{0.496} & \cellcolor[HTML]{42AC5A}\textcolor[HTML]{000000}{0.715} \\
en-kn\_IN & \cellcolor[HTML]{128A49}\textcolor[HTML]{FFFFFF}{0.796} & \cellcolor[HTML]{17934E}\textcolor[HTML]{FFFFFF}{0.778} & \cellcolor[HTML]{0A7B41}\textcolor[HTML]{FFFFFF}{0.826} & \cellcolor[HTML]{FFFCBA}\textcolor[HTML]{000000}{0.379} & \cellcolor[HTML]{138C4A}\textcolor[HTML]{FFFFFF}{0.790} & \cellcolor[HTML]{45AD5B}\textcolor[HTML]{000000}{0.714} & \cellcolor[HTML]{FEEA9B}\textcolor[HTML]{000000}{0.324} & \cellcolor[HTML]{4EB15D}\textcolor[HTML]{000000}{0.703} & \cellcolor[HTML]{FFFBB8}\textcolor[HTML]{000000}{0.375} & \cellcolor[HTML]{ADDC6F}\textcolor[HTML]{000000}{0.563} \\
en-ko\_KR & \cellcolor[HTML]{78C565}\textcolor[HTML]{000000}{0.645} & \cellcolor[HTML]{6BBF64}\textcolor[HTML]{000000}{0.667} & \cellcolor[HTML]{51B35E}\textcolor[HTML]{000000}{0.699} & \cellcolor[HTML]{60BA62}\textcolor[HTML]{000000}{0.680} & \cellcolor[HTML]{18954F}\textcolor[HTML]{FFFFFF}{0.774} & \cellcolor[HTML]{7AC665}\textcolor[HTML]{000000}{0.643} & \cellcolor[HTML]{A5D86A}\textcolor[HTML]{000000}{0.580} & \cellcolor[HTML]{78C565}\textcolor[HTML]{000000}{0.648} & \cellcolor[HTML]{B7E075}\textcolor[HTML]{000000}{0.547} & \cellcolor[HTML]{30A356}\textcolor[HTML]{FFFFFF}{0.738} \\
en-lt\_LT & \cellcolor[HTML]{118848}\textcolor[HTML]{FFFFFF}{0.798} & \cellcolor[HTML]{07753E}\textcolor[HTML]{FFFFFF}{0.837} & \cellcolor[HTML]{016A38}\textcolor[HTML]{FFFFFF}{0.858} & \cellcolor[HTML]{3CA959}\textcolor[HTML]{FFFFFF}{0.726} & \cellcolor[HTML]{097940}\textcolor[HTML]{FFFFFF}{0.828} & \cellcolor[HTML]{249D53}\textcolor[HTML]{FFFFFF}{0.755} & \cellcolor[HTML]{B1DE71}\textcolor[HTML]{000000}{0.556} & \cellcolor[HTML]{15904C}\textcolor[HTML]{FFFFFF}{0.783} & \cellcolor[HTML]{96D268}\textcolor[HTML]{000000}{0.601} & \cellcolor[HTML]{1E9A51}\textcolor[HTML]{FFFFFF}{0.762} \\
en-mas\_KE & \cellcolor[HTML]{54B45F}\textcolor[HTML]{000000}{0.694} & \cellcolor[HTML]{FEEA9B}\textcolor[HTML]{000000}{0.325} & \cellcolor[HTML]{FAFDB8}\textcolor[HTML]{000000}{0.403} & \cellcolor[HTML]{F67A49}\textcolor[HTML]{000000}{0.124} & \cellcolor[HTML]{E2F397}\textcolor[HTML]{000000}{0.460} & \cellcolor[HTML]{F8FCB6}\textcolor[HTML]{000000}{0.406} & \cellcolor[HTML]{FDBB6C}\textcolor[HTML]{000000}{0.223} & \cellcolor[HTML]{F16640}\textcolor[HTML]{000000}{0.096} & \cellcolor[HTML]{A50026}\textcolor[HTML]{FFFFFF}{-0.085} & \cellcolor[HTML]{BFE47A}\textcolor[HTML]{000000}{0.533} \\
en-mr\_IN & \cellcolor[HTML]{30A356}\textcolor[HTML]{FFFFFF}{0.738} & \cellcolor[HTML]{89CC67}\textcolor[HTML]{000000}{0.622} & \cellcolor[HTML]{15904C}\textcolor[HTML]{FFFFFF}{0.785} & \cellcolor[HTML]{FBA05B}\textcolor[HTML]{000000}{0.179} & \cellcolor[HTML]{91D068}\textcolor[HTML]{000000}{0.610} & \cellcolor[HTML]{5DB961}\textcolor[HTML]{000000}{0.685} & \cellcolor[HTML]{F67A49}\textcolor[HTML]{000000}{0.124} & \cellcolor[HTML]{9BD469}\textcolor[HTML]{000000}{0.595} & \cellcolor[HTML]{DE402E}\textcolor[HTML]{FFFFFF}{0.034} & \cellcolor[HTML]{FEE999}\textcolor[HTML]{000000}{0.320} \\
en-ro\_RO & \cellcolor[HTML]{7FC866}\textcolor[HTML]{000000}{0.634} & \cellcolor[HTML]{4BB05C}\textcolor[HTML]{000000}{0.707} & \cellcolor[HTML]{2AA054}\textcolor[HTML]{FFFFFF}{0.748} & \cellcolor[HTML]{128A49}\textcolor[HTML]{FFFFFF}{0.796} & \cellcolor[HTML]{249D53}\textcolor[HTML]{FFFFFF}{0.753} & \cellcolor[HTML]{89CC67}\textcolor[HTML]{000000}{0.619} & \cellcolor[HTML]{B7E075}\textcolor[HTML]{000000}{0.546} & \cellcolor[HTML]{78C565}\textcolor[HTML]{000000}{0.648} & \cellcolor[HTML]{AFDD70}\textcolor[HTML]{000000}{0.561} & \cellcolor[HTML]{1E9A51}\textcolor[HTML]{FFFFFF}{0.762} \\
en-ru\_RU & \cellcolor[HTML]{A5D86A}\textcolor[HTML]{000000}{0.580} & \cellcolor[HTML]{78C565}\textcolor[HTML]{000000}{0.647} & \cellcolor[HTML]{63BC62}\textcolor[HTML]{000000}{0.677} & \cellcolor[HTML]{36A657}\textcolor[HTML]{FFFFFF}{0.731} & \cellcolor[HTML]{4BB05C}\textcolor[HTML]{000000}{0.707} & \cellcolor[HTML]{BDE379}\textcolor[HTML]{000000}{0.534} & \cellcolor[HTML]{D1EC86}\textcolor[HTML]{000000}{0.499} & \cellcolor[HTML]{A7D96B}\textcolor[HTML]{000000}{0.575} & \cellcolor[HTML]{D1EC86}\textcolor[HTML]{000000}{0.500} & \cellcolor[HTML]{2DA155}\textcolor[HTML]{FFFFFF}{0.742} \\
en-sr\_Cyrl\_RS & \cellcolor[HTML]{51B35E}\textcolor[HTML]{000000}{0.699} & \cellcolor[HTML]{17934E}\textcolor[HTML]{FFFFFF}{0.775} & \cellcolor[HTML]{45AD5B}\textcolor[HTML]{000000}{0.714} & \cellcolor[HTML]{2DA155}\textcolor[HTML]{FFFFFF}{0.743} & \cellcolor[HTML]{33A456}\textcolor[HTML]{FFFFFF}{0.737} & \cellcolor[HTML]{A7D96B}\textcolor[HTML]{000000}{0.577} & \cellcolor[HTML]{A7D96B}\textcolor[HTML]{000000}{0.577} & \cellcolor[HTML]{73C264}\textcolor[HTML]{000000}{0.655} & \cellcolor[HTML]{6BBF64}\textcolor[HTML]{000000}{0.664} & \cellcolor[HTML]{54B45F}\textcolor[HTML]{000000}{0.696} \\
en-sr\_Latn\_RS & \cellcolor[HTML]{36A657}\textcolor[HTML]{FFFFFF}{0.731} & \cellcolor[HTML]{148E4B}\textcolor[HTML]{FFFFFF}{0.789} & \cellcolor[HTML]{3CA959}\textcolor[HTML]{FFFFFF}{0.724} & \cellcolor[HTML]{57B65F}\textcolor[HTML]{000000}{0.691} & \cellcolor[HTML]{128A49}\textcolor[HTML]{FFFFFF}{0.797} & \cellcolor[HTML]{66BD63}\textcolor[HTML]{000000}{0.672} & \cellcolor[HTML]{BFE47A}\textcolor[HTML]{000000}{0.532} & \cellcolor[HTML]{6EC064}\textcolor[HTML]{000000}{0.661} & \cellcolor[HTML]{ADDC6F}\textcolor[HTML]{000000}{0.564} & \cellcolor[HTML]{91D068}\textcolor[HTML]{000000}{0.610} \\
en-sv\_SE & \cellcolor[HTML]{6EC064}\textcolor[HTML]{000000}{0.662} & \cellcolor[HTML]{30A356}\textcolor[HTML]{FFFFFF}{0.738} & \cellcolor[HTML]{17934E}\textcolor[HTML]{FFFFFF}{0.777} & \cellcolor[HTML]{097940}\textcolor[HTML]{FFFFFF}{0.830} & \cellcolor[HTML]{16914D}\textcolor[HTML]{FFFFFF}{0.780} & \cellcolor[HTML]{7FC866}\textcolor[HTML]{000000}{0.634} & \cellcolor[HTML]{A9DA6C}\textcolor[HTML]{000000}{0.573} & \cellcolor[HTML]{4BB05C}\textcolor[HTML]{000000}{0.706} & \cellcolor[HTML]{7AC665}\textcolor[HTML]{000000}{0.641} & \cellcolor[HTML]{118848}\textcolor[HTML]{FFFFFF}{0.798} \\
en-th\_TH & \cellcolor[HTML]{0B7D42}\textcolor[HTML]{FFFFFF}{0.821} & \cellcolor[HTML]{05713C}\textcolor[HTML]{FFFFFF}{0.845} & \cellcolor[HTML]{07753E}\textcolor[HTML]{FFFFFF}{0.837} & \cellcolor[HTML]{69BE63}\textcolor[HTML]{000000}{0.667} & \cellcolor[HTML]{08773F}\textcolor[HTML]{FFFFFF}{0.831} & \cellcolor[HTML]{17934E}\textcolor[HTML]{FFFFFF}{0.775} & \cellcolor[HTML]{A2D76A}\textcolor[HTML]{000000}{0.585} & \cellcolor[HTML]{118848}\textcolor[HTML]{FFFFFF}{0.797} & \cellcolor[HTML]{7DC765}\textcolor[HTML]{000000}{0.639} & \cellcolor[HTML]{33A456}\textcolor[HTML]{FFFFFF}{0.735} \\
en-tr\_TR & \cellcolor[HTML]{4BB05C}\textcolor[HTML]{000000}{0.704} & \cellcolor[HTML]{219C52}\textcolor[HTML]{FFFFFF}{0.758} & \cellcolor[HTML]{45AD5B}\textcolor[HTML]{000000}{0.713} & \cellcolor[HTML]{75C465}\textcolor[HTML]{000000}{0.649} & \cellcolor[HTML]{15904C}\textcolor[HTML]{FFFFFF}{0.782} & \cellcolor[HTML]{89CC67}\textcolor[HTML]{000000}{0.619} & \cellcolor[HTML]{D1EC86}\textcolor[HTML]{000000}{0.498} & \cellcolor[HTML]{7AC665}\textcolor[HTML]{000000}{0.642} & \cellcolor[HTML]{C7E77F}\textcolor[HTML]{000000}{0.516} & \cellcolor[HTML]{30A356}\textcolor[HTML]{FFFFFF}{0.738} \\
en-uk\_UA & \cellcolor[HTML]{78C565}\textcolor[HTML]{000000}{0.646} & \cellcolor[HTML]{4BB05C}\textcolor[HTML]{000000}{0.704} & \cellcolor[HTML]{2AA054}\textcolor[HTML]{FFFFFF}{0.745} & \cellcolor[HTML]{1B9950}\textcolor[HTML]{FFFFFF}{0.763} & \cellcolor[HTML]{279F53}\textcolor[HTML]{FFFFFF}{0.752} & \cellcolor[HTML]{9BD469}\textcolor[HTML]{000000}{0.594} & \cellcolor[HTML]{B5DF74}\textcolor[HTML]{000000}{0.550} & \cellcolor[HTML]{7AC665}\textcolor[HTML]{000000}{0.643} & \cellcolor[HTML]{ABDB6D}\textcolor[HTML]{000000}{0.568} & \cellcolor[HTML]{18954F}\textcolor[HTML]{FFFFFF}{0.771} \\
en-vi\_VN & \cellcolor[HTML]{45AD5B}\textcolor[HTML]{000000}{0.714} & \cellcolor[HTML]{1E9A51}\textcolor[HTML]{FFFFFF}{0.762} & \cellcolor[HTML]{1E9A51}\textcolor[HTML]{FFFFFF}{0.762} & \cellcolor[HTML]{7DC765}\textcolor[HTML]{000000}{0.641} & \cellcolor[HTML]{097940}\textcolor[HTML]{FFFFFF}{0.827} & \cellcolor[HTML]{5DB961}\textcolor[HTML]{000000}{0.685} & \cellcolor[HTML]{CDEA83}\textcolor[HTML]{000000}{0.507} & \cellcolor[HTML]{51B35E}\textcolor[HTML]{000000}{0.698} & \cellcolor[HTML]{C5E67E}\textcolor[HTML]{000000}{0.522} & \cellcolor[HTML]{2DA155}\textcolor[HTML]{FFFFFF}{0.743} \\
en-zh\_CN & \cellcolor[HTML]{B1DE71}\textcolor[HTML]{000000}{0.557} & \cellcolor[HTML]{82C966}\textcolor[HTML]{000000}{0.633} & \cellcolor[HTML]{73C264}\textcolor[HTML]{000000}{0.653} & \cellcolor[HTML]{73C264}\textcolor[HTML]{000000}{0.653} & \cellcolor[HTML]{5AB760}\textcolor[HTML]{000000}{0.688} & \cellcolor[HTML]{A2D76A}\textcolor[HTML]{000000}{0.584} & \cellcolor[HTML]{C3E67D}\textcolor[HTML]{000000}{0.525} & \cellcolor[HTML]{A2D76A}\textcolor[HTML]{000000}{0.584} & \cellcolor[HTML]{C7E77F}\textcolor[HTML]{000000}{0.518} & \cellcolor[HTML]{2DA155}\textcolor[HTML]{FFFFFF}{0.744} \\
ja-zh\_CN & \cellcolor[HTML]{CBE982}\textcolor[HTML]{000000}{0.508} & \cellcolor[HTML]{B3DF72}\textcolor[HTML]{000000}{0.553} & \cellcolor[HTML]{70C164}\textcolor[HTML]{000000}{0.658} & \cellcolor[HTML]{33A456}\textcolor[HTML]{FFFFFF}{0.735} & \cellcolor[HTML]{16914D}\textcolor[HTML]{FFFFFF}{0.779} & \cellcolor[HTML]{7DC765}\textcolor[HTML]{000000}{0.639} & \cellcolor[HTML]{BFE47A}\textcolor[HTML]{000000}{0.532} & \cellcolor[HTML]{7DC765}\textcolor[HTML]{000000}{0.639} & \cellcolor[HTML]{B9E176}\textcolor[HTML]{000000}{0.545} & \cellcolor[HTML]{42AC5A}\textcolor[HTML]{000000}{0.718} \\
\bottomrule
\end{tabular}

\end{table*}


\end{document}

